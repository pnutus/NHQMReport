\documentclass[tikz=true]{standalone}
\usepackage{../../main/NHQM}

\begin{document}


   \newcommand{\circdist}{1.2}
   \newcommand{\circrad}{2}
   \begin{tikzpicture}[even odd rule]
     \foreach \angle/\colour in {90/red,-30/blue,210/green}
       \draw [fill=\colour] 
         (\angle:\circdist) circle (\circrad) circle (\circrad+0.5);

     \begin{scope}
     \clip (-30:\circdist) circle (1);
     \draw [fill=red] (90:\circdist) circle (\circrad) circle (\circrad+0.5);
     \end{scope}

     \begin{scope}
     	\clip (90:\circdist) ++ (180:\circrad) circle (1);
     	\draw [fill=red] (90:\circdist) circle (\circrad) circle (\circrad+0.5);	 
     \end{scope}
     
	  % \begin{scope}
	  % 		 \draw [opacity = 0](1,0) -- (0,-4.5) -- (-1,0) -- (0,-4.5) -- cycle;
	  % 			 %OBS OLA, detta är dålig lösning, gör inte det här hemma DÅÅÅLIG LÖSNING
	  % \end{scope}

	 
   \end{tikzpicture}
   
   \end{document}