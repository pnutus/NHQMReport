\documentclass[12pt,a4paper]{report}
\usepackage[english]{babel}

\usepackage{NHQM}

\begin{document}
  
\numberwithin{equation}{chapter}
\numberwithin{figure}{chapter}

\pagenumbering{gobble}
\listoftodos


\title{Resonances in Helium Nuclei\\ 
\todo{There is a proper front page somewhere, let's fix that}
\Large Bachelor Thesis in Physics}
\author{Jonathan Bengtsson \and Ola Embréus \and Vincent Ericsson \and Pontus Granström \and Nils Wireklint}
\date{\today}



\maketitle

\newpage
\begin{abstract}
\todo{Begin the first sentence by other thing than This thesis, this report, etc.}
This thesis studies loosely bound quantum systems by expanding their wavefunctions in a complex basis using numerical methods. This is made by studying
resonances in \He{5} and \He{6} nuclei. We explain the theory of basis
expansion and expand the \He{5} nucleus wave functions in the spherical harmonic oscillator in coordinate space
and plane wave bases in momentum space. We model our systems as a \He{4} nucleus with one and two orbiting neutrons, respectively.
To find the resonances and thus reproduce experimental results, we extend the spherical wave basis into the complex plane according to the 
theory of Tore Berggren. \todo{Better connection between Berggren and his basis.} The Berggren basis is then used as a single particle
basis for the \He{6} problem. \todo{Should we cut down on the amounts of We ...} We also explain many-body theory and the second
quantization formalism, providing details for its implementation.


Finally, a
Monte Carlo approach to reducing computation time for many-body 
calculations is investigated.
\todo{We have -ing form on five places in abstract, that is too many.}
\todo{Read through and correct the abstract.}
\end{abstract}
\newpage

\pagenumbering{roman}

\tableofcontents
\newpage

\pagenumbering{arabic}
\setcounter{page}{1}

\chapter{Introduction}
\label{cha:introduction}
\documentclass[../main/report.tex]{subfiles}
\begin{document}
\chapter{Introduction}
\label{cha:introduction}






The properties of a quantum mechanical system are determined by its Hamiltonian, consisting of a kinetic energy term and a potential term.
Particles in a potential well with infinitely high walls form localized bound states.
Such a system is called a \emph{closed quantum system}, since the number of particles is conserved and the particles are localized in a finite region. 
The energy of a closed quantum system can only take on discrete values, as illustrated in \cref{fig:closed_quantum_system} with the harmonic oscillator potential.

An \emph{open quantum system}, on the other hand, portrayed in \cref{fig:open_quantum_system}, is a system with a finite potential.
For a system with a vanishing potential particles can enter and exit the system and, consequently, there are unbound states. 
For a trivial potential these are just free particle states but with a non-negligible potential, these are denoted \emph{scattering states}.
Depending on the depth of the potential well of the open system, there can be a finite number of bound states.
The number of scattering states, however, is infinite. They can take on any positive value of energy and are said to be in the energy \emph{continuum}.

\begin{figure}[b!]
  \newcommand{\depth}{3}
  \pgfdeclareverticalshading{resonance}{100bp}
   {color(0bp)=(white); color(50bp)=(black!70); color(100bp)=(white)}
  \subfloat[A closed quantum system]{
    \label{fig:closed_quantum_system}
    \begin{tikzpicture}[
      scale = 1.6,
      domain=0:2,
      samples=200,
      ]
      \draw[->] (0, 0) -- (2, 0) node[above] {$r$};
      \draw[->] (0, -2.1) -- (0, 2) node[right] {$V$};
      \draw plot (\x, {\x*\x - 2});
      \foreach \y in {-1.3, -0.6, ..., 2} {
        \draw (0, \y) -- ($ ({ sqrt(\y + 2)} , \y) $);
        \draw[shorten >=0.5mm] (-1, 0)
          -- (0, \y);
      }
      \node[left, align=center, fill=white] 
        at (-0.2, 0)  {Bound\\states};
    \end{tikzpicture}
    
  } 
  \subfloat[An open quantum system]{
    \label{fig:open_quantum_system}
    \begin{tikzpicture}[
      scale=1.6,
      domain=0:2.95,
      samples=200,
      ]
      \begin{scope}[->]
        \draw[decorate, decoration={snake, post length=1mm}] 
          (0, 1) -- (3, 1);
        \draw[decorate, decoration={snake, post length=1mm,
                                    segment length = 2mm}] 
          (0, 1.4) -- (2.7, 1.4);
      \end{scope}
      
      \shade[shading=resonance]
        (0,0.5) ++ (0,-0.02) rectangle +(3,0.04);
      \begin{scope}[left, fill=white, align=right]
        \node at (0, 1.2) {Scattering\\states};
        \node at (0, 0.5) {Resonance};
      \end{scope}
     
      \draw[->] (0, 0) node[left] {$0$} -- (3, 0) node[below] {$r$};
      \draw[->] (0, -2.1) -- (0, 2) node[right] {$V$};
      \draw plot (\x,{-2/(1 + exp((\x-1)/0.3))});
      \foreach \x/\y in {1/-1, 1.33/-0.5} {
        \draw (0, \y)-- (\x, \y);
      }
      \node[left, align=center] 
        at (0,-0.75)  {Bound\\states};
        
        
      \draw[decorate, decoration={brace}] 
        (3.1,2) -- +(0,-2) node[midway, xshift=0.4cm, rotate=-90] {Continuum};
    \end{tikzpicture}
    
  }
  \caption{An open and a closed quantum system. The open system has an infinite number of bound, localized states, whereas the closed system has unbound scattering states and resonances in addition to a finite number of bound states.}
  \label{fig:potentials}
\end{figure}

In addition to the bound and unbound states, open quantum systems can harbor \emph{resonances}. 
These are \emph{quasi-bound} states that are neither bound nor unbound, but exhibit properties of both. 
They appear in the continuum, like unbound states, but are localized, like bound states.
However, the wavefunction of a resonance is only localized for a finite amount of time, as opposed to a bound, \emph{stationary} state that forever stays the same.
A resonance can be described with a complex energy, as the following argument suggests.

The state of a particle is described by its wavefunction $\psi$, which can be written as the product of a function of time and position
\begin{eq}
  \psi(t, \vec{r}) = \psi_t(t)\psi_{\vec{r}}(\vec{r}).
\end{eq}
The wavefunction evolves according to the \emph{Time-Dependent Schrödinger Equation} (TDSE)
\begin{eq}
  \label{eq:schrödinger}
  i\hbar\ddt\ket\psi = H \ket\psi.
\end{eq}
An eigenstate of the Hamiltonian $H$ with energy $E$, i.e. a solution to the \emph{Time-Independent Schrödinger Equation} (TISE)
\begin{eq}
  H \ket\psi = E \ket\psi
\end{eq}
has the simple time evolution
\begin{eq}
	\psi(t, \vec{r})
	= 
  \exp\p{-\frac{iE}{\hbar}t}\psi(0, \vec{r}).
\end{eq}
With the energy $E$ real, the exponential factor is just a phase 
and the probability $|\psi(t, \vec{r})|^2$ of finding the particle at a given $\vec{r}$ is unchanged over time. 
However, if we let the energy be complex
\begin{eq}
	E = E_0 - i\frac{\Gamma}{2},
\end{eq}
we get
\begin{eq}
  |\psi(t, \vec{r})|^2 
  =
  \absq{
    \exp\p{-\frac{iE_0}{\hbar} t} 
    \exp\p{- \frac{\Gamma}{2\hbar} t} 
    \psi(0, \vec{r})
  }
  =
  \exp\p{-\frac{\Gamma}{\hbar} t} \absq{\psi(0, \vec{r})},
\end{eq} 
describing a resonance state decaying exponentially with half-life $t_{1/2}=\hbar\ln 2/\Gamma$. This parameter $\Gamma$ is called the \emph{width} of the resonance.

We want to use the simpler formalism of the TISE, as opposed to the more general TDSE, and we see that this is possible by letting the resonance have a complex energy.
However, complex eigenvalues pose a problem in standard quantum mechanics, where operators are postulated to be Hermitian.
Hermitian operators can only have real eigenvalues, and are thus insufficient for treating resonances.

The aim of this thesis is to present methods for describing resonances using the TISE in a complex-energy framework.
The motivation behind the development of these methods is their ability to describe loosely bound or unbound systems, where other methods only work for stable, strongly bound systems.
The methods are used to study the simple nuclear systems \He{5} and \He{6}, comparing the results to experimental data.
The He nuclei are relevant because the ground state of \He{5} and the first excited state of both \He{5} and \He{6} are resonances. 
Despite the unbound nature of \He{5}, the ground state of \He{6} is actually bound, an example of a \emph{Borromean} system.

We solve the TISE numerically, but have in this thesis chosen to focus on the mathematical and physical aspects as the implementation has been rather straightforward in most cases.
Consequently, there will be no code nor pseudocode in the thesis, although the code we have written to produce the results is freely available at
\begin{quote}
  \url{https://github.com/pnutus/NHQM}.
\end{quote}
We used the \emph{Python} programming language with the libraries \emph{NumPy} and \emph{SciPy} to perform computations. 
The figures in the thesis were made using the \emph{TikZ} and \emph{pgfplots} LaTeX libraries together with the Python code.

The thesis can conceptually be divided into two parts, the first covering resonances in a simple two-body problem and the second part covering the first steps toward more complicated many-body systems. 
In \cref{cha:basis_expansion} the \emph{basis expansion} method for solving the Schrödinger equation is introduced.
The basis expansion method is then used in \cref{cha:two-body} to study a loosely bound two-body nuclear system, the \He{5} nucleus.
In \Cref{cha:berggren} we use the Berggren basis to reproduce the resonance in \He{5}.

\Cref{cha:many-body} is an introduction to many-body theory, focusing on fermionic systems. 
The many-body theory is then utilized in \cref{cha:three-body} 
to study a three-body problem, specifically the \He{6} nucleus.  
Finally, \cref{cha:outlook} is an outlook discussing further development of the methods.

\end{document}



\chapter{Basis Expansion}
\label{cha:basis expansion}
\documentclass[../main/report.tex]{subfiles}
\begin{document}
	
\chapter{The Basis Expansion Method}
\label{cha:basis_expansion}

\todo{Intro to chapter}
\todo{Subsection: Basis of basics expansion}

We want to study loosely bound quantum systems by solving the \emph{Time Independent Schrödinger Equation} (TISE)
\begin{eq}
  \label{eq:TISE}
  H \ket\psi = E \ket\psi,
\end{eq}
commonly written in the position basis as
\begin{eq}
  \label{eq:TISEpos}
  \p{-\frac{\hbar^2}{2m}\nabla^2 + V(\vec{r})}\psi(\vec{r}) = E\psi(\vec{r}).
\end{eq}
For the nuclear systems to be treated, the TISE has no known analytical solutions, and we need to use numerical methods to solve it.
We will use the basis expansion method, writing the equation as an eigenvalue problem
\begin{eq}
  \label{eq:matrix equation}
  \sum_j H_{ij}\psi_j = E \psi_i
\end{eq}
with a finite matrix $H$ that we can diagonalize to find the eigenvalues $E$.

To write the TISE as a matrix equation we use \emph{basis expansion}. 
Basis expansion is how we make sense of the abstract Hilbert spaces, operators and state vectors of Quantum Mechanics (QM).
By expanding these abstract objects in a basis, we can relate them to the physical world. 
For example, \cref{eq:TISEpos} is the TISE for a single particle, expanded in the position basis. 
We will not expand our problems in the position basis, but it will still be important, since it is the most natural basis to express the potential in.

Before we begin, we briefly recap some well known QM facts. 
First we need a \emph{complete basis}, either discrete, $\ket{n}$, or continuous, $\ket{x}$. 
A complete basis means that any state $\ket\psi$ can be written as a linear combination of the basis states
\begin{eq}
  \label{eq:lincomb}	
  \ket\psi = \sum_n \psi_n \ket{n}
  \quad
  \textup{or}
  \quad
  \ket\psi = \fint{x} \psi(x) \ket{x}.
\end{eq}
The  complete bases we will use in this thesis are the \emph{position basis} $\ket{\vec{r}}$, the \emph{momentum basis} $\ket{\vec{k}}$, the \emph{harmonic oscillator basis} $\ket{nlm}$ and the elusive \emph{Berggren basis} \cite{berggren}. 
All these bases are orthonormal, i.e.~all the basis vectors satisfy 
\begin{eq}
  \braket{n}{n'} = \delta_{nn'}
  \quad
  \textup{or}
  \quad
  \braket{x}{x'} = \delta(x - x'),
\end{eq}
depending on if the base is discrete or continuous.
With a complete basis $\ket{n}$, we get the very useful \emph{completeness relation}
\begin{eq}
  I = \sum_n \ket{n} \bra{n}
  \quad
  \textup{or}
  \quad
  I = \fint{x} \ket{x}\bra{x},
\end{eq}
where $I$ is the identity operator. This relation can thus be inserted anywhere in any equation, and will find frequent use in this thesis.

Let us now expand the TISE in the abstract $\ket{n}$ basis. We start by inserting the completeness relation for $\ket{n}$ in \cref{eq:TISE}
\begin{eq}
  \label{eq:expand1}
  H
  \p{
    \sum_{n'} \ket{n'} \bra{n'}
  }
  \ket\psi
  =
  \sum_{n'} H \ket{n'} \braket{n'}{\psi}
  =
  E \ket\psi.
\end{eq}
Multiplying \cref{eq:lincomb} with $\bra{n}$ from the left and using orthonormality, we see that $\braket{n'}{\psi} = \psi_{n'}$. If we now close \cref{eq:expand1} with $\bra{n}$ on the left
\begin{eq}
  \label{eq:expand2}
  \sum_{n'} \bra{n} H \ket{n'} \psi_{n'}
  = 
  E \braket{n}{\psi},
\end{eq}
and write $H_{nn'} = \bra{n} H \ket{n'}$, we get
\begin{eq}
  \label{eq:expand3}
  \sum_{n'} H_{nn'} \psi_{n'} = E \psi_n,
\end{eq}
which is equivalent to the matrix equation \cref{eq:matrix equation}. This is 
the basic method of expanding the TISE in a basis.

\section{Spherical Symmetry}
\label{sec:spherical symmetry}

We limit ourselves to spherically symmetric systems, i.e. systems with a potential $V(r)$ that only depends on the radial distance $r$.
If we considered three-dimensional systems with arbitrary potentials, the matrices would be very large and solving the problem would become infeasible.

Spherical symmetry allows us to write the wavefunction $\psi(\vec{r})$ as a product of a radial wavefunction $R(r)$ and a spherical harmonic $Y_l^m(\Omega_r)$
\begin{eq}
  \psi(\vec{r}) = R_{nl}(r) Y_l^m(\Omega_r).
\end{eq}
Here $l$ and $m$ are the quantum numbers for the orbital angular momentum and its projection along an arbitrary $z$-axis.
For the matrix elements we find, using the orthonormality of the spherical harmonics,
\begin{eq}
  \bra{nlm}V\ket{n'l'm'} 
  &= \fint[0][\inf]{r} 
    r^2 \conj{R_{nl}(r)}R_{n'l'}(r)V(r) 
    \fint{\Omega_r} 
      \conj{Y_l^m(\Omega_r)}Y_{l'}^{m'}(\Omega_r)
  \\ & = 
  \delta_{mm'}\delta_{ll'}\fint[0][\inf]{r} 
    r^2 \conj{R_{nl}(r)}R_{n'l'}(r)V(r)
\end{eq}
meaning that the matrix will be \emph{block diagonal}, illustrated in \cref{fig:bmatrix}. This means that systems with different $l$ and $m$ can be treated separately. We say that \emph{$H$ is diagonal in $l$ and $m$}.

\begin{figure}
\center
\includegraphics{../figures/block_matrix/matrix.pdf}	
\caption{An illustration of a block diagonal matrix. The eigenvalues of different blocks are independent of each other. Thus the blocks can be diagonalized separately.}
\label{fig:bmatrix}
\end{figure}

\section{The Harmonic Oscillator Basis}
\label{sec:harmosc}

We now expand the TISE in the spherically symmetric Harmonic Oscillator (HO) basis. The basis consists of the eigenstates $\ket{nlm}$ of the HO Hamiltonian
\begin{eq}
  \label{eq:HO_hamiltonian}
  H\sub{HO} = \frac{p^2}{2\mu} + \frac{\mu\omega^2 r^2}{2},
\end{eq}
where $\mu$ is the mass and $\omega$ is the angular frequency of the oscillator. 
The expansion procedure is the same as in \cref{eq:expand1,eq:expand2,eq:expand3} and gives us
\begin{eq}
  \sum_{n'l'm'} \bra{nlm} H \ket{n'l'm'} \psi_{n'l'm'} = E \psi_{nlm}
\end{eq}
and if we use the fact that $H$ is diagonal in $l$ and $m$ we get
\begin{eq}
  \sum_{n'} \bra{nlm} H \ket{n'lm} \psi_{n'lm} = E\psi_{nlm}.
\end{eq}
We now have a matrix equation, but we need to find the matrix elements $\bra{nlm} H \ket{n'lm}$. This require some calculation (details are in \cref{sec:HO matrix elements}) and the result is
\begin{eq}
  \label{eq:HO_matrix_elements}
  &
  \bra{nlm} H \ket{n'lm} =
	\frac{\hbar\omega}{2}
	\left(
    \p{2n+l+\frac{3}{2}} \delta_{nn'}
    +
		\sqrt{n(n+l+\frac{1}{2})} \delta_{n,n'-1}\right.
		\\ & + 
		\left.\sqrt{n'(n'+l+\frac{1}{2})} \delta_{n',n-1} 
	\right)
	+
	\fint[0][\inf]{r} 
    r^2 R_{nl}(r) V(r) R_{n'l}(r)
\end{eq}
where $R_{nl}$ are the radial wavefunctions of the harmonic oscillator,
\begin{eq}
  \label{eq:HO_radial_wavefunction}
	R_{nl}(r) 
	= 
 \sqrt{\frac{2 \, (\frac{n-l}{2})! }{\Gamma(\frac{n+l}{2}+\frac{3}{2})}}
\frac{r^l}{r_0^{l+{3\over 2}}} \exp\p{-\frac{r^2}{2r_0^2}}
	L_{(n-l) / 2}^{(l+\frac{1}{2})}\p{\frac{r^2}{r_0^2}},
\end{eq}
$r_0 = \sqrt{\hbar/\mu\omega}$ is the range and $L_\nu^\lambda(x)$ are the generalized Laguerre polynomials.
The radial wavefunction $R(r)$ of a state will be expressed as a linear combination of the harmonic oscillator radial wavefunctions:
\begin{eq}
  R(r) = \sum_n \psi_{nl} R_{nl}(r).
\end{eq}

\section{The Momentum Basis}
\label{sec:mom_space}
The momentum, or plane wave, basis has eigenstates $\ket{\vec{k}}$, each describing a free particle with momentum $\vec{p}$ or wavevector $\vec{k} = \vec{p}/\hbar$. We will only use $\vec{k}$ and refer to it as momentum.

The expansion is done in the same way as before, giving us
\begin{eq}
  \int \rd^3 \vec{k}' \bra{\vec{k}} H \ket{\vec{k}'} \Phi(\vec{k}')
  &= 
  E\Phi(\vec{k}),
\end{eq}
where we denote the momentum space wavefunctions with $\Phi$. 
As in position space, these can be separated into a radial and angular part.
This is shown in \cref{app:radial_mom_TISE} along with the fact that the Schrödinger equation can be written as
\begin{align}
  \label{eq:radial mom space TISE}
  H\phi(k)
  & =
  \frac{k^2}{2\mu}\phi(k) + \fint[0][\inf]{k'} k'^2 V(k,k') \phi(k') 
  =
  E\phi(k)
  \\
  V(k,k') 
  & = 
  \frac{2}{\pi}\fint[0][\inf]{r} r^2 V(r) j_l(kr) j_l(k'r),
\end{align}
\todo[inline]{No align, where before V(k,k)?}
where $\phi(k)$ is the radial part of the momentum space wavefunction
and $j_l(kr)$ are the spherical bessel functions of order $l$. 
The momentum space radial wavefunction $\phi(k)$ is related to the position space radial wavefunction by
\begin{eq}
  R(r)=i^l\sqrt{\frac{2}{\pi}} \fint[0][\inf]{k} k^2 \phi(k)j_l(kr).
  \label{eq:radial wavefunction}
\end{eq}

\subsection{Discretization and Symmetrization}
\label{sec:discretization}
The integral equation \cref{eq:radial mom space TISE} can be rewritten as a matrix equation through discretization, turning the integral into a sum over a finite set of points $k_j$ and $\rd{k}$ into a set of weights $w_j$:
\begin{eq}
  \label{eq:discrete_momentum}
  \frac{k_i^2}{2\mu} \phi(k_i)
  +
  \sum_{j=1}^N w_j
    k_j^2 V(k_i,k_j)
  \phi(k_j)
  =
  E \phi(k_i)
  .
\end{eq}
A particular set of points and corresponding weights is called a \emph{quadrature}, and the choice of quadrature greatly impacts the precision of the result. 
A naïve quadrature with evenly spaced $k_j = j\Delta k$ and a constant weight $w_j=\Delta k$ converges slowly, and should not be used.
We instead use the Gauss-Legendre quadrature, for details see \cref{app:gauss-legendre}.

With this discretization the Schrödinger equation may be written as 
\begin{eq}
  \sum_j H_{ij} \phi(k_j) &= E \phi(k_i)
\end{eq}
where
\begin{align}
  \label{eq:mom_matrix}
  H_{ij} &= \frac{k_i^2}{2\mu}\delta_{ij} + w_jk_j^2 V_{ij} \\
  \label{eq:potential matrix}
  V_{ij} &= \frac{2}{\pi} \fint[0][\inf]{r} r^2 V(r) j_l(k_i r) j_l(k_j r).
\end{align}

Because of the $k_j^2$ in the second term of the matrix elements 
\cref{eq:mom_matrix}, the $H_{ij}$ matrix will not be symmetric. 
Working with a symmetric matrix is faster, which will be explained in the following section. 
We perform the transformation
\begin{eq}
  \phi(k_i) &\mapsto
  \phi'(k_i) =  \sqrt{w_i} k_i \phi(k_i)
  \\
  H_{ij} &\mapsto
  H_{ij}' 
  = 
  \sqrt{\frac{w_i}{w_j}} \frac{k_i}{k_j}H_{ij},
\end{eq}
which gives us a symmetric matrix
\begin{eq}
  \label{eq:plane_wave_matrix_elements}
  H_{ij}' = \frac{k_i^2}{2\mu}\delta_{ij} + \sqrt{w_i w_j}k_i k_j V_{ij}.
\end{eq}
The Schrödinger equation then becomes
\begin{eq}
  \sum_j H'_{ij}\phi'(k_j) = E\phi'(k_j),
\end{eq}
with the same eigenvalues $E$, meaning that we can work with the symmetric $H'_{ij}$ matrix.
Another benefit of the symmetrization is that the norm of the $\phi'(k_i)$ incorporates the weights $w_i k_i^2$.
The real space radial wavefunction $R(r)$ is expressed in terms of $\phi'(k_j)$ as
\begin{eq}
  R(r)
  =
  i^l 
  \sqrt{\frac{2}{\pi}}
  \sum_{j=1}^N 
    \sqrt{w_j}k_j \phi'_j j_l(k_j r).
\end{eq}

\section{Numerical Considerations}
\todo{calculation vs computation (entire report)}

In order to perform basis expansion on a computer, we need to consider the numerical aspects of the problem. 
This includes truncation of the basis, matrix size reduction, numerical integration and eigensolver optimizations.
The momentum basis is continuous and thus requires special treatment.

The $\ket{nlm}$ and $\ket{k}$ bases are infinite, so we truncate them by only including a finite number $N$ of states in the basis. 
For an orthonormal basis, the best approximation is to include the $N$ first states.
The more states we include in the basis, the more accurate results we get.

The truncation gives us an $N \times N$ Hamiltonian matrix $H$ and the TISE can be written in matrix notation as
\begin{eq}
  \label{eq:matrix eq}
  H\psi = E\psi,
\end{eq}
where the $\psi$ are eigenvectors. We compute the matrix elements $H_{ij}$ with \cref{eq:HO_matrix_elements,eq:plane_wave_matrix_elements}, carrying out the integrals with the Gauss-Legendre quadrature rule (see \cref{app:gauss-legendre}) and setting the upper limit to a finite number.
If the matrix is hermitian or symmetric, we only need to consider the elements in the upper triangle including the diagonal, roughly halving the number of computed elements. 

When the matrix elements have been computed, the matrix is diagonalized using a standard eigensolver algorithm. For hermitian matrices there are specialized, faster algorithms.

\end{document}

\chapter{The \He{5} Nucleus}
\label{cha:he5}
We view the \He{5} nucleus as an alpha particle (\He{4}) interacting with a valence neutron.
Since the exact nucleon-nucleon interaction is unknown, we use the established
Woods-Saxon potential as an approximation. The Woods-Saxon potential (\cref{fig:woods-saxons}) is given by
\todo{What's the 4 for?}
\todo{From where do we get our $V_0$ and $V\sub{so}$?}
\begin{eq}
	V(r)=
	V_0f(r) - 4V\sub{so}\vec{l}\cdot\vec{s}\frac{1}{r}\frac{df}{dr}
\end{eq}
where 
\begin{eq}
	f(r)=\frac{1}{1+\exp\p{\frac{r-r_0}{d}}}.
\end{eq}
We use the depth $V_0 = \SI{-47}{MeV}$, spin-orbit coupling strength $V\sub{so} = \SI{-7.5}{Mev}$, range $r_0 = \SI{2}{fm}$ and diffuseness $d = \SI{0.65}{fm}$.
The spin-orbit coupling can be either attractive or repulsive depending on how the angular momenta couples
\begin{eq}
  \vec{l}\cdot\vec{s} 
  = 
  \frac{1}{2}
  \bigp{
    j(j+1)-l(l+1)-s(s+1)
  }
  =
  \begin{cases}
    l,    &\text{ if } j = l + \frac{1}{2}\\
    -l-1, &\text{ if } j = l - \frac{1}{2}\\
  \end{cases}
  .
\end{eq}

Since we approximate the system as a spherically symmetric interaction 
between two particles, we can reduce the problem to a one-dimensional equation by using the relative coordinate $r = r_\alpha - r_n$ and introducing the reduced mass
\begin{eq}
  \mu = \frac{m_\alpha m_n}{m_\alpha + m_n}.
\end{eq}

\begin{figure}
  \centering
  \begin{tikzpicture}
    \begin{axis}[
      domain = 0:5.8, 
      xmax = 5.9,
      ymin = -47, ymax = 9,
      xlabel = $r/\b{\si{fm}}$, ylabel = $V/\b{\si{MeV}}$,
      axis x line = middle,
      axis y line = left
      ]
      \addplot[black] {-47/(1 + e^((x-2)/0.65))};
    \end{axis}
  \end{tikzpicture}
  \caption{The Woods-Saxon potential for $l = 0$, with $V_0 = \SI{-47}{MeV}$, $r_0 = \SI{2}{fm}$ and $d = \SI{0.65}{fm}$}
  \label{fig:woods-saxons}
\end{figure}

\section{Convergence}
\todo{Is convergence even important?}
We can solve the problem using either HO expansion or discretizised momentum space. A comparison of convergence is shown in \cref{fig:HO vs mom}.
\todo{Skip zoom and title of plot.} 
\begin{figure}
  \centering
    \includegraphics[width = \textwidth]{figures/He5_convergence.pdf}
  \caption{}
  \label{fig:HO vs mom}
\end{figure}

\section{Wavefunctions}

\Cref{fig:resonance wavefunction} compares the radial probability distributions $r^2|R(r)^2|$ for two states as the potential is varied. We see that one solution is basically unchanged under variation of the potential. This can be interpreted as it being an unbound state in the energy continuum. We see that its probability distribution does not decrease for large $r$.

The other state has an energy $E>0$ meaning that it, too, must be unbound. However, its wavefunction is highly localized near the center, suggesting a quasi-stationary state. Additionally, the solution varies dramatically with the potential, meaning that it must be a feature of the system. 

\todo{Remove last however.}However, because we are working within the realm of real numbers, we can gain no further insight into the nature of these solutions yet, since we expect the resonance to be properly described only by complex energies. 


\begin{figure}[H]
  \pgfplotstableread{figures/wavefunctions/wavefunctions.data}\wavefunctions
  \centering{
  \pgfplotsset{
    width = 0.45\textwidth, height = 7cm,
    xlabel = $r/\b{\si{fm}}$, ylabel = $r^2\absq{R(r)}$,
    axis x line = bottom,
    axis y line = left,
    no markers,
    ytick = \empty,
    ymax = 1.7,
  }
  \subfloat[$V_0=\SI{-52}{MeV}$]{
    \begin{tikzpicture}
      \begin{axis}
        \addplot          table[x index=0, y index=1] {\wavefunctions};
        \addplot+[dashed] table[x index=0, y index=2] {\wavefunctions};
      \end{axis}
    \end{tikzpicture}
  }
  \subfloat[$V_0 = \SI{-47}{MeV}$]{
    \begin{tikzpicture}
      \begin{axis}
        \addplot          table[x index=0, y index=3] {\wavefunctions};
        \addplot+[dashed] table[x index=0, y index=4] {\wavefunctions};
      \end{axis}
    \end{tikzpicture}
  }
  }
  \caption{An unbound and resonant solution to the Woods-Saxon potential for different depths $V_0$, using the plane wave expansion.}
  \label{fig:resonance wavefunction}
\end{figure}
\todo{Start with a deeper well holding a bound state, and then decrease it to get resonances? Add another subplot?}

\section{Varying $\omega$}


\chapter{Non-Hermitian Quantum Mechanics} 
\label{cha:nhqm}
We have noticed that one of the \He{5} states behaves 
differently than the others. We suspect this is the 
resonance state, but we cannot yet quantify its width, 
or half-life. This means we need to find the complex 
energy of the state. We do this by using the theory of 
Tore Berggren and his \emph{Berggren basis} \cite{berggren}. 
The theory is involved and will not be fully explained 
here. Instead we present a heuristic argument.

If we relate the energies $E$ of a system to momenta $k$ as
\begin{eq}
  E = \frac{\hbar^2 k^2}{2\mu}
  \quad\quad
  \textup{or}
  \quad\quad
  k = \frac{\sqrt{2\mu E}}{\hbar},
\end{eq}
we can plot the energies as $k$ in the complex plane, see 
\cref{fig:complex plane}. We then expect bound states, with 
$E<0$, to be represented by $k$ along the imaginary axis---
whereas unbound, scattering states, with $E>0$, are found 
along the real axis. Resonance states, with complex 
$E = E_0 - i \Gamma /2$, would by this argument appear somewhere
in the fourth quadrant.

\todo{Maybe show results before complexifying? We could relate the mesh points to the solutions here.}

We now interpret these $k$ as poles, and our integration 
\begin{eq}
  \fint[0][\inf]{k} k'^2 V(k,k') \phi(k')
\end{eq}
\todo{Should we write the integral as a path integral rather than one along the real axis?}
as a contour integration around the upper half plane, 
see \cref{fig:simple contour}. The result of a contour 
integration depends on the poles it encircles by the 
residue theorem. Therefore, we expect something to happen if 
we let the contour encircle the pole of the resonance,
as in \cref{fig:berggren contour}.

\begin{figure}
  \tikzset{
    triangle/.style={regular polygon, regular polygon sides=3},
    nosep/.style={inner sep=0},
    bound/.style={circle,draw,minimum size=2mm,nosep},
    unbound/.style={rectangle,draw,minimum size=2mm,nosep},
    quasibound/.style={triangle,draw,minimum size=2.5mm,nosep}
  }
  \subfloat[]{
  \label{fig:simple contour}
  \begin{tikzpicture}[scale = 2.5]
    \draw[->] (-1.2, 0) -- (1.2, 0) node[right] {$\Re k$};
    \draw[->] (0, -0.5) -- (0, 1.2) node[above] {$\Im k$};
    \foreach \y in {0.1, 0.3}
      \node at (0, \y) [bound] {};
    \foreach \x in {0.25, -0.25}
      \node at (\x, -0.15) [quasibound] {};
    \draw[very thick, mid arrows] (1, 0) arc (0:90:1) arc (90:180:1) 
                                  -- (0,0) -- cycle;
  \end{tikzpicture}
  }
  \subfloat[]{
  \label{fig:berggren contour}
  \begin{tikzpicture}[scale = 2.5]
    \draw[->] (-1.2, 0) -- (1.2, 0) node[right] {$\Re k$};
    \draw[->] (0, -0.5) -- (0, 1.2) node[above] {$\Im k$};
    \foreach \y in {0.1, 0.3}
      \node at (0, \y) [bound] {};
    \foreach \x in {0.25, -0.25}
      \node at (\x, -0.15) [quasibound] {};
    \draw[very thick, mid arrows, radius=1]
      (1, 0) arc [start angle=0,  end angle=90]
             arc [start angle=90, end angle=180]
             -- (-0.5, 0) 
             -- (-0.25, 0.25) 
             -- (0.25, -0.25)
             -- (0.5, 0)
             -- cycle;
  \end{tikzpicture}
  }
  \caption{The complex $k$-plane. The circles represent 
  bound states and the triangles resonant states. Note the 
  mirroring of the states in the imaginary axis.}
  \label{fig:complex plane}
\end{figure}

\todo{Need to mention completeness of berggren basis. Important here or later?}

\todo{Where to put: "No conjugate on bras"?}

\section{The Complex Contour}

We choose to extend our integration along the real axis to 
the simplest possible complex contour, a triangle-shaped 
extrusion downwards. The tip of the triangle is placed directly 
below the hypothesized resonance pole.

Numerically, we are faced with the problem of how to choose 
the points and weights, now that the contour is complex. 
We consider each straight segment of the contour separately, 
and rescale the Gauss-Legendre points to each of the different segments.
An example of a contour is seen in 
\cref{fig:triangle contour}.

\begin{figure}[H]
  \centering
    \includegraphics[width = 0.7\textwidth]{figures/complex_contour.pdf}
  \caption{The complex contour used.}
  \label{fig:triangle contour}
\end{figure}

\section{Studying the Resonance}

\todo{Results and shit here}


\chapter{Many-Body Theory}
\label{cha:many-body}
\documentclass[../main/report.tex]{subfiles}
\begin{document}
  
\chapter{Many-Body Theory}
\label{cha:many-body}
\todo{mer röd tråd}  
A basic fact of quantum mechanics is that particles that look identical \emph{are} identical. 
It is also known that there are only two types of particles, bosons and fermions, and that systems of many bosons and fermions behave differently. 
When exhanging the coordinates of two particles, the wavefunction of a bosonic system is unchanged, whereas a fermionic wavefunction changes sign. 
We say that the bosonic wavefunction is \emph{symmetric} and that the fermionic wavefunction is \emph{antisymmetric}. 

The Schrödinger equation for a system of $N$ identical particles can be expressed as
\begin{eq}
  \p{
    \sum_{i} \frac{1}{2m}\nabla^2_i + \sum_{i < j} V(\vec{r}_i, \vec{r}_j)
  } \psi(\vec{r}_1, \vec{r}_2, ..., \vec{r}_N)
  =
  E \psi(\vec{r}_1, \vec{r}_2, ..., \vec{r}_N),
\end{eq}
but this is not a particularly convenient form for many-body manipulation. It is especially inconvenient when working with systems that do not have a fixed number of particles. This form is called the first quantization.

We would instead like to describe the system as a combination of single particle (sp) states and particles that occupy these states. 
The operators can then be expressed using creation and annihilation operators that allow us to work with an arbitrary number of particles.
\todo{first - second quantization. no follow up}
\todo{We should mention there is something similar for bosons.}
\section{Fermionic states}

\todo{Use $\alpha$ or something else?}
We begin with an orthonormal single particle (sp) basis $\ket\alpha$, where $\alpha$ represents all the quantum numbers that describe the state.
Next, consider $N$ identical particles, expressed in this basis. We form a product state
\begin{eq}
  \pket{\alpha_1\alpha_2\dots\alpha_N} 
  \equiv
  \ket{\alpha_1} \otimes \ket{\alpha_2} \otimes \dots \otimes \ket{\alpha_N}
  =
  \ket{\alpha_1}\ket{\alpha_2}\dots\ket{\alpha_N},
\end{eq}
which, by the orthonormality of the $\ket\alpha$, will be orthonormal as well
\begin{eq}
  \pbraket{\alpha_1\alpha_2\dots\alpha_N}{\alpha'_1\alpha'_2\dots\alpha'_N}
  =
  \braket{\alpha_1}{\alpha'_1}
  \braket{\alpha_2}{\alpha'_2}
  \dots
  \braket{\alpha_N}{\alpha'_N}.
\end{eq}
These states span the Hilbert space of the $N$-body problem, but do not incorporate the right symmetry for fermions. 
To obey the pauli principle, fermionic states need to be antisymmetric, i.e. change sign under exchange of quantum numbers for two particles 
\begin{eq}
  \ket{\alpha_1\alpha_2\dots\alpha_i\dots\alpha_j\dots\alpha_N} 
  = 
  - \ket{\alpha_1\alpha_2\dots\alpha_j\dots\alpha_i\dots\alpha_N}.
\end{eq}
Fermionic, antisymmetrized states can be expressed as a linear combination of the product states. 
For example, in the case of two particles, the correctly normalized antisymmetric state is
\begin{eq}
  \ket{\alpha_1\alpha_2} 
  = 
  \frac{1}{\sqrt{2}}
  \bigp{
    \pket{\alpha_1\alpha_2} - \pket{\alpha_2\alpha_1}
  }.
\end{eq}
Note the use of angular ket notation $\ket\dots$ for antisymmetric states, as opposed to $\pket\dots$ for product states.

Because the antisymmetric states change sign on exchange of two particles, should two particles be in the same state, the state will become 0. \todo{Is this really why?} 
Thus, no particles can occupy the same state and the Pauli principle is satisfied.

Important to note is that states with permuted quantum numbers, such as the states $\ket{\alpha_1\alpha_2}$ and $\ket{\alpha_2\alpha_1}$, represent the same physical state, as they only differ in sign (phase). 
This means that we have to make sure not to double count these states. 
We can do this by requiring the sp states to always appear in the same order in the ket. 
If they are not, we permute two sp states at a time until the correct ordering is reached \todo{"double count"?} \todo{achieved }
\begin{eq}
  \ket{\alpha_i\alpha_1\dots\alpha_{i-1}\alpha_{i+1}\dots\alpha_N}
  & =
  - \ket{\alpha_1\alpha_i\dots\alpha_{i-1}\alpha_{i+1}\dots\alpha_N}
  \\ & =
  (-1)^{i-2} 
  \ket{\alpha_1\dots\alpha_i\alpha_{i-1}\alpha_{i+1}\dots\alpha_N}
  \\ & =
  (-1)^{i-1} 
  \ket{\alpha_1\dots\alpha_{i-1}\alpha_i\alpha_{i+1}\dots\alpha_N}.
\end{eq}

\section{Creation and Annihilation Operators in Fock Space}
\todo{chapter name okay?}
\todo{how do we denote the ops? a? c?}

\todo{Fock space/second quantization/occupation number representation?}
So far we've looked at a system with a fixed number of particles, but we want to work with a system with an arbitrary number of identical particles. 
The construction that lets us do this is called the \emph{Fock space}. 
\todo{why?/cite/explain citation at start of chapter, covering all our bases :)}
A state in Fock space, a \emph{Fock state}, can contain any number of particles. 
 States with different number of particles are orthogonal to each other.

\todo{particle addition/removal operators?}
The simplest Fock state is the \emph{vacuum state} $\ket{0}$, which describes a system with no particles. 
All other states can be created from the vacuum state using the \emph{creation operator} $a_\alpha^\dag$, which adds a particle with quantum numbers $\alpha$ to a state
\begin{eq}
  a_{\alpha}^{\dagger} \ket{\alpha_1 \alpha_2 ... \alpha_N} 
  =
  \ket{\alpha \alpha_1 \alpha_2 ... \alpha_N}.
\end{eq}
The resulting state will not necessarily be ordered, and the ordering might contribute a sign:
\begin{eq}
  a_{\alpha_i}^{\dagger} 
  \ket{\alpha_1 \alpha_2 ... \alpha_{i-1} \alpha_{i+1}...\alpha_{N}} 
  =
  (-1)^{i-1} 
  \ket{\alpha_1 \alpha_2 ... \alpha_{i-1} \alpha_i \alpha_{i+1} ... \alpha_{N}}.
\end{eq}
Note that when $a_\alpha^\dag$ acts on a state that already contains a particle with quantum numbers $\alpha$, the result is 0, because of antisymmetry
\begin{eq}
  a_{\alpha}^{\dagger} \ket{\alpha\alpha_1 \alpha_2 ... \alpha_N} 
  =
  0.
  \label{eq:creation_zero}
\end{eq}

The adjoint of the creation operator is called the \emph{annihilation operator} $a_\alpha$. 
It can be shown to have the opposite effect, removing a particle, when acting on a state
\begin{eq}
  a_{\alpha} \ket{\alpha \alpha_1 \alpha_2 ... \alpha_N}
  =
  \ket{\alpha_1 \alpha_2 ... \alpha_N}.
\end{eq}
Here, too, a sign might appear from the ordering
\begin{eq}
  a_{\alpha_i}
  \ket{\alpha_1 \alpha_2 ... \alpha_{i-1} \alpha_i \alpha_{i+1} ... \alpha_N}
  =
  (-1)^{i-1}
  \ket{\alpha_1 \alpha_2 ... \alpha_{i-1} \alpha_{i+1}...\alpha_N}.
\end{eq}
Analogous to $a_\alpha^\dag$, when $a_\alpha$ acts on a state that does not contain a particle with the quantum numbers $\alpha$, the result is 0
\begin{eq}
  a_\alpha \ket{\alpha_1 \alpha_2 ... \alpha_N} 
  =
  0.
  \label{eq:annihilation_zero}
\end{eq}


\section{General Operators in Fock Space}

We can now express the state of any number of particles, but in order to do something with the states we also need to express operators in the Fock space formalism. It turns out that operators involving any number of particles can be expressed using the creation and annihilation operators. The operators can then act on a state with an arbitrary number of particles. We will only treat one- and two-body operators here, as they are all we need in this report.

\subsection{One-body operators}

A one-body operator $H_1$ which acts on a single sp state, is represented by the Fock space operator
\begin{eq}
  \hat{H}_1
  =
  \sum_{\alpha \beta} 
  \bra\alpha H_1 \ket\beta 
  a_\alpha^\dag a_\beta.
\end{eq}
It is important to note that, while the sum runs over the complete set of sp states twice, only a few terms will be non-zero, because of the operator rules in \cref{eq:creation_zero,eq:annihilation_zero}. 

\todo{$\alpha$ bad notation?}
If the sp-states are eigenstates to the one-body operator
\begin{eq}
  H_1 \ket{\alpha_i} = h_i \ket{\alpha_i}
\end{eq}
the matrix elements only exist on the diagonal, when $\alpha = \beta$, and we get
\begin{eq}
  \hat{H}_1
  =
  \sum_{\alpha} 
  \bra\alpha H_1 \ket\alpha
  a_\alpha^\dag a_\alpha.
\end{eq}
Sandwiched between a bra and a ket, this becomes
\begin{eq}
  \bra{\alpha_1\dots\alpha_N} \hat{H}_1 \ket{\alpha'_1\dots\alpha'_N}
  & =
  \sum_{\alpha} 
  \bra\alpha H_1 \ket\alpha
  \bra{\alpha_1\dots\alpha_N} 
  a_\alpha^\dag a_\alpha
  \ket{\alpha'_1\dots\alpha'_N}
  \\ & =
  \sum_{i = 1}^N 
  \bra{\alpha_i} H_1 \ket{\alpha_i}
  \braket{\alpha_1\dots\alpha_N}{\alpha'_1\dots\alpha'_N}
  \\ & =
  \p{
    h_1 + \dots + h_N
  }
  \delta_{\alpha_1\alpha'_1} \dots \delta_{\alpha_N\alpha'_N},
\label{eq:one-body interaction}
\end{eq}
the sum of the eigenvalues of the sp states in the bra and ket, but only if the bra and ket are the same. The Fock space operator $\hat{H}_1$ is thus also diagonal.

\subsection{Two-body operators}
\todo{notations}
A two-body operator in Fock space becomes
\begin{eq}
  \hat{H}_2
  =
  \frac{1}{2}\sum_{\alpha \beta \gamma \delta} 
  \pbra{\alpha \beta} H_2 \pket{\gamma \delta} 
  a_\alpha^\dag a_\beta^\dag a_\delta a_\gamma.
\end{eq}
Note that the ordering of the $\gamma$ and $\delta$ is different for the product states and the operators, so-called \emph{normal ordering}.
The factor \nicefrac{1}{2} stems from the fact that %%%%%%%%%%%%%%%%%%%%%%%%%%%
\begin{eq}
  \pbra{\alpha \beta} H_2 \pket{\gamma \delta} 
  = 
  \pbra{\beta \alpha} H_2 \pket{\delta \gamma}
\end{eq}
and we are counting both.

We can also express $\hat{H}_2$ using matrix elements between antisymmetric states
\begin{eq}
  \bra{\alpha\beta} H_2 \ket{\gamma\delta} 
  = 
  \pbra{\alpha\beta} H_2 \pket{\gamma\delta}
  -
  \pbra{\alpha\beta} H_2 \pket{\delta\gamma},
\end{eq}
but we will have to add another factor \nicefrac{1}{2} to compensate for double counting
\begin{eq}
  \hat{H}_2
  =
  \frac{1}{4}\sum_{\alpha \beta \gamma \delta} 
  \bra{\alpha \beta} H_2 \ket{\gamma \delta} 
  a_\alpha^\dag a_\beta^\dag a_\delta a_\gamma.
\end{eq}
The double counting can be avoided, however, by taking into account the ordering of the states
\begin{eq}
  \hat{H}_2
  =
  \sum_{\substack{\alpha < \beta \\ \gamma < \delta}} 
  \bra{\alpha \beta} H_2 \ket{\gamma \delta} 
  a_\alpha^\dag a_\beta^\dag a_\delta a_\gamma.
\end{eq}

For the case of two particles we have
\todo[inline]{Use $a, b, c, d$ instead of $\alpha_i$?}
\begin{eq}
  \bra{\alpha_1\alpha_2} \hat{H}_2 \ket{\alpha'_1\alpha'_2}
  & =
  \sum_{\substack{\alpha < \beta \\ \gamma < \delta}} 
  \bra{\alpha \beta} H_2 \ket{\gamma \delta} 
  \bra{\alpha_1\alpha_2} 
  a_\alpha^\dag a_\beta^\dag a_\delta a_\gamma
  \ket{\alpha'_1\alpha'_2}
  \\ & =
  \sum_{\substack{\alpha < \beta \\ \gamma < \delta}} 
  \bra{\alpha \beta} H_2 \ket{\gamma \delta}
  \delta_{\alpha\alpha_1}\delta_{\beta\alpha_2}
  \delta_{\gamma\alpha'_1}\delta_{\delta\alpha'_2}
  \\ & =
  \bra{\alpha_1\alpha_2} H_2 \ket{\alpha'_1\alpha'_2},
\end{eq}
as expected.

\section{Implementation}

In order to use the fock space formalism in numerical calculations

\end{document}

\section{Three-Body implementation}
In order to solve the three body-problem, or rather the He-6 system 
modeled as a He-4 nucleus with two orbiting neutrons, we had to 
translate the governing mathematical and physical relations to 
something better suited for numerical solutions. This Section gives a description of the models and simplifications we have used.

%\subsection{our special case and simplifications}
%//core - nucleus
%\subsection{multistep method}
We utilized the multistep method to find the solution to a 
multi-body (mb) problem through solutions of incremental many-body 
problems. We started out by solving the two body problem of a core 
(\He{4}) and a single orbiting particle (sp) (neutron, forming \He{5}
---there was no bound state, only resonances as can be expected, 
see ref ref ref) and used these solutions to form mb-states out of 
different combinations of the sp-states; see fock space section in 
mb-theory. We have only reached the second step in the multi-step 
method but this is the largest step, most of the operators that govern 
any mb-problem are needed for a three-particle problem, whereas a 
two-body particle can be reduced to a one-body-problem through the 
use of relative quantities.
 
\subsection{Many-body states}
The governing theory of mb-states is covered in \cref{sec:fock space}, this section will focus on one way of implementing these states.
%We enumerate the solutions to the He-5 problem as a basis for the mb states and include information about the state's angular momentum.
%Our first simplification is that we only regard states with $\alpha, \beta$, where $\alpha_{sp}$ and $\beta_{sp} $ are sp-states and $\alpha_{sp} <= \beta_{sp}$ in terms of enumeration so as to avoid oversumming, recall the relation between alpa, beta :: beta, alpha ref ref . 
%This will gives us a smaller matrix between the different mb states and saves us a lot of computation-time.

A many-body state is represented as a number representing the sign and a list of mb-states (????????).
 The mb-state is a set of sp-states, one for each orbiting particle (represented as the sp-state's k value). 
 The sp-states can also be extended to hold information about other relevant quantum numbers, we will treat states with (coupled) angular momentum in the calculations but the extra information is best supplied when it is used for calculations.
 
%each state is a set of numbers signifying the relevant quantum numbers of the state.
 %The state is a pair of numbers one indicates: which single-particle solution is represented (we store the k-value of the sp-state) and the other is the angular momentum which is in the range of $[-3/2,3/2]$.
 
 

The mb-state is formed by taking a vacuum-state, an empty list, and adding a single particle state along with any other quantum number needed for the calculations, this is described mathemtically in \cref{eq:creation}.
To avoid oversummation, the single particle states are ordered in a lexicographic order and our calculations only require one of the permutations of a set of given sp-states, an integer is used to keep track of the sign of a given permutation. 
This integer is calculated as in \cref{eq:creation sign}.
Because of the fermionic nature of our particles we do not allow two sp-states with the same quantum numbers and if one tries to create a state which is already present, as is the case in \cref{eq:creation zero}, the sign will be set to 0 and it will be regarded as a vacuum-state in further calculations.
The annihilation operator is implemented in a similar fashion, it removes a given state (set of quantum numbers) and returns a new FermionState without the given to-be-removed state.
 
% To keep track of all the many-body states and to perform sums over all possible many-body states we arrange a list with all the states and signify each many body state as the index representing the state in this list
 
\subsection{Generating many-body Hamiltonian}
To calculate the Hamilton matrix for the many-body states we generate a list of all possible mb-states and let the (i,j)-th element of the matrix be the hamiltonian contribution from the i-th mb-state in a bra and the j-th mb-state in a ket. 
 In the case of two orbiting particles the two-body operator will work on all combinations of bras and kets but in the four or more body problem only some combinations will contribute.
 To find which combinations of bras and kets contribute to the two-body operator one can take a ket and remove two sp-states and iteratively add all possible combinations of two sp-states to the new ket and check whether the bra is the same as the new ket.
 On the diagonal, however, there will also be a contribution from the one-body operator.
 The contributions from these two operators are discussed in detail below.

\subsubsection{The one-body opreator}
The one-body operator is the simple kinetic energy operator that we have known and loved since our first food fight. \todo{Too much joke.}
%Presented on the form of \cref{eq: stans i mb-teori} the one-body operator, \fockop{T} simply yields the kinetic energy of the two particles, but in a matrix representation 
 In our matrix representation the bra and the ket will be the same in a diagonal-element and thus have the same single-particle states.
 This operator returns the kinetic energy of these two single particle states. 
 The energies are eigenvalues to the sp-hamilton matrix and have already been calculated and can be retrieved.

\subsubsection{The two-body operator}
The two-body operator is the contribution from the neutron-neutron interaction and is computationally taxing. Although the potential is (ridiculously) simple this calculation requires sums over all the quantum numbers in both the bra and the ket; the mathematical expression for the hamiltonian is presented in \cref{two-body op sum}. 
With a simple contour of 15 points this would be a sum over $15^4*4^4 \approx 13$ million elements, which would be computationaly unfeasible.
In order to reduce the complexity we instead make use of coupled angular momentum for the many-body states, refer to ref ref ref.

%the (i,j)-th matrix element, $H_{ij}$, would be the interaction between the i-th and j-th mb-state. and for each pair of indexes calculate a matrix element; 


To determine the hamilton matrix \todo{is it called hamilton matrix or hamiltonian matrix} for the mb-system we generate a list of all possible (non-permutated) mb-states, constructed only from different sp-states.
For a given bra and ket we determine whether the fock-space relations allow for an interaction, in this case with two-orbiting particles this is trivially true.
This is where we introduce the different allowed m-quantum numbers, the original bra and ket are used to generate all possible mb-states with the given sets of sp-states but with an angular momentum for each of the sp-states.
Thus we can calculate the Clebsch-Gordan coefficients to treat the degeneracy in m-quantum number, recall that Clebsch-Gordan coefficients was covered in ref ref ref.

The calculation of clebsch gordan coeff (and fock relations) severly reduces the number of elements that actually give any contribution at all. 
When we know which bra and ket interactions that give a contribution it is time to calculate it. 
Mathematically this contribution from the sepparable n-n potential can be expressed like presented in ref ref.
However, this is not a suitable form for the computer, not even with the very powerful Gauss-Legendre contour for the integral; instead we rewrite it as a matrix equation:
$1+2+3+4+5+6+7+...+226+... = - \frac{1}{12}$\\
where the matrix is the same for each combination of bras and kets and need only be calculated once.

%only some (ALL) bra-ket combinations will live, governed by the creation / %annihilation relations. 
%n-n separable interaction, clever matrix multiplication scheme
%clebch-gordan coefficients :-()

\chapter{The \He{6} Nucleus}
\label{cha:he6}


\chapter{The \He{7} Nucleus}
\label{cha:he7}

\chapter{Monte Carlo Approach}
\label{cha:monte carlo}

\appendix

\todo{Gauss-Legendre?}

\chapter{Derivations}

\section{Harmonic Oscillator Matrix Elements}
\label{sec:HO matrix elements}

First, we combine \cref{eq:HO hamiltonian,eq:spherical hamiltonian} to get
\begin{eq}
  H = H\sub{HO} - \frac{\mu\omega^2 r^2}{2} + V(r).
\end{eq}
We then close this equation with $\bra{nlm}$ on the left and $\ket{n'lm}$ on the right to get three terms on the RHS, which we consider in turn. The first is just the eigenvalues of $H\sub{HO}$
\begin{eq}
  \bra{nlm} H\sub{HO} \ket{n'lm} 
  = 
  E_{nl} \braket{nlm}{n'lm} 
  = 
  \hbar\omega\p{2n + l + \frac{3}{2}} \delta_{nn'}.
\end{eq}
The second follows from the known identity % \cite{moshinsky} eller den moshinsky citear?
\begin{eq}
	& \bra{nlm} r^2 \ket{n'lm} 
	= \\
	& \frac{\hbar}{\mu\omega}
  \p{
  	\p{2n + l + \frac{3}{2}}\delta_{nn'}
    -
  	\sqrt{n(n+l+\frac{1}{2})}\delta_{n,n'-1}
  	-
  	\sqrt{n'(n'+l+\frac{1}{2})}\delta_{n',n-1}
  }.
\end{eq}
The third term is calculated in the position basis, with
\begin{eq}
  \braket{\vec{r}}{nlm} = R_{nl}(r) Y_l^m(\theta, \phi),
\end{eq}
where $R_{nl}$ is the HO radial wavefunctions (see \cref{eq:HO radial wavefunction}) and $Y_l^m$ are the spherical harmonics. Because $V(r)$ is spherically symmetric, the spherical harmonics integrate to 1 and we get
\begin{eq}
	\bra{nlm} V(r) \ket{n'lm} 
	=
	\fint[0][\inf]{r} r^2 R_{nl}(r)  V(r) R_{n'l}(r).
\end{eq}
Putting it all together, we have
\begin{eq}
  \bra{nlm} H \ket{n'lm} 
  & =
  \bra{nlm} H\sub{HO} - \frac{\mu\omega^2 r^2}{2} + V(r) \ket{n'lm}
  \\ & =
	\frac{\hbar\omega}{2}
	\left(
    \p{2n+l+\frac{3}{2}} \delta_{nn'}
    +
		\sqrt{n(n+l+\frac{1}{2})} \delta_{n,n'-1}\right.
		\\ & + 
		\left.\sqrt{n'(n'+l+\frac{1}{2})} \delta_{n',n-1} 
	\right)
	+
	\fint[0][\inf]{r} 
    r^2 R_{nl}(r) V(r) R_{n'l}(r).
\end{eq}

\section{Radial Momentum Space TISE}
\label{sec:radial mom space TISE}

\section{Radial Momentum Space TISE}
\label{app:radial_mom_TISE}

To find the momentum space Schrödinger equation, we need to write an explicit expression for
\begin{eq}
  \int \rd^3 \vec{k}' \bra{\vec{k}} H \ket{\vec{k}'} \Phi(\vec{k}')
  &= 
  E\Phi(\vec{k})
\end{eq}
To begin with, using the completeness relation with the position basis, we note that
\begin{eq}
  \Phi(\vec{k}) &= \braket{\vec{k}}{\psi} 
  = 
  \int \rd^3\vec{r} \braket{\vec{k}}{\vec{r}} \psi(\vec{r})
\end{eq}
Standard textbooks on quantum mechanics show 
\begin{eq}
  \braket{\vec{k}}{\vec{r}} 
  &= 
  \frac{1}{(2\pi)^{\frac{3}{2}}}e^{i\vec{k}\cdot\vec{r}}.
\end{eq}
For a spherically symmetric problem, solutions can be found on the form $\psi(\vec{r})=  R(r)Y_l^m(\Omega_r)$.
We can simplify the above integral by using the plane wave expansion \cite{mehrem}
\begin{eq}
  e^{i\vec{k}\cdot\vec{r}} 
  &= 
  4\pi \sum_{l=0}^\infty \sum_{m=-l}^l  i^l j_l(kr)Y_l^m(\Omega_k)\conj{Y_l^m(\Omega_r)}.
\end{eq}
Inserting this and using orthogonality of spherical harmonics
\begin{eq}
  \int \rd \Omega_r \conj{Y_{l'}^{m'}(\Omega_r)}Y_l^m(\Omega_r)
  =
  \delta_{mm'}\delta_{ll'} 
\end{eq}
we obtain
\begin{eq}
  \Phi(\vec{k}) &= \phi(k)Y_l^m(\Omega_k)
  =
  \sqrt{\frac{2}{\pi}} i^l Y_l^m(\Omega_k) \fint{r} r^2 R(r) j_l(kr).
\end{eq}
In a similar manner, we evaluate
\begin{eq}
  \bra{\vec{k}}V(r)\ket{\vec{k}'} 
  &= 
  \frac{1}{(2\pi)^3} \int \rd^3 \vec{r} V(r)  e^{i\vec{k}'\cdot\vec{r}} e^{-i\vec{k} \cdot \vec{r}} \\
  &=
  \frac{1}{(2\pi)^3} (4\pi)^2 \sum_{l,l'}\sum_{m,m'} 
  (-1)^l i^{(l+l')} Y_{l'}^{{m'}^*}(\Omega_{k'}) Y_l^m(\Omega_k)
  \\
  &\times
  \int \rd r \, 
    r^2 V(r) j_l(kr)j_{l'}(k'r)
  \int \rd \Omega_r \, 
    Y_{l'}^{m'}(\Omega_r)Y_l^{m^*}(\Omega_r)
\end{eq}
Here a factor $(-1)^l$ was introduced because of the parity of the spherical harmonics: $Y_l^m(-\Omega_k)=(-1)^lY_l^m(\Omega_k)$. 

Inserting all of this into the Schrödinger equation and again simplifying by using the orthogonality of the spherical harmonics twice, we obtain

\begin{eq}
  \int \rd^3 \vec{k}' \bra{\vec{k}} H \ket{\vec{k}'} \Phi(\vec{k}') 
  &= 
  \frac{k^2}{2\mu}\phi(k)Y_l^m(\Omega_k) 
  + 
  \fint[\R^3]{^3\vec{k}'} \bra{\vec{k}}V(r)\ket{\vec{k}'} \phi(k') Y_l^m(\Omega_{k'}) 
  \\
  &=
  \frac{k^2}{2\mu}\phi(k)Y_l^m(\Omega_k) + Y_l^m(\Omega_k) \fint[0][\inf]{k'} k'^2 \phi(k') V(k,k')
  \\
  &=
  E\phi(k)Y_l^m(\Omega_k)
\end{eq}
where
\begin{eq}
  V(k,k') = \frac{2}{\pi}\int \rd r \, r^2 V(r)j_l(kr)j_l(k'r).
\end{eq}
We now have a factor $Y_l^m(\Omega_k)$ on both sides, meaning that we can divide and finally arrive at \cref{eq:radial mom space TISE},
\begin{eq}
  \frac{k^2}{2\mu}\phi(k) + \fint[0][\inf]{k'} k'^2 V(k,k') \phi(k') 
  =
  E\phi(k).
\end{eq}


\chapter{Tools}

All calculations were made in the Python programming language 
using the libraries NumPy and SciPy. We used the matplotlib 
library to make the figures. The code was managed using Git 
and is available at
\begin{quote}
  \url{https://github.com/pnutus/NHQM}
\end{quote}

\bibliographystyle{ieeetr}
\bibliography{nhqm}{}

\end{document}
