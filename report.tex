\documentclass[12pt,a4paper]{report}
\usepackage[english]{babel}

\usepackage{NHQM}

\begin{document}
  
\numberwithin{equation}{chapter}
\numberwithin{figure}{chapter}


\title{Non-Hermitian Quantum Mechanics\\
\Large Bachelor Thesis Report}
\author{Jonathan Bengtsson  \and Ola Embréus \and Vincent Ericsson \and Pontus Granström \and Nils Wireklint}
\date{\today}



\maketitle

\newpage
\begin{abstract}
This is a bachelor thesis, and we've done some calculations on non-hermitian quantum mechanics 
\end{abstract}
\newpage

\tableofcontents

\newpage

\chapter{Introduction}
this is the introduction. There will be more interesting text here later.

\section{Background}

Quantum mechanics (QM) is a cornerstone of modern physics, as it describes the world on the smallest of scales. One topic often introduced in graduate QM courses is resonance scattering, a phenomenon where a system form so-called \emph{quasi-stationary} states. In contrast to \emph{stationary} (bound) states, in which a system remains indefinitely, these quasi-stationary states decay over time. Describing this process leads naturally to the introduction of complex energy eigenvalues. 

Complex eigenvalues pose a problem in standard QM. This is because observable quantities are regarded as real values and are described by \emph{Hermitian} operators. When working with complex eigenvalues one needs a \emph{non-Hermitian} formulation of the problem. While the idea of using non-Hermitian quantum mechanics (NHQM) has been around for quite some time, the technique is not widely used.

{\Large Helium-family \\ Numerical methods\\ basis expansion}

{\Large Purpose? \\ Method? \\ Reading Guide?}

\begin{itemize}
  \item resonans. varför är det vårt mål? jämföra med exp. data He-5[nuklidkarta]?
\end{itemize}

\section{Purpose}

The main purpose of this project is to study resonances in few-body problems using non-Hermitian quantum mechanics and numerical methods. We expand the problems in the Berggren basis, resulting in a non-Hermitian hamiltonian. To calculate and solve the hamiltonian matrix, we use numerical integration and eigensolvers.

\chapter{The Basis Expansion Method} % (fold)
\label{cha:the_basis_expansion_method}

\begin{itemize}
  \item Vi vill lösa i datorn ty omöjligt analytiskt
  \item Skriva som ändlig matrisekvation
  \item => Basexpansion + Trunkering
  \item Expansion i allmänna fallet
  \item HarmOsc
  \begin{itemize}
    \item Hatom – för att verifiera
    \item He5 – vad är det där? Resonans?
  \end{itemize}
  \iitem 
\end{itemize}

\chapter{Discretizing Momentum Space} % (fold)
\label{cha:discretizing_momentum_space}

\begin{itemize}
  \item Varför mom-space? 
  \begin{itemize}
    \item Naturligare ramverk
    \item Berggren
  \end{itemize}
  \item Skriva om TISE
  \begin{itemize}
    \item Resolution of identity
    \item fouriertransform
    \item All tråkig matte i appendix
    \item kan vi förklara bra, inte appendix
  \end{itemize}
  \item He5 (perhaps Hatom) in momspace
  \begin{itemize}
    \item Compare to harmonic oscillator
  \end{itemize}
\end{itemize}

\input{momspace.tex}

\chapter{Non-Hermitian Quantum Mechanics} % (fold)
\label{cha:non_hermitian_quantum_mechanics}

\begin{itemize}
  \item varför: analysera resonanser
  \item resonans? -koppling till radioaktivitet, transienta tillstånd. tid<=>energi heisenberg
  \item verktyg: komplex kontur. varför? cite bergren, förklara 
  \item resultatmed dessa verktyg: två-kropp, resonans i he-5 ! jämföra med vedertagna parametrar. 
  \item berggrenbasen, varför inkludera resonans i basen
\end{itemize}
teorin ovan kan lyftas till introduktion
%har folk läst subatomär?

% chapter non_hermitian_quantum_mechanics (end)

\chapter{Many-Body Theory} % (fold)
\label{cha:many_body_theory}


\begin{itemize}
  \item allmämnn flerkroppars teori
  \item implementation, dickhoff - fock space -FermionSTates
\end{itemize} 

% chapter many_body_theory (end)

\chapter{The Three-Body Problem} % (fold)
\label{cha:the_three_body_problem}

\begin{itemize}
  \item teorin i specifik bas, tillåtna förenklingar i konkret fall(ett, flera)
  \item 
\end{itemize}

Can't reduce to one dimension.

\section{Many-Body Theory} % (fold)
\label{sec:many_body_theory}

Or this could be a chapter.

% section many_body_theory (end)


\section{The \ce{^6He} Nucleus} % (fold)
\label{sec:the_^6he_nucleus}

Extending \ce{^5He} example.

% section the_^6he_nucleus (end)

\section{NHQM for three bodies}


% chapter the_three_body_problem (end)

\chapter{Four or more bodies} % (fold)
\label{cha:four_or_more_bodies}

\ce{^7He}, \ce{^8He} ... ?

% chapter four_or_more_bodies (end)

\chapter{Monte Carlo Approach}
\label{cha:monte_carlo_approach}

Using a statistical approach.

\chapter{Eigensolver Optimization}
\label{cha: eigensolver}

Adapt eigensolvers to work more efficiently.

\chapter{Discussion}
\label{discussion}

hello.

\chapter{Division of tasks}
\label{division}
This part is for the individual judgement.

\section{Who did what?}
no one did nothing, except for someone who did a little bit of something.
\section{Authors}
no one wrote nothing.


\begin{thebibliography}{9}

\bibitem{lamport94}
  Leslie Lamport,
  \emph{\LaTeX: A Document Preparation System}.
  Addison Wesley, Massachusetts,
  2nd Edition,
  1994.

\end{thebibliography}

\appendix

\chapter{}

\section{Simplifying the Two-Body Problem} % (fold)
\label{sec:the_two_body_problem}

Reducing to one dimension.

% section the_two_body_problem (end)

\section{Reduction to a radial problem} % (fold)
\label{sec:reduction_to_a_radial_problem}

Reducing to one dimension.

% section reduction_to_a_radial_problem (end)


\section{Code?} % (fold)
\label{sec:code}

% section code (end)

\end{document}
