\documentclass[12pt,a4paper]{report}
\usepackage[english]{babel}

\usepackage{NHQM}

\begin{document}
  
\numberwithin{equation}{chapter}
\numberwithin{figure}{chapter}

\pagenumbering{gobble}
\listoftodos


\title{Resonances in Helium Isotopes\\ 
\todo{There is a proper front page somewhere, let's fix that}
\Large Bachelor Thesis in Physics}
\author{Jonathan Bengtsson  \and Ola Embréus \and Vincent Ericsson \and Pontus Granström \and Nils Wireklint}
\date{\today}



\maketitle

\newpage
\begin{abstract}
This report aims to serve as an introduction to quantum resonance and 
numerical calculations in quantum mechanics. This is done by studying
resonances in \He{5} and \He{6} nuclei. We explain the theory of basis
expansion and expand the \He{5} nucleus in the spherical harmonic oscillator
and spherical wave bases. To quantify the \He{5} resonance, we extend the
spherical wave basis into the complex plane according to the 
theory of Tore Berggren. The Berggren basis is then used as a single particle
basis for the \He{6} problem. We also explain many-body theory and the second
quantization formalism, providing details for its implementation. Finally, a
Monte Carlo approach to reducing computation time for many-body 
calculations is investigated.

\todo{Read through and correct the abstract.}
\end{abstract}
\newpage

\pagenumbering{roman}

\tableofcontents
\newpage

\pagenumbering{arabic}
\setcounter{page}{1}

\chapter{Introduction}
\label{cha:introduction}
\emph{Quantum mechanics} (QM) is a cornerstone of modern physics, as it describes the world on the smallest of scales.
On this small level we observe a vast number of interesting phenomena, some of which in our ordinary large-scale world seem completely impossible.
One of these phenomena is called \emph{resonance} and is the focus of this report.
%To study these resonances we will expand the hamiltonian in a \emph{Berggren basis}, which makes the hamiltonian non-hermitian, leading us into the world of \emph{Non-Hermitian Quantum Mechanics} (NHQM).

Resonances, or \emph{quasi-stationary states}, are states in which a nucleon is bound to a nucleus, but only for a limited time. The probability of the nucleon remaining bound will decay exponentially over time. This can be explained as the state having a complex energy by the following argument:

The time-dependency of a stationary state $\psi$ is, by the Shrödinger equation,
\begin{eq}
	\psi(t)
	=e^{-\frac{iE}{\hbar}t}\psi(0).
\end{eq}
With the energy $E$ real, the exponential factor in front is just a phase and the probability $|\psi(t)|^2$ is unchanged over time. However, if we let the energy be complex
\begin{eq}
	E = E_0 - i\frac{\Gamma}{2},
\end{eq}
we get
\begin{eq}
  |\psi(t)|^2 
  =
  \absq{
    e^{-\frac{iE_0}{\hbar} t} e^{- \frac{\Gamma}{2\hbar} t} \psi(0)
  }
  =
  e^{-\frac{\Gamma}{\hbar} t} \absq{\psi(0)},
\end{eq} 
which describes a resonant state with half-life $t_{1/2}=\hbar\ln 2/\Gamma$.


\todo{This paragraph sort of breaks flow}
Resonances can be understood more easily by observing the potential of the nucleus.
The potentials for different nuclei look slightly different, but in all nuclei where resonances is observed we find a potential barrier.
Now combine this with the knowledge of QM and we can explain the previously unexpected behavior with another phenomenon: tunneling.
Tunneling says that when looking at a QM-system with a potential barrier a particle once found on one side of the barrier later has a probability of being found on the other.
This is what happens in the case of quasi-stationary states.

This report will study resonances in the Helium isopes \He{5} and \He{6}.
Helium is chosen since \He{4} is a very stable nucleus and thus can be treated a single (alpha) particle. \He{5} has a known resonance with half-life $t_{1/2} = \SI{700e-24}{s}$. \He{6} is a so-called Boromean nucleus which, because of the interaction between the valence neutrons, has a bound state.


\todo{Talk about width of resonance state?}

\todo{Reading guide? Explain the structure of the report}

\section{Tools}

All calculations were made in the Python programming language using the libraries NumPy and SciPy. We used the matplotlib library to make the figures. The code was managed using Git and is available at \url{https://github.com/pnutus/NHQM}.


\chapter{Basis Expansion}
\label{cha:basis expansion}
We want to study the nuclei of Helium isotopes by solving the time independent Schrödinger equation (TISE)
\begin{eq}
  \label{eq:TISE}
  H \ket\psi = E \ket\psi.
\end{eq}
The TISE is commonly written in the position basis as
\begin{eq}
  \label{eq:TISEpos}
  \p{-\frac{\hbar^2}{2m}\nabla^2 + V(\vec{r})}\psi(\vec{r}) = E\psi(\vec{r}),
\end{eq}
since this is the only basis where the potential operator $V$ is known.

For the nuclear systems we are looking at, the TISE has no known analytical solutions, and we need to use numerical methods to solve it. However, written as in \cref{eq:TISEpos}, it is not suitable for numerical calculations. Instead, we would  like to write it as a matrix equation
\begin{eq}
  \label{eq:matrix equation}
  \sum_j H_{ij}\psi_j = E \psi_i
\end{eq}
with a finite matrix $H$ that we can diagonalize to find the eigenvalues $E$.

To write the TISE as a matrix equation we use \emph{basis expansion}. Basis expansion is how we make any sense at all of the abstract Hilbert spaces, operators and state vectors of QM. By expanding these abstract objects in a basis, we can relate them to the physical world. For example, equation \cref{eq:TISEpos} is the TISE for a particle, expanded in the position basis. This is the only basis in which we can express the potential, why we will always have to relate our new bases to this one.

Before we begin, we briefly recap some well known QM facts. First we need a \emph{complete basis}, either discrete $\ket{n}$ or continuous $\ket{x}$. A discrete basis means that any state $\ket\psi$ can be written as a linear combination of the basis states
\begin{eq}
  \label{eq:lincomb}	
  \ket\psi = \sum_n \psi_n \ket{n}
  \quad
  \textup{or}
  \quad
  \ket\psi = \fint{x} \psi(x) \ket{x}.
\end{eq}
The complete bases we will use in this report are the \emph{position basis} $\ket{\vec{r}}$, the \emph{momentum basis} $\ket{\vec{k}}$, the \emph{harmonic oscillator basis} $\ket{nlm}$ and the elusive \emph{Berggren basis} \cite{berggren}. All these bases are orthonormal, i.e. all the basis vectors satisfy 
\begin{eq}
  \braket{n}{n'} = \delta_{nn'}
  \quad
  \textup{or}
  \quad
  \braket{x}{x'} = \delta(x - x').
\end{eq}
With a complete basis $\ket{n}$, we get the very useful \emph{completeness relation}
\begin{eq}
  I = \sum_n \ket{n} \bra{n}
  \quad
  \textup{or}
  \quad
  I = \fint{x} \ket{x}\bra{x},
\end{eq}
where $I$ is the identity operator. This relation can therefore be inserted anywhere in any equation, and will find frequent use in this report. 

Let's now expand the TISE in the abstract $\ket{n}$ basis. We start by inserting the completeness relation for $\ket{n}$ in \cref{eq:TISE}
\begin{eq}
  \label{eq:expand1}
  H
  \p{
    \sum_{n'} \ket{n'} \bra{n'}
  }
  \ket\psi
  =
  \sum_{n'} H \ket{n'} \braket{n'}{\psi}
  =
  E \ket\psi.
\end{eq}
By closing \cref{eq:lincomb} with $\bra{n}$ on the left side and using orthonormality, we see that $\braket{n'}{\psi} = \psi_{n'}$. Now we close \cref{eq:expand1} with $\bra{n}$ on the left
\begin{eq}
  \label{eq:expand2}
  \sum_{n'} \bra{n} H \ket{n'} \psi_{n'}
  = 
  E \braket{n}{\psi},
\end{eq}
and if we write $H_{nn'} = \bra{n} H \ket{n'}$, we get
\begin{eq}
  \label{eq:expand3}
  \sum_{n'} H_{nn'} \psi_{n'} = E \psi_n,
\end{eq}
which is equivalent to the matrix \cref{eq:matrix equation}. This is the basic method of expanding the TISE in a basis. Expanding in an abstract basis won't get us very far, however, so we move on to the epitomic example: the harmonic oscillator.

\section{The Spherical Harmonic Oscillator}
\label{sec:harm_osc}
The Helium nuclei we want to expand are, for our intents and purposes, spherically symmetric. A spherically symmetric basis is therfore preferable, and we begin with the spherical harmonic oscillator (HO). The treatment below is adapted from \cite{moshinsky}.

We have for the HO, the Hamiltonian
\begin{eq}
  \label{eq:HO hamiltonian}
  H\sub{HO} = \frac{p^2}{2\mu} + \frac{\mu\omega^2 r^2}{2},
\end{eq}
where $\mu$ is the mass of the problem and $\omega$ is the angular frequency of the oscillator. With this Hamiltonian, the TISE has the solutions
\begin{eq}
  H\sub{HO}\ket{nlm} = E_{nl}\ket{nlm} % E_{nl}???
\end{eq}
with
\begin{eq}
  E_{nl} = \hbar\omega(2n + l + \frac{3}{2}), \quad n = 0, 1, 2, \dots
\end{eq}
Since they are eigenstates of $H\sub{HO}$, $\ket{nlm}$ form a complete basis, and it is this basis we will expand our TISE in. The procedure is the same as in \cref{eq:expand1,eq:expand2,eq:expand3} and gives us
\begin{eq}
  \label{eq:HOexpanded}
  \sum_{n'l'm'} \bra{nlm} H \ket{n'l'm'} \psi_{n'l'm'} = E \psi_{nlm}\,.
\end{eq}
Since we are considering a spherically symmetric Hamiltonian
\begin{eq}
  \label{eq:spherical hamiltonian}
  H = \frac{p^2}{2\mu} + V(r),
\end{eq}
the $l$ and $m$ in $\ket{nlm}$ commute with the Hamiltonian and will therefore contribute with a $\delta_{ll'}\delta_{mm'}$ factor. This means we can drop the primes and write \cref{eq:HOexpanded} as
\begin{eq}
  \sum_{n'} \bra{nlm} H \ket{n'lm} \psi_{n'lm} = E\psi_{nlm}.
\end{eq}
If we see $\bra{nlm} H \ket{n'lm} = H_{nn'}$ as a matrix, this can be expressed as {\it $H$ is diagonal in $l$ and $m$}.

Now we have our matrix equation, but we need to find the matrix elements $\bra{nlm} H \ket{n'lm}$. These require some calculation (see \cref{sec:HO matrix elements} for the details) and the result is
\begin{eq}
  \label{eq:HO matrix elements}
  &
  \bra{nlm} H \ket{n'lm} =
	\frac{\hbar\omega}{2}
	\left(
    \p{2n+l+\frac{3}{2}} \delta_{nn'}
    +
		\sqrt{n(n+l+\frac{1}{2})} \delta_{n,n'-1}\right.
		\\ & + 
		\left.\sqrt{n'(n'+l+\frac{1}{2})} \delta_{n',n-1} 
	\right)
	+
	\fint[0][\inf]{r} 
    r^2 R_{nl}(r) V(r) R_{n'l}(r)
\end{eq}
where $R_{nl}$ are the radial wavefunctions of the harmonic oscillator
\begin{eq}
  \label{eq:HO radial wavefunction}
	R_{nl}(r) 
	= 
	N r^l e^{-\gamma r^2 / 2}
	L_{(n-l) / 2}^{(l+\frac{1}{2})}(\gamma r^2)
\end{eq}
with $\gamma = \mu\omega/\hbar$, $L_n^\alpha(x)$ are the generalized Laguerre polynomials and $N$ is a normalization constant,
\begin{eq}
	N = 
  \sqrt{\frac{
    2^{n+l+2}\gamma^{l+\frac{3}{2}}
  }{
    \sqrt{\pi}
  }}
  \sqrt{\frac{
    \p{\frac{1}{2}(n-l)}!
    \p{\frac{1}{2}(n+l)}!
  }{
    (n+l+1)!
  }}.
\end{eq}

\section{Implementation}

Since the HO basis is a discrete basis, implementing it in code is straightforward. There are a few considerations, however, and we mention them here. 

The $\ket{nlm}$ basis is infinite in $n$, but we need a finite matrix, so we truncate the basis at a finite number $N$ giving us an $N \times N$ matrix. Since the matrix is diagonal in $l$ and $m$, we do the calculation separately for each value of $l$ and $m$ to reduce the amount of computation needed to solve for the eigenvalues
The equation we are solving is then, for a given $l$ and $m$,
\begin{eq}
  \sum_{n'= 0}^N \bra{n} H \ket{n'} \psi_{n'} = E\psi_{n}
\end{eq}
or in linear algebra notation
\begin{eq}
  H\psi = E\psi.
\end{eq}
This matrix equation is solved using a standard eigensolver algorithm, which uses the fact that the matrix is hermitian to solve the equation faster.

We calculate the matrix elements using \cref{eq:HO matrix elements}. The integration is performed using Gauss-Legendre quadrature and setting the upper limit to a finite number.

\section{Expanding the \He{5} Nucleus}


\todo{Possible name change: Plane Wave Expansion?}
\section{Discretized Momentum Space}
\label{sec:mom_space}
The reason that we want to solve the problem in momentum space is that we are studying a system with a short-range potential supporting only a few, if any, bound solutions. This means that we will find multiple unbound solutions, corresponding to free particles of various energies. These are basically already eigenstates of the momentum operator, only slightly disturbed by the small potential well at $r=0$. 

The expansion is done in the same way as before, giving us
\begin{eq}
  \int \rd^3 \vec{k}' \bra{\vec{k}} H \ket{\vec{k}'} \Phi(\vec{k}')
  &= 
  E\Phi(\vec{k}) \, .
\end{eq}
This three-dimensional equation is practically unsolvable, but we can simplify it using the fact that we have a central problem. A rather involved calculation (see \cref{sec:radial_mom_space_TISE}) shows that the Schrödinger equation can be written as
\begin{eq} 
  \frac{k^2}{2\mu}\phi(k) + \int_0^\infty \rd k' \, k'^2 V(k,k') \phi(k') 
  &=
  E\phi(k) \, ,
\end{eq}
where $\phi(k)$ is the radial part of the momentum space wavefunction, 
\begin{eq}
  V(k,k') 
  &= 
  \frac{2}{\pi}\int_0^\infty \rd r \, r^2 V(r) j_l(kr) j_l(k'r) 
\end{eq}
and $j_l(kr)$ are the spherical bessel functions of order $l$. We see that the transformation turned the differential equation into an integral equation.

\subsection{Numerical Considerations}
As with the harmonic oscillator basis we want to rewrite the equation as a finite matrix equation. An integral equation can be rewritten as a matrix equation by approximating the integral with a numerical quadrature, 
\begin{eq}
  \label{eq:discrete_momentum}
  \int_0^\infty \rd k' \, k'^2 V(k,k')\phi(k') 
  \approx
  \sum_{j=1}^N w_j k_j^2 V(k,k_j)\phi(k_j)
\end{eq}
where $w_j$ are the quadrature weights. For the na\"{i}ve rectangular quadrature you would use a constant $w_j=\Delta k_j$, equal to the step length. However, this quadrature converges slowly to the correct value of the integral, and much better alternatives can be employed. We are using the G-L quadrature.

With this approximation the Schrödinger equation may be written
\begin{eq}
  \sum_j H_{ij} \phi_j &= E \phi_i
\end{eq}
where $\phi_i=\phi(k_i)$ and 
\begin{eq}
  H_{ij} &= \frac{k_i^2}{2\mu}\delta_{ij} + w_jk_j^2 V_{ij} \\
  V_{ij} &= \frac{2}{\pi} \int_0^\infty \rd r \, r^2 V(r) j_l(k_i r) j_l(k_j r)
\end{eq}
The equation is now written as a matrix equation of order N -- the number of states included in the basis. The energy eigenvalues $E$ will be obtained by calculating the matrix elements $H_{ij}$ and diagonalizing the resulting matrix. Since there is generally no analytic expression for the terms $V_{ij}$, they will need to be evaluated by numerical integration.

To speed up calculations we can transform the equation so that the matrix will be symmetric. This is achieved by the transformation
\begin{eq}
  \phi_i &\mapsto
  \phi_i' =  \sqrt{w_i} k_i \phi_i
  \\
  H_{ij} &\mapsto
  H_{ij}' 
  = 
  \sqrt{\frac{w_i}{w_j}} \frac{k_i}{k_j}H_{ij}
\end{eq}
We would then have
\begin{eq}
  \sum_j H'_{ij}\phi'_j = E\phi'_i\,,
\end{eq}
meaning that the eigenvalues could be just as well obtained by diagonalizing the symmetric $H'$ matrix, thus saving precious time. We have
\begin{eq}
  H_{ij}' = \frac{k_i^2}{2\mu}\delta_{ij} + \sqrt{w_i w_j}k_i k_j V_{ij}
\end{eq}
with $V_{ij}$ defined as above.



\section{The Hydrogen Atom} 

To test our HO and plane-wave bases, we expand the well known and analytically solvable hydrogen atom.
We compare the results in \cref{fig:ho_mom}.
Here we have the unit \emph{Hartree} for energy, which makes the sought value \unit{0.5}{H}.
We see that the plane-wave expansion converges faster than the harmonic oscillator, but towards the wrong value.
This is because the momentum space expansion not support the use of non-finite potentials and since the coulomb potentioal of the hydrogen atom is propotional to$\frac{1}{r}$, a small deviation is not unexpected.\todo{is expected or as it is?}




\todo{Fixa detta. Figur finns redan tror jag.}




\chapter{The \He{5} Nucleus}
\label{cha:he5}
\todo{Where is the theory for \He{5} potential? We should mention Woods-Saxon.}
A comparison of performance between HO and momentum basis expansion for the hydrogen atom and \He{5} problems is shown in \cref{fig:HO vs mom}.
\begin{figure}
  \centering
    \includegraphics[width = \textwidth]{figures/HOvsMom.pdf}
  \caption{}
  \label{fig:HO vs mom}
\end{figure}

Let us study the obtained solutions closer. If we transform the momentum wavefunctions $\phi(k)$ to radial position wavefunctions $R(r)$ according to \cref{sec:radial_mom_space_TISE},
\todo{Maybe wavefunctions are bit too technical for this section. Move to numerical section?}
\begin{eq}
R(r)=i^l\sqrt{\frac{2}{\pi}} \int_0^\infty \rd k \, k^2 \phi(k)j_l(kr) \, ,
\end{eq} 
we can see the spatial distribution of our solutions. In our discretized basis this would be written
\begin{eq}
R(r)=i^l\sqrt{\frac{2}{\pi}}\sum_{j=1}^N \sqrt{w_j}k_j\phi_j'j_l(k_j r) \, .
\end{eq}
\Cref{fig:momspace solutions} shows the wavefunctions $R(r)$ for a few of the states with lowest energy.

\begin{figure}
  \centering
  \includegraphics[width=1\textwidth]{mom_solutions.pdf}
  \caption{A few solutions to the Woods-Saxon potential with well depth $V0=\SI{-52}{MeV}$. The probability distributions $r^2|R(r)|^2$ are plotted relative to their energies. }
  \label{fig:momspace solutions}
\end{figure}

 While all of the states we see have energies $E>0$, and thus are unbound, we see that one solution is more localized near the center $r=0$. This is the sign of a quasi-bound state. To confirm this we may vary the depth $V0$ of the potential well and see how this affects the solutions. \Cref{fig:momspace solutions var} shows the solutions obtained with a different $V0$.
\begin{figure}
  \centering
  \includegraphics[width=1\textwidth]{mom_solutions_var.pdf}
  \caption{A few solutions for $V0=\SI{-47}{MeV}$.}
  \label{fig:momspace solutions var}
  \todo{This fig could be combined with previous}
\end {figure}
We see that the unbound states remain practically unchanged. This means that they basically correspond to free particles of energies $E_n=\frac{k_n^2}{2\mu}$, where $k_n$ are the momenta that were included in the discretization of the integrals. We will refer to these values of $k$ as our \emph{mesh points}. The quasi-bound state changed dramatically, which shows that this solution is a feature of the system we are studying.  
\todo{something here should lead to wanting NHQM}

\chapter{Non-Hermitian Quantum Mechanics} 
\label{cha:nhqm}
We have noticed that one of the \He{5} states behaves 
differently than the others. We suspect this is the 
resonance state, but we cannot yet quantify its width, 
or half-life. This means we need to find the complex 
energy of the state. We do this by using the theory of 
Tore Berggren and his \emph{Berggren basis} \cite{berggren}. 
The theory will not be fully explained here, instead we 
present a heuristic argument.

If we relate the energies $E$ of a system to momenta $k$ as
\begin{eq}
  E = \frac{\hbar^2 k^2}{2\mu}
  \quad\quad
  \textup{or}
  \quad\quad
  k = \frac{\sqrt{2\mu E}}{\hbar},
\end{eq}
we can plot the energies as $k$ in the complex plane, see 
\cref{fig:complex plane}. We then expect bound states, with 
$E<0$, to be represented by $k$ along the imaginary axis---
whereas unbound, scattering states, with $E>0$, are found 
along the real axis. Resonance states, with complex 
$E = E_0 - i \Gamma /2$, would by this argument appear 
in the fourth quadrant.

\todo{Maybe show results before complexifying? We could relate the mesh points to the solutions here.}

We now interpret these $k$ as poles, and our integration 
\begin{eq}
  \fint[0][\inf]{k} k'^2 V(k,k') \phi(k')
\end{eq}
as a contour integration around the upper half plane, see \cref{fig:simple contour}. The result of a contour 
integration depends on the poles it encircles by the residue 
theorem, so we expect something to happen if we let the 
contour encircle the pole of the resonance.

\begin{figure}[H]
  \centering
    \includegraphics[width = \textwidth]{figures/complex_plane.pdf}
  \caption{The complex $k$-plane. The circles represent 
  bound states, the diamonds unbound states and the 
  crosses resonant states. Note the mirroring in
  the imaginary axis.}
  \label{fig:complex plane}
\end{figure}

\section{The Complex Contour}

We choose to extend our integration along the real axis to 
the simplest possible complex contour, a triangle-shaped 
extrusion downwards. The tip of the triangle is placed directly 
below the hypothesized resonance pole.

Numerically, we are faced with the problem of how to choose 
the points and weights, now that the contour is complex. 
We consider each straight segment of the contour separately, 
and rescale the Gauss-Legendre points between the 
of the segment. An example of a contour is seen in 
\cref{fig:triangle contour}.

\begin{figure}
  \centering
    \includegraphics[width = \textwidth]{figures/complex_contour.pdf}
  \caption{The complex contour used.}
  \label{fig:triangle contour}
\end{figure}

\section{Studying the Resonance}



\todo{NHQM doesn't arise, we make it happen?}
NHQM is a must when you observe the binding momentum, related to the energy eigenvalues as
\begin{eq}
    E=\frac{p^2}{2m}.
\end{eq}
This energy will be complex if we expand the hamiltonian of a resonance state in a Berggren-basis. 
It is now the physics go non-hermitian.


We will now introduce a generalization of the momentum-space expansion which allows ut to calculate these complex energies.
First, if we look at our momentum-space expansion, we see that it consists of some discrete momentum-values along the positive real axis.
This is fine as long as the states we observe are bound.
If they on the other hand are quasi-bound resonance-states, they will appear as poles in the complex momentum-plane.
To calculate the behavior of these states we have to introduce a new basis that allows for complex values instead of just real ones.
This new basis is called a \emph{Berggren basis} after the Swedish mathematician Tore Berggren.
%We will refer to this new basis as a contour since we could replace our integration path by a closed contour that would give the same result \cite{Berggren}.

To do this generalization introduces some strange behaviors.
First of all we can now aquire complex eigenvalues when solving the Schrödinger-equation, which is the main reason for switching to a Berggren-basis.
There also happens a strange thing with the bras in the equations.
Their representation as wave-functions do no longer have conjugate signs during calculations, this is above our heads to comprehead why this happens, but is explained by Berggren in his paper \cite{berggren}.





%The way to do this is to instead of using a strictly real basis introduce a complex one, called a Berggren-basis (Should we cite here?).
%The Berggren-basis is expressed as a contour in the complex plane.
%The one we used, with slight modifications, is seen in \cref{fig:berggren contour}.
%By lowering this contour to contain the pole we include the resonance in the calculations and are thus able to predict its behavior.
%Last, one have to include the resonance pole in the basis for it to be complete \cite{berggren}.

\section{Implementation}
Implementing the Berggren-basis is pretty straight-forward when momentum-space is made.
The hardest part to do is to allow for different contours, which is easy if the code is well structured.

One important thing that may severly increase computation time is which points and weights to use.
Recall \cref{eq:discrete_momentum} from earlier, here we said that we could use equaly distanced points for the calculations.
\todo{Repeating benefits of G-L. We should rather mention how we use G-L on the segments of the contour.}
If we on the other hand change these equally distanced points to ones described by the Gauss-Legendre approximation to integrals we get a much faster convergence.
This way of selecting points is used once for each of the three parts of our contour since we not know of any godd way to do this for the whole contour at once.



\section{Results}
By remodelling our old mom-space solution to allow different contours, we were ready to get som results to see if this new model was correct. The values of our constants was given to us by our supervisors since they are derived from experiments.
\todo{From where do we get our $V_0$ and $V\sub{so}$}


\chapter{Many-Body Theory}
\label{cha:many-body}
\documentclass[../main/report.tex]{subfiles}
\begin{document}
  
\chapter{Many-Body Theory and Implementation}
\label{cha:many-body} 

We have solved the \He{5} nuclear two-body problem and studied its resonances.
The next step is to add another neutron and subsequently solve the three-body problem. 
However, while the two-body problem is reducible to a radial one-dimensional problem, the general many-body problem is not.
Instead, we need to use a single particle (sp) basis to construct many-body states.
This chapter covers the construction of such states.

First, \cref{sec:identical_particles} discusses the mathematical consequences of identical, indistinguishable particles, focusing on fermions.
This is followed in \cref{sec:second_quantization} by a short introduction to the second quantization formalism, which allows calculations with an arbitrary number of particles.
The line of reasoning as well as the notation of these sections is adapted from \cite{dickhoff}, to facilitate further reading for the interested reader. 
With the same intention, the style of \cite{suhonen} is employed in \cref{sec:coupling} where the concept of angular momentum coupling is briefly explained.
Finally, in \cref{sec:mb_implementation} we discuss the use of second quantization in computations and present a simple implementation for fermions.

\section{Identical Particles}
\label{sec:identical_particles}

A quirk of quantum mechanics is that particles that look identical \emph{are} identical.
For example, consider the nucleons in a nucleus. 
The neutrons and protons have different charge, and can thus be told apart, but distinguishing between individual neutrons is impossible.
This has to be taken into consideration when dealing with many-body states of identical particles, as we will see.

We begin with an orthonormal single particle (sp) basis $\ket{\alpha_i}$, where $\alpha_i$ represents all the quantum numbers that describe the state.
Next, consider $N$ identical particles, expressed in this basis. We form product states
\begin{eq}
  \pket{\alpha_1\alpha_2\dots\alpha_N} 
  \equiv
  \ket{\alpha_1} \otimes \ket{\alpha_2} \otimes \dots \otimes \ket{\alpha_N}
  =
  \ket{\alpha_1}\ket{\alpha_2}\dots\ket{\alpha_N},
\end{eq}
which, by the orthonormality of the $\ket\alpha$, are orthonormal as well
\begin{eq}
  \pbraket{\alpha_1\alpha_2\dots\alpha_N}{\alpha'_1\alpha'_2\dots\alpha'_N}
  =
  \delta_{\alpha_1\alpha'_1}
  \delta_{\alpha_2\alpha'_2}
  \dots
  \delta_{\alpha_N\alpha'_N}.
\end{eq}

\todo{orthonormality or completeness relation?}

Let us assume that the system of identical particles can be described by some linear combinations of these basis states, denoted
\begin{eq}
  \ket{\alpha_1\alpha_2\dots\alpha_N}.
\end{eq}
Since the particles are identical, and thus indistinguishable, we require the norm of the state to be unchanged when swapping the quantum numbers of two particles $\beta$ and $\gamma$
\begin{eq}
  \braket{\alpha_1\dots\beta\dots&\gamma\dots\alpha_N}
    {\alpha_1\dots\beta\dots\gamma\dots\alpha_N} \\
&=\braket{\alpha_1\dots\gamma\dots\beta\dots\alpha_N}
    {\alpha_1\dots\gamma\dots\beta\dots\alpha_N}
\end{eq}
These states can therefore only differ in phase $e^{i\varphi}$, and since another swap will bring us back to the original state, the phase has to be either $e^{i\varphi} = 1$ or $e^{i\varphi} = -1$.
Symmetric states with no phase change describe \emph{bosons}, whereas antisymmetric states that change sign describe \emph{fermions}.
In this thesis all sp states will be fermionic, hence we do not treat the bosonic case.


\subsection{Antisymmetric Fermion states}

We have now established that our fermion many-body states are a linear combination of product states that satisfy
\begin{eq}
  \ket{\alpha_1\dots\alpha_i\dots\alpha_j\dots\alpha_N}
  =
  -\ket{\alpha_1\dots\alpha_j\dots\alpha_i\dots\alpha_N}.
\end{eq}
For example, in the case of two particles, the correctly normalized antisymmetric state is
\begin{eq}
  \ket{\alpha_1\alpha_2} 
  = 
  \frac{1}{\sqrt{2}}
  \bigp{
    \pket{\alpha_1\alpha_2} - \pket{\alpha_2\alpha_1}
  }.
\end{eq}
We will henceforth use the angular ket notation $\ket\dots$ for antisymmetric states, as opposed to $\pket\dots$ for product states.

Let us ponder that two fermions occupy the same state. 
Exchanging the two particles and flipping the sign would then result in
\begin{eq}
  \ket{\alpha\alpha\alpha_1\dots\alpha_N} 
  = 
  - \ket{\alpha\alpha\alpha_1\dots\alpha_N}
\end{eq}
which can only be true if both states are 0. 
We conclude that two fermions can never occupy the same state, commonly referred to as the \emph{Pauli Principle}.

It is important to note that states with permuted quantum numbers, such as the states $\ket{\alpha_1\alpha_2}$ and $\ket{\alpha_2\alpha_1}$, represent the same physical state, as they only differ in sign (phase). 
This means that we have to make sure not to double count these states. 
We can do this by requiring that the sp states always appear in the same order in the ket. 
If they do not, we permute two sp states at a time until the correct ordering is reached
\begin{eq}
  \ket{\alpha_i\alpha_1\dots\alpha_{i-1}\alpha_{i+1}\dots\alpha_N}
  & =
  - \ket{\alpha_1\alpha_i\dots\alpha_{i-1}\alpha_{i+1}\dots\alpha_N}
  \\ & =
  \dots
  \\ & =
  (-1)^{i-2} 
  \ket{\alpha_1\dots\alpha_i\alpha_{i-1}\alpha_{i+1}\dots\alpha_N}
  \\ & =
  (-1)^{i-1} 
  \ket{\alpha_1\dots\alpha_{i-1}\alpha_i\alpha_{i+1}\dots\alpha_N}.
\end{eq}
With a well-defined ordering orthonormality of the normalized antisymmetric states can be stated simply
\begin{eq}
  \braket{\alpha_1\alpha_2\dots\alpha_N}{\alpha'_1\alpha'_2\dots\alpha'_N}
  =
  \delta_{\alpha_1\alpha'_1}
  \delta_{\alpha_2\alpha'_2}
  \dots
  \delta_{\alpha_N\alpha'_N}.
\end{eq}


\section{Second quantization}
\label{sec:second_quantization}

So far we've looked at a system with a fixed number of particles, but we want to work with a system hosting an arbitrary number of identical particles.
The \emph{second quantization} formalism lets us do this by introducing the \emph{Fock space}, a combination of Hilbert spaces of $0,1,2,...$ particles.
This means that a state in Fock space, a \emph{Fock state}, can contain any number of particles and that Fock states with different number of particles are orthogonal.

A disclaimer is in place here: while we introduce this powerful concept, we do not make full use of it in this thesis. 
It should instead be considered as a stepping stone for readers wishing to expand on the systems and methods we do use.

\subsection{Creation and Annihilation Operators}

The simplest Fock state is the \emph{vacuum state} $\ket{0}$, which describes a system with no particles. 
All other states can be created from the vacuum state using the \emph{creation operator} $a_\alpha^\dag$, which adds a particle with quantum numbers $\alpha$ to a state
\begin{eq}
  \label{eq:create}
  a_{\alpha}^{\dagger} \ket{\alpha_1 \alpha_2 ... \alpha_N} 
  =
  \ket{\alpha \alpha_1 \alpha_2 ... \alpha_N}.
\end{eq}
The resulting state will not necessarily be ordered, and the ordering might contribute a sign:
\begin{eq}
  \label{eq:create_ordered}
  a_{\alpha_i}^{\dagger} 
  \ket{\alpha_1 \alpha_2 ... \alpha_{i-1} \alpha_{i+1}...\alpha_{N}} 
  =
  (-1)^{i-1} 
  \ket{\alpha_1 \alpha_2 ... \alpha_{i-1} \alpha_i \alpha_{i+1} ... \alpha_{N}}.
\end{eq}
Note that when $a_\alpha^\dag$ acts on a state that already contains a particle with quantum numbers $\alpha$, the result is 0, because of the Pauli principle
\begin{eq}
  \label{eq:create_zero}
  a_{\alpha}^{\dagger} \ket{\alpha\alpha_1 \alpha_2 ... \alpha_N} 
  =
  0.
\end{eq}

The adjoint of the creation operator is called the \emph{annihilation operator} $a_\alpha$. 
It can be shown to have the opposite effect, removing a particle, when acting on a state
\begin{eq}
  \label{eq:annihilate}
  a_{\alpha} \ket{\alpha \alpha_1 \alpha_2 ... \alpha_N}
  =
  \ket{\alpha_1 \alpha_2 ... \alpha_N}.
\end{eq}
Here, too, a sign might appear from the ordering
\begin{eq}
  \label{eq:annihilate_ordered}
  a_{\alpha_i}
  \ket{\alpha_1 \alpha_2 ... \alpha_{i-1} \alpha_i \alpha_{i+1} ... \alpha_N}
  =
  (-1)^{i-1}
  \ket{\alpha_1 \alpha_2 ... \alpha_{i-1} \alpha_{i+1}...\alpha_N}.
\end{eq}
Analogous to $a_\alpha^\dag$, when $a_\alpha$ acts on a state that does not contain a particle with the quantum numbers $\alpha$, the result is 0
\begin{eq}
  \label{eq:annihilate_zero}
  a_\alpha \ket{\alpha_1 \alpha_2 ... \alpha_N} 
  =
  0.
\end{eq}


\subsection{General Operators in Fock Space}

We can now express the state of an arbitrary number of particles, but to have any use of the states we also need to express operators such as the Hamiltonian in the Fock space formalism. 
A general operator has can in Fock space be expressed using the creation and annihilation operators.
The Fock space equivalent of an operator is able act on a state with an arbitrary number of particles. 
We will only treat one- and two-body operators here, as they are sufficient for our purposes.

\subsubsection{One-Body Operators}

A one-body operator $H_1$ which acts on a single sp state, is represented by the Fock space operator
\begin{eq}
  \hat{H}_1
  =
  \sum_{\alpha \beta} 
  \bra\alpha H_1 \ket\beta 
  a_\alpha^\dag a_\beta.
\end{eq}
It is important to note that while the sum runs over the complete set of sp states twice, only a few terms will be non-zero, because of the operator rules in \cref{eq:create_zero,eq:annihilate_zero}. 

If the sp-states are eigenstates to the one-body operator
\begin{eq}
  H_1 \ket{\alpha} = h_\alpha \ket{\alpha}
\end{eq}
the matrix elements only exist on the diagonal, when $\alpha = \beta$, and we get
\begin{eq}
  \hat{H}_1
  =
  \sum_{\alpha} 
  \bra\alpha H_1 \ket\alpha
  a_\alpha^\dag a_\alpha.
\end{eq}
The many-body matrix element becomes
\begin{eq}
  \label{eq:one-body_matrix_elements}
  \bra{a_1\dots a_N} \hat{H}_1 \ket{b_1\dots b_N}
  & =
  \sum_{\alpha} 
  \bra\alpha H_1 \ket\alpha
  \bra{a_1\dots a_N} 
  a_\alpha^\dag a_\alpha
  \ket{b_1\dots b_N}
  \\ & =
  \sum_{i = 1}^N 
  \bra{a_i} H_1 \ket{a_i}
  \braket{\alpha_1\dots\alpha_N}{\alpha'_1\dots\alpha'_N}
  \\ & =
  \p{
    h_1 + \dots + h_N
  }
  \delta_{a_1 b_1} \dots \delta_{a_N b_N},
\end{eq}
the sum of the eigenvalues of the sp states in the bra or ket, but only if the bra and ket are the same. The Fock space operator $\hat{H}_1$ is thus also diagonal.

\subsubsection{Two-Body Operators}

A two-body operator in Fock space becomes
\begin{eq}
  \hat{H}_2
  =
  \frac{1}{2}\sum_{\alpha \beta \gamma \delta} 
  \pbra{\alpha \beta} H_2 \pket{\gamma \delta} 
  a_\alpha^\dag a_\beta^\dag a_\delta a_\gamma.
\end{eq}
Note that the ordering of the $\gamma$ and $\delta$ is different for the product states and the operators, so-called \emph{normal ordering}.
The factor \nicefrac{1}{2} stems from the fact that %%%%%%%%%%%%%%%%%%%%%%%%%%%
\begin{eq}
  \pbra{\alpha \beta} H_2 \pket{\gamma \delta} 
  = 
  \pbra{\beta \alpha} H_2 \pket{\delta \gamma},
\end{eq}
and we are counting both.

We can also express $\hat{H}_2$ using matrix elements between antisymmetric states
\begin{eq}
  \bra{\alpha\beta} H_2 \ket{\gamma\delta} 
  = 
  \pbra{\alpha\beta} H_2 \pket{\gamma\delta}
  -
  \pbra{\alpha\beta} H_2 \pket{\delta\gamma},
\end{eq}
but we will have to add another factor \nicefrac{1}{2} to compensate for double counting
\begin{eq}
  \hat{H}_2
  =
  \frac{1}{4}\sum_{\alpha \beta \gamma \delta} 
  \bra{\alpha \beta} H_2 \ket{\gamma \delta} 
  a_\alpha^\dag a_\beta^\dag a_\delta a_\gamma.
\end{eq}
The double counting can be avoided, however, by taking into account the ordering of the states
\begin{eq}
  \hat{H}_2
  =
  \sum_{\substack{\alpha < \beta \\ \gamma < \delta}} 
  \bra{\alpha \beta} H_2 \ket{\gamma \delta} 
  a_\alpha^\dag a_\beta^\dag a_\delta a_\gamma.
\end{eq}

For the special case of two particles we have
\begin{eq}
  \label{eq:two-body_matrix_elements}
  \bra{ab} \hat{H}_2 \ket{cd}
  & =
  \sum_{\substack{\alpha < \beta \\ \gamma < \delta}} 
  \bra{\alpha \beta} H_2 \ket{\gamma \delta} 
  \bra{ab} 
  a_\alpha^\dag a_\beta^\dag a_\delta a_\gamma
  \ket{cd}
  \\ & =
  \sum_{\substack{\alpha < \beta \\ \gamma < \delta}} 
  \bra{\alpha \beta} H_2 \ket{\gamma \delta}
  \delta_{\alpha a}\delta_{\beta b}
  \delta_{\gamma c}\delta_{\delta d}
  \\ & =
  \bra{ab} H_2 \ket{cd},
\end{eq}
as expected.

\section{Angular Momentum Coupling}
\label{sec:coupling}

We have now developed the theory we need to solve actual many-body problems.
Before applying these methods, however, we will discuss the concept of angular momentum coupling. 
This corresponds to making a change of basis, using the rotational symmetry of problems to make the Hamiltonian matrix block diagonal and significantly reducing its size. 
In principle, coupling is not neccesary to solve many-body problems, but its an invaluable tool.
We will limit the discussion to coupling of two particles, as we never work with more particles here.
A more complete description is found in \cite{suhonen}.

\subsection{The Two-Particle Coupled Basis}
In previous Chapters we have studied single-particle states on the form $\ket{Ejm}$, where we let the quantum number $E$ represent all quantum numbers needed to uniquely specify a state. From these we would form product states
\begin{eq}
  \pket{E_1j_1m_1, E_2j_2m_2} = \ket{E_1j_1m_1}\otimes\ket{E_2j_2m_2}
\end{eq}
that are eigenstates to the operators $\vec{J}_1 = \vec{J}\otimes\vec{1}$ and $\vec{J}_2 = \vec{1}\otimes\vec{J}$.

One often studies systems where the total angular momentum $\vec{J} = \vec{J}_1 + \vec{J}_2$ is conserved, but the individual angular momenta $\vec{J}_1$ and $\vec{J}_2$ are not. Conservation of $\vec{J}$ is equivalent to the entire system being symmetric under rotation, while $\vec{J}_1$ and $\vec{J}_2$ will only be conserved if one of the particles can be rotated independently of the other, without affecting the solutions. 
This is rarely the case when studying directly interacting particles.
It is then convenient to switch to a basis where the total angular momentum is well defined, but the individual momenta are not. 

We introduce the \emph{coupled} basis with states $\pket{E_1j_1, E_2j_2; JM}$ meant to be read as: the first particle is described by the quantum numbers $E_1 j_1$, the second by $E_2 j_2$, and they together have a total angular momentum $JM$ (the $M$ is sometimes left out).
The coupled states are related to the uncoupled by
\begin{eq}
  \pket{E_1j_1, E_2j_2; JM} 
  = 
  \sum_{m_1, m_2} c_{m_1 m_2} \pket{E_1j_1m_1, E_2j_2m_2} .
\end{eq}
where $c_{m_1 m_2}$ are the \emph{Clebsch-Gordan coefficients} (which also depend on $j_1, j_2, J$ and $M$, though this is suppressed for brevity). 
There are known expressions for these, and values can be found in standard tables or calculated in a fairly straight-forward way. 
A more in-depth treatment of the coefficients can be found in \cite{suhonen}.

Working with matrix elements between coupled basis states is called working in the \emph{coupled scheme} or \emph{$J$-scheme}.
If we study a system with rotational symmetry, the Hamiltonian will be block diagonal in $J$ and $M$, meaning that working in the coupled scheme will be much more efficient.

\subsection{Antisymmetrizing the Coupled Basis}

Since we are studying fermions, we need to use basis states that are antisymmetric with respect to exchange of all quantum numbers. Using the $m$ symmetry property of the Clebsch-Gordan coefficients,
\begin{eq}
  c_{m_2 m_1} = (-1)^{j_1 + j_2 - J} c_{m_1 m_2} 
\end{eq}
we can see that
\begin{eq}
  \pket{E_2j_2, E_1j_1; JM} 
  & = 
  \sum_{m_1, m_2} c_{m_1 m_2} \pket{E_2 j_2 m_1, E_1j_1 m_2} 
  \\ & = 
  (-1)^{j_1+j_2-J}\sum_{m_1, m_2} c_{m_1 m_2} \pket{E_2 j_2 m_2, E_1 j_1 m_1}
\end{eq}
Hence it is possible to form the antisymmetric basis vector
\begin{eq}
  \ket{E_1 j_1 E_2 j_2; JM} &= \frac{1}{\sqrt{2}}\bigp{\pket{E_1 j_1 E_2 j_2;JM} - (-1)^{j_1+j_2-J}\pket{E_2 j_2 E_1 j_1;JM}} \\
  &= \sum_{m_1, m_2} c_{m_1 m_2} \ket{E_1 j_1 m_1, E_2 j_2 m_2}.
\end{eq}

Consider the case where both particles occupy the same orbital, $E_1 j_1 = E_2 j_2 = E j$. 
Since, for fermions, $j$ is a half-integer we have $(-1)^{j_1+j_2 - J} = - (-1)^J$ and the result is
\begin{eq}
  \ket{(Ej)^2; JM} = \frac{1+(-1)^J}{\sqrt{2}}\pket{(Ej)^2; JM} .
\end{eq}
We see that this state is equal to zero for $J$ odd. However, for $J$ even, we find that the norm
\begin{eq}
  \braket{(Ej)^2; JM}{(Ej)^2; JM} = 2
\end{eq}
meaning that we have to normalize these states with an additional factor $\nicefrac{1}{\sqrt{2}}$. 
We can write this result succinctly as 
\begin{align}
  \ket{ab; JM} 
  &= 
  \N_{ab}\sum_{m_1, m_2} c_{m_1 m_2} \ket{E_1 j_1 m_1, E_2 j_2 m_2}
  \\ & 
  \N_{ab} = \frac{\sqrt{1+(-1)^{J}\delta_{ab}}}{1+\delta_{ab}}
\end{align}
where we have used the notation $(a, b)$ for $(E_1j_1, E_2j_2)$.

\todo{OLA OLA THEOREM}

\section{Second Quantization Implementation}
\label{sec:mb_implementation}

To use the Fock space formalism in numerical calculations we have to represent the quantum states using available data structures. 
We construct sp states and use them to form antisymmetric Fock states.
The creation and annihilation operators become functions that take the Fock states as arguments, and are used to create the Fock operators. 
Furthermore, the sum in the expression of the Fock operators can be optimized by only evaluating the non-zero terms.

\subsection{Single-Particle-State Objects}

We represent a single particle state with a record, i.e. an object with named fields that can be assigned values. It is natural to let each quantum number, such as $l$ and $j$ be a field. 
Moreover, we can include other information about the state, information that is taken for granted in the mathematical formulation.
This includes a unique index for each state, the eigenvector holding information about the wavefunction, the basis of the eigenvector and, in the case of the plane wave basis, the contour used.
Much of this extra information is redundant, as many states share the same information.
Nevertheless, we found that this representation significantly simplifies the structure of the program and makes it easier to understand.

\subsection{Fock State Objects and Operator Functions}

An antisymmetric many-body state $\ket{\alpha_1\dots}$ is represented by an ordered list of single particle objects and a sign. 
Since the sp state objects have a unique index, there is a well-defined sorting order. 
The sign can be $1$, $-1$ or $0$, representing $\ket{\alpha_1\dots}$, $-\ket{\alpha_1\dots}$ and $0$, respectively. 

The creation and annihilation operators are implemented as functions on the Fock state objects, obeying
\cref{eq:create,eq:create_ordered,eq:create_zero,eq:annihilate,eq:annihilate_ordered,eq:annihilate_zero}.
The creation operator function steps through the list, flipping the sign at each step, until the correct place for the new particle is found. 
If the particle is already part of the Fock state, the sign is set to 0.
The annihilation operator searches for a state, saves its index $k$ in the list, annihilates it and set the sign to $(-1)^k$. If the state not exists, the sign is set to 0.

\subsection{Fock Operator Matrix Elements}

When computing matrix elements of a Fock space operator, most terms in the sum vanish. 
This is because most states will become zero when acted on by the creation and annihilation operators (equations \cref{eq:create_zero,eq:annihilate_zero}).
Evaluating a sum of mostly zero elements is not very efficient, but this can be 
avoided by letting the sum run over just the sp states present in the bras and kets.

\todo{behöver detta vara tydligare? /pontus}

\end{document}

\section{Three-Body implementation}
In order to solve the three body-problem, or rather the He-6 system modeled as a He-4 nucleus with two orbiting neutrons, we had to translate the governing mathematical and physical relations to something better suited for numerical solutions.
 This Section gives a description of the models and simplifications we have used.

\subsection{our special case and simplifications}
%//core - nucleus
\subsection{multistep method}
We have utilized the multistep method to find the solution to a multi-body (mb) problem through solutions of incremental many-body problems.
 We started out by solving the two body problem of a core (He-4) and a single orbiting particle (sp) (neutron, forming He-5 -there was no bound state, only resonances as can be expected, see ref ref ref) and used these solutions to form mb-states out of different combinations of the sp-states; see the fock space section in mb-theory.
 We have only reached the second step in the multi-step method but this is the greatest (largest?) step, most of the operators that govern any mb-problem are needed for a three-particle problem whereas a two-body particle can be reduced to a one-body-problem through the use of relative quantities.
 
\subsection{Many-body states}
The governing theory of mb-states is covered in \ref{sec:fock_space} so this section will focus on one way of implementing the states. 
%We enumerate the solutions to the He-5 problem as a basis for the mb states and include information about the state's angular momentum.
%Our first simplification is that we only regard states with $\alpha, \beta$, where $\alpha_{sp}$ and $\beta_{sp} $ are sp-states and $\alpha_{sp} <= \beta_{sp}$ in terms of enumeration so as to avoid oversumming, recall the relation between alpa, beta :: beta, alpha ref ref . 
%This will gives us a smaller matrix between the different mb states and saves us a lot of computation-time.

A many-body state is represented as a number representing the sign and a list mb-states.
 The mb-state is a set of sp-states, one for each orbiting particle (represented as the sp-state's k value). 
 The sp-states can also be extended to hold information about other relevant quantum numbers, we will treat states with (coupled) angular momentum in the calculations but the extra information is best supplied when it's used for calculations.
 
%each state is a set of numbers signifying the relevant quantum numbers of the state.
 %The state is a pair of numbers one indicates: which single-particle solution is represented (we store the k-value of the sp-state) and the other is the angular momentum which is in the range of $[-3/2,3/2]$.
 
 

 The mb-state is formed by taking a vacuum-state, an empty list, and adding a single particle state along with any other quantum number needed for the calculations, this is described mathemtically in \ref{eq:creation}.
 To avoid oversumming the single particle states are ordered in a lexicographic order and our calculations only require one of the permutations of a set of given sp-states, an integer is used to keep track of the sign of a given permutation. 
 Because of the fermionic nature of our paticles we do not allow two sp-states with the same quantum numbers.
 This integer is calculated as in \ref{eq:creation sign} and if one tries to create a state which is already present, as is the case in \ref{eq:creation zero}, the sign will be set to 0 and it will be regarded as a vacuum-state in our calculations.
 The annihilation operator is implemented in a similar fashion but it instead removes a given state (set of quantum numbers) and returns a new FermionState without the given to-be-removed state.
 
% To keep track of all the many-body states and to perform sums over all possible many-body states we arrange a list with all the states and signify each many body state as the index representing the state in this list
 
\subsection{Generating many-body Hamiltonian}
To calculate the Hamilton matrix for the many-body states we generate a list of all possible mb-states and let the (i,j)-th element of the matrix be the hamiltonian contribution from the i-th mb-state in a bra and the j-th mb-state in a ket. 
 In the case of two orbiting particles the two-body operator will work on all combinations of bras and kets but in the four or more body problem only some combinations will contribute.
 To find which combinations of bras and kets contribute to the two-body operator one can take a ket and remove two sp-states and iteratively add all possible combinations of two sp-states to the new ket and check whether the bra is the same as the new ket.
 On the diagonal however there will also be a contribution from the one-body operator.
 The contributions from the two operators are discussed in detail below.

\subsubsection{One-body opreator}
The one-body operator is the simple kinetic energy operator that we have known and loved since our first food fight. 
%Presented on the form of \ref{eq: stans i mb-teori} the one-body operator, \fockop{T} simply yields the kinetic energy of the two particles, but in a matrix representation 
 In our matrix representation the bra and the ket will be the same in a diagonal-element and thus have the same single particle states.
 This operator returns the kinetic energy of these two single particle states. 
 These energies are eigenvalues to the sp-hamilton matrix and have already been calculated and can easily be retrieved.

one body >==< clebsxhgordan???

\subsubsection{Two-body operator}
The two-body operator is the contribution from the neutron-neutron interaction and is computationally taxing. Although the potential is (ridiculously) simple this calculation requires sums over all the quantum numbers in both the bra and the ket; the mathematical expression for the hamiltonian is presented in \ref{two-body op sum}. 
With a simple contour of 15 points this would be a sum over $15^4*4^4 \approx 13$ million elements which is computationaly unfeasible.
In order to reduce the complexity we instead make use of coupled angular momentum for the many-body states, refer to ref ref ref.

%the (i,j)-th matrix element, $H_{ij}$, would be the interaction between the i-th and j-th mb-state. and for each pair of indexes calculate a matrix element; 


In order to determine the hamilton matrix for the mb-system we generate a list of all possible (non-permutated) mb-states, constructed only from different sp-states.
For a given bra and ket we determine whether the fock-space relations allow for an interaction, in this case with two-orbiting particles this is trivially true.
This is where we introduce the different allowed m-quantum numbers, the original bra and ket are used to generate all possible mb-states with the given sets of sp-states but with an angular momentum for each of the sp-states.
Thus we can calculate the Clebsch-Gordan coefficients to treat the degeneracy in m-quantum number, recall that Clebsch-Gordan coefficients was covered in ref ref ref.

The calculation of clebsch gordan coeff (and fock relations) severly reduces the number of elements that actually give any contribution at all. 
When we know which bra and ket interactions that give a contribtion it is time to calculate it. 
Mathematically this contribution, from the sepparable n-n potential can be expressed like presented in ref ref
However this is not a suitable form for the computer, not even with the very powerful Gauss-Legendre contour for the integral; instead we rewrite it as a matrix equation:
$1+2+3+4+5+6+7+...+226+... = - \frac{1}{12}$ because fuck you that's why\\
where the matrix is the same for each combination of bra's and ket's and need only be calculated once.

only some (ALL) bra-ket combinations will live, governed by the creation / annihilation relations. 
n-n separable interaction, clever matrix multiplication scheme
clebch-gordan coefficients :-()

\chapter{The \He{6} Nucleus}
\label{cha:he6}


\chapter{The \He{7} Nucleus}
\label{cha:he7}

\chapter{Monte Carlo Approach}
\label{cha:monte carlo}

\appendix

\todo{Gauss-Legendre?}

\chapter{Derivations}

\section{Harmonic Oscillator Matrix Elements}
\label{sec:HO matrix elements}

First, we combine \cref{eq:HO hamiltonian,eq:spherical hamiltonian} to get
\begin{eq}
  H = H\sub{HO} - \frac{\mu\omega^2 r^2}{2} + V(r).
\end{eq}
We then close this equation with $\bra{nlm}$ on the left and $\ket{n'lm}$ on the right to get three terms on the RHS, which we consider in turn. The first is just the eigenvalues of $H\sub{HO}$
\begin{eq}
  \bra{nlm} H\sub{HO} \ket{n'lm} 
  = 
  E_{nl} \braket{nlm}{n'lm} 
  = 
  \hbar\omega\p{2n + l + \frac{3}{2}} \delta_{nn'}.
\end{eq}
The second follows from the known identity % \cite{moshinsky} eller den moshinsky citear?
\begin{eq}
	& \bra{nlm} r^2 \ket{n'lm} 
	= \\
	& \frac{\hbar}{\mu\omega}
  \p{
  	\p{2n + l + \frac{3}{2}}\delta_{nn'}
    -
  	\sqrt{n(n+l+\frac{1}{2})}\delta_{n,n'-1}
  	-
  	\sqrt{n'(n'+l+\frac{1}{2})}\delta_{n',n-1}
  }.
\end{eq}
For the third term, $\bra{nlm} V(r) \ket{n'lm}$, a little more work is required. We begin by inserting a completeness relation on each side of the potential
\begin{eq}
  \label{eq:potential element}
  &\bra{nlm}
  \p{ 
    \fint[\R^3]{^3\vec{r}}
    \ket{\vec{r}}\bra{\vec{r}}
  }
  V
  \p{ 
    \fint[\R^3]{^3\vec{r}'}
    \ket{\vec{r}'}\bra{\vec{r}'}
  }
  \ket{n'lm}
  \\ = & 
  \fint[\R^3]{^3\vec{r}}
  \fint[\R^3]{^3\vec{r}'}
  \braket{nlm}{\vec{r}}
  \bra{\vec{r}} V \ket{\vec{r}'}
  \braket{\vec{r}'}{n'lm}.
\end{eq}
We get the position representation of the potential with
\begin{eq}
  \bra{\vec{r}} V \ket{\vec{r}'} = V(r) \delta^3(\vec{r} - \vec{r}')
\end{eq}
 $\braket{\vec{r}}{nlm}$ are the HO eigenstates in the position basis, which we know to be
\begin{eq}
  \braket{\vec{r}}{nlm} = R_{nl}(r)Y_l^m(\theta, \phi),
\end{eq}
where $Y_l^m$ are the spherical harmonics and 
\begin{eq}
	R_{nl}(r) 
	= 
	N r^l e^{-\gamma r^2 / 2}
	L_{(n-l) / 2}^{(l+\frac{1}{2})}(\gamma r^2)
\end{eq}
with $\gamma = \mu\omega/\hbar$, $L_n^\alpha(x)$ are the generalized Laguerre polynomials and $N$ is a normalization constant,
\begin{eq}
	N = 
  \sqrt{\frac{
    2^{n+l+2}\gamma^{l+\frac{3}{2}}
  }{
    \sqrt{\pi}
  }}
  \sqrt{\frac{
    \p{\frac{1}{2}(n-l)}!
    \p{\frac{1}{2}(n+l)}!
  }{
    (n+l+1)!
  }}.
\end{eq}
Plugging this into \cref{eq:potential element} we get
\begin{eq}
  \fint[\R^3]{^3\vec{r}}
  \fint[\R^3]{^3\vec{r}'}
  \p{
    R_{nl}(r) Y_l^m(\theta, \phi)
  }^*
  V(r) \delta^3(\vec{r} - \vec{r}')
  R_{n'l}(r) Y_l^m(\theta, \phi)
  \braket{\vec{r}'}{n'lm}
\end{eq}



\section{Radial Momentum Space TISE}
\label{sec:radial mom space TISE}

\section{Radial Momentum Space TISE}
\label{app:radial_mom_TISE}

To find the momentum space Schrödinger equation, we need to write an explicit expression for
\begin{eq}
  \int \rd^3 \vec{k}' \bra{\vec{k}} H \ket{\vec{k}'} \Phi(\vec{k}')
  &= 
  E\Phi(\vec{k})
\end{eq}
To begin with, using the completeness relation with the position basis, we note that
\begin{eq}
  \Phi(\vec{k}) &= \braket{\vec{k}}{\psi} 
  = 
  \int \rd^3\vec{r} \braket{\vec{k}}{\vec{r}} \psi(\vec{r})
\end{eq}
Standard textbooks on quantum mechanics show 
\begin{eq}
  \braket{\vec{k}}{\vec{r}} 
  &= 
  \frac{1}{(2\pi)^{\frac{3}{2}}}e^{i\vec{k}\cdot\vec{r}}.
\end{eq}
For a spherically symmetric problem, solutions can be found on the form $\psi(\vec{r})=  R(r)Y_l^m(\Omega_r)$. \todo{why?/cite/explain ol}
We can simplify the above integral by using the plane wave expansion \cite{mehrem}
\begin{eq}
  e^{i\vec{k}\cdot\vec{r}} 
  &= 
  4\pi \sum_{l=0}^\infty \sum_{m=-l}^l  i^l j_l(kr)Y_l^m(\Omega_k)Y_l^m(\Omega_r)^*
\end{eq}
where you can choose either factor to conjugate. \todo{"complex" necessary?}

Inserting this and using orthogonality of spherical harmonics
\begin{eq}
  \int \rd \Omega_r Y_{l'}^{{m'}^*}(\Omega_r)Y_l^m(\Omega_r)
  =
  \delta_{mm'}\delta_{ll'} 
\end{eq}
you end up with \todo{ordval}
\begin{eq}
  \Phi(\vec{k}) &= \phi(k)Y_l^m(\Omega_k)
  =
  \sqrt{\frac{2}{\pi}} i^l Y_l^m(\Omega_k) \fint{r} r^2 R(r) j_l(kr).
\end{eq}

In a similar manner, we evaluate
\begin{eq}
  \bra{\vec{k}}V(r)\ket{\vec{k}'} 
  &= 
  \frac{1}{(2\pi)^3} \int \rd^3 \vec{r} V(r)  e^{i\vec{k}'\cdot\vec{r}} e^{-i\vec{k} \cdot \vec{r}} \\
  &=
  \frac{1}{(2\pi)^3} (4\pi)^2 \sum_{l,l'}\sum_{m,m'} 
  (-1)^l i^{(l+l')} Y_{l'}^{{m'}^*}(\Omega_{k'}) Y_l^m(\Omega_k)
  \\
  &\times
  \int \rd r \, 
    r^2 V(r) j_l(kr)j_{l'}(k'r)
  \int \rd \Omega_r \, 
    Y_{l'}^{m'}(\Omega_r)Y_l^{m^*}(\Omega_r)
\end{eq}
Here, a factor $(-1)^l$ was introduced because of the parity of the spherical harmonics: $Y_l^m(-\Omega_k)=(-1)^lY_l^m(\Omega_k)$. 

Inserting all of this into the Schrödinger equation and again simplifying by using the orthogonality of the spherical harmonics twice, you immediatly obtain the equation given in the text. \todo{akta så att vi inte får en loop i att texten ger att bilagan ger att texten ger... ar stringent med var saken ges och var den återberättas}

\begin{eq}
  \int \rd^3 \vec{k}' \bra{\vec{k}} H \ket{\vec{k}'} \Phi(\vec{k}') 
  &= 
  \frac{k^2}{2\mu}\phi(k)Y_l^m(\Omega_k) 
  + 
  \fint[\R^3]{^3\vec{k}'} \bra{\vec{k}}V(r)\ket{\vec{k}'} \phi(k') Y_l^m(\Omega_{k'}) 
  \\
  &=
  \frac{k^2}{2\mu}\phi(k)Y_l^m(\Omega_k) + Y_l^m(\Omega_k) \fint[0][\inf]{k'} k'^2 \phi(k') V(k,k')
  \\
  &=
  E\phi(k)Y_l^m(\Omega_k)
\end{eq}
where
\begin{eq}
  V(k,k') = \frac{2}{\pi}\int \rd r \, r^2 V(r)j_l(kr)j_l(k'r).
\end{eq}


\bibliographystyle{ieeetr}
\bibliography{nhqm}{}

\end{document}
