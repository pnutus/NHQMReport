\documentclass[12pt,a4paper]{report}
\usepackage[english]{babel}

\usepackage{NHQM}

\begin{document}
  
\numberwithin{equation}{chapter}
\numberwithin{figure}{chapter}

\pagenumbering{gobble}
\listoftodos


\title{Resonances in Helium Nuclei\\ 
\todo{There is a proper front page somewhere, let's fix that}
\Large Bachelor Thesis in Physics}
\author{Jonathan Bengtsson \and Ola Embréus \and Vincent Ericsson \and Pontus Granström \and Nils Wireklint}
\date{\today}



\maketitle

\newpage
\begin{abstract}
\todo{Begin the first sentence by other thing than This thesis, this report, etc.}
\todo{Alternative sencences in parentheses.}
(Loosely bound quantum systems is intering in todays nuclear physics. This theses studies these loosely bound ...)

(Loosely bound quantum systems is an interesting topic studied in this thesis. This is made by expanding the wavefunction of these systems in a complex ...)

This thesis studies loosely bound quantum systems by expanding their wavefunctions in a complex basis using numerical methods. 
(We implement this theory to study the \He{5} and \He{6} nuclei.)
This is made by studying
resonances in \He{5} and \He{6} nuclei. We explain the theory of basis
expansion and expand the \He{5} nucleus wave functions in the spherical harmonic oscillator in coordinate space
and plane wave bases in momentum space. We model our systems as a \He{4} nucleus with one and two orbiting neutrons, respectively.
To find the resonances and thus reproduce experimental results, we extend the spherical wave basis into the complex plane according to the 
theory of Tore Berggren. \todo{Better connection between Berggren and his basis.}
(This complex Berggren basis ...) The Berggren basis is then used as a single particle
basis for the \He{6} problem. \todo{Should we cut down on the amounts of We ...}
We also explain many-body theory and the second quantization formalism, providing details for its implementation.

Finally, a Monte Carlo approach to reduce computation time for many-body calculations is investigated.

\todo{We have -ing form on five places in abstract, that is too many.}
\todo{Read through and correct the abstract.}
\end{abstract}
\newpage

\pagenumbering{roman}

\tableofcontents
\newpage

\pagenumbering{arabic}
\setcounter{page}{1}

\chapter{Introduction}
\label{cha:introduction}
\subfile{chapters/introduction.tex}


\chapter{Basis Expansion}
\label{cha:basis expansion}
\subfile{chapters/basis_expansion.tex}

\chapter{The \He{5} Nucleus}
\label{cha:he5}
\subfile{chapters/he5.tex}

\chapter{Non-Hermitian Quantum Mechanics} \todo{Discuss whether to use NHQM as chapter name.}
\label{cha:nhqm}
\subfile{chapters/nhqm_alt.tex}

\chapter{Many-Body Theory}
\label{cha:many-body}
\subfile{chapters/mb_theory.tex}

\chapter{The \He{6} Nucleus}
\label{cha:he6}
\subfile{chapters/he6.tex}

\chapter{The \He{7} Nucleus}
\label{cha:he7}

\chapter{Monte Carlo Approach}
\label{cha:monte carlo}
\subfile{chapters/monte_carlo.tex}

\chapter{Discussion}
\label{cha:disc}
\subfile{chapters/discussion.tex}

\appendix

\todo{Gauss-Legendre appendix?}

\chapter{Derivations}

\section{Harmonic Oscillator Matrix Elements}
\label{sec:HO matrix elements}

\subfile{appendices/HO_elements.tex}

\section{Radial Momentum Space TISE}
\label{sec:radial mom space TISE}

\subfile{appendices/radial_mom_TISE.tex}

\chapter{Tools}

All calculations were made in the Python programming language 
using the libraries NumPy and SciPy. We used the matplotlib 
library to make the figures. The code was managed using Git 
and is available at
\begin{quote}
  \url{https://github.com/pnutus/NHQM}
\end{quote}

\bibliographystyle{ieeetr}
\bibliography{nhqm}{}

\end{document}
