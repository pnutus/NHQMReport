We want to rewrite the Schrödinger equation in momentum basis/space, because yada yada.

This essentially means that we describe a state in terms of its momentum -- or equivalently its wavenumber $\vec{k}$ -- distribution, $\Phi(\vec{k}) = \braket{\vec{k}}{\psi}$.
In the same way that the squared wavefunction $|\psi(\vec{r})|^2$ describes the probability density of a particle to be found at position $\vec{r}$, the squared momentum space wavefunction describes the probability density of a particle to be found with momentum $\vec{k}$. 
To find the equations of motion in terms of this wavefunction we simply use the completeness of $\ket\vec{k}$ to expand the TISE.

\begin{eq}
  H\ket{\psi} &= E\ket{\psi} 

  \\
  
  \int \rd^3 \vec{k}' \bra{\vec{k}} H \ket{\vec{k}'} \Phi(\vec{k}')
  = 
  E\Phi(\vec{k})
\end{eq} 

If we consider a central problem, with $H=\frac{k^2}{2m} + V(r)$, a cumbersome calculation, found in Appendix [XXXXX??], shows that the equation can be written
\begin{eq} 
  \frac{k^2}{2m}\phi(k) + \int_0^\infty \rd k' \, k'^2 V(k,k') \phi(k') 
  &=
  E\phi(k)
\end{eq}
where
\begin{eq}
  V(k,k') 
  &= 
  \frac{2}{\pi}\int_0^\infty \rd r \, r^2 V(r) j_l(kr) j_l(k'r) 
\end{eq}
where $j_l(kr)$ are the spherical bessel functions. The total wavefunction $\Phi(\vec{k}) = \phi(k)Y_l^m(\Omega_{\vec{k}})$, where $\Omega_{\vec{k}}$ denotes the angular coordinates of $\vec{k}$.

\subsection{Numerical solution of the MSTISE}
The equation is solved numerically by writing it as a matrix equation. This is achieved by approximating the integral with a numerical quadrature, 
\begin{eq}
  \int_0^\infty \rd k' \, k'^2 V(k,k')\phi(k') 
  \approx
  \sum_{n'=1}^N w(n') k_{n'}^2 V(k,k_{n'})\phi(k_{n'})
\end{eq}
where $w(n')$ are the quadrature weights. For the naive rectangular quadrature you would use a constant $w(m')=\Delta k_{n'}$ equal to the step length. However, this quadrature converges slowly to the correct value of the integral, and much better alternatives can be employed. 

With this approximation the Schrödinger equation may be written
\begin{eq}
  \sum_{n'} H_{n,n'} \phi_{n'} &= E \phi_n
\end{eq}
where $\phi_n=\phi(k_n)$ and 
\begin{eq}
  H_{n,n'} &= \frac{k_n^2}{2m}\delta_{n,n'} + w(n')k_{n'}^2 V_{n,n'} \\
  V_{n,n'} &= \frac{2}{\pi} (-1)^l \int_0^\infty \rd r \, r^2 V(r) j_l(k_n r) j_l(k_{n'} r)
\end{eq}
The equation is now written as a matrix equation of order N (the number of k values used). The energy eigenvalues $E$ will be obtained by calculating the matrix elements $H_{n,n'}$ and diagonalizing the resulting matrix. Since there is generally no analytic expression for the terms $V_{n,n'}$, they will need to be evaluated by numerical integration.




\subsection{APPENDIX}

To find the momentum space Schrödinger equation, we need to write an explicit expression for
\begin{eq}
  \int \rd^3 \vec{k}' \bra{\vec{k}} H \ket{\vec{k}'} \Phi(\vec{k}')
  &= 
  E\Phi(\vec{k})
\end{eq}
To begin with, using the completeness relation for the position basis, we note that
\begin{eq}
  \Phi(\vec{k}) &= \braket{\vec{k}}{\psi} 
  = 
  \int \rd^3 \braket{\vec{k}}{\vec{r}} \psi(\vec{r})
\end{eq}
Standard textbooks on quantum mechanics show 
\begin{eq}
  \braket{\vec{k}}{\vec{r}} 
  &= 
  \frac{1}{(2\pi)^{\frac{3}{2}}}e^{i\vec{k}\cdot\vec{r}}
\end{eq}
For a spherically symmetric problem, eigensolutions can be found on the form $\psi(\vec{r})= R(r)Y_l^m(\Omega_r)$. We can simplify the above integral by using the plane wave expansion \cite{mehrem2011},
\begin{eq}
  e^{i\vec{k}\cdot\vec{r}} 
  &= 
  4\pi \sum_{l=0}^\infty \sum_{m=-l}^l  i^l j_l(kr)Y_l^m(\Omega_k)Y_l^m(\Omega_r)^*
\end{eq}
where you have the freedom of choice to put the complex conjugation on either factor. 

Inserting this and using the orthogonality of the spherical harmonics, you get
\begin{eq}
  \Phi(\vec{k}) &= \phi(k)Y_l^m(\Omega_k)
  =
  \sqrt{\frac{2}{\pi}} i^l Y_l^m(\Omega_k) \int \rd r \, r^2 R(r) j_l(kr) 
\end{eq}

In a similar way, we evaluate
\begin{eq}
  \bra{\vec{k}}V(r)\ket{\vec{k}'} &= \frac{1}{(2\pi)^3} \int \rd^3 \vec{r} V(r)  e^{i\vec{k}'\cdot\vec{r}} e^{-i\vec{k} \cdot \vec{r}} \\
  &= \frac{1}{(2\pi)^3} (4\pi)^2 \sum_{l,l'}\sum_{m,m'} 
  \b{
    (-1)^l i^{(l+l')} Y_{l'}^{m'}(\Omega_{k'})^*Y_l^m(\Omega_k)* 
    \\
    &\p{
      \int \rd r \, r^2 V(r) j_l(kr)j_{l'}(k'r) \int \rd \Omega_r Y_l'^m'(\Omega_r)Y_l^m(\Omega_r)^*
       }
    }
\end{eq}
Here, a factor $(-1)^l$ was introduced because of the parity of the spherical harmonics: $Y_l^m(-\Omega_k)=(-1)^lY_l^m(\Omega_k)$. 

Inserting all of this into the Schrödinger equation and again simplifying by using the orthogonality of the spherical harmonics twice, you immediatly obtain the equation given in the text.

\begin{eq}
  \int \rd^3 \vec{k}' \bra{\vec{k}} H \ket{\vec{k}'} \Phi(\vec{k}') 
  &= 
  \frac{k^2}{2m}\phi(k)Y_l^m(\Omega_k) + \int \rd^3 \vec{k}' \bra{\vec{k}}V(r)\ket{\vec{k}'} \phi(k') Y_l^m(\Omega_{k'}) 
  
  \\
  
  &=
  \frac{k^2}{2m}\phi(k)Y_l^m(\Omega_k) + Y_l^m(\Omega_k) \frac{2}{\pi}\int \rd k' \, k'^2 \phi(k') V(k,k') = E\psi(k)Y_l^m(\Omega_k)
\end{eq}