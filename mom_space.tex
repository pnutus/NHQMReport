A more natural framework to work in for the systems we are studying is that of momentum space. This is done by expanding the Schrödinger equation in momentum basis, demonstrated below. 

SPECULATION:

The reason that we want to solve the problem in momentum space is that we are studying a system with a short-range potential supporting only a few, if any, bound solutions. This means that we will find multiple unbound solutions, corresponding to free particles of various energies. These are basically already eigenstates of the momentum operator, only slightly disturbed by the small potential well at $r=0$. 

Additionally, we want to construct a Berggren basis consisting of complex scattering states, resonances and eventual bound states. Because of the close relation between momentum and energy for scattered (unbound) states, $E=\frac{k^2}{2m}$, a complex contour in the energy plane will correspond to a complex contour in momentum space. [Perhaps we shouldn't talk about contours until next chapter?]

END OF SPECULATION


LAME:
 
We want to rewrite the Schrödinger equation in momentum basis/space, because yada yada.

This essentially means that we describe a state in terms of its momentum distribution, $\Phi(\vec{k})=\braket{\vec{k}}{\psi} $. In the same way that the squared wavefunction $|\psi(\vec{r})|^2$ describes the probability density of a particle to be found at position $\vec{r}$, the squared momentum space wavefunction describes the probability density of a particle to be found with momentum $\vec{k}$. To find the equations of motion in terms of this wavefunction we simply use the completeness of $\ket{\vec{k}}$ to expand the TISE.

END OF LAME

\begin{eq}
  H\ket{\psi} &= E\ket{\psi} 
  \\
  \int \rd^3 \vec{k}' \bra{\vec{k}} H \ket{\vec{k}'} \Phi(\vec{k}')
  &= 
  E\Phi(\vec{k})
\end{eq} 

If we consider a central problem, with $H=\frac{k^2}{2m} + V(r)$, a cumbersome calculation, found in \cref{sec:radial_mom_space_TISE}, shows that the Schrödinger equation in this representation can be written
\begin{eq} 
  \frac{k^2}{2\mu}\phi(k) + \int_0^\infty \rd k' \, k'^2 V(k,k') \phi(k') 
  &=
  E\phi(k)
\end{eq}
where
\begin{eq}
  V(k,k') 
  &= 
  \frac{2}{\pi}\int_0^\infty \rd r \, r^2 V(r) j_l(kr) j_l(k'r) 
\end{eq}
and $j_l(kr)$ are the spherical bessel functions of order $l$. The total wavefunction can be written $\Phi(\vec{k}) = \phi(k)Y_l^m(\Omega_{\vec{k}})$, where $\Omega_{\vec{k}}$ denotes the angular coordinates of $\vec{k}$.


\subsection{Numerical solution in momentum space}
The equation is solved numerically by writing it as a matrix equation. This is achieved by approximating the integral with a numerical quadrature, 
\begin{eq}
  \int_0^\infty \rd k' \, k'^2 V(k,k')\phi(k') 
  \approx
  \sum_{j=1}^N w_j k_j^2 V(k,k_j)\phi(k_j)
\end{eq}
where $w_j$ are the quadrature weights. For the naive rectangular quadrature you would use a constant $w_j=\Delta k_j$ equal to the step length. However, this quadrature converges slowly to the correct value of the integral, and much better alternatives can be employed. 

With this approximation the Schrödinger equation may be written
\begin{eq}
  \sum_j H_{ij} \phi_j &= E \phi_i
\end{eq}
where $\phi_i=\phi(k_i)$ and 
\begin{eq}
  H_{ij} &= \frac{k_i^2}{2\mu}\delta_{ij} + w_jk_j^2 V_{ij} \\
  V_{ij} &= \frac{2}{\pi} \int_0^\infty \rd r \, r^2 V(r) j_l(k_i r) j_l(k_j r)
\end{eq}
The equation is now written as a matrix equation of order N -- the number of states included in the basis. The energy eigenvalues $E$ will be obtained by calculating the matrix elements $H_{ij}$ and diagonalizing the resulting matrix. Since there is generally no analytic expression for the terms $V_{ij}$, they will need to be evaluated by numerical integration.

A comparison of performance between HO basis expansion and momentum space for the hydrogen atom and Helium-5 problems is shown in \cref{fig:HO vs mom}. We are using Gauss-Legendre quadrature to approximate the integrals. 

{figure}

\label{fig:HO vs mom}
{end figure}

However, let us study the obtained solutions closer. If we transform the momentum wavefunctions $\phi(k)$ to radial position wavefunctions $R(r)$ according to \cref{sec:radial_mom_space_TISE},
\begin{eq}
R(r)=\sqrt{\frac{2}{\pi}}i^l \int_0^\infty \rd k \, k^2 \phi(k)j_l(kr) \, ,
\end{eq} 
we can see the spatial distribution of our solutions. \Cref{fig:momspace solutions} shows the wavefunctions $R(r)$ for a few of the states with lowest energy.

{figure}

\label{fig:momspace solutions}
{end figure}

 While all of the states we see have energies $E>0$, and thus are unbound, we see that one solution is more localized near the center $r=0$. This is the sign of a quasi-bound state. To confirm this we may vary the depth $V0$ of the potential well and see how this affects the solutions. \Cref{fig:momspace solutions var} shows the solutions obtained with a different $V0$.

{figure}

\label{fig:momspace solutions var}
{end figure}

We find that the unbound states remained unchanged, and basically correspond to free particles of energies $E_n=\frac{k_n^2}{2\mu}$, where $k_n$ are the momenta that was included in the discretization of the integrals. We will refer to these values of $k$ as our \emph{mesh points}. The quasi-bound state changed, which shows that this solution is a feature of the potential.  

