\documentclass[../main/report.tex]{subfiles}
\begin{document}
	

\newcommand{\backgroundpic}[3]{%
    \put(#1,#2){
        \parbox[b][\paperheight]{\paperwidth}{%
    		\centering
			\includegraphics[width=\paperwidth,height=\paperheight,keepaspectratio]{#3}
			\vfill
}}}

\begin{titlepage}

\AddToShipoutPicture{\backgroundpic{-4}{56.7}{framsida.pdf}}
\mbox{}
\vfill
\addtolength{\voffset}{1cm}

   

\begin{figure}[h!]
   \newcommand{\circdist}{1.2}
   \newcommand{\circrad}{2}
   \centering
   \begin{tikzpicture}[even odd rule]
     \foreach \angle/\colour in {90/red,-30/blue,210/green}
       \draw [fill=\colour] 
         (\angle:\circdist) circle (\circrad) circle (\circrad+0.5);

     \begin{scope}
     \clip (-30:\circdist) circle (1);
     \draw [fill=red] (90:\circdist) circle (\circrad) circle (\circrad+0.5);
     \end{scope}

     \begin{scope}
     	\clip (90:\circdist) ++ (180:\circrad) circle (1);
     	\draw [fill=red] (90:\circdist) circle (\circrad) circle (\circrad+0.5);	 
     \end{scope}
     
	  \begin{scope}
	  		 \draw [opacity = 0](1,0) -- (0,-4.5) -- (-1,0) -- (0,-4.5) -- cycle;
			 %OBS OLA, detta är dålig lösning, gör inte det här hemma DÅÅÅLIG LÖSNING
	  \end{scope}

	 
   \end{tikzpicture}
   
\end{figure}
%\\[1 cm] denna är värdelös, rör ej

\begin{flushleft}

    {\noindent {\Huge A Study of Quantum Resonances \\ in a Complex-Momentum Basis} \\[0.5 cm]

    %{\Large Underrubrik} \\[0.5cm]

    \emph{\Large Bachelor Thesis in Physics} \\[0.5 cm]

    

	{\Large Jonathan Bengtsson, Ola Embréus, Vincent Ericsson, Pontus Granström, Nils Wireklint}\\[1 cm]

	

	{\Large Department of Fundamental Physics \\

	\textsc{Chalmers University of Technology} \\

	Göteborg, Sverige 2013 \\

    FUFX02-13-XX\\

	} 

	}

\end{flushleft}



\end{titlepage}



\ClearShipoutPicture



\newpage 

\end{document}
