\documentclass[12pt,a4paper]{report}
\usepackage[english]{babel}

\usepackage{../main/NHQM}

\begin{document}
  
  
\numberwithin{equation}{chapter}
\numberwithin{figure}{chapter}

\listoftodos

\subfile{titlepage.tex}

\todo{finare färger på framsidan}
\begin{abstract}
Resonances are important features of open quantum systems.
We study, in particular, unbound and loosely bound nuclear systems.
We model \He{5} and \He{6} in a few-body picture, consisting of an alpha-particle core with one and two valence neutrons respectively.
Basis-expansion theory is briefly explained and then used to expand the nuclear system in the harmonic oscillator and momentum bases.
We extend the momentum basis into the complex plane, obtaining the so-called Berggren basis. 
With the complex-momentum method we are able to reproduce the observed resonances in \He{5}.
The \He{5} Berggren basis solutions are used as a single-particle basis to create many-body states in which we expand the \He{6} system.
For the two-body interaction between the neutrons, we use two different phenomenological models: a gaussian and a Surface Delta Interaction (SDI).
The strength of each interaction is fitted to reproduce the \He{6} ground state energy.
With the gaussian interaction we do not obtain the \He{6} resonance, whereas with the SDI we do.
The relevant parts of the second quantization formalism is summarized, and we provide details for its implementation.
\todo{finish this}

\end{abstract}

\setcounter{page}{1}
\pagenumbering{roman}

\tableofcontents

\newpage

\pagenumbering{arabic}

\subfile{../chapters/introduction.tex}

\subfile{../chapters/basis_expansion.tex}

\subfile{../chapters/two-body.tex}

\subfile{../chapters/berggren.tex}

\subfile{../chapters/mb_theory.tex}

\subfile{../chapters/three-body.tex}

\subfile{../chapters/outlook.tex}

\bibliographystyle{ieeetr}
\bibliography{../main/nhqm.bib}

\appendix

\chapter{Derivations}

\subfile{../appendices/HO_elements.tex}

\subfile{../appendices/radial_mom_TISE.tex}

\chapter{Numerical Integration}

\subfile{../appendices/gauss_legendre.tex}

\chapter{Tools}

All the computations were implemented in the Python programming language 
using the libraries NumPy and SciPy. 
We used the TikZ and pgfplots libraries to make the figures. 
The code was managed using Git and is available at
\begin{quote}
  \url{https://github.com/pnutus/NHQM}
\end{quote}
\todo{Move to outlook?}

\end{document}