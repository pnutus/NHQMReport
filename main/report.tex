\documentclass[12pt,a4paper]{report}
\usepackage[english]{babel}

\usepackage{../main/NHQM}

\begin{document}
  
\numberwithin{equation}{chapter}
\numberwithin{figure}{chapter}

\listoftodos

\documentclass[../main/report.tex]{subfiles}
\begin{document}
	

\newcommand{\backgroundpic}[3]{%
    \put(#1,#2){
        \parbox[b][\paperheight]{\paperwidth}{%
    		\centering
			\includegraphics[width=\paperwidth,height=\paperheight,keepaspectratio]{#3}
			\vfill
}}}

\begin{titlepage}

\AddToShipoutPicture{\backgroundpic{-4}{56.7}{framsida.pdf}}
\mbox{}
\vfill
\addtolength{\voffset}{1cm}

   

\begin{figure}[h!]
   \newcommand{\circdist}{1.2}
   \newcommand{\circrad}{2}
   \centering
   \begin{tikzpicture}[even odd rule]
     \foreach \angle/\colour in {90/red,-30/blue,210/green}
       \draw [fill=\colour] 
         (\angle:\circdist) circle (\circrad) circle (\circrad+0.5);

     \begin{scope}
     \clip (-30:\circdist) circle (1);
     \draw [fill=red] (90:\circdist) circle (\circrad) circle (\circrad+0.5);
     \end{scope}

     \begin{scope}
     	\clip (90:\circdist) ++ (180:\circrad) circle (1);
     	\draw [fill=red] (90:\circdist) circle (\circrad) circle (\circrad+0.5);	 
     \end{scope}
     
	  \begin{scope}
	  		 \draw [opacity = 0](1,0) -- (0,-4.5) -- (-1,0) -- (0,-4.5) -- cycle;
			 %OBS OLA, detta är dålig lösning, gör inte det här hemma DÅÅÅLIG LÖSNING
	  \end{scope}

	 
   \end{tikzpicture}
   
\end{figure}
%\\[1 cm] denna är värdelös, rör ej

\begin{flushleft}

    {\noindent {\Huge A Study of Quantum Resonances \\ in a Complex-Momentum Basis} \\[0.5 cm]

    %{\Large Underrubrik} \\[0.5cm]

    \emph{\Large Bachelor Thesis in Physics} \\[0.5 cm]

    

	{\Large Jonathan Bengtsson, Ola Embréus, Vincent Ericsson, Pontus Granström, Nils Wireklint}\\[1 cm]

	

	{\Large Department of Fundamental Physics \\

	\textsc{Chalmers University of Technology} \\

	Göteborg, Sverige 2013 \\

    FUFX02-13-XX\\

	} 

	}

\end{flushleft}



\end{titlepage}



\ClearShipoutPicture



\newpage 

\end{document}




\begin{abstract}
Resonance is a feature of open, loosely bound quantum systems, such as the atomic nucleus.
We study resonances using as model systems \He{5} and \He{6}, treated as an alpha-particle core with one and two valence neutrons respectively.
Basis expansion theory is briefly explained and then used to expand the nuclear system in the harmonic oscillator and plane wave bases.
To find the resonances we extend the plane wave basis into the complex plane, obtaining the so-called Berggren basis. 
With the Berggren basis, we reproduce the known resonances in \He{5} within 15\%.
The relevant details of the occupation number representation is summarized, and we provide details for its implementation.
The \He{6} problem is then expanded using the \He{5} Berggren basis solutions as a single particle basis. 
For the two-body interaction between the neutrons, we use a gaussian and a Surface Delta Interaction (SDI).
The strength of the interactions is fitted to the \He{6} ground state energy.
With the gaussian interaction we cannot reproduce the \He{6} resonances, whereas with the SDI we can, within XXX\%.
Finally, a Monte Carlo approach to reduce computation time for many-body calculations is investigated.
The method is found to be nonsensical and we deem it unusable.
\todo{finnish abstract}

\end{abstract}

\tableofcontents

\subfile{../chapters/introduction.tex}

\subfile{../chapters/basis_expansion.tex}

\subfile{../chapters/two-body.tex}

\subfile{../chapters/berggren.tex}

\subfile{../chapters/mb_theory.tex}

\subfile{../chapters/three-body.tex}

\subfile{../chapters/monte_carlo.tex}

\subfile{../chapters/outlook.tex}

\appendix

\chapter{Derivations}

\subfile{../appendices/HO_elements.tex}

\subfile{../appendices/radial_mom_TISE.tex}

\chapter{Numerical Integration}

\subfile{../appendices/gauss_legendre.tex}

\chapter{Tools}

All calculations were made in the Python programming language 
using the libraries NumPy and SciPy. We used the matplotlib 
library to make the figures. The code was managed using Git 
and is available at
\begin{quote}
  \url{https://github.com/pnutus/NHQM}
\end{quote}

\bibliographystyle{ieeetr}
\bibliography{../main/nhqm.bib}

\end{document}
