We have noticed that one of the \He{5} states behaves 
differently than the others. We suspect this is the 
resonance state, but we cannot yet quantify \todo{Investigate, identify, find} its width, 
or half-life. This means we need to find the complex 
energy of the state. We do this by using the theory of 
Tore Berggren and his \emph{Berggren basis} \cite{berggren}. 

Another reason for moving to the complex energy plane is the fact that when we later move to more than two bodies we want to use a basis consisting of our two-body states.
A basis consisting of all bound, free and resonant complex states forms a complete set  after the \emph{Berggren Comleteness Relation}, this theory is also described by Tore Berggren.

The theory is involved and will not be fully explained 
here. Instead we present a heuristic argument.

If we relate the energies $E$ of a system to momenta $k$ as
\begin{eq}
  E = \frac{\hbar^2 k^2}{2\mu}
  \quad\quad
  \textup{or}
  \quad\quad
  k = \frac{\sqrt{2\mu E}}{\hbar},
\end{eq}
we can plot the energies as $k$ in the complex plane, see 
\cref{fig:complex plane}. We then expect bound states, with 
$E<0$, to be represented by $k$ along the imaginary axis---
whereas unbound, scattering states, with $E>0$, are found 
along the real axis. Resonance states, with complex 
$E = E_0 - i \Gamma /2$, would by this argument appear somewhere
in the fourth quadrant.

\todo{Maybe show results before complexifying? We could relate the mesh points to the solutions here.}

We now interpret these $k$ as poles, and our integration 
\begin{eq}
  \fint[0][\inf]{k} k'^2 V(k,k') \phi(k')
\end{eq}
\todo{Should we write the integral as a path integral rather than one along the real axis?}
as a contour integration around the upper half plane, 
see \cref{fig:simple contour}. The result of a contour 
integration depends on the poles it encircles by the 
residue theorem. Therefore, we expect something to happen if 
we let the contour encircle the pole of the resonance,
as in \cref{fig:berggren contour}.

\begin{figure}
  \tikzset{
    triangle/.style={regular polygon, regular polygon sides=3},
    nosep/.style={inner sep=0},
    bound/.style={circle,draw,minimum size=2mm,nosep},
    unbound/.style={rectangle,draw,minimum size=2mm,nosep},
    quasibound/.style={triangle,draw,minimum size=2.5mm,nosep}
  }
  \subfloat[]{
  \label{fig:simple contour}
  %\tikzset{external/remake next}
\tikzsetnextfilename{simple_contour}
  \begin{tikzpicture}[scale = 2.5]
    \draw[->] (-1.2, 0) -- (1.2, 0) node[right] {$\Re k$};
    \draw[->] (0, -0.5) -- (0, 1.2) node[above] {$\Im k$};
    \foreach \y in {0.1, 0.3}
      \node at (0, \y) [bound] {};
    \foreach \x in {0.25, -0.25}
      \node at (\x, -0.15) [quasibound] {};
    \draw[very thick, mid arrows] (1, 0) arc (0:90:1) arc (90:180:1) 
                                  -- (0,0) -- cycle;
  \end{tikzpicture}
  }
  \subfloat[]{
  \label{fig:berggren contour}
  %\tikzset{external/remake next}
\tikzsetnextfilename{berggren_contour}
  \begin{tikzpicture}[scale = 2.5]
    \draw[->] (-1.2, 0) -- (1.2, 0) node[right] {$\Re k$};
    \draw[->] (0, -0.5) -- (0, 1.2) node[above] {$\Im k$};
    \foreach \y in {0.1, 0.3}
      \node at (0, \y) [bound] {};
    \foreach \x in {0.25, -0.25}
      \node at (\x, -0.15) [quasibound] {};
    \draw[very thick, mid arrows, radius=1]
      (1, 0) arc [start angle=0,  end angle=90]
             arc [start angle=90, end angle=180]
             -- (-0.5, 0) 
             -- (-0.25, 0.25) 
             -- (0.25, -0.25)
             -- (0.5, 0)
             -- cycle;
  \end{tikzpicture}
  }
  \caption{The complex $k$-plane. The circles represent 
  bound states and the triangles resonant states. Note the 
  mirroring of the states in the imaginary axis.}
  \label{fig:complex plane}
\end{figure}

\todo{Need to mention completeness of berggren basis. Important here or later?}

\todo{Where to put: "No conjugate on bras"?}

\section{The Complex Contour}

We choose to extend our integration along the real axis to 
the simplest possible complex contour, a triangle-shaped 
extrusion downwards. The tip of the triangle is placed directly 
below the hypothesized resonance pole.

Numerically, we are faced with the problem of how to choose 
the points and weights, now that the contour is complex. 
We consider each straight segment of the contour separately, 
and rescale the Gauss-Legendre points to each of the different segments.
An example of a contour is seen in 
\cref{fig:triangle contour}.

%\tikzset{external/remake next}
\tikzsetnextfilename{triangle_contour}
\begin{figure}[H]
  \centering
  \begin{tikzpicture}
    \begin{axis}[
      width = \textwidth,
      height = 7cm,
      xlabel=Re $k$,
      ylabel=Im $k$,
		  axis lines = middle,
      ymax = 0.1,
      enlargelimits,
      only marks,
      ticks = none,
      ]
      \addplot table {figures/numerical_contour/numerical_contour.data};
    \end{axis}
  \end{tikzpicture}
  \caption{The complex contour used. The points are distributed on each segment according to the Gauss-Legendre quadrature rule.}
  \label{fig:triangle contour}
\end{figure}

\section{Studying the Resonance}
Armed with the tools listed above we can now continue our study of the resonances in a better way.
Especially the fact that we now are able to localize the resonance is appreciable.
We can now vary our potential depth and width, which changes the position of the pole.
This is used to fit the pole after experimental data.
When this is made we have fixed our basis, meaning that we are fairly shure that our description of \He{5} is good.

One thing to think of when locating a resonance is that it is not the energy eigenvalue with lowest real part or norm.
The easiest way to find it is to plot the wavefunctions in momentum or position space where they show different behavior than the bond and free ones.
An example of this is shown in \cref{????????????} and \cref{???????????????????????}.

We also noticed some interesting behavior of this pole by playing around with our different parameters.
This gave way to some nice looking convergence patterns which can be seen in \cref{fig:some_fig}.

\todo{At another place?} As a measure of strength of the Gauss-Legendre quadrature we compared it to a normal, evenly spaced quadrature and found that the convergence rate was significanly faster for Gauss-Legendre.
\todo{This text is bad now, I'll fix it later.}
\todo{Results and shit here}

%\tikzset{external/remake next}
\tikzsetnextfilename{pole(V0)}
\begin{figure}
  \centering	
  \begin{tikzpicture}
      \begin{axis}[
        xlabel=Re $k/\b{\si{fm^{-1}}}$,
        ylabel=Im $k/\b{\si{fm^{-1}}}$,
  		  axis x line = middle,
        axis y line = left,
        every axis y label/.style={
          at = {(current axis.above origin)},
          anchor = north west,
        },
        every axis x label/.style={
          at = {(current axis.right of origin)},
          anchor = north east,
        },
        every x tick label/.append style = {anchor = south, yshift = 3pt},
        xmax=1,
        ytickmax = 0.3, xtickmax = 0.9,
        enlarge y limits,
        no markers,
        ]
      	\addplot+[very thick, ->] table  {figures/res_pole(V0)/poles.data};
      	    \addlegendentry{Pole position}
      	\addplot+[very thick] table {figures/res_pole(V0)/contour.data};
          \addlegendentry{Contour}
      \end{axis}
  \end{tikzpicture}
  \caption{The pole position as a function of $V_0$.}
  \label{fig:pole(V0)}

\end{figure}
