We have noticed that one of the \He{5} states behaves 
differently than the others. We suspect this is the 
resonance state, but we cannot yet quantify its width, 
or half-life. This means we need to find the complex 
energy of the state. We do this by using the theory of 
Tore Berggren and his \emph{Berggren basis} \cite{berggren}. 
The theory is involved and will not be fully explained 
here. Instead we present a heuristic argument.

If we relate the energies $E$ of a system to momenta $k$ as
\begin{eq}
  E = \frac{\hbar^2 k^2}{2\mu}
  \quad\quad
  \textup{or}
  \quad\quad
  k = \frac{\sqrt{2\mu E}}{\hbar},
\end{eq}
we can plot the energies as $k$ in the complex plane, see 
\cref{fig:complex plane}. We then expect bound states, with 
$E<0$, to be represented by $k$ along the imaginary axis---
whereas unbound, scattering states, with $E>0$, are found 
along the real axis. Resonance states, with complex 
$E = E_0 - i \Gamma /2$, would by this argument appear 
in the fourth quadrant.

\todo{Maybe show results before complexifying? We could relate the mesh points to the solutions here.}

We now interpret these $k$ as poles, and our integration 
\begin{eq}
  \fint[0][\inf]{k} k'^2 V(k,k') \phi(k')
\end{eq}
as a contour integration around the upper half plane, 
see \cref{fig:simple contour}. The result of a contour 
integration depends on the poles it encircles by the 
residue theorem, so we expect something to happen if 
we let the contour encircle the pole of the resonance.

\begin{figure}
  \subfloat[]{
  \begin{tikzpicture}
    [ scale = 2.5,
      triangle/.style={regular polygon, regular polygon sides=3},
      nosep/.style={inner sep=0},
      bound/.style={circle,draw,minimum size=2mm,nosep},
      unbound/.style={rectangle,draw,minimum size=2mm,nosep},
      quasibound/.style={triangle,draw,minimum size=2.5mm,nosep}]
    \draw[->] (-1.2, 0) -- (1.2, 0) node[right] {$\Re k$};
    \draw[->] (0, -0.5) -- (0, 1.2) node[above] {$\Im k$};
    \foreach \y in {0.1, 0.3}
      \node at (0, \y) [bound] {};
    \node at (0.25, -0.15) [quasibound] {};
    \draw[very thick, 
          decoration={markings,
            mark=at position 0.17 with {\arrow{latex}},
            mark=at position 0.47 with {\arrow{latex}},
            mark=at position 0.72 with {\arrow{latex}},
            mark=at position 0.92 with {\arrow{latex}}},
          postaction={decorate}] 
    (1, 0) arc (0:180:1) -- cycle;
  \end{tikzpicture}
  }
  \subfloat[]{
  \begin{tikzpicture}
    [ scale = 2.5,
      triangle/.style={regular polygon, regular polygon sides=3},
      nosep/.style={inner sep=0},
      bound/.style={circle,draw,minimum size=2mm,nosep},
      unbound/.style={rectangle,draw,minimum size=2mm,nosep},
      quasibound/.style={triangle,draw,minimum size=2.5mm,nosep}]
    \draw[->] (-1.2, 0) -- (1.2, 0) node[right] {$\Re k$};
    \draw[->] (0, -0.5) -- (0, 1.2) node[above] {$\Im k$};
    \foreach \y in {0.1, 0.3}
      \node at (0, \y) [bound] {};
    \node at (0.25, -0.15) [quasibound] {};
    \draw[very thick, 
          decoration={markings,
            mark=at position 0.15 with {\arrow{latex}},
            mark=at position 0.45 with {\arrow{latex}},
            mark=at position 0.69 with {\arrow{latex}},
            mark=at position 0.88 with {\arrow{latex}}},
          postaction={decorate}] 
    (1, 0) arc (0:180:1) -- (-0.5, 0) 
                         -- (-0.25, 0.25) 
                         -- (0.25, -0.25)
                         -- (0.5, 0)
                         -- cycle;
  \end{tikzpicture}
  }
  \caption{}
  \label{fig:simple contour}
\end{figure}

\begin{figure}[H]
  \centering
    \includegraphics[width = \textwidth]{figures/complex_plane.pdf}

  \caption{The complex $k$-plane. The circles represent 
  bound states, the diamonds unbound states and the 
  crosses resonant states. Note the mirroring in
  the imaginary axis.}
  \label{fig:complex plane}
\end{figure}

\todo{Need to mention completeness of berggren basis. Important here or later?}

\todo{Where to put: "No conjugate on bras"?}

\section{The Complex Contour}

We choose to extend our integration along the real axis to 
the simplest possible complex contour, a triangle-shaped 
extrusion downwards. The tip of the triangle is placed directly 
below the hypothesized resonance pole.

Numerically, we are faced with the problem of how to choose 
the points and weights, now that the contour is complex. 
We consider each straight segment of the contour separately, 
and rescale the Gauss-Legendre points between the 
of the segment. An example of a contour is seen in 
\cref{fig:triangle contour}.

\begin{figure}[H]
  \centering
    \includegraphics[width = \textwidth]{figures/complex_contour.pdf}
  \caption{The complex contour used.}
  \label{fig:triangle contour}
\end{figure}

\section{Studying the Resonance}

\todo{Results and shit here}
