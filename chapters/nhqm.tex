\documentclass[../main/report.tex]{subfiles}
\begin{document}

\chapter{Non-Hermitian Quantum Mechanics} \todo{Discuss whether to use NHQM as chapter name.}
\label{cha:nhqm}

The resonant state has a complex energy and we need to extend our basis into the complex plane to study it. 
We do this by using the theory of Tore Berggren and his \emph{Berggren basis} \cite{berggren}.
This allows us to determine the energy of the state which charcterizes its width and half-life.

Another reason for moving to the complex energy plane is the fact that when we later move to more than two bodies we want to use a basis consisting of our two-body states.
A basis consisting of all bound, free and resonant complex states forms a complete set  after the \emph{Berggren Comleteness Relation}, this theory is also described by Tore Berggren.

\todo{comment on why all free states can be approximated by a few free states along a contour with low energy.}

The theory is involved and will not be fully explained 
here. Instead we present a heuristic argument.

If we relate the energies $E$ of a system to momenta $k$ as
\begin{eq}
  E = \frac{\hbar^2 k^2}{2\mu}
  \quad\quad
  \textup{or}
  \quad\quad
  k = \frac{\sqrt{2\mu E}}{\hbar},
\end{eq}
we can plot the energies as $k$ in the complex plane, see 
\cref{fig:complex plane}. We then expect bound states, with \todo{introduce k - momenta, basis mom here?}
$E<0$, to be represented by $k$ along the imaginary axis---
whereas unbound, scattering states, with $E>0$, are found 
along the real axis. Resonance states, with complex 
$E = E_0 - i \Gamma /2$, would by this argument appear somewhere
in the fourth quadrant.

\todo{maybe show real conour before complex for comparison, röd tråd, mycket figurer för att förstå}

We now interpret these complex momenta, $k$, as poles and our integration 

\begin{eq}
  \fint[0][\inf]{k} k'^2 V(k,k') \phi(k')
\end{eq}
\todo{resiidy stycke, integration behöver kärlek}
is evaluated as a contour integration around the upper half plane, 
see \cref{fig:simple contour}. The result of a contour 
integration depends on the poles it encircles by the 
residue theorem. Therefore, we expect something to happen if 
we let the contour encircle the pole of the resonance,
as in \cref{fig:berggren contour}.
\todo{tydligare förankring av beggren teoris}
An interesting thing seen when moving to a complex contour is the fact that we no longer use a conjugate on the bra when computing the matrix elements.
This is a confusing piece of theory stemming, again, from Berggren.


\begin{figure}
  \tikzset{
    triangle/.style={regular polygon, regular polygon sides=3},
    nosep/.style={inner sep=0},
    bound/.style={circle,draw,minimum size=2mm,nosep},
    unbound/.style={rectangle,draw,minimum size=2mm,nosep},
    quasibound/.style={triangle,draw,minimum size=2.5mm,nosep}
  }
  \subfloat[]{
  \label{fig:simple contour}
  %\tikzset{external/remake next}
\tikzsetnextfilename{simple_contour}
  \begin{tikzpicture}[scale = 2.5]
    \draw[->] (-1.2, 0) -- (1.2, 0) node[right] {$\Re k$};
    \draw[->] (0, -0.5) -- (0, 1.2) node[above] {$\Im k$};
    \foreach \y in {0.1, 0.3}
      \node at (0, \y) [bound] {};
    \foreach \x in {0.25, -0.25}
      \node at (\x, -0.15) [quasibound] {};
    \draw[very thick, mid arrows] (1, 0) arc (0:90:1) arc (90:180:1) 
                                  -- (0,0) -- cycle;
  \end{tikzpicture}
  }
  \subfloat[]{
  \label{fig:berggren contour}
  %\tikzset{external/remake next}
\tikzsetnextfilename{berggren_contour}
  \begin{tikzpicture}[scale = 2.5]
    \draw[->] (-1.2, 0) -- (1.2, 0) node[right] {$\Re k$};
    \draw[->] (0, -0.5) -- (0, 1.2) node[above] {$\Im k$};
    \foreach \y in {0.1, 0.3}
      \node at (0, \y) [bound] {};
    \foreach \x in {0.25, -0.25}
      \node at (\x, -0.15) [quasibound] {};
    \draw[very thick, mid arrows, radius=1]
      (1, 0) arc [start angle=0,  end angle=90]
             arc [start angle=90, end angle=180]
             -- (-0.5, 0) 
             -- (-0.25, 0.25) 
             -- (0.25, -0.25)
             -- (0.5, 0)
             -- cycle;
  \end{tikzpicture}
  }
  \caption{The complex $k$-plane. The circles represent 
  bound states and the triangles resonant states. Note the 
  mirroring of the states in the imaginary axis.}
  \label{fig:complex plane}
\end{figure}

\todo{Need to mention completeness of berggren basis. Important here or later?}

\todo{Where to put: "No conjugate on bras"?}

\section{The Complex Contour}

We choose to extend our integration along the real axis to 
the simplest possible complex contour, a triangle-shaped downward extrusion. The tip of the triangle is placed directly 
below the hypothesized resonance pole.

Numerically, we are faced with the problem of how to choose 
the points and weights. 
We consider each straight segment of the contour separately, 
and rescale the Gauss-Legendre points to each of the different segments.
An example of a contour is seen in
\cref{fig:triangle contour}, note the concentration of the points around the ends of each segment, the points near the center are given more weight though.

%\tikzset{external/remake next}
\tikzsetnextfilename{triangle_contour}
\begin{figure}[H]
  \centering
  \begin{tikzpicture}
    \begin{axis}[
      width = \textwidth,
      height = 7cm,
      xlabel=Re $k$,
      ylabel=Im $k$,
		  axis lines = middle,
      ymax = 0.1,
      enlargelimits,
      only marks,
      ticks = none,
      ]
      \addplot table {../figures/numerical_contour/numerical_contour.data};
    \end{axis}
  \end{tikzpicture}
  \caption{The complex contour used. The points are distributed on each segment according to the Gauss-Legendre quadrature rule.}
  \label{fig:triangle contour}
\end{figure}

\section{Studying the Resonance}
Armed with the tools listed above we can now continue our study of the resonances.
We now need a way to determine which solution corresponds to the resonnance. 
It is not as simple as picking the solution with the lowest energy or norm. 
Instead we need to consider the solutions in momentum space.
Bound states have negative energy and can easily be identified, that leaves unbound and resonant states. 
The unbound solutions would be dominated by a single momentum, i e, their probability density would closely resemble sine waves in position space. 
Whereas a more localized resonant state would have a greater uncertainty in its momentum by the Heisenberg relation, and can thus be identified by picking the most diffuse solution.

We can now proceed to fit our models to experimental data provided by \cite{inte jimmy i alla fall}. 
We also found it worthwhile to plot the resonnance momentum as a function of the strength of the potential. 
In the \He{5} system no resonant state can co-exist with a bound state for a given potential strength, \cite{spill?}.
Hence we could single out the least localized solution in momentum space, this allows us to study how the bound system becomes weaker and gradually dissipates into a resonance. 
The result is presented in \cref{fig:pole(V0)}, in this figure the potential was varried from \SI{-70}{MeV}, bound, to \SI{-50}{MeV}, resonant.

\Cref{fig:pole(cont)} on the other hand depicts the pole's stability with respect to different complex contours. And that's about it 

%\tikzset{external/remake next}
\tikzsetnextfilename{pole(V0)}
\begin{figure}
  \centering
  \begin{tikzpicture}
    \begin{axis}[
      width = \textwidth,
      height = 9cm,
        xlabel=Re $k/\b{\si{fm^{-1}}}$,
        ylabel=Im $k/\b{\si{fm^{-1}}}$,
  		  axis x line = middle,
        axis y line = left,
        every axis y label/.style={
          at = {(current axis.above origin)},
          anchor = north west,
        },
        every axis x label/.style={
          at = {(current axis.right of origin)},
          anchor = north east,
        },
 	      yticklabel style={/pgf/number format/fixed,
 	                     /pgf/number format/precision=3},
        every x tick label/.append style = {anchor = south, yshift = 3pt},
        xmax=0.5,
        ytickmax = 0.3, xtickmax = 0.9,
        enlarge y limits,
        no markers,
        ]
      	\addplot+[very thick, ->] table  {../figures/res_pole(V0)/poles.data};
      	    \addlegendentry{Pole position}
      	\addplot+[very thick] table {../figures/res_pole(V0)/contour.data};
          \addlegendentry{Contour}
      \end{axis}
  \end{tikzpicture}
  \caption{The pole position as a function of $V_0$.}
  \label{fig:pole(V0)}

\end{figure}

%%%%%%%%%%%%%%%%%%%%%%%%%V0 (CONTOUR) FIGURE
  
\begin{figure}
   \centering{
   \pgfplotsset{
     width = \textwidth,
     height = 7cm,
     axis lines = middle,
     xmin = -0.01, xmax = 0.5,
     xtickmin = 0.15,
     max space between ticks=60pt
     enlargelimits,
     ylabel=Im $k/\b{\si{fm^{-1}}}$,
     every axis y label/.style={
     at = {(current axis.above origin)},
     anchor = north west,
     },
     yticklabel style={/pgf/number format/fixed,
     /pgf/number format/precision=3},
     every axis x label/.style={
     at = {(current axis.right of origin)},
     anchor = north,
     },
     every x tick label/.append style = {anchor = south, yshift = 3pt},
   }
     \subfloat[Square Contour]{
       \tikzset{external/remake next}
   \tikzsetnextfilename{sqrcont}
       \begin{tikzpicture}
         \begin{axis}[
			 ymax = 0.04,
			 legend style={at={(0.6,0.4)}, anchor=north,legend columns=1},
			 xlabel=Re $k/\b{\si{fm^{-1}}}$]
         	\addplot+[only marks,very thick] table [x index =2, y index =3] {../figures/poles(contour)/square.data};
         	    \addlegendentry{Pole position}
         	\addplot+[no marks, very thick] table [x index =0, y index =1] {../figures/poles(contour)/square.data};
             \addlegendentry{Contour}
 
         \end{axis}
       \end{tikzpicture}
     }
 
   \subfloat[Triangle Contour]{
     \tikzset{external/remake next}
 \tikzsetnextfilename{trigcont}
     \begin{tikzpicture}
       \begin{axis}[
		   ymax = 0.04,
		   legend style={at={(0.8,0.2)}, anchor=north,legend columns=1},
		   xlabel=Re $k/\b{\si{fm^{-1}}}$]
         	\addplot+[only marks, very thick] table [x index =2, y index =3] {../figures/poles(contour)/triangle.data};
         	    \addlegendentry{Pole position}
         	\addplot+[no marks, very thick] table [x index =0, y index =1] {../figures/poles(contour)/triangle.data};
			             \addlegendentry{Contour}

       \end{axis}
     \end{tikzpicture}
   }
   
   \subfloat[Sierpinski Contour]{
     \tikzset{external/remake next}
 \tikzsetnextfilename{sierpcont}
     \begin{tikzpicture}
       \begin{axis}[
		   ymax = 0.04,
		   legend style={at={(0.8,0.2)}, anchor=north,legend columns=1},
		   xlabel=Re $k/\b{\si{fm^{-1}}}$]
         	\addplot+[only marks, very thick] table [x index =2, y index =3] {../figures/poles(contour)/sierpinski.data};
         	    \addlegendentry{Pole position}
         	\addplot+[no marks, very thick] table [x index =0, y index =1] {../figures/poles(contour)/sierpinski.data};
			             \addlegendentry{Contour}

       \end{axis}
     \end{tikzpicture}
   }
   }
  \caption{yadayoda} 
   \label{fig:pole(cont)}  
\end{figure}

%%%%%%%%%%%%%%%%%%%%%%%%%V0 (CONTOUR) end

\end{document}