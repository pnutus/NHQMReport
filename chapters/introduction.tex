this is the introduction. There will be more interesting text here later. Hopefully it has started to be later now.

\section{Background}
Quantum mechanics is a cornerstone of modern physics, as it describes the world on the smallest of scales.
On this small level we observe a vast number of interesting phenomena, some wich in our ordinary large-scale world seems completely impossible.
One of these phenomena is called \emph{resonance} and in this report we will study these resonances in Helium.
Helium is choosen since it is has a small nucleus whith several interesting isotopes showing different behaviors.
Another nucleus often choosen is Lithium, which displays similar behavior.
To study these resonances we will expand the hamiltonian in a \emph{Berggren basis}, which makes the hamiltonian non-hermitian, leading us into the world of \emph{Non-Hermitian Quantum Mechanics} (NHQM).

%%%%%%%%%%%%%%%% Only say that we want to study resonances and the rest comes later????????

\section{Resonances}
Resonances is a phenomenon with its origins in scattering theory.
Once upon a time... When phycisists scattered neutrons against various nuclei and observed the obtained particles they found that some neutrons seemed to dissappear for a while, only to be seen again shortly after.
This was a mystery until (Name of smart scientist, year smart happened) came with an explanation to this problem. (Might not have been a mystery, but the smart phycisist was probably Fermi.)
This explanation introduced so called quasi-bound states, states that behaved as a bound state for a while, but then decayed into smaller pieces.
These quasi-bound states is often reffered to as resonances, even though some phycisists consider resonances only as a peak in scattering cross-section.
They can be understood more easily by observing the potential of the nucleus.
The potentials for different nucleus look slightly different, but in all nuclei where resonances is observed we find a potential barrier.
Now combine this with the knowledge of QM and we can explain the previously unexpected behavior with another phenomenon: tunneling.
Tunneling says that when looking at a QM-system with a potential barrier a particle once found on one side of the barrier later has a probability of being found on the other.
This is exactly what happens in the case of quasi-bound states.
With some statistics one can now see that these particles have a mean life-time similar to that of radioactive deccay.
This is not a coincidence since they as well are quasi-bound resonance states.

The connection between resonances and NHQM comes from the fact that the barrier width and the mean life-time of the system depends directly on the complex binding momentum.
To see this, we start by the known fact that the time-dependent solution to the Schrödinger-Equation can be expressed as:



\section{NHQM}
While the problem of standard, Hermition QM, is widely known among phisicists, the NHQM way of thinking lacks an easy-to-process theoretical foundation.
In this report we describe how we addressed this issue by calculating the (complex) eigenvalues of some Helium nuclei.

Helium is choosen since it is has a small nucleus whith several interesting isotopes showing different behavior.
Another nucleus which often is choosen is Lithium, which displays similar behavior.

Since the solution to the schroedinger equation 








Quantum mechanics (QM) is a cornerstone of modern physics, as it describes the world on the smallest of scales.
One topic often introduced in graduate QM courses is resonance scattering, a phenomenon where a system form so-called \emph{quasi-stationary} states.
In contrast to \emph{stationary} (bound) states, in which a system remains indefinitely, these quasi-stationary states decay over time.
Describing this process leads naturally to the introduction of complex energy eigenvalues. 

Complex eigenvalues pose a problem in standard QM.
This since observable quantities are regarded as real values and are described by \emph{Hermitian} operators.
When working with complex eigenvalues one thus needs a \emph{non-Hermitian} formulation of the problem.
While the idea of using non-Hermitian quantum mechanics (NHQM) has been around for quite some time, the technique is not widely used.