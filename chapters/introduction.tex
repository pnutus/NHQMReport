\documentclass[../main/report.tex]{subfiles}
\begin{document}
\chapter{Introduction}
\label{cha:introduction}






The properties of a quantum mechanical system are determined by its Hamiltonian, consisting of a kinetic energy term and a potential term.
Particles in a potential well with infinitely high walls form localized bound states.
Such a system is called a \emph{closed quantum system}, since the number of particles is conserved and the particles are localized in a finite region. 
The energy of a closed quantum system can only take on discrete values, as illustrated in \cref{fig:closed_quantum_system} with the harmonic oscillator potential.

An \emph{open quantum system}, on the other hand, portrayed in \cref{fig:open_quantum_system}, is a system with a finite potential.
For a system with a vanishing potential particles can enter and exit the system and, consequently, there are unbound states. 
For a trivial potential these are just free particle states but with a non-negligible potential, these are denoted \emph{scattering states}.
Depending on the depth of the potential well of the open system, there can be a finite number of bound states.
The number of scattering states, however, is infinite. They can take on any positive value of energy and are said to be in the energy \emph{continuum}.

\begin{figure}[b!]
  \newcommand{\depth}{3}
  \pgfdeclareverticalshading{resonance}{100bp}
   {color(0bp)=(white); color(50bp)=(black!70); color(100bp)=(white)}
  \subfloat[A closed quantum system]{
    \label{fig:closed_quantum_system}
    \begin{tikzpicture}[
      scale = 1.6,
      domain=0:2,
      samples=200,
      ]
      \draw[->] (0, 0) -- (2, 0) node[above] {$r$};
      \draw[->] (0, -2.1) -- (0, 2) node[right] {$V$};
      \draw plot (\x, {\x*\x - 2});
      \foreach \y in {-1.3, -0.6, ..., 2} {
        \draw (0, \y) -- ($ ({ sqrt(\y + 2)} , \y) $);
        \draw[shorten >=0.5mm] (-1, 0)
          -- (0, \y);
      }
      \node[left, align=center, fill=white] 
        at (-0.2, 0)  {Bound\\states};
    \end{tikzpicture}
    
  } 
  \subfloat[An open quantum system]{
    \label{fig:open_quantum_system}
    \begin{tikzpicture}[
      scale=1.6,
      domain=0:2.95,
      samples=200,
      ]
      \begin{scope}[->]
        \draw[decorate, decoration={snake, post length=1mm}] 
          (0, 1) -- (3, 1);
        \draw[decorate, decoration={snake, post length=1mm,
                                    segment length = 2mm}] 
          (0, 1.4) -- (2.7, 1.4);
      \end{scope}
      
      \shade[shading=resonance]
        (0,0.5) ++ (0,-0.02) rectangle +(3,0.04);
      \begin{scope}[left, fill=white, align=right]
        \node at (0, 1.2) {Scattering\\states};
        \node at (0, 0.5) {Resonance};
      \end{scope}
     
      \draw[->] (0, 0) node[left] {$0$} -- (3, 0) node[below] {$r$};
      \draw[->] (0, -2.1) -- (0, 2) node[right] {$V$};
      \draw plot (\x,{-2/(1 + exp((\x-1)/0.3))});
      \foreach \x/\y in {1/-1, 1.33/-0.5} {
        \draw (0, \y)-- (\x, \y);
      }
      \node[left, align=center] 
        at (0,-0.75)  {Bound\\states};
        
        
      \draw[decorate, decoration={brace}] 
        (3.1,2) -- +(0,-2) node[midway, xshift=0.4cm, rotate=-90] {Continuum};
    \end{tikzpicture}
    
  }
  \caption{An open and a closed quantum system. The open system has an infinite number of bound, localized states, whereas the closed system has unbound scattering states and resonances in addition to a finite number of bound states.}
  \label{fig:potentials}
\end{figure}

In addition to the bound and unbound states, open quantum systems can harbor \emph{resonances}. 
These are \emph{quasi-bound} states that are neither bound nor unbound, but exhibit properties of both. 
They appear in the continuum, like unbound states, but are localized, like bound states.
However, the wavefunction of a resonance is only localized for a finite amount of time, as opposed to a bound, \emph{stationary} state that forever stays the same.
A resonance can be described with a complex energy, as the following argument suggests.

The state of a particle is described by its wavefunction $\psi$, which can be written as the product of a function of time and position
\begin{eq}
  \psi(t, \vec{r}) = \psi_t(t)\psi_{\vec{r}}(\vec{r}).
\end{eq}
The wavefunction evolves according to the \emph{Time-Dependent Schrödinger Equation} (TDSE)
\begin{eq}
  \label{eq:schrödinger}
  i\hbar\ddt\ket\psi = H \ket\psi.
\end{eq}
An eigenstate of the Hamiltonian $H$ with energy $E$, i.e. a solution to the \emph{Time-Independent Schrödinger Equation} (TISE)
\begin{eq}
  H \ket\psi = E \ket\psi
\end{eq}
has the simple time evolution
\begin{eq}
	\psi(t, \vec{r})
	= 
  \exp\p{-\frac{iE}{\hbar}t}\psi(0, \vec{r}).
\end{eq}
With the energy $E$ real, the exponential factor is just a phase 
and the probability $|\psi(t, \vec{r})|^2$ of finding the particle at a given $\vec{r}$ is unchanged over time. 
However, if we let the energy be complex
\begin{eq}
	E = E_0 - i\frac{\Gamma}{2},
\end{eq}
we get
\begin{eq}
  |\psi(t, \vec{r})|^2 
  =
  \absq{
    \exp\p{-\frac{iE_0}{\hbar} t} 
    \exp\p{- \frac{\Gamma}{2\hbar} t} 
    \psi(0, \vec{r})
  }
  =
  \exp\p{-\frac{\Gamma}{\hbar} t} \absq{\psi(0, \vec{r})},
\end{eq} 
describing a resonance state decaying exponentially with half-life $t_{1/2}=\hbar\ln 2/\Gamma$. This parameter $\Gamma$ is called the \emph{width} of the resonance.

We want to use the simpler formalism of the TISE, as opposed to the more general TDSE, and we see that this is possible by letting the resonance have a complex energy.
However, complex eigenvalues pose a problem in standard quantum mechanics, where operators are postulated to be Hermitian.
Hermitian operators can only have real eigenvalues, and are thus insufficient for treating resonances.

The aim of this thesis is to present methods for describing resonances using the TISE in a complex-energy framework.
The motivation behind the development of these methods is their ability to describe loosely bound or unbound systems, where other methods only work for stable, strongly bound systems.
The methods are used to study the simple nuclear systems \He{5} and \He{6}, comparing the results to experimental data.
The He nuclei are relevant because the ground state of \He{5} and the first excited state of both \He{5} and \He{6} are resonances. 
Despite the unbound nature of \He{5}, the ground state of \He{6} is actually bound, an example of a \emph{Borromean} system.

We solve the TISE numerically, but have in this thesis chosen to focus on the mathematical and physical aspects as the implementation has been rather straightforward in most cases.
Consequently, there will be no code nor pseudocode in the thesis, although the code we have written to produce the results is freely available at
\begin{quote}
  \url{https://github.com/pnutus/NHQM}.
\end{quote}
We used the \emph{Python} programming language with the libraries \emph{NumPy} and \emph{SciPy} to perform computations. 
The figures in the thesis were made using the \emph{TikZ} and \emph{pgfplots} LaTeX libraries together with the Python code.

The thesis can conceptually be divided into two parts, the first covering resonances in a simple two-body problem and the second part covering the first steps toward more complicated many-body systems. 
In \cref{cha:basis_expansion} the \emph{basis expansion} method for solving the Schrödinger equation is introduced.
The basis expansion method is then used in \cref{cha:two-body} to study a loosely bound two-body nuclear system, the \He{5} nucleus.
In \Cref{cha:berggren} we use the Berggren basis to reproduce the resonance in \He{5}.

\Cref{cha:many-body} is an introduction to many-body theory, focusing on fermionic systems. 
The many-body theory is then utilized in \cref{cha:three-body} 
to study a three-body problem, specifically the \He{6} nucleus.  
Finally, \cref{cha:outlook} is an outlook discussing further development of the methods.

\end{document}
