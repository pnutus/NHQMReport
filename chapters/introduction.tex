\documentclass[../main/report.tex]{subfiles}
\begin{document}
\chapter{Introduction}

% The dynamics of a quantum mechanical system is governed by the Shrödinger equation
% \begin{eq}
%   \label{eq:schrödinger}
%   i\hbar\ddt\ket\psi = H \ket\psi.
% \end{eq}
% Of interest are states with well-defined energy $E$, i.e. the eigenstates, or \emph{spectrum}, of the Hamiltonian $H$
% \begin{eq}
%   H \ket\psi = E \ket\psi.
% \end{eq}
% The spectrum can be discrete, continuous or both. States with energy values in the discrete part of the spectrum correspond to bound states, whereas states in the continuum are unbound. The type of spectrum depends largely on the form of the potential. 
% 
% \Cref{fig:harmosc} displays the potential of a \emph{closed} system, in this case the harmonic oscillator. It has an infinitely high potential well from which nothing can escape, thus a particle in the well will be bound
% and the spectrum only consists of discrete states. 

The properties of a quantum mechanical system is largely determined by its potential. 
Depending on the type of potential, a system can give rise to bound states, unbound states or both.

A particle in a potential well with infinitely high walls is completely bound and cannot leave the potential well.
Such a system is a called a \emph{closed} quantum system, since it does not interact with the environment. 
The energy of a closed quantum system can only take on discrete values, as illustrated in \cref{fig:closed_quantum_system} with the harmonic oscillator potential.
The wavefunction of a bound state is localized inside the potential well.

The polar opposite of a system with an infinitely high potential is no potential at all, i.e.~a free particle propagating through space (\cref{fig:free_particle}).
This system is not restricted to discrete energy values.
In fact, it can take on a any energy value, continuously. \todo{"continously"}
An unbound, free state is therefore said to be \emph{in the continuum}.
The wavefunction of an unbound state is unlocalized and has infinite range.

Between the closed and completely free systems is the \emph{open} quantum system, portrayed in  \cref{fig:open_quantum_system}. \todo{"between"}
Here the potential has a finite range, i.e. it levels off toward zero.
Thus, particles can enter and exit the system and, consequently, there are unbound states. 
In addition to the unbound states, the open system can have a finite number of bound states.


We are interested in a third kind of state, that is neither bound nor unbound, a so-called \emph{resonance}. 
These are \emph{quasi-bound} states that exhibit properties of both bound and unbound states. 
A resonance wavefunction is localized, but will escape the potential well after a short time. \todo{blanda in geometri här? at det finns sannolikhet utanför}
This can be expressed mathematically by letting the resonance have a complex energy.

\begin{figure}
  \newcommand{\depth}{3}
  \subfloat[Closed quantum system]{
    \label{fig:closed_quantum_system}
    \begin{tikzpicture}[
      scale=1.6,
      domain=-2:2,
      samples=200,
      ]
      %\draw[->] (-3, 0) -- (3, 0) node[above] {$r$};
      \draw[->] (0, 0) -- (0, 4) node[right] {$V$};
      \draw plot (\x, \x*\x);
      \foreach \y in {0.7, 1.4, ..., 4}
        \draw ($ ({-sqrt(\y)} , \y) $) 
           -- ($ ({ sqrt(\y)} , \y) $);
        \node[above] at (-0.04, 2.1) {Bound states};
    \end{tikzpicture}
  }
  \subfloat[Free particle]{
    \label{fig:free_particle}
    \begin{tikzpicture}[
      scale=1.6
      ]
      %\draw[->] (-3, 0) -- (3, 0) node[above] {$r$};
      \draw[->] (0, -2) -- (0, 2) node[right] {$V$};
      \draw[->, decorate, decoration={snake, post length=1mm}] 
        (-2, 0) -- (2, 0) node[above left, yshift=1mm] {Unbound};
    \end{tikzpicture}
  }
  \\
  \subfloat[Open quantum system]{
    \label{fig:open_quantum_system}
    \begin{tikzpicture}[
      scale=2,
      xscale=0.5, 
      domain=-6:6,
      samples=200,
      ]
      %\draw[->] (-6, 0) -- (6, 0) node[above] {$r$};
      \draw (-6, 0) -- (6, 0)
        node[right] {$V = 0$};
      \draw[->] (0, -3.1) -- (0, 2) node[right] {$V$};
      \draw plot (\x,{-\depth*exp(- \x*\x/6)});
      \foreach \y in {-2.3, -1.7, -1}
      \draw ($ ({sqrt(-6*ln(-\y/\depth))}, \y)$)
         -- ($ ({-sqrt(-6*ln(-\y/\depth))}, \y)$);
      \draw[densely dashed] 
        (-6,0.5) -- (6,0.5)
        node[midway, above right] {Resonance state};
      \begin{scope}
        [->, decoration={snake, post length=1mm}, gray]
        \draw[decorate] (-6, 1) -- (6, 1) 
          node[black, above left] {Unbound scattering states};
        \draw[decorate] (-6, 1.4) -- (5.5, 1.4);
      \end{scope} 
      \draw[decorate, decoration={brace}] 
        (-6.1,0) -- +(0,2) node[midway, xshift=-0.4cm, rotate=90] {Continuum};
    \end{tikzpicture}
  }
  \caption{}
  \label{fig:potentials}
\end{figure}



The time-dependency of a bound state $\psi$ with a energy $E$ in the discrete spectrum is
\begin{eq}
	\psi(t)
	= 
  \exp\p{-\frac{iE}{\hbar}t}\psi(0).
\end{eq}
With the energy $E$ real, the exponential factor is just a phase 
and the probability $|\psi(t)|^2$ is unchanged over time (hence the name
stationary). However, if we let the energy be complex
\begin{eq}
	E = E_0 - i\frac{\Gamma}{2},
\end{eq}
we get
\begin{eq}
  |\psi(t)|^2 
  =
  \absq{
    \exp\p{-\frac{iE_0}{\hbar} t} \exp\p{- \frac{\Gamma}{2\hbar} t} \psi(0)
  }
  =
  \exp\p{-\frac{\Gamma}{\hbar} t} \absq{\psi(0)}
\end{eq} 
which describes a resonant state with half-life 
$t_{1/2}=\hbar\ln 2/\Gamma$ and so-called \emph{width} $\Gamma$. \todo{Too many so-called in introduction.}

\todo{Is non-hermitian even important?}
It seems, then, that we need complex energies to describe resonant 
states. However, complex eigenvalues pose a problem in standard QM. 
This is because observable quantities are regarded as real values
and are described by \emph{Hermitian} operators. When working with 
complex eigenvalues one needs a \emph{non-Hermitian} formulation of 
the problem, which we encounter in \cref{cha:nhqm}.

The systems we have chosen to study %using numerical calculations
are the nuclei of the  Helium isotopes \He{5} and \He{6}. We chose Helium  
because \He{4} is a very stable nucleus that can be treated 
as a single (alpha) particle. \He{5} and \He{6} are then modeled
as an alpha particle core with one or two valence neutrons, 
respectively. 

\He{5} has a known resonance with half-life $t_{1/2} = \SI{700e-24}{s}$,
that we use to set the problem parameters. \todo{We don't verify it, we use it to fix our basis.} In addition to resonant states, \He{6} has a bound state because of the attraction between the valence neutrons. Our goal is to find these resonances and bound states to verify the methods used.

\todo{Where does width stuff go?}
 Heisenberg's uncertainty principle gives a relation between energy and time
 \begin{equation}
	 \Delta E \Delta t \ge \frac{\hbar}{2}.
 \end{equation}
 Hence a state with finite life time must have an uncertainty in its energy spectrum, this is called the \emph{width} of a resonant state. It is this width that is measured in experiments.

This report can be thought of as divided into two parts, the first covering resonances in a simple two-body problem and the second part covering the first steps toward more complicated many-body systems. 
In \cref{cha:basis_expansion} the mathematical foundation for the calculations in this thesis, \emph{basis expansion}, is introduced.
The basis expansion method is then used in \cref{cha:two-body} to study a loosely bound nuclear system, the \He{5} nucleus.
In \Cref{cha:berggren} we use the Berggren basis to reproduce the resonance in \He{5}.

\Cref{cha:many-body} is an introduction to many-body theory, focusing on fermionic systems. 
The many-body theory is then utilized in \cref{cha:three-body} 
to study a three-body problem, specifically the \He{6} nucleus.  
In \cref{cha:monte_carlo} a Monte Carlo method for reducing the basis size is investigated. 
Finally, \cref{cha:outlook} discusses the results and methods and suggests further avenues of inquiry.

\end{document}