\documentclass[../main/report.tex]{subfiles}
\begin{document}
\chapter{Introduction}

The properties of a quantum mechanical system is largely determined by its potential. 
Depending on the shape of the potential, a system can give rise to bound states, unbound states or both.
A \emph{closed} quantum system with an infinite potential well, such as the harmonic oscillator, has only bound solutions. 
Bound means that the wavefunction stays localized indefinitely inside the potential well. 
An \emph{open} quantum system, on the other hand, has unbound solutions, where the wavefunction passes through the system only slightly perturbed. 
The potentials of an open and a closed system can be seen in \cref{fig:potentials}.

\begin{figure}
  \newcommand{\axes}[6]{
    \draw[->] (#2, 0) -- (#3, 0) node[right] {#1};
    \draw[->] (0, #5) -- (0, #6) node[above] {#4};
  }
  \subfloat[]{
    \label{fig:harmosc}
    \begin{tikzpicture}
      \begin{axis}[
        xlabel = $r$,
        ylabel = $V$,
        axis lines = middle,
        ymin = -5.1, ymax = 2,
        ticks = none
        ]
        \addplot[black] {x^2 - 5};
      \end{axis}
    \end{tikzpicture}
  }
  \subfloat[]{
    \label{fig:woods-saxon}
    \begin{tikzpicture}
      \begin{axis}[
        xlabel = $r$,
        ylabel = $V$,
        axis lines = middle,
        ymin = -5.1, ymax = 2,
        ticks = none
        ]
        \addplot[black] {-5/(1+e^((abs(x)-2)/0.65))};
      \end{axis}
    \end{tikzpicture}
  }
  \caption{The harmonic oscillator (a) and Woods-Saxon (b) potentials. The harmonic oscillator is a closed system, with an infinitely high potential well. It has a discrete spectrum with only bound solutions. The Woods-Saxon potential describes an open system, with a continuous spectrum of unbound states as well as possible bound states and quasi-bound resonances.}
  \label{fig:potentials}
\end{figure}

We are interested in a third kind of solution to the equation, so-called \emph{resonances}. 
These are \emph{quasi-bound} states which exhibit properties of both bound and unbound states. 
A resonance wavefunction is localized, but will escape the potential well after a short time.
This can be expressed mathematically by letting the resonance have a complex energy.

The time-dependency of a bound state $\psi$ with a energy $E$ in the discrete spectrum is
\begin{eq}
	\psi(t)
	= 
  \exp\p{-\frac{iE}{\hbar}t}\psi(0).
\end{eq}
With the energy $E$ real, the exponential factor is just a phase 
and the probability $|\psi(t)|^2$ is unchanged over time (hence the name
stationary). However, if we let the energy be complex
\begin{eq}
	E = E_0 - i\frac{\Gamma}{2},
\end{eq}
we get
\begin{eq}
  |\psi(t)|^2 
  =
  \absq{
    \exp\p{-\frac{iE_0}{\hbar} t} \exp\p{- \frac{\Gamma}{2\hbar} t} \psi(0)
  }
  =
  \exp\p{-\frac{\Gamma}{\hbar} t} \absq{\psi(0)}
\end{eq} 
which describes a resonant state with half-life 
$t_{1/2}=\hbar\ln 2/\Gamma$ and so-called \emph{width} $\Gamma$. \todo{Too many so-called in introduction.}

\todo{Is non-hermitian even important?}
It seems, then, that we need complex energies to describe resonant 
states. However, complex eigenvalues pose a problem in standard QM. 
This since observable quantities are regarded as real values 
and are described by \emph{Hermitian} operators. When working with 
complex eigenvalues one needs a \emph{non-Hermitian} formulation of 
the problem, which we encounter in \cref{cha:nhqm}.

The systems we have chosen to study using numerical calculations
are the nuclei of the  Helium isotopes \He{5} and \He{6}. We chose Helium  
because \He{4} is a very stable nucleus that can be treated 
as a single (alpha) particle. \He{5} and \He{6} are then modeled
as an alpha particle core with one or two valence neutrons, 
respectively. 

\He{5} has a known resonance with half-life $t_{1/2} = \SI{700e-24}{s}$,
which we verify. \todo{which we verify ... which we seek to verify} \todo{We don't verify it, we use it to fix our basis.} In addition to resonant states, \He{6} has a bound state because of the attraction between the valence neutrons. Our goal is to find these resonances and bound states to verify the methods used.

\todo{Where does width stuff go?}
 Heisenberg's uncertainty principle gives a relation between energy and time
 \begin{equation}
	 \Delta E \Delta t \ge \frac{\hbar}{2}.
 \end{equation}
 Hence a state with finite life time must have an uncertainty in its energy spectrum, this is called the \emph{width} of a resonant state. It is this width that is measured in experiments.

This report can be thought of as divided into two parts, the first covering resonances in a simple two-body problem and the second part covering the first steps toward more complicated many-body systems. 
In \cref{cha:basis_expansion} the mathematical foundation of the calculations in this thesis, basis expansion, is introduced.
The basis expansion method is then used in \cref{cha:he5} to study a loosely bound nuclear system, the \He{5} nucleus.
In \Cref{cha:nhqm} we use the Berggren basis to reproduce the resonance in \He{5}.

\Cref{cha:many-body} is an introduction to many-body theory, focusing on fermionic systems. 
The many-body theory is then utilized in \cref{cha:he6} 
to study a three-body problem, specifically the \He{6} nucleus.  
In \cref{cha:monte_carlo} a Monte Carlo method for reducing the basis size is investigated. 
Finally, \cref{cha:outlook} discusses the results and methods and suggests further avenues of inquiry.

\end{document}