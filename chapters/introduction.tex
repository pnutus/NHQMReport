\documentclass[../main/report.tex]{subfiles}
\begin{document}
\chapter{Introduction}

The properties of a quantum mechanical system is largely determined by its potential. 
Depending on the type\todo{structure of the,kind of, nature of the} of potential, a system can give rise to bound states, unbound states or both.

A particle in a potential well with infinitely high walls is completely bound and cannot leave the potential well.
Such a system is a called a \emph{closed} quantum system, since it does not interact with the environment. 
The energy of a closed quantum system can only take on discrete values, as illustrated in \cref{fig:closed_quantum_system} with the harmonic oscillator potential.
The wavefunction of a bound state is localized inside the potential well.
\todo{Talk about positive/negative energy?}

The polar opposite of a system with an infinitely high potential is no potential at all, i.e.~a free particle propagating through space (\cref{fig:free_particle}).
This system is not restricted to discrete energy values and can, in fact, take on any energy value.
An unbound, free state is therefore said to be \emph{in the continuum}.
The wavefunction of an unbound state is unlocalized and has infinite range.

Between the closed and completely free systems is the \emph{open} quantum system, portrayed in \cref{fig:open_quantum_system}. \todo[inline]{between?}
Here, the potential has a finite range, i.e. it levels off toward zero.
Thus, particles can enter and exit the system and, consequently, there are unbound states. \todo[inline]{paricles can thus?}
Depending on the depth of the potential well of the open system, there can be a finite number of bound states.

In addition to the bound and unbound states, an open quantum system can harbor \emph{resonances}. 
These are \emph{quasi-bound} states that are neither bound nor unbound, but exhibit properties of both. 
It appears in the continuum, like an unbound state, but its wavefunction is localized, like that of a bound state.
Perhaps most curious is that the norm of the resonance state decreases exponentially with time. 
This can be expressed mathematically if we let the resonance correspond to a complex energy, as the following argument suggests.

\begin{figure}
  \newcommand{\depth}{3}
  \subfloat[Closed quantum system]{
    \label{fig:closed_quantum_system}
    \begin{tikzpicture}[
      scale=1.6,
      domain=-2:2,
      samples=200,
      ]
      %\draw[->] (-3, 0) -- (3, 0) node[above] {$r$};
      \draw[->] (0, 0) -- (0, 4) node[right] {$V$};
      \draw plot (\x, \x*\x);
      \foreach \y in {0.7, 1.4, ..., 4}
        \draw ($ ({-sqrt(\y)} , \y) $) 
           -- ($ ({ sqrt(\y)} , \y) $);
        \node[above] at (-0.04, 2.1) {Bound states};
    \end{tikzpicture}
  } 
  \subfloat[Free particle]{
    \label{fig:free_particle}
    \begin{tikzpicture}[
      scale=1.6
      ]
      %\draw[->] (-3, 0) -- (3, 0) node[above] {$r$};
      \draw[->] (0, -2) -- (0, 2) node[right] {$V$};
      \draw[->, decorate, decoration={snake, post length=1mm}] 
        (-2, 0) -- (2, 0) node[above left, yshift=1mm] {Unbound};
    \end{tikzpicture}
  }
  \\
  \subfloat[Open quantum system]{
    \label{fig:open_quantum_system}
    \begin{tikzpicture}[
      scale=2,
      xscale=0.5, 
      domain=-6:6,
      samples=200,
      ]
      %\draw[->] (-6, 0) -- (6, 0) node[above] {$r$};
      \draw (-6, 0) -- (6, 0)
        node[right] {$V = 0$};
      \draw[->] (0, -3.1) -- (0, 2) node[right] {$V$};
      \draw plot (\x,{-\depth*exp(- \x*\x/6)});
      \foreach \y in {-2.3, -1.7, -1}
      \draw ($ ({sqrt(-6*ln(-\y/\depth))}, \y)$)
         -- ($ ({-sqrt(-6*ln(-\y/\depth))}, \y)$);
      \draw[densely dashed] 
        (-6,0.5) -- (6,0.5)
        node[midway, above right] {Resonance state};
      \begin{scope}
        [->, decoration={snake, post length=1mm}, gray]
        \draw[decorate] (-6, 1) -- (6, 1) 
          node[black, above left] {Unbound scattering states};
        \draw[decorate] (-6, 1.4) -- (5.5, 1.4);
      \end{scope} 
      \draw[decorate, decoration={brace}] 
        (-6.1,0) -- +(0,2) node[midway, xshift=-0.4cm, rotate=90] {Continuum};
    \end{tikzpicture}
  }
  \caption{Three types of quantum systems: closed, completely free and open.}
  \label{fig:potentials}
\end{figure}
\todo[inline]{bound state in last figure?}
\todo[inline]{Colored energy levels? y-axis to the left of the potentials? add x-axis specifying position?}
\todo{fixa klart figuren}

The time evolution of a quantum system is governed by the Schrödinger equation
\begin{eq}
  \label{eq:schrödinger}
  i\hbar\ddt\ket\psi = H \ket\psi.
\end{eq}
The time-dependency of an eigenstate to the Hamiltonian with energy $E$ is 
\begin{eq}
	\psi(t)
	= 
  \exp\p{-\frac{iE}{\hbar}t}\psi(0).
\end{eq}
With the energy $E$ real, the exponential factor is just a phase 
and the norm $|\psi(t)|^2$ is unchanged over time. 
However, if we let the energy be complex
\begin{eq}
	E = E_0 - i\frac{\Gamma}{2},
\end{eq}
we get
\begin{eq}
  |\psi(t)|^2 
  =
  \absq{
    \exp\p{-\frac{iE_0}{\hbar} t} \exp\p{- \frac{\Gamma}{2\hbar} t} \psi(0)
  }
  =
  \exp\p{-\frac{\Gamma}{\hbar} t} \absq{\psi(0)}
\end{eq} 
which describes a state that decays exponentially with half-life 
$t_{1/2}=\hbar\ln 2/\Gamma$. $\Gamma$ is called the \emph{width} of the resonance.

\todo[inline]{Talk about width here? Needs to be better. Figure of experimental resonance?}
 % Heisenberg's uncertainty principle gives a relation between energy and time
 % \begin{equation}
 %    \Delta E \Delta t \ge \frac{\hbar}{2}.
 % \end{equation}
 % Hence a state with finite life time must have an uncertainty in its energy spectrum, called the \emph{width} of a resonant state. It is this width that is measured in experiments.
 
\todo[inline]{We should mention the parallel to tunneling and radioactive decay}

It seems, then, that we need complex energies to describe resonances. 
However, complex eigenvalues pose a problem in standard quantum mechanics, where operators are postulated to be Hermitian.
Hermitian operators can only have real eigenvalues, and are thus insufficient for treating resonances.

The aim of this thesis is to present methods for dealing with the complex energies that describe resonances. To demonstrate the methods, we will use 
the simple nuclear systems \He{5} and \He{6}, which are both known to have resonances.
\todo[inline]{Many ``this thesis'' in last and next paragraph.}

The thesis can be thought of as divided into two parts, the first covering resonances in a simple two-body problem and the second part covering the first steps toward more complicated many-body systems. 
In \cref{cha:basis_expansion} the mathematical foundation for the calculations in this thesis, \emph{basis expansion}, is introduced.
The basis expansion method is then used in \cref{cha:two-body} to study a loosely bound two-body nuclear system, the \He{5} nucleus.
In \Cref{cha:berggren} we use the Berggren basis to reproduce the resonance in \He{5}.

\Cref{cha:many-body} is an introduction to many-body theory, focusing on fermionic systems. 
The many-body theory is then utilized in \cref{cha:three-body} 
to study a three-body problem, specifically the \He{6} nucleus.  
In \cref{cha:monte_carlo} a Monte Carlo method for reducing the basis size\todo{and reducing calculation time?} is investigated. 
Finally, \cref{cha:outlook} discusses the methods and results, and suggests further avenues of inquiry.

\end{document}