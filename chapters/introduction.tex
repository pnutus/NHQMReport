\todo{One to threes sentences to lead our way even better.}
\todo{Alternative introduction, old one is commented out in the source-file. It is not much corrected, but should be read through first.}
In Quantum Mechanics, the dynamics of a system is described by its wave function, given by the Schrödinger equation
\begin{eq}
i\hbar\ddt{}\psi=H\psi.
\end{eq}
The solutions here are dependent on the potential of the Hamiltonian and can be observed to have different behavior.
Normaly they are either a set of discrete, bound, solutions as in the case of the harmonic oscillator or a set of continous eigenvalues which can describe free particles.
It is also common to observe a combination of these two where some states are descrete and bound and some free in the continuum.
This behavior is known since the Hydrogen atom.
Sometimes, however, indications towards a third type of solutions are noticed.
These solutions shows a complex energy, describing a particle whose wavefunction is temporary localized, but quickly becomes ``unlocalized'' (free).
A system with this  property is said to be \emph{quasi-bound} state, or a \emph{resonance}.
The probability for these quasi-bound systems of staying localized will decrease exponentially over time, which follows through the following argument using a complex energy.

%These functions are observed to be either discrete with negative energies (bound) or continous (free) with positive energies.
%However, there is a third possibility where the energy is complex.
%Here, the wavefunction is temporary localized, but after a short time it can be seen as free.
%This kind of state is called a \emph{quasi-bound} state or a \emph{resonance}.
%The probability of a quasi-bound particle to stay localized will decrease exponentially over time.
%This can be described with complex energies, as we said earlier, and the following argument

%In Quantum Mechanics (QM), a particle is often said to be either bound or free.
%However, there is a third possibility, where the particle is bound, but
%only for a limited time, after which it escapes. This kind of state is 
%called a \emph{quasi-bound} state, or a \emph{resonance}. The 
%probability of a quasi-bound particle remaining bound will decay 
%exponentially over time. This can be explained as the state 
%having a complex energy by the following argument:

\begin{figure}
  \subfloat[]{
  \label{fig:coloumb}
  \begin{tikzpicture}[domain=0.2:6, samples=100]
    \draw[->] (-0.1, 0) -- (6, 0) node[right] {$r$};
    \draw[->] (0, -5) -- (0, 0.5) node[above] {$V$};
    \draw plot (\x,{-1/\x});
  \end{tikzpicture}
  }
  \subfloat[]{
  \label{fig:woods-saxon}
  \begin{tikzpicture}[domain=0:6, samples=100]
    \draw[->] (-0.1, 0) -- (6, 0) node[right] {$r$};
    \draw[->] (0, -5) -- (0, 0.5) node[above] {$V$};
    \draw plot (\x,{-5/(1 + exp((\x - 2)/0.65))});
  \end{tikzpicture}
  }
  \caption{The Coulomb (a) and Woods-Saxon (b) potentials.}
  \label{fig:potentials}
\end{figure}


The time-dependency of a bound, stationary state $\psi$ is
\begin{eq}
	\psi(t)
	= 
  \exp\p{-\frac{iE}{\hbar}t}\psi(0).
\end{eq}
With the energy $E$ real, the exponential factor is just a phase 
and the probability $|\psi(t)|^2$ is unchanged over time (hence the name
stationary). However, if we let the energy be complex
\begin{eq}
	E = E_0 - i\frac{\Gamma}{2},
\end{eq}
we get
\begin{eq}
  |\psi(t)|^2 
  =
  \absq{
    \exp\p{-\frac{iE_0}{\hbar} t} \exp\p{- \frac{\Gamma}{2\hbar} t} \psi(0)
  }
  =
  \exp\p{-\frac{\Gamma}{\hbar} t} \absq{\psi(0)}
\end{eq} 
which describes a resonant state with half-life 
$t_{1/2}=\hbar\ln 2/\Gamma$ and so-called \emph{width} $\Gamma$.

\todo{Is non-hermitian even important?}
It seems, then, that we need complex energies to describe resonant 
states. However, complex eigenvalues pose a problem in standard QM. 
This since observable quantities are regarded as real values 
and are described by \emph{Hermitian} operators. When working with 
complex eigenvalues one needs a \emph{non-Hermitian} formulation of 
the problem, which we encounter in \cref{cha:nhqm}.

The systems we have chosen to study using numerical calculations
are the nuclei of the  Helium isotopes \He{5} and \He{6}. We chose Helium  
because \He{4} is a very stable nucleus that can be treated 
as a single (alpha) particle. \He{5} and \He{6} are then modeled
as an alpha particle core with one or two valence neutrons, 
respectively. 

\He{5} has a known resonance with half-life $t_{1/2} = \SI{700e-24}{s}$,
which we verify. In addition to resonant states, \He{6} has a bound state because of the attraction between the valence neutrons. Our goal is to find these resonances and bound states to verify the methods used.

\todo{Should we have results in the introduction.}

\todo{Tunneling is when $E<V\sub{max}$. This is $E>V\sub{max}$. But we can relate them!}

\todo{Where does width stuff go?}
% Heisenberg's uncertainty principle gives a relation between energy and time
% \begin{equation}
% \Delta E \Delta t \ge \frac{\hbar}{2}.
% \end{equation}
% Hence a state with finite life time must have an uncertainty in its energy spectrum, this is called the \emph{width} of a resonant state. It is this width that is measured in experiments.

In \cref{cha:basis expansion} the concept of 
basis expansion, which is the mathematical foundation for all the
following calculations, is introduced. The basis expansion method is then used in 
\cref{cha:he5} to study the \He{5} nucleus, looking for a resonance.
\Cref{cha:nhqm} introduces the Berggren basis, which
is used to reproduce the resonance.

\todo{Lessen the times introduction, introduces, osv is used.}

\Cref{cha:many-body} is an introduction to many-body theory, focusing 
on fermionic systems. The many-body theory is then utilized in \cref{cha:he6} 
to study the \He{6} nucleus, a three-body problem. \Cref{cha:he7}
continues with the four-body \He{7} nucleus, and in \cref{cha:monte carlo}
a Monte Carlo method is used to speed up calculations.
