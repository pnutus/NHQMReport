In quantum mechanics, a particle is often said to be either bound or free.
However, there is a third possibility, where the particle is bound, but
only for a limited time, after which it escapes. This kind of state is 
called a \emph{quasi-bound} state, or a \emph{resonance}. The 
probability of a quasi-bound particle remaining bound will decay 
exponentially over time. This can be explained as the state 
having a complex energy by the following argument:

The time-dependency of a bound, stationary state $\psi$ is
\begin{eq}
	\psi(t)
	= 
  \exp\p{-\frac{iE}{\hbar}t}\psi(0).
\end{eq}
With the energy $E$ real, the exponential factor is just a phase 
and the probability $|\psi(t)|^2$ is unchanged over time (hence the name
stationary). However, if we let the energy be complex
\begin{eq}
	E = E_0 - i\frac{\Gamma}{2},
\end{eq}
we get
\begin{eq}
  |\psi(t)|^2 
  =
  \absq{
    \exp\p{-\frac{iE_0}{\hbar} t} \exp\p{- \frac{\Gamma}{2\hbar} t} \psi(0)
  }
  =
  \exp\p{-\frac{\Gamma}{\hbar} t} \absq{\psi(0)}
\end{eq} 
which describes a resonant state with half-life 
$t_{1/2}=\hbar\ln 2/\Gamma$.

\todo{Is non-hermitian even important?}
It seems, then, that we need complex energies to describe resonant 
states. However, complex eigenvalues pose a problem in standard QM. 
This is because observable quantities are regarded as real values 
and are described by \emph{Hermitian} operators. When working with 
complex eigenvalues one needs a \emph{non-Hermitian} formulation of 
the problem, which we encounter in \cref{cha:nhqm}.

The systems we have chosen to study using numerical calculations
are the nuclei of the  Helium isotopes \He{5} and \He{6}. We chose Helium  
because \He{4} has a very stable nucleus that can be treated 
as a single (alpha) particle. \He{5} and \He{6} are then modeled
as an alpha particle core with one and two valence neutrons, 
respectively. 

\He{5} has a known resonance with half-life $t_{1/2} = \SI{700e-24}{s}$,
which we verify. In addition to resonant states, \He{6} has a bound state.
It is a so-called Boromean nucleus because of the attraction between the valence neutrons.

\todo{Tunneling is when $E<V\sub{max}$. This is $E>V\sub{max}$. But we can relate them!}

\todo{Where does width stuff go?}
% Heisenberg's uncertainty principle gives a relation between energy and time
% \begin{equation}
% \Delta E \Delta t \ge \frac{\hbar}{2}.
% \end{equation}
% Hence a state with finite life time must have an uncertainty in its energy spectrum, this is called the \emph{width} of a resonant state. It is this width that is measured in experiments.

\section{Tools}

All calculations were made in the Python programming language 
using the libraries NumPy and SciPy. We used the matplotlib 
library to make the figures. The code was managed using Git 
and is available at
\begin{quote}
  \url{https://github.com/pnutus/NHQM}
\end{quote}

\section{Reading Guide}
We begin in \cref{cha:basis expansion} by introducing the concept of 
basis expansion, which is the mathematical foundation for all
following calculations. The basis expansion method is then used in 
\cref{cha:he5} to study the \He{5} nucleus, looking for a resonance.
In \cref{cha:nhqm} we extend our methods to the complex plane, which
lets us quantify the resonance.

\Cref{cha:many-body} is an introduction to many-body theory, focusing 
on fermionic systems. The theory is then used in \cref{cha:he6} to study
the \He{6} nucleus, a three-body problem. In \cref{cha:he7} we continue 
with the four-body \He{7} nucleus, and use a Monte Carlo method in
\cref{cha:monte carlo} to speed up calculations.
