\documentclass[../main/report.tex]{subfiles}
\begin{document}

\chapter{Introduction}
\label{cha:introduction}

In Quantum Mechanics, the dynamics of a system is described by its wave function, given by the Schrödinger equation
\begin{eq}
i\hbar\ddt{}\psi=H\psi.
\end{eq}
The solutions here are dependent on the potential of the Hamiltonian and can be observed to have different behaviors.
Normaly they are either a set of discrete, bound, solutions as in the case of the harmonic oscillator or a set of continous eigenvalues which describes free particles.
It is also common to observe a combination of these two where some states are discrete and bound and some are free in the continuum.
This behavior is known since the Hydrogen atom.
Sometimes, however, indications towards a third type of solutions are noticed.
These solutions shows a complex energy, describing a particle whose wavefunction is temporary localized, but quickly becomes ``unlocalized'' (free).
A system with this  property is said to be \emph{quasi-bound} state, or a \emph{resonance}.
The probability for these quasi-bound systems of staying localized will decrease exponentially over time, which follows through the following argument using a complex energy.

\newcommand{\axes}[6]{
  \draw[->] (#2, 0) -- (#3, 0) node[right] {#1};
  \draw[->] (0, #5) -- (0, #6) node[above] {#4};
}

\begin{figure}
  \subfloat[]{
    %\tikzset{external/remake next}
\tikzsetnextfilename{harmosc_potential}
    \begin{tikzpicture}[domain=0:4.5, samples=100]
      %\clip (-0.2, -5.1) rectangle (7, 1);
      \axes{$r$}{-0.1}{6}{$V$}{-5.1}{0.5}
      \draw plot (\x, 0.3*\x*\x - 4.8);
    \end{tikzpicture}
  }
  \subfloat[]{
    %\tikzset{external/remake next}
\tikzsetnextfilename{woods-saxon_potential}
    \begin{tikzpicture}[domain=0:6, samples=100]
      \draw[->] (-0.1, 0) -- (6, 0) node[right] {$r$};
      \draw[->] (0, -5.1) -- (0, 0.5) node[above] {$V$};
      \draw plot (\x,{-5/(1 + exp((\x - 2)/0.65))});
    \end{tikzpicture}
  }
  % \\
  % \subfloat[]{
  %   \label{fig:coloumb}
  %   \begin{tikzpicture}[domain=0.2:6, samples=100]
  %     \draw[->] (-0.1, 0) -- (6, 0) node[right] {$r$};
  %     \draw[->] (0, -5) -- (0, 0.5) node[above] {$V$};
  %     \draw plot (\x,{-1/\x});
  %   \end{tikzpicture}
  % }
  \caption{The Harmonic Oscillator (a) and Woods-Saxon (b) potentials.}
  \label{fig:potentials}
\end{figure}


The time-dependency of a bound, stationary state $\psi$ is
\begin{eq}
	\psi(t)
	= 
  \exp\p{-\frac{iE}{\hbar}t}\psi(0).
\end{eq}
With the energy $E$ real, the exponential factor is just a phase 
and the probability $|\psi(t)|^2$ is unchanged over time (hence the name
stationary). However, if we let the energy be complex
\begin{eq}
	E = E_0 - i\frac{\Gamma}{2},
\end{eq}
we get
\begin{eq}
  |\psi(t)|^2 
  =
  \absq{
    \exp\p{-\frac{iE_0}{\hbar} t} \exp\p{- \frac{\Gamma}{2\hbar} t} \psi(0)
  }
  =
  \exp\p{-\frac{\Gamma}{\hbar} t} \absq{\psi(0)}
\end{eq} 
which describes a resonant state with half-life 
$t_{1/2}=\hbar\ln 2/\Gamma$ and so-called \emph{width} $\Gamma$.

\todo{Is non-hermitian even important?}
It seems, then, that we need complex energies to describe resonant 
states. However, complex eigenvalues pose a problem in standard QM. 
This since observable quantities are regarded as real values 
and are described by \emph{Hermitian} operators. When working with 
complex eigenvalues one needs a \emph{non-Hermitian} formulation of 
the problem, which we encounter in \cref{cha:nhqm}.

The systems we have chosen to study using numerical calculations
are the nuclei of the  Helium isotopes \He{5} and \He{6}. We chose Helium  
because \He{4} is a very stable nucleus that can be treated 
as a single (alpha) particle. \He{5} and \He{6} are then modeled
as an alpha particle core with one or two valence neutrons, 
respectively. 

\todo{Rewrite to correctness.}
\He{5} has a known resonance with half-life $t_{1/2} = \SI{700e-24}{s}$,
which we use to fix our two-body basis. In addition to resonant states, \He{6} has a bound state because of the attraction between the valence neutrons. Our goal is to find these resonances using the bound state to verify the methods used.
\todo{Should we have results in the introduction.}
\todo{Tunneling is when $E<V\sub{max}$. This is $E>V\sub{max}$. But we can relate them!}
\todo{Where does width stuff go?}
% Heisenberg's uncertainty principle gives a relation between energy and time
% \begin{equation}
% \Delta E \Delta t \ge \frac{\hbar}{2}.
% \end{equation}
% Hence a state with finite life time must have an uncertainty in its energy spectrum, this is called the \emph{width} of a resonant state. It is this width that is measured in experiments.

The report is full of surprises. In \cref{cha:basis expansion} the mathematical foundation of the calculations in this thesis, basis expansion, is introduced.
The basis expansion method is then used in \cref{cha:he5} to study a loosely bound nuclear system, the \He{5} nucleus.
In \Cref{cha:nhqm} we use the Berggren basis to reproduce the resonance in \He{5}.

\Cref{cha:many-body} is an introduction to many-body theory, focusing 
on fermionic systems. 
The many-body theory is then utilized in \cref{cha:he6} 
to study a three-body problem, specifically the \He{6} nucleus.  
In \cref{cha:monte carlo} a Monte Carlo method for reducing the basis size is investigated. 
Finally, \cref{cha:outlook} discusses the work and suggest further avenues of inquiry.

\end{document}