\documentclass[../main/report.tex]{subfiles}
\begin{document}
\chapter{Introduction}

The properties of a quantum mechanical system are determined by its Hamiltonian. 
Particles in a potential well with infinitely high walls form localized bound states.
Such a system is a called a \emph{closed quantum system}, since the number of particles is conserved and the particles are localized in a finite region. 
The energy of a closed quantum system can only take on discrete values, as illustrated in \cref{fig:closed_quantum_system} with the harmonic oscillator potential.

An \emph{open quantum system}, on the other hand, portrayed in \cref{fig:open_quantum_system}, is neither closed nor completely free.
Here, the potential vanishes at infinity.
Particles can thus enter and exit the system and, consequently, there are unbound states.
Depending on the depth of the potential well of the open system, there can be a finite number of bound states.

In addition to the bound and unbound states, some open quantum systems harbor \emph{resonances}. 
These are \emph{quasi-bound} states that are neither bound nor unbound, but exhibit properties of both. 

They appear in the continuum, like unbound states, but are localized, like bound states.
However, the wavefunction of a resonance is only localized for a finite amount of time, as opposed to a bound, \emph{stationary} state that forever stays the same.
A resonance can be described as a \emph{quasi-stationary} state with a complex energy, as the following argument suggests.

\todo{Vet inte var/hur vi förklarar det exponentiella avtagandet. -behöver vi det?}

\begin{figure}
  \newcommand{\depth}{3}
  \subfloat[Closed quantum system]{
    \label{fig:closed_quantum_system}
    \begin{tikzpicture}[
      scale=1.6,
      domain=-2:2,
      samples=200,
      ]
      \draw[->] (-2, 2) -- (2, 2) node[above] {$r$};
      \draw[->] (0, 0) -- (0, 4) node[right] {$V$};
      \draw plot (\x, \x*\x);
      \foreach \y in {0.7, 1.4, ..., 4}
        \draw ($ ({-sqrt(\y)} , \y) $) 
           -- ($ ({ sqrt(\y)} , \y) $);
        \node[above] at (-0.04, 2.1) {Bound states};
    \end{tikzpicture}
  } 
  \subfloat[Free particle]{
    \label{fig:free_particle}
    \begin{tikzpicture}[
      scale=1.6
      ]
      \draw[->] (-2, 0) -- (2, 0) node[above] {$r$};
      \draw[->] (0, -2) -- (0, 2) node[right] {$V$};
      \draw[->, decorate, decoration={snake, post length=1mm}] 
        (-2, 1) -- (2, 1) node[above left, yshift=1mm] {Unbound};
    \end{tikzpicture}
  }
  \\
  \subfloat[Open quantum system]{
    \label{fig:open_quantum_system}
    \begin{tikzpicture}[
      scale=2,
      xscale=0.5, 
      domain=-6:6,
      samples=200,
      ]  
    \shade[top color=white, bottom color=black!70]
    (-6,0.5) rectangle +(12,0.05);
    \shade[bottom color=white, top color=black!70]
    (-6,0.5) rectangle +(12,-0.05)
	 node[midway, above right] {Resonance state};
      \draw[->] (-6, 0) -- (6, 0) node[above] {$r$};
      \draw[->] (0, -3.1) -- (0, 2) node[right] {$V$};
      \draw plot (\x,{-\depth*exp(- \x*\x/6)});
      \foreach \y in {-2.3, -1}
      \draw ($ ({sqrt(-6*ln(-\y/\depth))}, \y)$)
         -- ($ ({-sqrt(-6*ln(-\y/\depth))}, \y)$);
		 
         \draw ($ ({sqrt(-6*ln(1/\depth))}, -1)$)
            -- ($ ({-sqrt(-6*ln(1/\depth))}, -1)$)
			        node[midway, above left] {Bound state};
      \begin{scope}
        [->, decoration={snake, post length=1mm}, gray]
        \draw[decorate] (-6, 1) -- (6, 1) 
          node[black, above left] {Unbound scattering states};
        \draw[decorate] (-6, 1.4) -- (5.5, 1.4);
      \end{scope} 
      \draw[decorate, decoration={brace}] 
        (-6.1,0) -- +(0,2) node[midway, xshift=-0.4cm, rotate=90] {Continuum};
    \end{tikzpicture}
  }
  \caption{Three types of quantum systems: closed, completely free and open.}
  \label{fig:potentials}
\end{figure}

The state of a particle is described by its wavefunction $\psi$, which can be written as the product of a function of time and position
\begin{eq}
  \psi(t, \vec{r}) = \psi_t(t)\psi_{\vec{r}}(\vec{r}).
\end{eq}
The wavefunction evolves according to the Time-Dependent Schrödinger Equation (TDSE)
\begin{eq}
  \label{eq:schrödinger}
  i\hbar\ddt\ket\psi = H \ket\psi.
\end{eq}
An eigenstate of the Hamiltonian $H$ with energy $E$, i.e. a solution to the Time-Independent Schrödinger Equation (TISE)
\begin{eq}
  H \ket\psi = E \ket\psi
\end{eq}
has the simple time evolution
\begin{eq}
	\psi(t, \vec{r})
	= 
  \exp\p{-\frac{iE}{\hbar}t}\psi_t(0)\psi_{\vec{r}}(\vec{r}).
\end{eq}
With the energy $E$ real, the exponential factor is just a phase 
and the probability $|\psi(t, \vec{r})|^2$ of finding the particle at a given $\vec{r}$ is unchanged over time. 
However, if we let the energy be complex
\begin{eq}
	E = E_0 - i\frac{\Gamma}{2},
\end{eq}
we get
\begin{eq}
  |\psi(t, \vec{r})|^2 
  =
  \absq{
    \exp\p{-\frac{iE_0}{\hbar} t} 
    \exp\p{- \frac{\Gamma}{2\hbar} t} 
    \psi(0, \vec{r})
  }
  =
  \exp\p{-\frac{\Gamma}{\hbar} t} \absq{\psi(0, \vec{r})}
\end{eq} 
describing a resonance state decaying exponentially with half-life $t_{1/2}=\hbar\ln 2/\Gamma$. $\Gamma$ is called the \emph{width} of the resonance.

\todo{width/scattering argument?}

We want to use the simpler formalism of the TISE, as opposed to the more general TDSE, and we see that this is possible by letting the resonance have a complex energy.
However, complex eigenvalues pose a problem in standard quantum mechanics, where operators are postulated to be Hermitian.
Hermitian operators can only have real eigenvalues, and are thus insufficient for treating resonances.

\todo{aim?}
The aim of this thesis is to present methods for describing resonances using the TISE and a complex energy. 
To demonstrate the methods, we will study the simple nuclear systems \He{5} and \He{6}.
They are relevant because the ground state of \He{5} and the first excited state of both \He{5} and \He{6} are resonances. 
Despite the unbound nature of \He{5}, the ground state of \He{6} is bound, an example of a \emph{Borromean} system, named after the Borromean rings depicted on the cover of this thesis.

The thesis can conceptually be divided into two parts, the first covering resonances in a simple two-body problem and the second part covering the first steps toward more complicated many-body systems. 
In \cref{cha:basis_expansion} the \emph{basis expansion} method for solving the Schrödinger equation is introduced.
The basis expansion method is then used in \cref{cha:two-body} to study a loosely bound two-body nuclear system, the \He{5} nucleus.
In \Cref{cha:berggren} we use the Berggren basis to reproduce the resonance in \He{5}.

\Cref{cha:many-body} is an introduction to many-body theory, focusing on fermionic systems. 
The many-body theory is then utilized in \cref{cha:three-body} 
to study a three-body problem, specifically the \He{6} nucleus.  
In \cref{cha:monte_carlo} a Monte Carlo method for reducing computation time is investigated. 
Finally, \cref{cha:outlook} is an outlook discussing further development of the methods.

\end{document}