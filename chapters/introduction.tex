\emph{Quantum mechanics} (QM) is a cornerstone of modern physics, as it describes the world on the smallest of scales.
On this small level we observe a vast number of interesting phenomena, some of which in our ordinary large-scale world seem completely impossible.
One of these phenomena is called \emph{resonance} and in this report we will study these nuclear resonances using \emph{Non-Hermitian Quantum Mechanics} (NHQM).
%To study these resonances we will expand the hamiltonian in a \emph{Berggren basis}, which makes the hamiltonian non-hermitian, leading us into the world of \emph{Non-Hermitian Quantum Mechanics} (NHQM).

Resonances arise from scattering experiments where neutrons of certain positive energies is seen to have a drastically increased phase shift.
This can be interpreted as the neutron forming a temporary bound state that quickly decays.
These short-lived states are called \emph{quasi-stationary} states and are a property of resonances.

The connection between resonances and NHQM stems from the fact that the mean life-time of the system depends directly on the complex binding momentum.
\todo{the concept of 'complex binding momentum/energy' is probably unfamiliar to the reader -- explain/rephrase?}
To see this, we start by the well-known fact that the time-dependent solution to the Schrödinger equation for a bound state can be expressed as
\begin{eq}
	\psi(t)
	=e^{-iEt}\psi(0).
\end{eq}
We see that the exponential factor in front is just a phase and since the energy $E$ is real in ordinary (Hermitian) QM, the probability $|\Psi(t)|^2$ is unchanged over time.
The quasi-bound states on the other hand are observed to decay over time, which would lead us to believe that the probability instead would take the form of
\begin{eq}
  |\Psi(t)|^2 
  = 
  e^{-\frac{\Gamma}{\hbar}t} |\Psi(0)|^2.
\end{eq} 
To get this result we would need to have a complex energy with negative imaginary part.
This is where we have to enter the world of NHQM to be able to describe the physics.
In the above case we would have a complex energy written as
\begin{eq}
	E=E_0-\frac{i\Gamma}{2},
\end{eq}
where $E_0$ is the binding energy and $t_{1/2}=\hbar\ln 2/\Gamma$ is the half-life of the resonance. 


\todo{Should we talk something about tunneling here?}
They can be understood more easily by observing the potential of the nucleus.
The potentials for different nuclei look slightly different, but in all nuclei where resonances is observed we find a potential barrier.
Now combine this with the knowledge of QM and we can explain the previously unexpected behavior with another phenomenon: tunneling.
Tunneling says that when looking at a QM-system with a potential barrier a particle once found on one side of the barrier later has a probability of being found on the other.
This is what happens in the case of quasi-stationary states.


\todo{Talk about width of resonance state?}

\todo{NHQM is more opportunity than obstacle, IMO}
This new introduction of NHQM poses some trouble for us since the problem of standard, Hermition QM, is widely known among physicists\todo{Is there a problem with standard QM? I think this paragraph is a bit fuzzy}, but the NHQM way of thinking lacks an easy-to-process theoretical foundation.
In this report we describe how we addressed this issue by calculating the complex eigenvalues of Helium isotopes using Non-Hermitian Quantum Mechanics.

\todo{Put this Helium stuff somewhere smart.}
Helium is chosen since it is has a small nucleus\todo{or is it chosen because the alpha core is extremely stable and can easily be treated as 1 particle?} with several isotopes showing interesting behavior.
A small nucleus is preferable since it is easier to make calculations on and the interesting behaviors are stronger for small nuclei.
Another nucleus often choosen for this kind of calculations is Lithium, which displays similar behavior.
