\todo{Wow, QM is amaaazing, etc. Do we need to say that QM seems impossible? Shorten this?}
\emph{Quantum mechanics} (QM) is a cornerstone of modern physics, as it describes the world on the smallest of scales.
On this small level we observe a number of interesting phenomena, some of which in our ordinary large-scale world seem completely impossible.
One of these phenomena is called \emph{resonance} and is the focus of this report.
%To study these resonances we will expand the hamiltonian in a \emph{Berggren basis}, which makes the hamiltonian non-hermitian, leading us into the world of \emph{Non-Hermitian Quantum Mechanics} (NHQM).

Resonances, or \emph{quasi-stationary states}, are states in which a nucleon is bound to a nucleus, but only for a limited time. 
The probability of the nucleon remaining bound will decay exponentially over time. 
This can be explained as the state having a complex energy by the following argument:

The time-dependency of a stationary state $\psi$ is, by the Shrödinger equation,
\begin{eq}
	\psi(t)
	=e^{-\frac{iE}{\hbar}t}\psi(0).
\end{eq}
With the energy $E$ real, the exponential factor in front is just a phase and the probability $|\psi(t)|^2$ is unchanged over time. However, if we let the energy be complex
\begin{eq}
	E = E_0 - i\frac{\Gamma}{2},
\end{eq}
we get
\begin{eq}
  |\psi(t)|^2 
  =
  \absq{
    e^{-\frac{iE_0}{\hbar} t} e^{- \frac{\Gamma}{2\hbar} t} \psi(0)
  }
  =
  e^{-\frac{\Gamma}{\hbar} t} \absq{\psi(0)}
\end{eq} 
which describes a resonant state with half-life $t_{1/2}=\hbar\ln 2/\Gamma$.

It seems, then, that we need complex energies to describe resonant states. However, complex eigenvalues pose a problem in standard QM. 
This is because observable quantities are regarded as real values and are described by \emph{Hermitian} operators. 
When working with complex eigenvalues one needs a \emph{non-Hermitian} formulation of the problem. This leads us to \emph{non-Hermitian quantum mechanics} (NHQM), which is what we use to find resonances in this report.

\todo{Tunneling is when $E<V\sub{max}$. This is $E>V\sub{max}$. But we can relate them!}
Resonances can be understood more easily by observing the potential of the nucleus.
The potentials for different nuclei look slightly different, but in all nuclei where resonances is observed we find a potential barrier.
Now combine this with the knowledge of QM and we can explain the previously unexpected behavior with another phenomenon: tunneling.
Tunneling says that when looking at a QM-system with a potential barrier a particle once found on one side of the barrier later has a probability of being found on the other.
This is what happens in the case of quasi-stationary states.

This report will study resonances in the Helium isopes \He{5} and \He{6}.
Helium is chosen since \He{4} is a very stable nucleus and thus can be treated as a single (alpha) particle.
\He{5} has a known resonance with half-life $t_{1/2} = \SI{700e-24}{s}$.
\He{6} is a so-called Boromean nucleus which, because of the interaction between the valence neutrons, has a bound state.


\todo{Talk about width of resonance state?}

\todo{Reading guide? Explain the structure of the report}

\section{Tools}

All calculations were made in the Python programming language using the libraries NumPy and SciPy. We used the matplotlib library to make the figures. The code was managed using Git and is available at \url{https://github.com/pnutus/NHQM}.


\section{Reading Guide}
In Chapter 2 we introduce the concept of basis expansion. This is a mathematical tool that lets us rewrite the Schrödinger equation as a matrix equation. A short introduction to the theory is included, followed by an example where we are using the method to find the hydrogen atom ground state.

In Chapter 3 we introduce the first problem that will be studied in the report: the \He{5} nucleus. We describe the model that is used, and solve the equation. This is followed by a discussion of the solutions.

In Chapter 4 we extend our methods to the complex plane, using this to find the resonance in \He{5}.

In Chapter 5 an introduction to many-body theory is included, focusing on fermionic systems.  

[WIP] In Chapter 6 the solutions of \He{5} is used as single-particle basis to solve the \He{6} problem. 
