\documentclass[../main/report.tex]{subfiles}
\begin{document}

\chapter{The Nuclear Three-Body Problem}
\label{cha:three-body}

\todo{interaction/potential potential confusion SPILL}
\todo{potential depth, interaction strength OLA}

Equipped with the many-body theory of \cref{cha:many-body} we are now ready to study an open three-body system.
The natural way to proceed is to add another neutron to our two-body \He{5} system and form \He{6}, seen as an alpha particle core with two valence neutrons ($\alpha + n + n$).
We now have to take into account the attractive two-body interaction between the neutrons, which allows for bound states in \He{6} where there were none in \He{5}.
Nuclei with this property are called \emph{Borromean}, after the Borromean rings (\cref{fig:borromean}). 
The Borromean rings are three rings interlocked in such a way that, if any one ring is removed, the other two will fall apart, as is the case with the core and the neutrons.

\begin{figure}[h]
  \newcommand{\circdist}{1.2}
  \newcommand{\circrad}{2}
  \centering
  \begin{tikzpicture}[even odd rule]
    \foreach \angle/\colour in {90/red,-30/blue,210/green}
      \draw [fill=\colour] 
        (\angle:\circdist) circle (\circrad) circle (\circrad+0.5);

    \begin{scope}
    \clip (-30:\circdist) circle (1);
    \draw [fill=red] (90:\circdist) circle (\circrad) circle (\circrad+0.5);
    \end{scope}

    \clip (90:\circdist) ++ (180:\circrad) circle (1);
    \draw [fill=red] (90:\circdist) circle (\circrad) circle (\circrad+0.5);
  \end{tikzpicture}
  \caption{The Borromean rings}
  \label{fig:borromean}
\end{figure}

\section{The \He{6} Nucleus}
 
We treat the \He{6} system as a combination of two identical \He{5} systems of core and neutron, i.e. two ``one-particle'' systems with the same reduced mass $\mu$ as in \cref{cha:two-body}.
This means that, even though we are studying a three-body system, we will use the same reduced mass for the neutrons as in the two-body case.

The validity of this approach is not immediately obvious, but it has been shown \cite{suzuki} that it is valid if the neutrons have the same parity. 
Since parity depends solely on $l$, and we only consider neutrons in the $p$ orbitals ($l=1$), the approach is valid.

The Hamiltonian takes on the form
\begin{eq}
  \hat{H} = \hat{H}_1 + \hat{V}\sub{res}
\end{eq}
where $\hat{H}_1$ is the Hamiltonian of the two-body problem and $\hat{V}\sub{res}$ is the residual neutron-neutron interaction.
To compute the Hamiltonian matrix we will use the eigensolutions $\ket{E_i}$ from the \He{5} problem as single-particle basis states. 
We can do this because of the Berggren completeness relation \cref{eq:berggren_completeness_relation}, proving the completeness of this basis. 
Using the Berggren basis in this way to build more complex nuclei is referred to as the \emph{Gamow Shell Model}.

The neutrons are fermions, so we form antisymmetric two-particle Fock states $\ket{E_i, E_j}$, according to the theory in \cref{cha:many-body}.
Because our basis consists of the eigenstates of $\hat{H}_1$, we can use \cref{eq:one-body_matrix_elements,eq:two-body_matrix_elements} to write the matrix elements as
\begin{eq}
  \bra{ab} \hat{H} \ket{cd} 
  = 
  \delta_{ac}\delta_{bd}(E_a + E_b)
  +
  \bra{ab} V\sub{res} \ket{cd},
\end{eq}
where $E_\alpha$ are the energy eigenvalues of the two-body problem. 
Finally, the states are coupled according to \cref{sec:coupling} to reduce the size of the Hamiltonian matrix.
The coupled matrix elements are
\begin{eq}
  \label{eq:coupled_matrix_elements}
  \bra{ab; J} H \ket{cd; J}
  & =
  \frac{1 + (-1)^J\delta_{ab}}{1 + \delta_{ab}}
  \delta_{ac}\delta_{bd}(E_a + E_b)
  +
  \bra{ab; J} V\sub{res} \ket{cd; J}.
\end{eq}


\section{Neutron-Neutron Interaction}

The interaction between nucleons is complex and there is no known analytical expression for the potential. It arises from the strong force between the quarks that make up the nucleons. The strong force is well known at high energies but less so at nuclear energy levels.
The study of this interaction at the nuclear level is therefore an active field of research \cite{edge}.

A convenient approximation is a \emph{separable} interaction,
\begin{eq}
  V(r_1, r_2) = v(r_1)v(r_2),
\end{eq}
a product of functions of $r_1$ and $r_2$, the radii of each neutron. The interaction can be trivially separable as above or might require a \emph{multipole expansion}, covered in \cite{suhonen}. We study two different types of separable two-body interactions, a gaussian interaction and a surface delta interaction.

\subsection{Gaussian Interaction}


Initially we investigate a trivially separable gaussian interaction
\begin{eq}
  V(r_1, r_2) 
  = 
  -V\sub{GI} \exp\p{-\frac{r_1^2}{R^2}} \exp\p{-\frac{r_2^2}{R^2}}.
\end{eq}
The range $R$ and strength $V\sub{GI}$ are fitting parameters.

In \cref{fig:gaussian} we see that the potential in a rough sense satisfies the expected properties of the interaction. If at least one neutron is far from the core, the other will experience little attraction. If both neutrons are in the vicinity of the core, they will experience a stronger attraction.


\begin{figure}[t]
  \centering
    \begin{tikzpicture}
      \begin{axis}[
        xlabel = $r_1$,
        ylabel = V,
        axis x line = middle,
        axis y line = left,
        xtick=\empty,
        ytick={0},
        yticklabels={0},
        domain=0:3,
        ymin=-1.2, ymax=0.5,
        every axis y label/.style={
          at = {(current axis.above origin)},
          anchor = north west,
        },
		after end axis/.code={
		               \draw[anchor=west, dashed] (axis cs:0,-1) -- (axis cs:1,-1) 
					   node [] {$\displaystyle V\sub{GI}\exp\p{-\frac{r_2^2}{R^2}}$};
		             },
        ]
        \addplot[thick, black] {-e^(-x^2)};		
      \end{axis}
    \end{tikzpicture}
  \caption{The Gaussian interaction potential as seen by one of the neutrons. The position of the other neutron determines the depth of the potential well.}
  \label{fig:gaussian}
\end{figure}
Because the potential is separable
\begin{eq}
  V(r_1, r_2) 
  = 
  -V\sub{GI} V\sub{sep}(r_1) V\sub{sep}(r_2),
  \quad
  V\sub{sep}(r) = e^{- \frac{r^2}{R^2}}
\end{eq}
we can write the two-body matrix elements as
\begin{eq}
  \pbra{ab} V \pket{cd}
  =
  -V\sub{GI} 
  \bra{a} V\sub{sep} \ket{c} 
  \bra{b} V\sub{sep} \ket{d}
\end{eq}
which in the coupled scheme becomes
\begin{eq}
  & \bra{ab; J} V \ket{cd; J}
  = \\
  & -V\sub{GI} 
  \N_{ab} \N_{cd}
  \p{
    \bra{a} V\sub{sep} \ket{c} 
    \bra{b} V\sub{sep} \ket{d}
    - (-1)^{j_1 + j_2 + J}
    \bra{a} V\sub{sep} \ket{d} 
    \bra{b} V\sub{sep} \ket{c}
  }.
\end{eq}
The $\bra{a} V\sub{sep}(r) \ket{c}$ are calculated by expanding $V\sub{sep}$ in the same basis as the sp states. 
In the momentum basis this results in
\begin{eq}
  \bra{a} V\sub{sep} \ket{c}
  &=
  \sum_i \sqrt{w_i}k_i \phi'_a(k_i) \sum_j \sqrt{w_j}k_j \phi'_c(k_j) V\sub{sep}(k_i, k_j) 
  \\ &=
  \sum_i a_i \sum_j V_{ij} c_j  ,
\end{eq}
where
\begin{eq}
  a_i = \sqrt{w_i}k_i\phi'_a(k_i),& 
  \hspace{1cm}
  c_j = \sqrt{w_j}k_j\phi'_c(k_j)
  \\
  V_{ij} 
  = 
  V\sub{sep}(k_i, k_j) 
  = 
  \frac{2}{\pi} &\fint[0][\inf]{r} r^2 V(r) j_l(k_i r)j_l(k_j r).
\end{eq}
We see that the matrix elements $V_{ij}$ of $V\sub{sep}$ are evaluated as described in \cref{cha:basis_expansion} for the momentum Schrödinger equation.


\subsection{Surface Delta Interaction}

Another possible interaction is the surface delta interaction (SDI)
\begin{eq}
  V(\vec{r}_1, \vec{r}_2) 
  = 
  -V\sub{SDI} 
  \delta(\vec{r}_1 - \vec{r}_2) 
  \delta(r_2 - r_0)
\end{eq}
where $V\sub{SDI}$ is the strength and $r_0$ is the range, chosen to have the same value as the range of the Woods-Saxon potential.

The short-range strong force is thus approximated as a point interaction. 
The physical motivation of the $\delta(r-r_0)$ term is the experimental fact that the scattering cross-section between neutrons is inversely proportional to their kinetic energy.
Since the kinetic energy has a minimum near the surface of the nucleus (at the range $r_0$ of the Woods-Saxon potential), we can approximate the interaction as focused entirely in that shell.

The SDI can be expanded into separable multipole radial components
\begin{eq}
  v_l(r) = \frac{\delta(r-r_0)}{r},
\end{eq}
and with a complicated calculation (see \cite{suhonen}) one reaches the following expression for the coupled scheme matrix elements:
\todo{definiera $\N$ i mb theory och här, mention wigner 3j OLA}
\begin{eq}
  \bra{ab; J} V \ket{cd; J} 
  = 
  & - K_{abcd} \N_{ab}(J) \N_{cd}(J) 
  (-1)^{l_a + l_c + j_b + j_d}
  \\ & \times
  \b{1 + (-1)^{l_a + l_b + l_c + l_d}}
  \b{1 + (-1)^{l_c + l_d + J}}
  \\ & \times
  \widehat{j_a} \widehat{j_b} \widehat{j_c} \widehat{j_d}
  \begin{pmatrix}
    j_a & j_b & J \\
    \frac{1}{2} & -\frac{1}{2} & 0
  \end{pmatrix}
  \begin{pmatrix} 
    j_c & j_d & J \\
    \frac{1}{2} & -\frac{1}{2} & 0
  \end{pmatrix}
  \\
  \N_{\alpha\beta} = & \frac{\sqrt{1+(-1)^{J}\delta_{\alpha\beta}}}{1+\delta_{\alpha\beta}}
  \\
  K_{abcd} 
  = &
  - \frac{V_0 r_0^2}{16\pi}
  \psi_a(r_0) \psi_b(r_0) \psi_c(r_0) \psi_d(r_0)
  \\
  \widehat{j_\alpha} = & \sqrt{2j_\alpha + 1}
\end{eq}
$\psi_\alpha(r)$ being the radial wavefunction and the $\begin{pmatrix}
    j_1 & j_2 & j_3 \\
    m_1 & m_2 & m_3
  \end{pmatrix}$ are the Wigner 3j symbols.


\section{The \He{6} Solutions}
%TUNL's förslag på hur dom vill bli citerade: TUNL Nuclear Data Evaluation Project, "Energy Level Diagram, \He{6}". Available WWW: http://www.tunl.duke.edu/nucldata/HTML/A=6/06_01_2002.pdf 
%\He{6} experimental:
%0^+ E = -0.975
%2^+ E = 0.8 - i0.55

We solve the \He{6} Schrödinger equation in the coupled scheme, using \cref{eq:coupled_matrix_elements} for the matrix elements. We first consider the Gaussian interaction, followed by the SDI.
The parameters of our interactions are fitted to the known $0^+$ ground state of \He{6}. Using the fitted paramaters, we solve for the excited $2^+$ state -- a resonance. 
To find this resonance we have to use the Berggren basis, including the resonance of \He{5} in the sp basis. 

\subsection{Identifying the Resonance}
The solutions $\ket{\psi}$ expanded in the coupled basis are written 
\begin{eq}
\ket{\psi} = \sum_{E_1 E_2} \Psi(E_1, E_2) \ket{E_1 E_2; J}
\end{eq}
where the coefficients $\Psi(E_1, E_2)$ can be considered the ``wavefunction''. These are not easily visualized because of the two variables,
but they can still be used to obtain information about the solutions. 
Consider the component 
\begin{eq}
\Psi(E\sub{res}, E\sub{res}) = \braket{E\sub{res}^2;J}{\psi},
\end{eq}
being the overlap between a solution and the two-particle state corresponding to both neutrons being in the \He{5} resonance state (here $E\sub{res}$ denotes the energy of the \He{5} $p_{3/2}$ resonance). 
It was found that the $2^+$ resonance in \He{6} predominantly consists of this component, allowing for an effective way to single out the relevant solution from a large set of scattering states.

\subsection{Using the Gaussian Interaction}
We begin by setting the range $R$ of the Gaussian potential to the same value as the Woods-Saxon range, \SI{2}{fm}. The strength $V\sub{GI}$ is fitted to reproduce the \He{6} ground state. Because the Gaussian interaction is spherically symmetric it is degenerate in $J$, and there will be no distinction between the ground and excited states. 

Decreasing the range to $R = \SI{1}{fm}$, we find a wide resonace ($\Gamma > \SI{6}{MeV}$), but this requires large values of $k$ to be included in the basis and thus more scattering states are required to reach convergence. Although it might be possible to vary $R$ and $V_0$ to obtain a resonance closer to experimental data, we do not pursue this method further because of computation times quickly getting out of hand.

\subsection{Using the Surface Delta Interaction}
The SDI has only one parameter, the strength $V\sub{SDI}$. 
However, the solutions will also always depend on the truncation parameter $k\sub{max}$ of the momentum basis (an artifact of the delta functions). 
We choose the smallest possible $k\sub{max}$ for which the \He{5} solutions converge, $k\sub{max} = \SI{2.5}{fm^{-1}}$.
The strength is then fitted to the ground state of \He{6} as before.
The result is presented in \cref{tab:He6_results}.

\Cref{fig:he6_momenta} shows all $2^+$ solutions $k=\sqrt{2\mu E}/\hbar$ obtained, plotted in the complex momentum plane.
Most of the solutions correspond to energies on the form
\begin{eq}
E = E_1 + E_2 \quad \text{or} \quad k=\sqrt{k_1^2 + k_2^2}
\end{eq} 
where $E_1$, $E_2$ (and thus $k_1$, $k_2$) are the energies of two \He{5} eigenstates. These solutions can be interpreted as two unbound particles, barely interacting with each other. This also explains the pattern that is seen -- the solutions are combinations of contour points that was used in the solution of \He{5} to generate the sp basis. 

\begin{figure}[H]
	\centering

      \includegraphics[]{../figures/he6_momenta/he6mom.pdf}

   \caption{Momentum solutions for \He{6}, $2^+$. The resonance is located at $k=\SI{0.241-0.037i}{fm^{-1}}$. We have used the fitted Woods-Saxon parameters $V_0 = \SI{47.05}{MeV}$, $V\sub{so} = \SI{-7.04}{MeV}$ and $k\sub{max} = \SI{2.5}{fm^{-1}}$. The SDI interaction was used with $V_0=\SI{998}{MeV}$. }
\label{fig:he6_momenta}  
\end{figure}

\subsection{\He{6} in the Real Momentum Basis}
While it is required for the \He{5} resonance to be included in the basis to find the excited $2^+$ state, the ground state can be found without it. For a purely real contour, with all other parameters staying the same, a bound state with energy \SI{-0.976}{MeV} is found -- almost identical to the result obtained with the Berggren basis.

\begin{table}
\caption{Experimental \He{6} resonance data\cite{tunl} and computed values 
         with fitted Woods-Saxon parameters $V_0 = 47.05$ and $V\sub{so}=-7.04$. For the SDI, we used interaction strength $V_0 = \SI{998}{MeV}$ and range $r_0 = 2$. 
         The Gaussian interaction used parameters $V_0 = \SI{99}{MeV}$ and $r_0 = 2$. 24 points were used on the contour for each basis. All values are in \si{MeV}.}
\label{tab:He6_results}
\begin{center}
\begin{tabular}{r S S S}
  \toprule
$J^\pi$ & {Experiment} & {SDI} & {Gaussian} \\ 
\midrule
 $0^+$  &   -0.975  &  -0.98    & -0.97  \\
 $2^+$  &  0.8-0.55i & 1.47-0.47i & -0.97 \\
 \bottomrule
\end{tabular}
\end{center}
\end{table}




\end{document}