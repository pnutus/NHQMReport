\documentclass[../main/report.tex]{subfiles}
\begin{document}

\chapter{The Nuclear Three-Body Problem}
\label{cha:three-body}

Equipped with the many-body theory of \cref{cha:many-body} we are now ready to study an open three-body system.
The natural way to proceed is to add another neutron to our two-body \He{5} system and end up with \He{6}, seen as an alpha particle core with two valence neutrons ($\alpha + n + n$).
We now have to take into account the attractive two-body interaction between the neutrons, which allows for bound states in \He{6} where there were none in \He{5}.
Nuclei with this property are called \emph{Borromean}, after the Borromean rings (\cref{fig:borromean}). 
The Borromean rings are three rings interlocked in such a way that, if any one ring is removed, the other two will fall apart, as is the case with the core and the neutrons.

\begin{figure}[h]
  \newcommand{\circdist}{1.2}
  \newcommand{\circrad}{2}
  \centering
  \begin{tikzpicture}[even odd rule]
    \foreach \angle/\colour in {90/red,-30/blue,210/green}
      \fill [\colour] (\angle:\circdist) circle (\circrad) circle (\circrad+0.5);

    \begin{scope}
    \clip (-30:\circdist) circle (1);
    \fill [red] (90:\circdist) circle (\circrad) circle (\circrad+0.5);
    \end{scope}

    \clip (90:\circdist) ++ (180:\circrad) circle (1);
    \fill [red] (90:\circdist) circle (\circrad) circle (\circrad+0.5);
  \end{tikzpicture}
  \caption{The Borromean rings}
  \label{fig:borromean}
\end{figure}

\section{Section title?}
\todo{section title?}

We see the three-body system as a combination of two identical two-body systems of core and neutron, i.e. two ``one-particle'' systems with the same reduced mass as in \cref{cha:two-body}.
The validity of this approach is not immediately obvious, but 
if the two neutron-core pairs have like parity there will be no recoil of the core \cite{suzuki}. 
The parity depends solely on $l$ and we only consider neutrons in the $p$ orbitals ($l=1$), so the approach is valid.

The Hamiltonian takes on the form
\begin{eq}
  \hat{H} = \hat{H}_1 + \hat{V}\sub{res}
\end{eq}
where $\hat{H}_1$ is the Hamiltonian of the two-body problem and $\hat{V}\sub{res}$ is the residual neutron-neutron interaction.
To compute the Hamiltonian matrix we will use the eigensolutions $\ket{\psi_i}$ from the previous two-body problem as single-particle basis states. 
The neutron-core pairs are fermions, so we form antisymmetric two-particle Fock states $\ket{\psi_i,\, \psi_j}$.
Because our basis consists of the eigenstates of $\hat{H}_1$, we can use \cref{eq:one-body_matrix_elements,eq:two-body_matrix_elements} to write the matrix elements as
\begin{eq}
  \bra{ab} \hat{H} \ket{cd} 
  = 
  \delta_{ac}\delta_{bd}(E_a + E_b)
  +
  \bra{ab} V\sub{res} \ket{cd},
\end{eq}
where $E_\alpha$ are the energy eigenvalues of the two-body problem. 
Finally, the states are coupled according to \cref{sec:coupling} to reduce the size of the Hamiltonian matrix.
The coupled matrix elements are
\begin{eq}
  \bra{ab; J} H \ket{cd; J}
  & =
  \frac{1 + (-1)^J\delta_{ab}}{1 + \delta_{ab}}
  \delta_{ac}\delta_{bd}(E_a + E_b)
  +
  \bra{ab; J} V\sub{res} \ket{ab; J}.
\end{eq}


\section{Neutron-Neutron Interaction}

\todo{maybe refer to some article that covers this topic and then conclude that we'd rather use a dumbed-down approximation}
The interaction between nucleons is complex and there is no known analytical expression for the potential. It arises from the strong interaction between the quarks that make up the nucleons, which is well known at high energies (\si{TeV}),
\todo{how high energies?}
but less so at nuclear energy levels (\si{MeV}).
The study of this interaction is an active field of research \cite{living on the edge?}.
\todo[inline]{mention ab initio?}

A commonly used approximation is an interaction on the form
\begin{eq}
  V = V(\vec{r}_1 - \vec{r}_2).
\end{eq}
To make the calculations more managable, a \emph{separable} interaction can be chosen. An arbitrary function $V = V(\vec{r}_1 - \vec{r}_2)$ can be expanded in a \emph{multipole expansion},
\todo[inline]{what conj?}
\begin{eq}
  V(\vec{r}_1 - \vec{r}_2) 
  = 
  \sum_{lm} v_l(r_1, r_2) 
  Y_l^m(\Omega_{r_1})\conj{Y_l^m}(\Omega_{r_2})
\end{eq}
using the symmetry around the axis connecting the two particles. A separable interaction is one where 
\begin{eq}
  V_l(r_1, r_2) = v_l(r_1)v_l(r_2)
\end{eq}
We study two different types of separable two-body interactions, a trivially separable gaussian interaction and a surface delta interaction.


\subsection{Gaussian Interaction}
Initially we investigate the simplest form of separable interaction, a product of two functions of $r_1$ and $r_2$, the distance from each neutron to the core
\begin{eq}
  V(r_1, r_2) 
  = 
  -V\sub{GI} \exp\p{-\frac{r_1^2}{R^2}} \exp\p{-\frac{r_2^2}{R^2}}.
\end{eq}
The range $R$ and strength $V\sub{GI}$ are fitting parameters.

In \cref{fig:gaussian} we see that the potential in a rough sense satisfies the expected properties of the interaction. If at least one neutron is far from the core, the other will experience little attraction. If both neutrons are in the vicinity of the core, they will experience a stronger attraction.


\begin{figure}[h]
  \centering
    \begin{tikzpicture}
      \begin{axis}[
        xlabel = $r_1$,
        ylabel = V,
        axis x line = middle,
        axis y line = left,
        xtick=\empty,
        ytick={0, -1},
        yticklabels={0, $\displaystyle V\sub{GI}\exp\p{-\frac{r_2^2}{R^2}}$},
        domain=0:3,
        ymin=-1.2, ymax=0.5,
        every axis y label/.style={
          at = {(current axis.above origin)},
          anchor = north west,
        },
        ]
        \addplot[black] {-e^(-x^2)};
      \end{axis}
    \end{tikzpicture}
  \caption{The Gaussian interaction potential as seen by one of the neutrons. The position of the other neutron determines the depth of the potential well.}
  \label{fig:gaussian}
\end{figure}

Because the potential is separable
\begin{eq}
  V(r_1, r_2) 
  = 
  -V\sub{GI} V\sub{sep}(r_1) V\sub{sep}(r_2),
  \quad
  V\sub{sep}(r) = e^{- \frac{r^2}{R^2}}
\end{eq}
we can write the two-body matrix elements as
\begin{eq}
  \pbra{ab} V \pket{cd}
  =
  -V\sub{GI} 
  \bra{a} V\sub{sep} \ket{c} 
  \bra{b} V\sub{sep} \ket{d}
\end{eq}
which in the coupled scheme becomes
\begin{eq}
  & \bra{ab; J} V \ket{cd; J}
  = \\
  & -V\sub{GI} 
  \N_{ab} \N_{cd}
  \p{
    \bra{a} V\sub{sep} \ket{c} 
    \bra{b} V\sub{sep} \ket{d}
    - (-1)^{j_1 + j_2 + J}
    \bra{a} V\sub{sep} \ket{d} 
    \bra{b} V\sub{sep} \ket{c}
  }.
\end{eq}
The $\bra{a} V\sub{sep}(r) \ket{c}$ are calculated by expanding $V\sub{sep}$ in the same basis as the sp states. In the plane wave basis this is
\begin{eq}
  \bra{a} V\sub{sep} \ket{c}
  =
  \sum_i \sqrt{w_i}k_i \phi'_a(k_i) \sum_j \sqrt{w_j}k_j \phi'_b(k_j) V\sub{sep}(k_i, k_j),
\end{eq}
with
\begin{eq}
  V\sub{sep}(k_i, k_j) 
  = 
  \frac{2}{\pi} \fint[0][\inf]{r} r^2 V(r) j_l(k_i r)j_l(k_j r),
\end{eq}
as before.
\todo{numerical considerations}

\subsection{Surface Delta Interaction}
Another possible separable interaction is the surface delta interaction (SDI)
\begin{eq}
  V(\vec{r}_1, \vec{r}_2) 
  = 
  -V\sub{SDI} 
  \delta(\vec{r}_1 - \vec{r}_2) 
  \delta(r_2 - r_0)
\end{eq}
where $V\sub{SDI}$ is the strength and $r_0$ is the range, chosen to have the same value as the range of the Woods-Saxon potential. 

The short-range strong force is thus approximated as a point interaction. 
The physical motivation of the $\delta(r-r_0)$ term is the experimental fact that the scattering cross-section between neutrons is inversely proportional to their kinetic energy.
Since the kinetic energy has a minimum near the surface of the nucleus (at the range $r_0$ of the Woods-Saxon potential), we can approximate the interaction as focused entirely in that shell. 
\todo{maybe a citation here?}

The SDI has multipole radial components
\begin{eq}
  v_l(r) = \frac{\delta(r-r_0)}{r},
\end{eq}
and with a complicated calculation (see \cite{suhonen}) one reaches the following expression for the coupled scheme matrix elements
\begin{eq}
  \bra{ab; J} V \ket{cd; J} 
  = 
  & - K_{abcd} \N_{ab} \N_{cd} 
  (-1)^{l_a + l_c + j_b + j_d}
  \\ & \times
  \b{1 + (-1)^{l_a + l_b + l_c + l_d}}
  \b{1 + (-1)^{l_c + l_d + J}}
  \\ & \times
  \widehat{j_a} \widehat{j_b} \widehat{j_c} \widehat{j_d}
  \begin{pmatrix}
    j_a & j_b & J \\
    \frac{1}{2} & -\frac{1}{2} & 0
  \end{pmatrix}
  \begin{pmatrix}
    j_c & j_d & J \\
    \frac{1}{2} & -\frac{1}{2} & 0
  \end{pmatrix}.
\end{eq}
with $\widehat{j_\alpha} = \sqrt{2j_\alpha + 1}$ and
\begin{eq}
  K_{abcd} 
  = 
  - \frac{V_0 r_0^2}{16\pi}
  \psi_a(r_0) \psi_b(r_0) \psi_c(r_0) \psi_d(r_0),
\end{eq}
$\psi_\alpha(r)$ being the radial wavefunction.

\section{The \He{6} Solutions}

\end{document}