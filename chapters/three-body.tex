\documentclass[../main/report.tex]{subfiles}
\begin{document}

\chapter{The Nuclear Three-Body Problem}
\label{cha:three-body}

\todo{interaction/potential potential confusion SPILL}
\todo{potential depth, interaction strength OLA}

Equipped with the many-body theory of \cref{cha:many-body} we are now ready to study an open three-body system.
The natural way to proceed is to add another neutron to our two-body \He{5} system and form \He{6}, seen as an alpha particle core with two valence neutrons ($\alpha + n + n$).
We now have to take into account the attractive two-body interaction between the neutrons, which allows for bound states in \He{6} where there were none in \He{5}.
Nuclei with this property are called \emph{Borromean}, after the Borromean rings (\cref{fig:borromean}). 
The Borromean rings are three rings interlocked in such a way that, if any one ring is removed, the other two will fall apart, as is the case with the core and the neutrons.

\begin{figure}[h]
  \newcommand{\circdist}{1.2}
  \newcommand{\circrad}{2}
  \centering
  \begin{tikzpicture}[even odd rule]
    \foreach \angle/\colour in {90/red,-30/blue,210/green}
      \draw [fill=\colour] 
        (\angle:\circdist) circle (\circrad) circle (\circrad+0.5);

    \begin{scope}
    \clip (-30:\circdist) circle (1);
    \draw [fill=red] (90:\circdist) circle (\circrad) circle (\circrad+0.5);
    \end{scope}

    \clip (90:\circdist) ++ (180:\circrad) circle (1);
    \draw [fill=red] (90:\circdist) circle (\circrad) circle (\circrad+0.5);
  \end{tikzpicture}
  \caption{The Borromean rings}
  \label{fig:borromean}
\end{figure}

\section{The \He{6} Nucleus}
\todo{He6 approach to writing PONTUS}

We treat the three-body system as a combination of two identical two-body systems of core and neutron, i.e. two ``one-particle'' systems with the same reduced mass $\mu$ as in \cref{cha:two-body}.
The validity of this approach is not immediately obvious, but 
if the two neutrons have like parity there will be no recoil of the core \cite{suzuki}. 
The parity depends solely on $l$ and we only consider neutrons in the $p$ orbitals ($l=1$), so the approach is valid.

The Hamiltonian takes on the form
\begin{eq}
  \hat{H} = \hat{H}_1 + \hat{V}\sub{res}
\end{eq}
where $\hat{H}_1$ is the Hamiltonian of the two-body problem and $\hat{V}\sub{res}$ is the residual neutron-neutron interaction.
To compute the Hamiltonian matrix we will use the eigensolutions $\ket{\psi_i}$ from the previous two-body problem as single-particle basis states. 
The neutron-core pairs are fermions, so we form antisymmetric two-particle Fock states $\ket{\psi_i, \psi_j}$.
Because our basis consists of the eigenstates of $\hat{H}_1$, we can use \cref{eq:one-body_matrix_elements,eq:two-body_matrix_elements} to write the matrix elements as
\begin{eq}
  \bra{ab} \hat{H} \ket{cd} 
  = 
  \delta_{ac}\delta_{bd}(E_a + E_b)
  +
  \bra{ab} V\sub{res} \ket{cd},
\end{eq}
where $E_\alpha$ are the energy eigenvalues of the two-body problem. 
Finally, the states are coupled according to \cref{sec:coupling} to reduce the size of the Hamiltonian matrix.
The coupled matrix elements are
\begin{eq}
  \bra{ab; J} H \ket{cd; J}
  & =
  \frac{1 + (-1)^J\delta_{ab}}{1 + \delta_{ab}}
  \delta_{ac}\delta_{bd}(E_a + E_b)
  +
  \bra{ab; J} V\sub{res} \ket{cd; J}.
\end{eq}


\section{Neutron-Neutron Interaction}

The interaction between nucleons is complex and there is no known analytical expression for the potential \cite{suhonen}. It arises from the strong force between the quarks that make up the nucleons, which is well known at high energies (\si{TeV}),
\todo{how high energies? OLA}
but less so at nuclear energy levels (\si{MeV}).
The study of this interaction is an active field of research \cite{living_on_the_edge}.

A commonly used approximation is a \emph{separable} interaction 
\begin{eq}
  V = V(|\vec{r}_1 - \vec{r}_2|),
\end{eq}
depending only on the distance between the nucleons. Using the symmetry around the axis connecting the two particles, the interaction can be expanded in a \emph{multipole expansion},
\todo[inline]{what conj?}
\begin{eq}
  V(|\vec{r}_1 - \vec{r}_2|) 
  & = 
  \sum_{lm} V_l(r_1, r_2) 
  Y_l^m(\Omega_{r_1})\conj{Y_l^m(\Omega_{r_2})}
  \\
  V_l(r_1, r_2) 
  & = 
  2\pi \fint[-1][1]{(\cos{\theta_{12}})} 
  P_l(\cos{\theta_{12}}) V(|\vec{r}_1 - \vec{r}_2|)
\end{eq}
where $\theta_{12}$ is the angle between $\vec{r}_1$ and $\vec{r}_2$. To make calculations even more managable, one can use a \emph{separable} interaction that can be expressed as a product of functions of $r_1$ and $r_2$, either before or after the multipole expansion
\begin{eq}
  V(r_1, r_2) = v(r_1)v(r_2)
  \quad\textup{or}\quad
  V_l(r_1, r_2) = v_l(r_1) v_l(r_2)
\end{eq}
We study two different types of separable two-body interactions, a trivially separable gaussian interaction and a surface delta interaction.


\subsection{Gaussian Interaction}
\todo{rewrite first sentence, separabel redan nämnt OLA}
Initially we investigate the simplest form of separable interaction, a product of two functions of $r_1$ and $r_2$, the distance from each neutron to the core
\begin{eq}
  V(r_1, r_2) 
  = 
  -V\sub{GI} \exp\p{-\frac{r_1^2}{R^2}} \exp\p{-\frac{r_2^2}{R^2}}.
\end{eq}
The range $R$ and strength $V\sub{GI}$ are fitting parameters.

In \cref{fig:gaussian} we see that the potential in a rough sense satisfies the expected properties of the interaction. If at least one neutron is far from the core, the other will experience little attraction. If both neutrons are in the vicinity of the core, they will experience a stronger attraction.


\begin{figure}[h]
  \centering
    \begin{tikzpicture}
      \begin{axis}[
        xlabel = $r_1$,
        ylabel = V,
        axis x line = middle,
        axis y line = left,
        xtick=\empty,
        ytick={0},
        yticklabels={0},
        domain=0:3,
        ymin=-1.2, ymax=0.5,
        every axis y label/.style={
          at = {(current axis.above origin)},
          anchor = north west,
        },
		after end axis/.code={
		               \draw[anchor=west] (axis cs:0,-1) -- (axis cs:1.5,-1) 
					   node [] {$\displaystyle V\sub{GI}\exp\p{-\frac{r_2^2}{R^2}}$};
		             },
        ]
        \addplot[thick, black] {-e^(-x^2)};		
      \end{axis}
    \end{tikzpicture}
  \caption{The Gaussian interaction potential as seen by one of the neutrons. The position of the other neutron determines the depth of the potential well.}
  \label{fig:gaussian}
\end{figure}
Because the potential is separable
\begin{eq}
  V(r_1, r_2) 
  = 
  -V\sub{GI} V\sub{sep}(r_1) V\sub{sep}(r_2),
  \quad
  V\sub{sep}(r) = e^{- \frac{r^2}{R^2}}
\end{eq}
we can write the two-body matrix elements as
\begin{eq}
  \pbra{ab} V \pket{cd}
  =
  -V\sub{GI} 
  \bra{a} V\sub{sep} \ket{c} 
  \bra{b} V\sub{sep} \ket{d}
\end{eq}
which in the coupled scheme becomes
\begin{eq}
  & \bra{ab; J} V \ket{cd; J}
  = \\
  & -V\sub{GI} 
  \N_{ab} \N_{cd}
  \p{
    \bra{a} V\sub{sep} \ket{c} 
    \bra{b} V\sub{sep} \ket{d}
    - (-1)^{j_1 + j_2 + J}
    \bra{a} V\sub{sep} \ket{d} 
    \bra{b} V\sub{sep} \ket{c}
  }.
\end{eq}
The $\bra{a} V\sub{sep}(r) \ket{c}$ are calculated by expanding $V\sub{sep}$ in the same basis as the sp states. In the momentum basis this is
\begin{eq}
  \bra{a} V\sub{sep} \ket{c}
  =
  \sum_i \sqrt{w_i}k_i \phi'_a(k_i) \sum_j \sqrt{w_j}k_j \phi'_b(k_j) V\sub{sep}(k_i, k_j),
\end{eq}
with
\begin{eq}
  V\sub{sep}(k_i, k_j) 
  = 
  \frac{2}{\pi} \fint[0][\inf]{r} r^2 V(r) j_l(k_i r)j_l(k_j r),
\end{eq}
as before.
\todo[inline]{many new terms, whitch not are explained?}


\subsection{Surface Delta Interaction}
\todo{motivering först OLA}
Another possible separable interaction is the surface delta interaction (SDI)
\begin{eq}
  V(\vec{r}_1, \vec{r}_2) 
  = 
  -V\sub{SDI} 
  \delta(\vec{r}_1 - \vec{r}_2) 
  \delta(r_2 - r_0)
\end{eq}
where $V\sub{SDI}$ is the strength and $r_0$ is the range, chosen to have the same value as the range of the Woods-Saxon potential.

The short-range strong force is thus approximated as a point interaction. 
The physical motivation of the $\delta(r-r_0)$ term is the experimental fact that the scattering cross-section between neutrons is inversely proportional to their kinetic energy.
Since the kinetic energy has a minimum near the surface of the nucleus (at the range $r_0$ of the Woods-Saxon potential), we can approximate the interaction as focused entirely in that shell.

The SDI has multipole radial components
\begin{eq}
  v_l(r) = \frac{\delta(r-r_0)}{r},
\end{eq}
and with a complicated calculation (see \cite{suhonen}) one reaches the following expression for the coupled scheme matrix elements
\todo{definiera $\N$ i mb theory och här, mention wigner 3j OLA}
\begin{eq}
  \bra{ab; J} V \ket{cd; J} 
  = 
  & - K_{abcd} \N_{ab} \N_{cd} 
  (-1)^{l_a + l_c + j_b + j_d}
  \\ & \times
  \b{1 + (-1)^{l_a + l_b + l_c + l_d}}
  \b{1 + (-1)^{l_c + l_d + J}}
  \\ & \times
  \widehat{j_a} \widehat{j_b} \widehat{j_c} \widehat{j_d}
  \begin{pmatrix}
    j_a & j_b & J \\
    \frac{1}{2} & -\frac{1}{2} & 0
  \end{pmatrix}
  \begin{pmatrix} 
    j_c & j_d & J \\
    \frac{1}{2} & -\frac{1}{2} & 0
  \end{pmatrix}.
\end{eq}
with $\widehat{j_\alpha} = \sqrt{2j_\alpha + 1}$ and
\begin{eq}
  K_{abcd} 
  = 
  - \frac{V_0 r_0^2}{16\pi}
  \psi_a(r_0) \psi_b(r_0) \psi_c(r_0) \psi_d(r_0),
\end{eq}
$\psi_\alpha(r)$ being the radial wavefunction.

\section{The \He{6} Solutions}
\todo{With/without resonance in basis?}
Now that we know the interaction matrix elements, we can proceed and solve the Schrödinger equation for the \He{6} nucleus. 
The parameters of our interactions will be fit to the known ground state\cite{tunlhe6 alt ajzenberg} of \He{6}, and a prediction for the excited $2^+$ state --- a resonance --- can be made. 
\Cref{tab:he6_resonance_data} summarizes the results of this investigation. 

%TUNL's förslag på hur dom vill bli citerade: TUNL Nuclear Data Evaluation Project, "Energy Level Diagram, \He{6}". Available WWW: http://www.tunl.duke.edu/nucldata/HTML/A=6/06_01_2002.pdf 
%\He{6} experimental:
%0^+ E = -0.975
%2^+ E = 0.8 - i0.55


It is clearly seen that the Gaussian interaction is unsuitable to describe this system. 
It is degenerate in J and will thus not reproduce the excited $2^+$ state of \He{6}. 
It is possible to fit $V_0$ to reproduce the ground state energy. 
However, with the range $r_0$ set to \SI{2}{fm}, it does not produce a resonance. 
A wide resonance ($\Gamma > \SI{6}{MeV}$) forms when the range is decreased to \SI{1}{fm}, but doing so requires larger values of $k$ to be included in the basis. 
This rapidly increases computation time, making the method uneffective.

On the other hand, the SDI managed to reproduce the solutions remarkably well. \todo{bla bla.. Or let the table do the talkin'?}


It is also possible to consider the wavefunctions. The solutions $\ket{\psi}$ will be written on the form
\begin{eq}
\ket{\psi} = \sum_{E_1 E_2} \Psi(E_1, E_2) \ket{E_1 E_2; J}
\end{eq}
where the coefficients $\Psi(E_1, E_2)$ can be considered the wavefunction. These are hard to visualize effectively. 
However, it can be of interest to investigate the component 
\begin{eq}
\Psi(E\sub{r}, E\sub{r}) = \braket{E\sub{r}^2;J}{\psi},
\end{eq}
the overlap between a solution and the two-particle state corresponding to two independent \He{5} resonances (where $E\sub{r}$ denotes the energy of the \He{5} $p_{3/2}$ resonance). For the $2^+$ resonance, we obtain a value of $0.99$ for this coefficient, while the bound $0^+$ state gives a value of $0.70$. These are both very high relative to the background continuum states, meaning that this is a allows for an efficient method of identifying relevant solutions. 

A plot showing all solutions $k=\sqrt{2\mu E}/\hbar$ obtained, plotted in the complex momentum plane, is shown in \cref{fig:he6_momenta}. It can be verified that most of the solutions correspond to energieson the form
\begin{eq}
E = E_1 + E_2 \quad \text{or} \quad k=\sqrt{k_1^2 + k_2^2}
\end{eq} 
where $E_1$, $E_2$, $k_1$ and $k_2$ are the energies of two \He{5} eigenstates or their corresponding momenta. These solutions can be interpreted as two unbound particles, barely interacting with eachother. This also explains the pattern that is seen -- the momenta are all combinations of contour points that was used in the solution of \He{5} to generate the basis. 


\begin{table}
\caption{Experimental \He{6} resonance data\cite{tunlhe6} and computed values 
         with fitted Woods-Saxon parameters $V_0 = 47.05$ and $V\sub{so}=-7.04$. For the SDI, we used interaction strength $V_0 = \SI{998}{MeV}$ and range $r_0 = 2$. 
         The Gaussian interaction used parameters $V_0 = \SI{99}{MeV}$ and $r_0 = 2$. 24 points were included on the contour for both bases. All values are in \si{MeV}.}
\begin{center}
\begin{tabular}{r| r c l}
$J^\pi$ &   Experiment &   SDI     & Gaussian \\ \hline
 $0^+$  &   -0.975  &  -0.98    & -0.97  \\
 $2^+$  &  0.8-0.55i & 1.47-0.47i & -0.97 
\end{tabular}
\end{center}
\end{table}
\end{document}