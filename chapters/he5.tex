When we move on to the problem of the \He{5} nucleus, we see it as an alpha core (\He{4}) with a valence neutron which moves in the potential of this alpha core.
We model the potential with what is called a Woods-Saxon potential, which is relatively easy to use in numerical calculations and is a sufficient approximation for the strong nuclear force of the nuclei.
The expression for the Woods-Saxon potential is
\begin{eq}
	V(r)=
	V_0f(r)-4V\sub{so}\vec{l}\cdot\vec{s}\frac{1}{r}\frac{df}{dr}
\end{eq}
where
\begin{eq}
	f(r)=\frac{1}{1+e^{\frac{r-r_0}{d}}}.
\end{eq}
A visualisation of this potential can be seen in \cref{fig:woods-saxon}.

The implementation of \He{5} is the same as for the hydrogen atom except from the new potential and we can solve the problem using either the HO expansion or discretizised momentum space.
\todo{Reduction to effective one-body problem?}

A comparison of performance between HO and momentum basis expansion for the hydrogen atom and \He{5} problems is shown in \cref{fig:HO vs mom}.
\begin{figure}
  \centering
    \includegraphics[width = \textwidth]{figures/He5_convergence.pdf}
  \caption{}
  \label{fig:HO vs mom}
\end{figure}
\Cref{fig:resonance wavefunction} shows the radial wavefunctions $R(r)$ for a few of the states with lowest energy.
\begin{figure}
  \centering
  \includegraphics[width = \textwidth]{figures/resonance_wavefunction.pdf}
  \caption{A few solutions to the Woods-Saxon potential with well depth $V0=\SI{-52}{MeV}$.}
  \label{fig:resonance wavefunction}
\end{figure}
While all of the states we see have energies $E>0$, and thus are unbound, we see that one solution is more localized near the center $r=0$. 
This is the sign of a quasi-bound state. To confirm this we may vary the depth $V0$ of the potential well and see how this affects the solutions. 

We see that the unbound states remain practically unchanged. This means that they basically correspond to free particles of energies $E_n=\frac{k_n^2}{2\mu}$, where $k_n$ are the momentum that were included in the discretization of the integrals. We will refer to these values of $k$ as our \emph{mesh points}. The quasi-bound state, however, changed dramatically, which shows that this solution is a feature of the system we are studying.

To study these resonances more thoroughly we need to move on to a new framework of QM called Non-Hermitian Quantum Mechanics.
\todo{something here should lead to wanting NHQM}