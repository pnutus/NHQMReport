When we move on to the problem of the \He{5} nucleus, we view it as composed of an alpha particle (\He{4}) interacting with an orbiting valence neutron.
Since the exact interaction is unknown, we are using the established Woods-Saxon potential to approximate it, which is relatively easy to use in numerical calculations and is a sufficient approximation for the strong nuclear force acting between the particles.
The expression for this potential is
\begin{eq}
	V(r)=
	V_0f(r) - 4V\sub{so}\vec{l}\cdot\vec{s}\frac{1}{r}\frac{df}{dr}
\end{eq}
where \todo{exponentiak function, frac is so small ;)}
\begin{eq}
	f(r)=\frac{1}{1+e^{\frac{r-r_0}{d}}}.
\end{eq}
A visualisation of this potential can be seen in \cref{fig:woods-saxon}. Notice its short range, stemming from the $e^{-r}$ dependence on distance.

Since we approximate the problem as a spherically symmetric interaction between two particles, the problem can be reduced to a one-dimensional radial one.\todo{Explicitly describe the reduction to effective one-body problem?} The implementation of \He{5}  will thus be the same as for the hydrogen atom, except for the new potential being drastically different because of its short range. 

We can solve the problem using either the HO expansion or discretizised momentum space. A comparison of performance between HO and momentum basis expansion for the hydrogen atom and \He{5} problems is shown in \cref{fig:HO vs mom}.
\begin{figure}
  \centering
    \includegraphics[width = \textwidth]{figures/He5_convergence.pdf}
  \caption{}
  \label{fig:HO vs mom}
\end{figure}

\Cref{fig:resonance wavefunction} compares the radial probability distributions $r^2|R(r)^2|$ for two states as the potential is varied. We see that one solution is basically unchanged as the potential is varied, corresponding to an unbound state in the continuum.  

The other state has an energy $E>0$ meaning that it, too, is unbound. However, its wavefunction is highly localized near the center, hinting at a quasi-stationary state. Additionally, the solution varies dramatically with the potential, meaning that it is a feature of the system. 

However, because we are working within the realm of real numbers, we can gain no further insight into the nature of these solutions, since we expect the resonance to be properly described by complex energies. 

\begin{figure}
  \centering
  \includegraphics[width = \textwidth]{figures/resonance_wavefunction.pdf}
  \caption{Two solutions to the Woods-Saxon potential with well depth $V0=\SI{-52}{MeV}$ and $\SI{-47}{MeV}$, using the momentum space method.}
  \label{fig:resonance wavefunction}
\end{figure}
\todo{Start with a deeper well holding a bound state, and then decrease it to get resonances? Add another subplot? Also, plot unbound state with dashed line?}

%We see that the unbound states remain practically unchanged. This means that they basically correspond to free particles of energies $E_n=\frac{k_n^2}{2\mu}$, where $k_n$ are the momentum that were included in the discretization of the integrals. We will refer to these values of $k$ as our \emph{mesh points}. The quasi-bound state, however, changed dramatically, which shows that this solution is a feature of the system we are studying.

%To study these resonances more thoroughly we need to move on to a new framework of QM called Non-Hermitian Quantum Mechanics.
\todo{something here should lead to wanting NHQM}