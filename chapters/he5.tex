\todo{Where is the theory for \He{5} potential? We should mention Woods-Saxon.}
A comparison of performance between HO and momentum basis expansion for the hydrogen atom and \He{5} problems is shown in \cref{fig:HO vs mom}.
\begin{figure}
  \centering
    \includegraphics[width = \textwidth]{figures/HOvsMom.pdf}
  \caption{}
  \label{fig:HO vs mom}
\end{figure}

Let us study the obtained solutions closer. If we transform the momentum wavefunctions $\phi(k)$ to radial position wavefunctions $R(r)$ according to \cref{sec:radial_mom_space_TISE},
\todo{Maybe wavefunctions are bit too technical for this section. Move to numerical section?}
\begin{eq}
R(r)=i^l\sqrt{\frac{2}{\pi}} \int_0^\infty \rd k \, k^2 \phi(k)j_l(kr) \, ,
\end{eq} 
we can see the spatial distribution of our solutions. In our discretized basis this would be written
\begin{eq}
R(r)=i^l\sqrt{\frac{2}{\pi}}\sum_{j=1}^N \sqrt{w_j}k_j\phi_j'j_l(k_j r) \, .
\end{eq}
\Cref{fig:momspace solutions} shows the wavefunctions $R(r)$ for a few of the states with lowest energy.

\begin{figure}
  \centering
  \includegraphics[width=1\textwidth]{mom_solutions.pdf}
  \caption{A few solutions to the Woods-Saxon potential with well depth $V0=\SI{-52}{MeV}$. The probability distributions $r^2|R(r)|^2$ are plotted relative to their energies. }
  \label{fig:momspace solutions}
\end{figure}

 While all of the states we see have energies $E>0$, and thus are unbound, we see that one solution is more localized near the center $r=0$. This is the sign of a quasi-bound state. To confirm this we may vary the depth $V0$ of the potential well and see how this affects the solutions. \Cref{fig:momspace solutions var} shows the solutions obtained with a different $V0$.
\begin{figure}
  \centering
  \includegraphics[width=1\textwidth]{mom_solutions_var.pdf}
  \caption{A few solutions for $V0=\SI{-47}{MeV}$.}
  \label{fig:momspace solutions var}
  \todo{This fig could be combined with previous}
\end {figure}
We see that the unbound states remain practically unchanged. This means that they basically correspond to free particles of energies $E_n=\frac{k_n^2}{2\mu}$, where $k_n$ are the momenta that were included in the discretization of the integrals. We will refer to these values of $k$ as our \emph{mesh points}. The quasi-bound state changed dramatically, which shows that this solution is a feature of the system we are studying.  
\todo{something here should lead to wanting NHQM}