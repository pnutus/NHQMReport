\todo{Where is the theory for \He{5} potential? We should mention Woods-Saxon.}

When moving on to the real problem of \He{5} we no longer have the coulomb potential to work with.
Instead we have to model the potential of the strong nuclear force in the \He{5} nucleus.
The strong nuclear force is extremly short ranged and can almost be seen as constant inside the nucleaus and zero outside.
Even though this sometimes is enough, it is not sufficient for our needs and we have to introduce a more realistic potential.
The potential we used is called a \emph{Woods-Saxon potential} and see frequent use of describing nuclear potentials.

\todo{Plot and equation of the Woods-Saxon potential.}

A comparison of performance between HO and momentum basis expansion for the hydrogen atom and \He{5} problems is shown in \cref{fig:HO vs mom}.
\begin{figure}
  \centering
    \includegraphics[width = \textwidth]{figures/HOvsMom.pdf}
  \caption{}
  \label{fig:HO vs mom}
\end{figure}

Let us study the obtained solutions closer. \Cref{fig:momspace solutions} shows the radial wavefunctions $R(r)$ for a few of the states with lowest energy.

\begin{figure}
  \centering
  \includegraphics[width=1\textwidth]{mom_solutions.pdf}
  \caption{A few solutions to the Woods-Saxon potential with well depth $V0=\SI{-52}{MeV}$. The probability distributions $r^2|R(r)|^2$ are plotted relative to their energies. }
  \label{fig:momspace solutions}
\end{figure}

 While all of the states we see have energies $E>0$, and thus are unbound, we see that one solution is more localized near the center $r=0$. This is the sign of a quasi-bound state. To confirm this we may vary the depth $V0$ of the potential well and see how this affects the solutions. \Cref{fig:momspace solutions var} shows the solutions obtained with a different $V0$.
\begin{figure}
  \centering
  \includegraphics[width=1\textwidth]{mom_solutions_var.pdf}
  \caption{A few solutions for $V0=\SI{-47}{MeV}$.}
  \label{fig:momspace solutions var}
  \todo{This fig could be combined with previous}
\end {figure}
We see that the unbound states remain practically unchanged. This means that they basically correspond to free particles of energies $E_n=\frac{k_n^2}{2\mu}$, where $k_n$ are the momenta that were included in the discretization of the integrals. We will refer to these values of $k$ as our \emph{mesh points}. The quasi-bound state changed dramatically, which shows that this solution is a feature of the system we are studying.  
\todo{something here should lead to wanting NHQM}