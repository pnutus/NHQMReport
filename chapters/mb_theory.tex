\documentclass[../main/report.tex]{subfiles}
\begin{document}
  
\chapter{Many-Body Theory and Implementation}
\label{cha:many-body} 

We have solved the \He{5} nuclear two-body problem and studied its resonances.
The next step is to add another neutron and subsequently solve the three-body problem. 
However, while the two-body problem is reducible to a radial one-dimensional problem, the general many-body problem is not.
Instead, we need to use a single particle (sp) basis to construct many-body states.
This chapter covers the construction of such states.

First, \cref{sec:identical_particles} discusses the mathematical consequences of identical, indistinguishable particles, focusing on fermions.
This is followed in \cref{sec:second_quantization} by a short introduction to the second quantization formalism, which allows calculations with an arbitrary number of particles.
The line of reasoning as well as the notation of these sections is adapted from \cite{dickhoff}, to facilitate further reading for the interested reader. 
With the same intention, the style of \cite{suhonen} is employed in \cref{sec:coupling} where the concept of angular momentum coupling is briefly explained.
Finally, in \cref{sec:mb_implementation} we discuss the use of second quantization in computations and present a simple implementation for fermions.

\section{Identical Particles}
\label{sec:identical_particles}

A quirk of quantum mechanics is that particles that look identical \emph{are} identical.
For example, consider the nucleons in a nucleus. 
The neutrons and protons have different charge, and can thus be told apart, but distinguishing between individual neutrons is impossible.
This has to be taken into consideration when dealing with many-body states of identical particles, as we will see.

We begin with an orthonormal single particle basis $\ket{\alpha_i}$, where $\alpha_i$ represents all the quantum numbers that describe the state.
Next, consider $N$ identical particles, expressed in this basis. We form product states
\begin{eq}
  \pket{\alpha_1\alpha_2\dots\alpha_N} 
  \equiv
  \ket{\alpha_1} \otimes \ket{\alpha_2} \otimes \dots \otimes \ket{\alpha_N}
  =
  \ket{\alpha_1}\ket{\alpha_2}\dots\ket{\alpha_N},
\end{eq}
which, by the orthonormality of the $\ket\alpha$, are orthonormal as well
\begin{eq}
  \pbraket{\alpha_1\alpha_2\dots\alpha_N}{\alpha'_1\alpha'_2\dots\alpha'_N}
  =
  \delta_{\alpha_1\alpha'_1}
  \delta_{\alpha_2\alpha'_2}
  \dots
  \delta_{\alpha_N\alpha'_N}.
\end{eq}

\todo{orthonormality or completeness relation?}

Let us assume that the system of identical particles can be described by some linear combinations of these basis states, denoted
\begin{eq}
  \ket{\alpha_1\alpha_2\dots\alpha_N}.
\end{eq}
Since the particles are identical, and thus indistinguishable, we require the norm of the state to be unchanged when swapping the quantum numbers of two particles $\beta$ and $\gamma$
\begin{eq}
  \braket{\alpha_1\dots\beta\dots&\gamma\dots\alpha_N}
    {\alpha_1\dots\beta\dots\gamma\dots\alpha_N} \\
&=\braket{\alpha_1\dots\gamma\dots\beta\dots\alpha_N}
    {\alpha_1\dots\gamma\dots\beta\dots\alpha_N}
\end{eq}
These states can therefore only differ in phase $e^{i\varphi}$, and since another swap will bring us back to the original state, the phase has to be either $e^{i\varphi} = 1$ or $e^{i\varphi} = -1$.
Symmetric states with no phase change describe \emph{bosons}, whereas antisymmetric states that change sign describe \emph{fermions}.
In this thesis all sp states will be fermionic, hence we do not treat the bosonic case.


\subsection{Antisymmetric Fermion states}

We have now established that our fermion many-body states are a linear combination of product states that satisfy
\begin{eq}
  \ket{\alpha_1\dots\alpha_i\dots\alpha_j\dots\alpha_N}
  =
  -\ket{\alpha_1\dots\alpha_j\dots\alpha_i\dots\alpha_N}.
\end{eq}
For example, in the case of two particles, the correctly normalized antisymmetric state is
\begin{eq}
  \ket{\alpha_1\alpha_2} 
  = 
  \frac{1}{\sqrt{2}}
  \bigp{
    \pket{\alpha_1\alpha_2} - \pket{\alpha_2\alpha_1}
  }.
\end{eq}
We will henceforth use the angular ket notation $\ket\dots$ for antisymmetric states, as opposed to $\pket\dots$ for product states.

Let us ponder that two fermions occupy the same state. 
Exchanging the two particles and flipping the sign would then result in
\begin{eq}
  \ket{\alpha\alpha\alpha_1\dots\alpha_N} 
  = 
  - \ket{\alpha\alpha\alpha_1\dots\alpha_N}
\end{eq}
which can only be true if both states are 0. 
We conclude that two fermions can never occupy the same state, commonly referred to as the \emph{Pauli Principle}.

It is important to note that states with permuted quantum numbers, such as the states $\ket{\alpha_1\alpha_2}$ and $\ket{\alpha_2\alpha_1}$, represent the same physical state, as they only differ in sign (phase). 
This means that we have to make sure not to double count these states. 
We can do this by requiring that the sp states always appear in the same order in the ket. 
If they do not, we permute two sp states at a time until the correct ordering is reached
\begin{eq}
  \ket{\alpha_i\alpha_1\dots\alpha_{i-1}\alpha_{i+1}\dots\alpha_N}
  & =
  - \ket{\alpha_1\alpha_i\dots\alpha_{i-1}\alpha_{i+1}\dots\alpha_N}
  \\ & =
  \dots
  \\ & =
  (-1)^{i-2} 
  \ket{\alpha_1\dots\alpha_i\alpha_{i-1}\alpha_{i+1}\dots\alpha_N}
  \\ & =
  (-1)^{i-1} 
  \ket{\alpha_1\dots\alpha_{i-1}\alpha_i\alpha_{i+1}\dots\alpha_N}.
\end{eq}
With a well-defined ordering orthonormality of the normalized antisymmetric states can be stated simply
\begin{eq}
  \braket{\alpha_1\alpha_2\dots\alpha_N}{\alpha'_1\alpha'_2\dots\alpha'_N}
  =
  \delta_{\alpha_1\alpha'_1}
  \delta_{\alpha_2\alpha'_2}
  \dots
  \delta_{\alpha_N\alpha'_N}.
\end{eq}


\section{Second quantization}
\label{sec:second_quantization}

So far we've looked at a system with a fixed number of particles, but we want to work with a system hosting an arbitrary number of identical particles.
The \emph{second quantization} formalism lets us do this by introducing the \emph{Fock space}, a combination of Hilbert spaces of $0,1,2,...$ particles.
This means that a state in Fock space, a \emph{Fock state}, can contain any number of particles and that Fock states with different number of particles are orthogonal.

A disclaimer is in place here: while we introduce this powerful concept, we do not make full use of it in this thesis. 
It should instead be considered as a stepping stone for readers wishing to expand on the systems and methods we do use.

\subsection{Creation and Annihilation Operators}

The simplest Fock state is the \emph{vacuum state} $\ket{0}$, which describes a system with no particles. 
All other states can be created from the vacuum state using the \emph{creation operator} $a_\alpha^\dag$, which adds a particle with quantum numbers $\alpha$ to a state
\begin{eq}
  \label{eq:create}
  a_{\alpha}^{\dagger} \ket{\alpha_1 \alpha_2 ... \alpha_N} 
  =
  \ket{\alpha \alpha_1 \alpha_2 ... \alpha_N}.
\end{eq}
The resulting state will not necessarily be ordered, and the ordering might contribute a sign:
\begin{eq}
  \label{eq:create_ordered}
  a_{\alpha_i}^{\dagger} 
  \ket{\alpha_1 \alpha_2 ... \alpha_{i-1} \alpha_{i+1}...\alpha_{N}} 
  =
  (-1)^{i-1} 
  \ket{\alpha_1 \alpha_2 ... \alpha_{i-1} \alpha_i \alpha_{i+1} ... \alpha_{N}}.
\end{eq}
Note that when $a_\alpha^\dag$ acts on a state that already contains a particle with quantum numbers $\alpha$, the result is 0, because of the Pauli principle
\begin{eq}
  \label{eq:create_zero}
  a_{\alpha}^{\dagger} \ket{\alpha\alpha_1 \alpha_2 ... \alpha_N} 
  =
  0.
\end{eq}

The adjoint of the creation operator is called the \emph{annihilation operator} $a_\alpha$. 
It can be shown to have the opposite effect, removing a particle, when acting on a state
\begin{eq}
  \label{eq:annihilate}
  a_{\alpha} \ket{\alpha \alpha_1 \alpha_2 ... \alpha_N}
  =
  \ket{\alpha_1 \alpha_2 ... \alpha_N}.
\end{eq}
Here, too, a sign might appear from the ordering
\begin{eq}
  \label{eq:annihilate_ordered}
  a_{\alpha_i}
  \ket{\alpha_1 \alpha_2 ... \alpha_{i-1} \alpha_i \alpha_{i+1} ... \alpha_N}
  =
  (-1)^{i-1}
  \ket{\alpha_1 \alpha_2 ... \alpha_{i-1} \alpha_{i+1}...\alpha_N}.
\end{eq}
Analogous to $a_\alpha^\dag$, when $a_\alpha$ acts on a state that does not contain a particle with the quantum numbers $\alpha$, the result is 0
\begin{eq}
  \label{eq:annihilate_zero}
  a_\alpha \ket{\alpha_1 \alpha_2 ... \alpha_N} 
  =
  0.
\end{eq}


\subsection{General Operators in Fock Space}

We can now express the state of an arbitrary number of particles, but to have any use of the states we also need to express operators such as the Hamiltonian in the Fock space formalism. 
A general operator has can in Fock space be expressed using the creation and annihilation operators.
The Fock space equivalent of an operator is able act on a state with an arbitrary number of particles. 
We will only treat one- and two-body operators here, as they are sufficient for our purposes.

\subsubsection{One-Body Operators}

A one-body operator $H_1$ which acts on a single sp state, is represented by the Fock space operator
\begin{eq}
  \hat{H}_1
  =
  \sum_{\alpha \beta} 
  \bra\alpha H_1 \ket\beta 
  a_\alpha^\dag a_\beta.
\end{eq}
It is important to note that while the sum runs over the complete set of sp states twice, only a few terms will be non-zero, because of the operator rules in \cref{eq:create_zero,eq:annihilate_zero}. 

If the sp-states are eigenstates to the one-body operator
\begin{eq}
  H_1 \ket{\alpha} = h_\alpha \ket{\alpha}
\end{eq}
the matrix elements only exist on the diagonal, when $\alpha = \beta$, and we get
\begin{eq}
  \hat{H}_1
  =
  \sum_{\alpha} 
  \bra\alpha H_1 \ket\alpha
  a_\alpha^\dag a_\alpha.
\end{eq}
The many-body matrix element becomes
\begin{eq}
  \label{eq:one-body_matrix_elements}
  \bra{a_1\dots a_N} \hat{H}_1 \ket{b_1\dots b_N}
  & =
  \sum_{\alpha} 
  \bra\alpha H_1 \ket\alpha
  \bra{a_1\dots a_N} 
  a_\alpha^\dag a_\alpha
  \ket{b_1\dots b_N}
  \\ & =
  \sum_{i = 1}^N 
  \bra{a_i} H_1 \ket{a_i}
  \braket{\alpha_1\dots\alpha_N}{\alpha'_1\dots\alpha'_N}
  \\ & =
  \p{
    h_1 + \dots + h_N
  }
  \delta_{a_1 b_1} \dots \delta_{a_N b_N},
\end{eq}
the sum of the eigenvalues of the sp states in the bra or ket, but only if the bra and ket are the same. The Fock space operator $\hat{H}_1$ is thus also diagonal.

\subsubsection{Two-Body Operators}

A two-body operator in Fock space becomes
\begin{eq}
  \hat{H}_2
  =
  \frac{1}{2}\sum_{\alpha \beta \gamma \delta} 
  \pbra{\alpha \beta} H_2 \pket{\gamma \delta} 
  a_\alpha^\dag a_\beta^\dag a_\delta a_\gamma.
\end{eq}
Note that the ordering of the $\gamma$ and $\delta$ is different for the product states and the operators, so-called \emph{normal ordering}.
The factor \nicefrac{1}{2} stems from the fact that %%%%%%%%%%%%%%%%%%%%%%%%%%%
\begin{eq}
  \pbra{\alpha \beta} H_2 \pket{\gamma \delta} 
  = 
  \pbra{\beta \alpha} H_2 \pket{\delta \gamma},
\end{eq}
and we are counting both.

We can also express $\hat{H}_2$ using matrix elements between antisymmetric states
\begin{eq}
  \bra{\alpha\beta} H_2 \ket{\gamma\delta} 
  = 
  \pbra{\alpha\beta} H_2 \pket{\gamma\delta}
  -
  \pbra{\alpha\beta} H_2 \pket{\delta\gamma},
\end{eq}
but we will have to add another factor \nicefrac{1}{2} to compensate for double counting
\begin{eq}
  \hat{H}_2
  =
  \frac{1}{4}\sum_{\alpha \beta \gamma \delta} 
  \bra{\alpha \beta} H_2 \ket{\gamma \delta} 
  a_\alpha^\dag a_\beta^\dag a_\delta a_\gamma.
\end{eq}
The double counting can be avoided, however, by taking into account the ordering of the states
\begin{eq}
  \hat{H}_2
  =
  \sum_{\substack{\alpha < \beta \\ \gamma < \delta}} 
  \bra{\alpha \beta} H_2 \ket{\gamma \delta} 
  a_\alpha^\dag a_\beta^\dag a_\delta a_\gamma.
\end{eq}

For the special case of two particles we have
\begin{eq}
  \label{eq:two-body_matrix_elements}
  \bra{ab} \hat{H}_2 \ket{cd}
  & =
  \sum_{\substack{\alpha < \beta \\ \gamma < \delta}} 
  \bra{\alpha \beta} H_2 \ket{\gamma \delta} 
  \bra{ab} 
  a_\alpha^\dag a_\beta^\dag a_\delta a_\gamma
  \ket{cd}
  \\ & =
  \sum_{\substack{\alpha < \beta \\ \gamma < \delta}} 
  \bra{\alpha \beta} H_2 \ket{\gamma \delta}
  \delta_{\alpha a}\delta_{\beta b}
  \delta_{\gamma c}\delta_{\delta d}
  \\ & =
  \bra{ab} H_2 \ket{cd},
\end{eq}
as expected.

\section{Angular Momentum Coupling}
\label{sec:coupling}

We have now developed the theory we need to solve actual many-body problems.
Before applying these methods, however, we will discuss the concept of angular momentum coupling. 
This corresponds to making a change of basis, using the rotational symmetry of problems to make the Hamiltonian matrix block diagonal and significantly reducing its size. 
In principle, coupling is not neccesary to solve many-body problems, but its an invaluable tool.
We will limit the discussion to coupling of two particles, as we never work with more particles here.
A more complete description is found in \cite{suhonen}.

\subsection{The Two-Particle Coupled Basis}
In previous Chapters we have studied single-particle states on the form $\ket{Ejm}$, where we let the quantum number $E$ represent all quantum numbers needed to uniquely specify a state. From these we would form product states
\begin{eq}
  \pket{E_1j_1m_1, E_2j_2m_2} = \ket{E_1j_1m_1}\otimes\ket{E_2j_2m_2}
\end{eq}
that are eigenstates to the operators $\vec{J}_1 = \vec{J}\otimes\vec{1}$ and $\vec{J}_2 = \vec{1}\otimes\vec{J}$.

One often studies systems where the total angular momentum $\vec{J} = \vec{J}_1 + \vec{J}_2$ is conserved, but the individual angular momenta $\vec{J}_1$ and $\vec{J}_2$ are not. Conservation of $\vec{J}$ is equivalent to the entire system being symmetric under rotation, while $\vec{J}_1$ and $\vec{J}_2$ will only be conserved if one of the particles can be rotated independently of the other, without affecting the solutions. 
This is rarely the case when studying directly interacting particles.
It is then convenient to switch to a basis where the total angular momentum is well defined, but the individual momenta are not. 

We introduce the \emph{coupled} basis with states $\pket{E_1j_1, E_2j_2; JM}$ meant to be read as: the first particle is described by the quantum numbers $E_1 j_1$, the second by $E_2 j_2$, and they together have a total angular momentum $JM$ (the $M$ is sometimes left out).
The coupled states are related to the uncoupled by
\begin{eq}
  \pket{E_1j_1, E_2j_2; JM} 
  = 
  \sum_{m_1, m_2} c_{m_1 m_2} \pket{E_1j_1m_1, E_2j_2m_2} .
\end{eq}
where $c_{m_1 m_2}$ are the \emph{Clebsch-Gordan coefficients} (which also depend on $j_1, j_2, J$ and $M$, though this is suppressed for brevity). 
There are known expressions for these, and values can be found in standard tables or calculated in a fairly straight-forward way. 
A more in-depth treatment of the coefficients can be found in \cite{suhonen}.

Working with matrix elements between coupled basis states is called working in the \emph{coupled scheme} or \emph{$J$-scheme}.
If we study a system with rotational symmetry, the Hamiltonian will be block diagonal in $J$ and $M$, meaning that working in the coupled scheme will be much more efficient.

\subsection{Antisymmetrizing the Coupled Basis}

Since we are studying fermions, we need to use basis states that are antisymmetric with respect to exchange of all quantum numbers. Using the $m$ symmetry property of the Clebsch-Gordan coefficients,
\begin{eq}
  c_{m_2 m_1} = (-1)^{j_1 + j_2 - J} c_{m_1 m_2} 
\end{eq}
we can see that
\begin{eq}
  \pket{E_2j_2, E_1j_1; JM} 
  & = 
  \sum_{m_1, m_2} c_{m_1 m_2} \pket{E_2 j_2 m_1, E_1j_1 m_2} 
  \\ & = 
  (-1)^{j_1+j_2-J}\sum_{m_1, m_2} c_{m_1 m_2} \pket{E_2 j_2 m_2, E_1 j_1 m_1}
\end{eq}
Hence it is possible to form the antisymmetric basis vector
\begin{eq}
  \ket{E_1 j_1 E_2 j_2; JM} &= \frac{1}{\sqrt{2}}\bigp{\pket{E_1 j_1 E_2 j_2;JM} - (-1)^{j_1+j_2-J}\pket{E_2 j_2 E_1 j_1;JM}} \\
  &= \sum_{m_1, m_2} c_{m_1 m_2} \ket{E_1 j_1 m_1, E_2 j_2 m_2}.
\end{eq}

Consider the case where both particles occupy the same orbital, $E_1 j_1 = E_2 j_2 = E j$. 
Since, for fermions, $j$ is a half-integer we have $(-1)^{j_1+j_2 - J} = - (-1)^J$ and the result is
\begin{eq}
  \ket{(Ej)^2; JM} = \frac{1+(-1)^J}{\sqrt{2}}\pket{(Ej)^2; JM} .
\end{eq}
We see that this state is equal to zero for $J$ odd. However, for $J$ even, we find that the norm
\begin{eq}
  \braket{(Ej)^2; JM}{(Ej)^2; JM} = 2
\end{eq}
meaning that we have to normalize these states with an additional factor $\nicefrac{1}{\sqrt{2}}$. 
We can write this result succinctly as 
\begin{align}
  \ket{ab; JM} 
  &= 
  \N_{ab}\sum_{m_1, m_2} c_{m_1 m_2} \ket{E_1 j_1 m_1, E_2 j_2 m_2}
  \\ & 
  \N_{ab} = \frac{\sqrt{1+(-1)^{J}\delta_{ab}}}{1+\delta_{ab}}
\end{align}
where we have used the notation $(a, b)$ for $(E_1j_1, E_2j_2)$.


\section{Implementation}
When solving problems in practice we need to form matrix elements in the coupled basis. 
Using the notation $\ket{E_1 E_2} = \ket{E_1 j_1 E_2 j_2; JM}$ for short, we would have
\begin{eq}
  \bra{E_1 E_2} \hat{H} \ket{E_1'E_2'} =  \N_{E_1E_2}\N_{E_1'E_2'} \sum_{m_1,m_2} \sum_{m_1',m_2'} \conj{c_{m_1 m_2}}c_{m_1' m_2'} \bra{E_1m_1 E_2m_2}\hat{H}\ket{E_1'm_1'E_2'm_2'}
\end{eq}
meaning that the matrix elements in the new basis are given by linear combinations of the elements in uncoupled basis. 
We only need to evaluate the elements $\bra{E_1m_1 E_2m_2}\hat{H}\ket{E_1'm_1'E_2'm_2'}$, and be done. 
Since we do not know what the two-body interaction may look like, we cannot do a general treatment of that contribution. 
Instead we focus on the one-body term, which was treated in \cref{eq:one-body_matrix_elements}. There is a subtle detail, however, that needs to be considered.

Even if both single-particles are in the same state in the coupled basis $E_1 = E_2 = E$, we will still sum over all $m_1$, $m_2$ in the ket $\ket{E m_1 E m_2}$, meaning that we will \emph{not} order our states in that calculation. Instead, for $E_1 = E_2$ \cref{eq:one-body_matrix_elements} will read
\begin{eq}
  \bra{E_1m_1 E_2m_2}\hat{H}_1\ket{E_1'm_1'E_2'm_2'} 
  = 
  (E_1 + E_2)\delta_{E_1 E_1'}\delta_{E_2 E_2'}\p{
    \delta_{m_1m_1'}\delta_{m_2 m_2'} - \delta_{E_1 E_2} \delta_{m_1 m_2'}\delta_{m_2 m_1'}
  }.
\end{eq}    
Inserted into the matrix element equation in the coupled basis, this will result in
\begin{eq}
  \bra{E_1 E_2} \hat{H}_1 \ket{E_1'E_2'} 
  &=
  \N_{E_1E_2}\N_{E_1'E_2'}(E_1 + E_2)(1 + \delta_{E_1 E_2}(-1)^J) \delta_{E_1E_1'}\delta_{E_2E_2'} \\
  &= 
  \frac{1+ (-1)^J \delta_{E_1 E_2}}{1+\delta_{E_1 E_2}} \delta_{E_1 E_1'}\delta_{E_2 E_2'} (E_1 + E_2) .
\end{eq}
Remember that $E_1$ is short for all quantum numbers specifying that single-particle state.


\section{Second Quantization Implementation}
\label{sec:mb_implementation}

To use the Fock space formalism in numerical calculations we have to represent the quantum states using available data structures. 
We construct sp states and use them to form antisymmetric Fock states.
The creation and annihilation operators become functions that take the Fock states as arguments, and are used to create the Fock operators. 
Furthermore, the sum in the expression of the Fock operators can be optimized by only evaluating the non-zero terms.

\subsection{Single-Particle-State Objects}

We represent a single particle state with a record, i.e. an object with named fields that can be assigned values. It is natural to let each quantum number, such as $l$ and $j$ be a field. 
Moreover, we can include other information about the state, information that is taken for granted in the mathematical formulation.
This includes a unique index for each state, the eigenvector holding information about the wavefunction, the basis of the eigenvector and, in the case of the plane wave basis, the contour used.
Much of this extra information is redundant, as many states share the same information.
Nevertheless, we found that this representation significantly simplifies the structure of the program and makes it easier to understand.

\subsection{Fock State Objects and Operator Functions}

An antisymmetric many-body state $\ket{\alpha_1\dots}$ is represented by an ordered list of single particle objects and a sign. 
Since the sp state objects have a unique index, there is a well-defined sorting order. 
The sign can be $1$, $-1$ or $0$, representing $\ket{\alpha_1\dots}$, $-\ket{\alpha_1\dots}$ and $0$, respectively. 

The creation and annihilation operators are implemented as functions on the Fock state objects, obeying
\cref{eq:create,eq:create_ordered,eq:create_zero,eq:annihilate,eq:annihilate_ordered,eq:annihilate_zero}.
The creation operator function steps through the list, flipping the sign at each step, until the correct place for the new particle is found. 
If the particle is already part of the Fock state, the sign is set to 0.
The annihilation operator searches for a state, saves its index $k$ in the list, annihilates it and set the sign to $(-1)^k$. If the state not exists, the sign is set to 0.

\subsection{Fock Operator Matrix Elements}

When computing matrix elements of a Fock space operator, most terms in the sum vanish. 
This is because most states will become zero when acted on by the creation and annihilation operators (equations \cref{eq:create_zero,eq:annihilate_zero}).
Evaluating a sum of mostly zero elements is not very efficient, but this can be 
avoided by letting the sum run over just the sp states present in the bras and kets.

\todo{behöver detta vara tydligare? /pontus  -nej /vincent}

\end{document}
