\documentclass[../main/report.tex]{subfiles}
\begin{document}

\chapter{The Complex-Momentum Basis}
\label{cha:berggren}
\todo{read through OLA SPILL}

In \cref{cha:two-body} we studied the \He{5} system and found a special state in the continuum, the resonance.
We will in this chapter detail a method of extending the momentum basis to the complex momentum plane, which gives a more complete description of resonances.
We also state the important the important Berggren completeness relation.
The \He{5} problem is examined once more using the complex-momentum basis and we fit the model parameters to experimental resonance data.

\section{The Complex Momentum Plane}

When solving the Schrödinger equation in the momentum basis, we know that the solutions form a complete basis, expressed as a completeness relation
\begin{eq}
  \label{eq:momentum_completeness_relation}
  \sum\sub{bound} \ket{E_n}\bra{E_n} + \fint[0][\inf]{k} k^2 \ket{E_k}\bra{E_k} = 1,
\end{eq}
where $E_n$ are discrete bound states and $E_k$ are continuous.

If we relate the energies $E$ to momenta $k$ as
\begin{eq}
  E = \frac{\hbar^2 k^2}{2\mu}
  \quad\quad
  \textup{or}
  \quad\quad
  k = \frac{\sqrt{2\mu E}}{\hbar},
\end{eq}
we can plot the solutions as $k$ in the complex plane, see 
\cref{fig:complex plane}. 
We then expect bound states, with $E<0$, to be represented by discrete $k$ along the imaginary axis---whereas unbound states with $E>0$, are found continuously along the real axis. 
Resonance states, with complex $E = E_0 - i \Gamma /2$, would by this argument appear somewhere in the fourth quadrant.

\begin{figure}
  \tikzset{
    triangle/.style={regular polygon, regular polygon sides=3},
    nosep/.style={inner sep=0},
    bound/.style={circle,draw,minimum size=2mm,nosep},
    unbound/.style={rectangle,draw,minimum size=2mm,nosep},
    quasibound/.style={triangle,draw,minimum size=2.5mm,nosep}
  }
  \centering{
    \subfloat[]{
    \label{fig:simple_contour}
    %\tikzset{external/remake next}
  \tikzsetnextfilename{simple_contour}
    \begin{tikzpicture}[scale = 2.7]
      \draw[->] (-1.2, 0) -- (1.2, 0) node[below left] {$\Re k$};
      \draw[->] (0, -0.5) -- (0, 1.2) node[above] {$\Im k$};
      
	    \tikzstyle{every pin edge}=[shorten <=1pt,]
      \tikzstyle{every pin}=[fill=white,]

        \node at (0, 0.1) [bound, pin=above left:Bound state,] {};
		\node at (0, 0.3) [bound] {};
		
        \node at (-0.25, -0.15) [quasibound] {};
        \draw[very thick, mid arrows] 
        (0,0) -- (1,0);
        \draw[thick, mid arrows, dashed]
          (1,0) arc (0:90:1) arc (90:180:1) -- (0,0);
		  
		\node at (0, 0.1) [bound, pin=above left:Bound State,] {};
		\node at (0.25, -0.15) [quasibound, pin=below right:Resonance,] {};
    \end{tikzpicture}
    }
    \subfloat[]{
    \label{fig:berggren_contour}
    %\tikzset{external/remake next}
  \tikzsetnextfilename{berggren_contour}
    \begin{tikzpicture}[scale = 2.7]
      \draw[->] (-1.2, 0) -- (1.2, 0) node[below left] {$\Re k$};
      \draw[->] (0, -0.5) -- (0, 1.2) node[above] {$\Im k$};
      \foreach \y in {0.1, 0.3}
        \node at (0, \y) [bound] {};
      \foreach \x in {0.25, -0.25}
        \node at (\x, -0.15) [quasibound] {};
        
      \draw[very thick, mid arrows] (0,0) 
         -- (0.25, -0.25) 
         -- (0.5, 0) node[pos=0.25, pin=below right:$L_+$] {} 
         -- (1, 0);
      \draw[thick, mid arrows, dashed, radius=1] (1, 0) 
                arc [start angle=0,  end angle=90]
                arc [start angle=90, end angle=180]
                -- (-0.5, 0) 
                -- (-0.25, 0.25)
                -- (0, 0);
      
    \end{tikzpicture}
    }
    \label{fig:contours}
  }
  \caption{The complex $k$-plane. The circles represent 
  bound states and the triangles resonant states. Note the 
  mirroring of the states in the imaginary axis.}
  \label{fig:complex plane}
\end{figure}

It is known that bound and resonant states indeed correspond to poles in the complex momentum plane at $k=\sqrt{2\mu E}/\hbar$.
The details of this is treated in scattering theory, and is beyond the scope of this thesis. 
We interpret the bound and resonant $k$ as complex poles and treat the unbound continuum as a contour, mirrored in the imaginary axis, encircling the upper half plane (\cref{fig:simple_contour}).


The integral to be evaluated along the contour is the radial momentum space Schrödinger equation
\begin{eq}
  \frac{k^2}{2\mu}\phi(k) + \fint[0][\inf]{k'} k'^2 V(k,k') \phi(k') 
  &=
  E\phi(k).
\end{eq}
The result of a contour integration depends on the poles it encircles by the residue theorem. 
The contour in \cref{fig:simple_contour} encircles the bound states, but not the resonance.
We suspect a deformation of the contour, such that it goes below the resonance, might have an effect on the solutions.

\subsection{The Berggren Completeness Relation}

In fact, this is correct, and was proven in 1968 by Berggren \cite{berggren}.
The contour, dubbed $L_+$, can be deformed to surround the resonance poles, as illustrated in \cref{fig:berggren_contour}. 
The continuum states along $L_+$ combined with the encircled bound and resonant states form a complete basis, the \emph{Berggren basis}. 
This result can be stated succinctly with the \emph{Berggren completeness relation} (compare with \cref{eq:momentum_completeness_relation})
\begin{eq}
  \sum_{\substack{\text{bound} \\ \text{resonant}}} \ket{E_n}\bra{E_n} 
  + \fint[L_+]{k} k^2 \ket{E_k}\bra{E_k} = 1.
  \label{eq:berggren_completeness_relation}
\end{eq}
This is an important result, because of the inclusion of the resonances in the basis.
The Berggren basis is what allows us to find resonances in systems with more particles.

An observant reader may have noticed that the resonant poles in \cref{fig:contours} are mirrored in the imaginary axis and that the contour has a shadow on the left half plane.
Berggren showed that this symmetry allows us to restrict ourselves to the contour segment from 0 to $\inf$. 
However, doing this leads to a scalar product \emph{without conjugation} that must be used
\begin{eq}
  \label{eq:berggren_product}
  \braket{\phi}{\phi'} = \fint[0][\inf]{k} k^2 \phi(k)\phi'(k).
\end{eq}
Naturally, this also affects the norm.



\todo{figure markers, romb, kvadrat, cirkel, triangel SPILL}
\section{\He{5} Revisited}

With the complex-momentum basis we can continue our study of the \He{5} system. 
We use the same Woods-Saxon parameters as before, but now use a complex contour.
The solutions will be examined in a similar fashion to \cref{cha:two-body}.

\subsection{The Discretized Complex Contour}

We deform the previously real contour by a triangle-shaped downward extrusion, as in \cref{fig:discretized_contour}. 
The tip of the triangle is placed below the expected position of the resonance pole.
To use the contour in computations it has to be discretized, as before in \cref{sec:discretization}.
We use the Gauss-Legendre quadrature, but now consider each segment of the contour separately: This requires us to rescale the evaluation points and weights between each pair of complex endpoints according to \cref{app:gauss-legendre}.
The discretized contour is seen in \cref{fig:discretized_contour}. 
Note the concentration of points near the ends of each segment, characteristic of the Gauss-Legendre quadrature.

The discretized Schrödinger equation \cref{eq:plane_wave_matrix_elements} is unchanged from before
\begin{eq}
  \label{eq:nhqm matrix element}
  H_{ij}' = \frac{k_i^2}{2\mu}\delta_{ij} + \sqrt{w_i w_j}k_i k_j V_{ij},
\end{eq}
but now the $k$ and $w$ are complex. 
However, when normalizing the obtained eigenvectors, one must make sure to use the new scalar product defined in \cref{eq:berggren_product}.

%\tikzset{external/remake next}
\tikzsetnextfilename{triangle_contour}
\begin{figure}[h]
  \centering
  \begin{tikzpicture}
    \begin{axis}[
      width = \textwidth,
      height = 7cm,
      xlabel=Re $k$,
      ylabel=Im $k$,
		  axis lines = middle,
      ymax = 0.1,
      enlargelimits,
      only marks,
      ticks = none,
      every axis x label/.style={
        at = {(current axis.right of origin)},
        anchor = north,
      },
      ]
      \addplot[color=darkgray] table {../figures/numerical_contour/numerical_contour.data};
    \end{axis}
  \end{tikzpicture}
  \caption{A schematic picture of the discretized complex contour. The points are distributed on each segment according to the Gauss-Legendre quadrature rule.}
  \label{fig:discretized_contour}
\end{figure}

\subsection{The \He{5} Resonances}

We expect the position $k_r$ of the resonance poles to be fairly close to the experimental data in \cref{tab:resonance_data}, hence we choose a triangular contour that extends below the expected position. \Cref{tab:contours} presents the poles and the contours used to identify them.

\todo{fix table width, cite exp vals. tunl finns i bibtex}

\begin{table}[b]
\caption{Contours used to identify \He{5} resonances.}
  \label{tab:contours}
  \hspace{-2cm}\begin{tabular}{c|c|c|c}
    Wave      & Experimental $k_r$ & Computed $k_r$  & Contour vertices \\
    \hline
    $p_{3/2}$ & 0.1788 -  0.0349i & 0.1778 - 0.0376i &
      $(0, 0) \to (0.17, -0.2) \to (0.34, 0) \to (2.5, 0)$ \si{fm^{-1}} \\
    $p_{1/2}$ & 0.3267 - 0.1643i & 0.3286 - 0.1630i &
      $(0, 0) \to (0.35, -0.4) \to (0.7, 0) \to (2.5, 0)$ \si{fm^{-1}} \\
  \end{tabular}
\end{table}
\todo[inline]{varför skiljer sig polerna i tabell och stycke?}
We solve the Schrödinger equation using the contours and represent the energy solutions by their momenta $k=\sqrt{2\mu E}$.
 The result for the $p_{3/2}$ wave is shown in \cref{fig:triangle_contour}. 
We see that most solutions follow the contour, corresponding to non-resonant continuum states, similarly to the real case.
There is one solution that does not lie on the contour, however. 
It has $k = \SI{0.173-0.0357i}{fm^{-1}}$, which is reasonably close to what we expected for the resonance.
If we have indeed found the resonance, we expect it to be unchanged when the contour is varied.
In \cref{fig:rect_contour}, a rectangular contour is used instead of the triangular, yet the pole is completely stable.
We also found that we could vary the downward extrusion of the contour to some extent without the resonance pole moving. 
But with sufficiently large imaginary parts, the matrix element integrals started to diverge.
Barring such numerical errors, we found that any contour that runs below and not too close to the pole gives a stable resonance solution.



\begin{figure}
  \centering{
    \subfloat[Triangle Contour]{
      \label{fig:triangle_contour}
      \includegraphics[page=2]{../figures/poles(contour)/poles.pdf}
    }
    \\
    \subfloat[Rectangular Contour]{
      \label{fig:rect_contour}
      \includegraphics[page=1]{../figures/poles(contour)/poles.pdf}
    }
  }
   \caption{Momentum solutions for a \He{5} $p_{3/2}$ wave for different contours. The resonance is located at $0.173, -0.0356i$ for both contours. we have used $V0 =$ \SI{47}{MeV} with $k_{max} =$ \SI{5}{fm^{-1}} with 60 points on both contours.} 
\label{fig:pole(cont)}  
\end{figure}


%\tikzset{external/remake next}
\tikzsetnextfilename{pole(V0)}
\begin{figure}
  \centering
  \begin{tikzpicture}
    \begin{axis}[
      width = \textwidth,
      height = 13 cm,
        xlabel=Re $k/\b{\si{fm^{-1}}}$,
        ylabel=Im $k/\b{\si{fm^{-1}}}$,
  		  axis x line = middle,
        axis y line = left,
        every axis y label/.style={
          at = {(current axis.above origin)},
          anchor = north west,
        },
        every axis x label/.style={
          at = {(current axis.right of origin)},
          anchor = north east,
        },
 	      yticklabel style={/pgf/number format/fixed,
 	                     /pgf/number format/precision=3},
        every x tick label/.append style = {anchor = south, yshift = 3pt},
        xmax=0.5,
        ytickmax = 0.3, xtickmax = 0.9,
		xtick={0.1,0.2,...,0.4},
        enlarge y limits,
		     xtickmin = 0.1,
		     xticklabel style={/pgf/number format/fixed,
		     /pgf/number format/precision=3},legend style={at={(0.8,0.65)}, anchor=north,legend columns=1},
        ]
      	\addplot+[color = red, very thick, ->, no markers,] table  {../figures/res_pole(V0)/poles.data};
      	    \addlegendentry{Resonance pole position}
      	\addplot+[color= gray, very thick, no markers,] table [x index = 0, y index = 1] {../figures/res_pole(V0)/contour.data};
		
\addplot+[color= gray, only marks] table  {../figures/res_pole(V0)/mark.data};
\node[coordinate,pin=above right:{$V_0=54.3$}, mark=*] 
at (axis cs:0,0.116) {};

\node[coordinate,pin=below right:{$V_0=51.2$}] 
at (axis cs:0.0817,-0.00752) {};

\node[coordinate,pin=right:{$V_0=47.1$}] 
at (axis cs:0.17,-0.0346) {};

\node[coordinate,pin=right:{$V_0=65.0$}] 
at (axis cs:0,0.351) {};

\node[coordinate,pin=below left:{$V_0=40.0$}] 
at (axis cs:0.257,-0.0860) {};




          \addlegendentry{Contour}
      \end{axis}
  \end{tikzpicture}
  \caption{The position of the resonance pole as a function of $V_0$. The pole begins at \SI{70}{MeV} as a bound state on the imaginary axis, gradually becomes less and less bound, jumps into the fourth quadrant ($V_0 \approx \SI{52}{MeV}$), passes the \He{5} resonance and continues further down and to the right.}
  \label{fig:pole(V0)}
\end{figure}
%%%%%%%%%%%%%%%%%%%%%%%%% V0(CONTOUR) FIGURE

\subsubsection{Momentum Space Wavefunctions}

As in \cref{cha:two-body} we can study the momentum wavefunctions obtained in the diagonalization.
\Cref{fig:complex_mom_wavefunctions} shows the momentum distribution of an arbitrary unbound solution and the resonance. We also show the corresponding situation when using the real basis, for comparison. Note how the resonance wavefunction is much more distinguished in the complex basis.  
The unbound solution still correspond to one definite (now complex) momentum. 
On the other hand, the resonance has a wide distribution, reflecting the localized nature of the solution 
(remember Heisenberg---a wide momentum wavefunction allows a localized position wavefunction). 
This also allows for automatically finding the resonance among a large set of solutions by singling out the one with the widest (or lowest) wavefunction.

%%%%%%%%%%%%%%%%%%%%%%%%%WF real/complex FIGURE

\begin{figure}
\centering{
	\subfloat[Momentum space wavefunctions with a real contour.]{
  		\includegraphics[page=1]{../figures/eigvecs_real_comp/eigvecs.pdf}
	}
	\\
	\subfloat[Momentum space wavefunctions with a complex contour.]{
  		\includegraphics[page=2]{../figures/eigvecs_real_comp/eigvecs.pdf}
	}
  }
\caption{\He{5} momentum probability distributions for a real and complex contour. The resonance is more prominent when solving along the complex contour but is significantly wider than the surrounding states in both cases. Both contours comprise 60 points in total, but the points in the complex contour are concentrated near the origin, stemming from the Gauss-Legendre distribution on the individual segments.} 
\label{fig:complex_mom_wavefunctions}
\end{figure}

%%%%%%%%%%%%%%%%%%%%%%%%%WF real/complex end

\subsubsection{Varying the Potential Depth}
\todo{SPILL}
We repeat the procedure from \cref{sub:momentum_basis}, lowering the potential well to \SI{70}{MeV} and increasing it gradually, but we now plot the position of the bound/resonance pole in the complex plane.
\Cref{fig:pole(V0)} we see how the pole starts on the imaginary axis at $V_0 = \SI{70}{MeV}$. 
It then moves downwards, becoming less and less strongly bound.
At $V_0 \approx \SI{52}{MeV}$ the pole jumps to the fourth quadrant, becoming a resonance.
As the potential well grows even less attractive, we see that both the binding energy $E_0$ and the width $\Gamma$ of the resonance increases.
\todo{vad händer sen?}

\subsection{Fitting to Experimental Resonance Data} 

Having studied the solutions and verified that we have found the resonances, we now proceed to fit our model of \He{5} to experimental data.
The data, taken from \cite{tunl}, is presented in \cref{tab:resonance_data} along with simulated results. Fitting the Woods-Saxon parameters after the four experimental data points resulted in optimal values $V_0 = \SI{47.05}{MeV}$ and $V\sub{so} = \SI{-7.04}{MeV}$.


\hspace{-2cm}\begin{table}[H]
\caption{Experimental \He{5} resonance data \cite{tunl} and computed values with fitted Woods-Saxon parameters $V_0 = \SI{47.05}{MeV}$ and $V\sub{so} = \SI{-7.04}{MeV}$. All values are in \si{MeV}.}
\label{tab:resonance_data} 
\begin{center}
\resizebox{10cm}{!} {
\begin{tabular}{c c c c c}  \hline\hline
 \multicolumn{1}{c|}{ \multirow{2}{*}{Wave} }  &\multicolumn{2}{c|}{Experimental data}    & \multicolumn{2}{c}{Computed values} \\ 

 \multicolumn{1}{c|}{} &       $E_0$        & \multicolumn{1}{c|}{$\Gamma$} &      $E_0$      &    $\Gamma$     \\ \hline
       $p_{3/2}$:       &       0.798        &              0.648            &      0.783      &     0.695       \\  
       $p_{1/2}$:       &       2.068        &              5.57             &      2.111      &     5.560       \\ \hline\hline
\end{tabular}
}
\end{center}
\end{table}
\todo{figures where they belong. Wednesday night: fixa uppslag med figur + tillhörande text}

\todo{in captions: solutions in the complex momentum plane SPILL}

\end{document}
