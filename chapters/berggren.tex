\documentclass[../main/report.tex]{subfiles}
\begin{document}

\chapter{The Complex-Momentum Basis}
\label{cha:berggren}

In \cref{cha:two-body} we studied the \He{5} system and found a special state in the continuum, which should be the resonance.
We will in this chapter present a method of extending the momentum basis to the complex plane, which lets us use the complex-energy description of resonances introduced in \cref{cha:introduction}.
We also state the important Berggren completeness relation.
Our ambition here is not to go into much detail, let alone prove anything. 
We instead try to give an intuitive argument for why the method works.
Finally, the \He{5} problem is examined once more using the complex-momentum basis and we fit the model parameters to experimental resonance data.

\section{The Complex-Momentum Plane}

When solving the Schrödinger equation in the momentum basis, we know that the solutions form a complete basis, expressed as a completeness relation
\begin{eq}
  \label{eq:momentum_completeness_relation}
  \sum\sub{bound} \ket{E_n}\bra{E_n} + \fint[0][\inf]{k} k^2 \ket{E_k}\bra{E_k} = 1,
\end{eq}
where $E_n$ are discrete bound states and $E_k$ are continuous.

If we relate the energies $E$ to momenta $k$ as
\begin{eq}
  E = \frac{\hbar^2 k^2}{2\mu}
  \quad\quad
  \textup{or}
  \quad\quad
  k = \frac{\sqrt{2\mu E}}{\hbar},
\end{eq}
we can plot the solutions as $k$ in the complex-momentum plane, see 
\cref{fig:complex plane}. 
We then expect bound states, with $E<0$, to be represented by discrete $k$ along the imaginary axis -- whereas unbound states with $E>0$, are found continuously along the real axis. 
Resonance states, with complex $E = E_0 - i \Gamma /2$, would by this argument appear somewhere in the fourth quadrant.

\begin{figure}
  \tikzset{
    triangle/.style={regular polygon, regular polygon sides=3},
    nosep/.style={inner sep=0},
    bound/.style={fill=white,circle,draw,minimum size=2mm,nosep},
    unbound/.style={rectangle,draw,minimum size=2mm,nosep},
    quasibound/.style={triangle,draw,minimum size=2.5mm,nosep}
  }
  \centering{
    \subfloat[]{
    \label{fig:simple_contour}
    %\tikzset{external/remake next}
  \tikzsetnextfilename{simple_contour}
    \begin{tikzpicture}[scale = 2.7]
      \draw[->] (-1.2, 0) -- (1.2, 0) node[below left] {$\Re k$};
      \draw[->] (0, -0.5) -- (0, 1.2) node[above] {$\Im k$};
      
	    \tikzstyle{every pin edge}=[shorten <=1pt,]
      \tikzstyle{every pin}=[fill=white,]

      \node at (0, 0.3) [bound] {};
		
        \node at (-0.25, -0.15) [quasibound] {};
        \draw[decorate, decoration={brace}, yshift=0.3mm]
          (0,0) -- (1,0) node[midway, yshift=0.3cm] {Continuum};
        \draw[very thick, mid arrows] 
        (0,0) -- (1,0);
        \draw[thick, mid arrows, dashed]
          (1,0) arc (0:90:1) arc (90:180:1) -- (0,0);
		  
		\node at (0, 0.1) [bound, pin=above left:Bound State,] {};
		\node at (0.25, -0.15) [quasibound, pin=below right:Resonance,] {};
    \end{tikzpicture}
    }
    \subfloat[]{
    \label{fig:berggren_contour}
    %\tikzset{external/remake next}
  \tikzsetnextfilename{berggren_contour}
    \begin{tikzpicture}[scale = 2.7]
      \draw[->] (-1.2, 0) -- (1.2, 0) node[below left] {$\Re k$};
      \draw[->] (0, -0.5) -- (0, 1.2) node[above] {$\Im k$};
      \foreach \y in {0.1, 0.3}
        \node at (0, \y) [bound] {};
      \foreach \x in {0.25, -0.25}
        \node at (\x, -0.15) [quasibound] {};
        \tikzstyle{every pin edge}=[thin]
      \draw[very thick, mid arrows] (0,0) 
         -- (0.25, -0.25) 
         -- (0.5, 0) node[semithick, pos=0.25, pin=below right:$L_+$] {} 
         -- (1, 0);
      \draw[thick, mid arrows, dashed, radius=1] (1, 0) 
                arc [start angle=0,  end angle=90]
                arc [start angle=90, end angle=180]
                -- (-0.5, 0) 
                -- (-0.25, 0.25)
                -- (0, 0);
      
    \end{tikzpicture}
    }
    \label{fig:contours}
  }
  \caption{The complex $k$-plane. The circles represent 
  bound states and the triangles resonant states. Note the 
  mirroring of the states in the imaginary axis.}
  \label{fig:complex plane}
\end{figure}

We now interpret the bound and resonant $k$ as complex poles, the details of which is treated in scattering theory and is beyond the scope of this thesis. We direct the curious reader to a QM textbook such as \cite{sakurai}. 
Furthermore, the unbound continuum is interpreted as a contour encircling the upper half plane (\cref{fig:simple_contour}).
The integral to be evaluated along the contour is in the radial-momentum Schrödinger equation
\begin{eq}
  \label{eq:rad_mom_TISE}
  \frac{k^2}{2\mu}\phi(k) + \fint[0][\inf]{k'} k'^2 V(k,k') \phi(k') 
  &=
  E\phi(k).
\end{eq}

The result of a contour integration depends on the poles it encloses by the residue theorem. 
The contour in \cref{fig:simple_contour} encloses the bound states, but not the resonance.
We suspect a deformation of the contour, such that it goes below the resonance, might have an effect on the solutions.

\subsection{The Berggren Completeness Relation}

In fact, this is correct, and was proven in 1968 by Berggren \cite{berggren}.
The real contour segment can be deformed into a complex one dubbed $L_+$, illustrated in \cref{fig:berggren_contour}. 
When solving \cref{eq:rad_mom_TISE} using the complex contour, we get the resonance as a solution with a complex energy.
All the solutions -- the continuum states along $L_+$ combined with the encircled bound and resonant states -- form a complete basis, the \emph{Berggren basis}. 
This result can be stated succinctly with the \emph{Berggren completeness relation} (compare with \cref{eq:momentum_completeness_relation})
\begin{eq}
  \sum_{\substack{\text{bound} \\ \text{resonant}}} \ket{E_n}\bra{E_n} 
  + \fint[L_+]{k} k^2 \ket{E_k}\bra{E_k} = 1.
  \label{eq:berggren_completeness_relation}
\end{eq}
This is an important result, because of the inclusion of the resonances in the basis.
The Berggren basis is what allows us to find resonances in systems with more particles.

An observant reader may have noticed that the resonant poles in \cref{fig:contours} are mirrored in the imaginary axis and that the contour has a shadow on the left half plane.
Berggren showed that this symmetry allows us to restrict the integration to the contour segment from 0 to $\inf$ and that this leads to a scalar product \emph{without conjugation}
\begin{eq}
  \label{eq:berggren_product}
  \braket{\phi}{\phi'} = \fint[0][\inf]{k} k^2 \phi(k)\phi'(k).
\end{eq}
Naturally, this also affects the norm.

\section{\He{5} Revisited}

With the complex-momentum basis we can continue our study of the \He{5} system. 
We use the same Woods-Saxon parameters as before, but now use a complex contour.
The solutions will be examined in a similar fashion to \cref{cha:two-body}.
We will be focusing on the $p_{3/2}$ resonance, as the method is general and only differs in the choice of contour.

\subsection{The Discretized Complex Contour}

We deform the previously real contour by a triangle-shaped downward extrusion, as in \cref{fig:berggren_contour}. 
The tip of the triangle is placed below the expected position of the resonance pole, using experimental data as a guide.
To use the contour in computations it has to be discretized and truncated as was done in \cref{cha:basis_expansion} with the real momentum basis.
We use the Gauss-Legendre quadrature, but now consider each segment of the contour separately: This requires us to rescale the evaluation points and weights between each pair of complex endpoints according to \cref{app:gauss-legendre}.
The discretized contour is seen in \cref{fig:discretized_contour}. 
Note the concentration of points near the ends of each segment, characteristic of the Gauss-Legendre quadrature.

The discretized Schrödinger equation \cref{eq:plane_wave_matrix_elements} is unchanged from before
\begin{eq}
  \label{eq:nhqm matrix element}
  H_{ij}' = \frac{k_i^2}{2\mu}\delta_{ij} + \sqrt{w_i w_j}k_i k_j V_{ij},
\end{eq}
but now the $k$ and $w$ are complex. 
However, when normalizing the obtained eigenvectors, one must make sure to use the new scalar product defined in \cref{eq:berggren_product}.

%\tikzset{external/remake next}
\tikzsetnextfilename{triangle_contour}
\begin{figure}[t]
  \centering
  \begin{tikzpicture}
    \begin{axis}[
      width = \textwidth,
      height = 7cm,
      xlabel=Re $k$,
      ylabel=Im $k$,
		  axis lines = middle,
      ymax = 0.1,
      enlargelimits,
      only marks,
      ticks = none,
      every axis x label/.style={
        at = {(current axis.right of origin)},
        anchor = north,
      },
      ]
      \addplot[color=darkgray] table {../figures/numerical_contour/numerical_contour.data};
    \end{axis}
  \end{tikzpicture}
  \caption{A schematic picture of the discretized complex contour. The points are distributed on each segment according to the Gauss-Legendre quadrature rule.}
  \label{fig:discretized_contour}
\end{figure}

\subsection{The \He{5} Resonances}

We solve the Schrödinger equation using the contours in \cref{tab:contours} and represent the energy solutions by their momenta $k=\sqrt{2\mu E}$.
The result for the $p_{3/2}$ wave is shown in \cref{fig:triangle_contour}. 
We see that most solutions follow the contour, corresponding to non-resonant continuum states, similarly to the real case.
There is one solution that does not lie on the contour, however. 
It has $k = \SI{0.173-0.0357i}{fm^{-1}}$, which is reasonably close to what we expected for the resonance.

\begin{table}[b]
\caption{Contours used to identify \He{5} resonances.}
  \label{tab:contours}
  \centering
  \begin{tabular}{cc}
    \toprule
    Wave      & Contour vertices \\
    \midrule
    $p_{3/2}$ & $(0, 0) \to (0.17, -0.2) \to (0.34, 0) \to (2.5, 0)$ \si{fm^{-1}} \\
    $p_{1/2}$ & $(0, 0) \to (0.35, -0.4) \to (0.70, 0) \to (2.5, 0)$ \si{fm^{-1}} \\
    \bottomrule
  \end{tabular}
\end{table}

\begin{figure}[b!]
  \centering{
    \subfloat[Triangle Contour]{
      \label{fig:triangle_contour}
      \includegraphics[page=2]{../figures/poles(contour)/poles.pdf}
    }
    \\
    \subfloat[Rectangular Contour]{
      \label{fig:rect_contour}
      \includegraphics[page=1]{../figures/poles(contour)/poles.pdf}
    }
  }
   \caption{Complex-momentum solutions for the \He{5} $p_{3/2}$ wave for a triangle and a rectangle contour, both truncated at $k\sub{max} = \SI{5}{fm^{-1}}$ and using 60 points. The resonance is located at the same $k_r = \SI{0.173 -0.0356i}{fm^{-1}}$ in both cases.} 
\label{fig:pole(cont)}  
\end{figure}

%%%%%%%%%%%%%%%%%%%%%%%%%WF real/complex FIGURE

\begin{figure}[p!]
\centering{
	\subfloat[Momentum space wavefunctions with a real contour.]{
  		\includegraphics[page=1]{../figures/eigvecs_real_comp/eigvecs.pdf}
	}
	\\
	\subfloat[Momentum space wavefunctions with a complex contour.]{
  		\includegraphics[page=2]{../figures/eigvecs_real_comp/eigvecs.pdf}
	}
  }
\caption{\He{5} momentum probability distributions for a real and complex contour. The resonance state is significantly lower in the complex case, and thus easier to identify. Both contours comprise 60 points in total, but the points in the complex contour are concentrated near the origin, stemming from the Gauss-Legendre point distribution on the individual segments of the contour.} 
\label{fig:complex_mom_wavefunctions}
\end{figure}

%%%%%%%%%%%%%%%%%%%%%%%%%WF real/complex end

If we have indeed found the resonance, we expect it to be unchanged when the contour is varied.
In \cref{fig:rect_contour}, a rectangular contour is used instead of the triangular, yet the pole is completely stable.
We also found that we could vary the downward extrusion of the contour to some extent without the resonance pole moving. 
But with sufficiently large imaginary parts, the matrix element integrals started to diverge.
Barring such numerical errors, we found that any contour that runs below and not too close to the pole gives a stable resonance solution.


\subsubsection{Momentum Wavefunctions}

As in \cref{cha:two-body} we can study the momentum wavefunctions obtained in the diagonalization.
\Cref{fig:complex_mom_wavefunctions} shows the radial-momentum wavefunctions using a real and a complex contour respectively.
Note how the resonance wavefunction is much more distinguished when using the complex basis.  
The unbound solution still correspond to one definite (now complex) momentum. 
On the other hand, the resonance has a wide distribution, reflecting the localized nature of the solution 
(remember Heisenberg -- a wide momentum wavefunction allows a localized position wavefunction). 
This also allows for automatically finding the resonance among a large set of solutions by singling out the one with the widest (or lowest) wavefunction.


\todo{lägg till data punkten "0	0	-5.25" i res_pole(v0)/poles.data så att den går in till origo som den ska}
%%%%%%%%%%%%%%%%%%%%%%%%% V0(CONTOUR) FIGURE

\subsubsection{Varying the Potential Depth}
We repeat the procedure from \cref{sub:momentum_basis}, lowering the potential well to \SI{70}{MeV} and increasing it gradually, but we now plot the position of the bound/resonance pole in the complex plane.
In \Cref{fig:pole(V0)} we see how the pole starts on the imaginary axis at $V_0 = \SI{70}{MeV}$. 
It then moves downwards, becoming less and less strongly bound.
At $V_0 \approx \SI{52}{MeV}$ the pole jumps to the fourth quadrant, becoming a resonance.
As the potential well grows even less attractive, we see that both the energy $E_0$ and width $\Gamma$ of the resonance increases.
With an even shallower potential well the resonance is so wide that it can no longer be detected.

\subsection{Fitting to Experimental Resonance Data} 

Having studied the solutions and verified that we have found the resonances, we now proceed to fit our model of \He{5} to experimental data.
We use $E_0$ and $\Gamma$ for the $p_{3/2}$ and $p_{1/2}$ waves, giving us four data points to fit to.
To simplify the fitting process, we choose to only fit the depth $V_0$ and spin-orbit coupling strength $V\sub{so}$ of the potential.
The fitting gives us the optimal values $V_0 = \SI{47.05}{MeV}$ and $V\sub{so} = \SI{-7.04}{MeV}$.
The experimental data, taken from \cite{tunl}, is presented in \cref{tab:resonance_data} along with our computed values after fitting. 


\begin{table}[t!]
\caption{Experimental \He{5} resonance data \cite{tunl} and computed values with fitted Woods-Saxon parameters $V_0 = \SI{47.05}{MeV}$ and $V\sub{so} = \SI{-7.04}{MeV}$. All values are in \si{MeV}.}
\label{tab:resonance_data} 
\centering
\begin{tabular}{c c c c c}
  \toprule
    &\multicolumn{2}{c}{Experimental data}  & \multicolumn{2}{c}{Computed values} \\ \cmidrule(lr){2-3} \cmidrule(lr){4-5}
    Wave & $E_0$        & $\Gamma$ &  $E_0$ & $\Gamma$ \\
 \midrule
       $p_{3/2}$       &       0.798        &              0.648            &      0.783      &     0.695       \\  
       $p_{1/2}$       &       2.068        &              5.57             &      2.111      &     5.560       \\
       \bottomrule
\end{tabular}
\end{table}

\begin{figure}[b!]
  \centering
  \begin{tikzpicture}
    \begin{axis}[
      width = \textwidth,
      height = 12 cm,
        xlabel=Re $k/\b{\si{fm^{-1}}}$,
        ylabel=Im $k/\b{\si{fm^{-1}}}$,
  		  axis x line = middle,
        axis y line = left,
        every axis y label/.style={
          at = {(current axis.above origin)},
          anchor = north west,
        },
        every axis x label/.style={
          at = {(current axis.right of origin)},
          anchor = north east,
        },
 	      yticklabel style={/pgf/number format/fixed,
 	                     /pgf/number format/precision=3},
        every x tick label/.append style = {anchor = south, yshift = 3pt},
        xmax=0.5,
        ytickmax = 0.3, xtickmax = 0.9,
		xtick={0.1,0.2,...,0.4},
        enlarge y limits,
		     xtickmin = 0.1,
		     xticklabel style={/pgf/number format/fixed,
		     /pgf/number format/precision=3},legend style={at={(0.8,0.65)}, anchor=north,legend columns=1},
        ]
      	\addplot+[color = red, very thick, ->, no markers,] table  {../figures/res_pole(V0)/poles.data};
      	    \addlegendentry{Resonance pole position}
      	\addplot+[color= gray, very thick, no markers,] table [x index = 0, y index = 1] {../figures/res_pole(V0)/contour.data};
		
\addplot+[color= gray, only marks] table  {../figures/res_pole(V0)/mark.data};
\node[coordinate,pin=above right:{$V_0=54.3$}, mark=*] 
at (axis cs:0,0.116) {};

\node[coordinate,pin=above:{$V_0=51.2$}] 
at (axis cs:0.0817,-0.00752) {};

\node[coordinate,pin=right:{$V_0=47.1$}] 
at (axis cs:0.17,-0.0346) {};

\node[coordinate,pin=right:{$V_0=65.0$}] 
at (axis cs:0,0.351) {};

\node[coordinate,pin=below left:{$V_0=40.0$}] 
at (axis cs:0.257,-0.0860) {};
          \addlegendentry{Contour}
      \end{axis}
  \end{tikzpicture}
  \caption{The position of the resonance pole in the complex-momentum plane as a function of $V_0$ for the \He{5}-like Woods-Saxon system. The pole begins at \SI{65}{MeV} as a bound state on the imaginary axis, gradually becomes less and less bound, jumps into the fourth quadrant ($V_0 \approx \SI{52}{MeV}$), passes the \He{5} resonance and continues further down and to the right.}
  \label{fig:pole(V0)}
\end{figure}


\end{document}
