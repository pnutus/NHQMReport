\documentclass[../main/report.tex]{subfiles}
\begin{document}

\chapter{The Complex Berggren Basis}

%\todo{I would like an argument akin to "Since we are only treating momenta in the fourth quadrant in the complex plane the square root can be regarded as injective (except in the case of a bound state with zero real part, but these are easy to identify and treat sepparately). Our states can thus be described by energy or momentum and we will use both in the following sections." somewhere}

\label{cha:berggren}
As we argued in the introduction, resonances can only be properly described by a complex energy, $E=E_0-i\frac{\Gamma}{2}$. \todo{I like the connection to the introduction but I am not so thrilled about the active, exclusive, we}
We may relate the energy of a system to a momentum, according to
\begin{eq}
  E = \frac{\hbar^2 k^2}{2\mu}
  \quad\quad
  \textup{or}
  \quad\quad
  k = \frac{\sqrt{2\mu E}}{\hbar},
\end{eq} \todo{Perhaps add a discussion/motivation of this ''definition''? See .tex for more info}
%Reason behind E=k^2/2m : schrödinger equation says Hu = Eu, H=k^2/2m + V(r), open system => finite range on the potential, meaning V(r) mostly 0.
%(This is especially true for unbound solutions that mostly live far away from the core)
%Therefore, far away from the core (''in the lab''?) all solutions satisfy (k^2/2m)u = Eu, because V(r) = 0, making it a reasonable (?) definition
%Should something along these lines be included? 
This tells us that the resonant states have complex momenta and that we have to extend the plane wave basis into the complex momentum plane in order to find them.
To clarify what this means, consider again the momentum space Schrödinger equation, \cref{eq:radial mom space TISE}:\todo{Momentum space Schrödinger equation? Plane wave SE? Spherical wave SE?}
\begin{eq}
  \frac{k^2}{2\mu}\phi(k) + \fint[0][\inf]{k} k'^2 V(k,k') \phi(k') 
  &=
  E\phi(k) \, .
\end{eq}
Using a complex momentum basis means that we evaluate the integral above along a complex contour. 
In 1968, Berggren\cite{berggren} first showed that the solutions that are obtained using this scheme --- bound states, resonant states and the non-resonant continuum --- form a complete orthonormal basis. 
This is a major result, proving that resonances are included as proper solutions to open quantum systems. This will be of importance later, when we consider systems of multiple interacting particles. 
In this Chapter, we will instead focus on using complex momentum contours to find the resonances of \He{5}. \todo{complex momentum contour, while technically correct aren't we mixing up the expressions a bit?}

We will start by mentioning some of the key features of the theory behind the complex momentum representation. 
However, we will not give a comprehensive or rigorous introduction to it, since the mathematical nature of the theory is beyond the scope of this report.

It is known\cite{Some source supporting this statement. Probably not berggren?} that bound state solutions correspond to poles on the positive imaginary axis, $k=i\kappa$, related to them having negative energies and
\begin{eq}
  E = \frac{\hbar^2(i\kappa)^2}{2\mu} = -\frac{\hbar^2 \kappa^2}{2\mu} \, ,
\end{eq} 
while resonances, with energies on the form $E=E_0 - i \Gamma /2$, correspond to poles in the fourth quadrant. 
This allows us to use the energy and momentum interchangealy to describe a state, since the square root is injective when restricted to a single quadrant. 
This will not apply to bound states however, but these are easy to identified and can be treated separately to avoid confusion. 
We will primarily use the energy of a state in our mathematical expressions, but the momentum is better suited to present a state graphically.

%The resonant state has a complex energy and we need to extend our basis into the complex plane to study it. 
%We do this by using the theory of Tore Berggren and his \emph{Berggren basis} \cite{berggren}.
%This allows us to determine the energy of the state which charcterizes its width and half-life.

%Another reason for moving to the complex energy plane is the fact that when we later move to more than two bodies we want to use a basis consisting of our two-body states.
%A basis consisting of all bound, free and resonant complex states forms a complete set  after the \emph{Berggren Comleteness Relation}, this theory is also described by Tore Berggren.

%\todo{comment on why all free states can be approximated by a few free states along a contour with low energy.}

%The theory is involved and will not be fully explained 
%here. Instead we present a heuristic argument.

%If we relate the energies $E$ of a system to momenta $k$ as
%\begin{eq}
%  E = \frac{\hbar^2 k^2}{2\mu}
%  \quad\quad
%  \textup{or}
%  \quad\quad
%  k = \frac{\sqrt{2\mu E}}{\hbar},
%\end{eq}
%we can plot the energies as $k$ in the complex plane, see 
%\cref{fig:complex plane}. We then expect bound states, with \todo{introduce k - momenta, basis mom here?}
%$E<0$, to be represented by $k$ along the imaginary axis---
%whereas unbound, scattering states, with $E>0$, are found 
%along the real axis. Resonance states, with complex 
%$E = E_0 - i \Gamma /2$, would by this argument appear somewhere
%in the fourth quadrant.

%\todo{maybe show real conour before complex for comparison, röd tråd, mycket figurer för att förstå}

%We now interpret these complex momenta, $k$, as poles and our integration 

To find the resonance, a mathematical quirk requires that the complex $k$ contour passes below the pole corresponding to the resonance. \todo{I wouldn't say quirk :P better go into the residue  math details}
A sample contour is visualized in \cref{fig:berggren contour}. Note that the complex plane is mirrored in the y-axis. \todo{why is this important? elaborate further?}
Solutions along or near the real axis are to be interpreted as 'ingoing' scattering solutions, while solutions on the left side can be interpreted as 'outgoing'. \todo{be more elaborate on the scattering theory?}
However, use of this closed contour is mainly a mathematical technique used to prove the completeness relation. 
In practice on the other hand, symmetry in our specific problems lets us restrict ourselves to the segment between $k=0$ and $k=+\infty$. 
Note that our original momentum space equation corresponded to a contour along the real axis, $k \in [0,\, \infty]$.


%\begin{eq}
%  \fint[0][\inf]{k} k'^2 V(k,k') \phi(k')
%\end{eq}
%\todo{resiidy stycke, integration behöver kärlek}
%is evaluated as a contour integration around the upper half plane, 
%see \cref{fig:simple contour}. The result of a contour 
%integration depends on the poles it encircles by the 
%residue theorem. Therefore, we expect something to happen if 
%we let the contour encircle the pole of the resonance,
%as in \cref{fig:berggren contour}.
%\todo{tydligare förankring av beggren teoris}
%An interesting thing seen when moving to a complex contour is the fact that we no longer use a conjugate on the bra when computing the matrix elements.
%This is a confusing piece of theory stemming, again, from Berggren.


\begin{figure}
  \tikzset{
    triangle/.style={regular polygon, regular polygon sides=3},
    nosep/.style={inner sep=0},
    bound/.style={circle,draw,minimum size=2mm,nosep},
    unbound/.style={rectangle,draw,minimum size=2mm,nosep},
    quasibound/.style={triangle,draw,minimum size=2.5mm,nosep}
  }
  \subfloat[]{
  \label{fig:simple contour}
  %\tikzset{external/remake next}
\tikzsetnextfilename{simple_contour}
  \begin{tikzpicture}[scale = 2.5]
    \draw[->] (-1.2, 0) -- (1.2, 0) node[right] {$\Re k$};
    \draw[->] (0, -0.5) -- (0, 1.2) node[above] {$\Im k$};
    \foreach \y in {0.1, 0.3}
      \node at (0, \y) [bound] {};
    \foreach \x in {0.25, -0.25}
      \node at (\x, -0.15) [quasibound] {};
    \draw[very thick, mid arrows] (1, 0) arc (0:90:1) arc (90:180:1) 
                                  -- (0,0) -- cycle;
  \end{tikzpicture}
  }
  \subfloat[]{
  \label{fig:berggren contour}
  %\tikzset{external/remake next}
\tikzsetnextfilename{berggren_contour}
  \begin{tikzpicture}[scale = 2.5]
    \draw[->] (-1.2, 0) -- (1.2, 0) node[right] {$\Re k$};
    \draw[->] (0, -0.5) -- (0, 1.2) node[above] {$\Im k$};
    \foreach \y in {0.1, 0.3}
      \node at (0, \y) [bound] {};
    \foreach \x in {0.25, -0.25}
      \node at (\x, -0.15) [quasibound] {};
    \draw[very thick, mid arrows, radius=1]
      (1, 0) arc [start angle=0,  end angle=90]
             arc [start angle=90, end angle=180]
             -- (-0.5, 0) 
             -- (-0.25, 0.25) 
             -- (0.25, -0.25)
             -- (0.5, 0)
             -- cycle;
  \end{tikzpicture}
  }
  \caption{The complex $k$-plane. The circles represent 
  bound states and the triangles resonant states. Note the 
  mirroring of the states in the imaginary axis.}
  \label{fig:complex plane}
\end{figure}

\todo{Need to mention completeness of berggren basis. Important here or later?}

\todo{Where to put: "No conjugate on bras"?}

\section{The Complex Contour}
\todo{In this section I will add additional context and connect it better with previous section}
We choose to extend our integration along the real axis to 
the simplest possible complex contour, a triangle-shaped downward extrusion. The tip of the triangle is placed directly 
below the hypothesized resonance pole.

Numerically, we are faced with the problem of how to choose 
the points and weights. 
We consider each straight segment of the contour separately, 
and rescale the Gauss-Legendre points to each of the different segments.
An example of a contour is seen in
\cref{fig:triangle contour}, note the concentration of the points around the ends of each segment, the points near the center are given more weight though.

%\tikzset{external/remake next}
\tikzsetnextfilename{triangle_contour}
\begin{figure}[H]
  \centering
  \begin{tikzpicture}
    \begin{axis}[
      width = \textwidth,
      height = 7cm,
      xlabel=Re $k$,
      ylabel=Im $k$,
		  axis lines = middle,
      ymax = 0.1,
      enlargelimits,
      only marks,
      ticks = none,
      ]
      \addplot table {../figures/numerical_contour/numerical_contour.data};
    \end{axis}
  \end{tikzpicture}
  \caption{The complex contour used. The points are distributed on each segment according to the Gauss-Legendre quadrature rule.}
  \label{fig:triangle contour}
\end{figure}

\section{Studying the Resonance}
Armed with the tools listed above we can now continue our study of the resonances.
We now need a way to determine which solution corresponds to the resonnance. 
It is not as simple as picking the solution with the lowest energy or norm. 
Instead we need to consider the solutions in momentum space.
Bound states have negative energy and can easily be identified, that leaves unbound and resonant states. 
The unbound solutions would be dominated by a single momentum, i e, their probability density would closely resemble sine waves in position space. 
Whereas a more localized resonant state would have a greater uncertainty in its momentum by the Heisenberg relation, and can thus be identified by picking the most diffuse solution.

We can now proceed to fit our models to experimental data provided by \cite{inte jimmy i alla fall}. 
We also found it worthwhile to plot the resonnance momentum as a function of the strength of the potential. 
In the \He{5} system no resonant state can co-exist with a bound state for a given potential strength, \cite{spill?}. \todo{rewrite this paragraph}
Hence we could single out the least localized solution in momentum space, this allows us to study how the bound system becomes weaker and gradually dissipates into a resonance. 
The result is presented in \cref{fig:pole(V0)}, in this figure the potential was varried from \SI{-70}{MeV}, bound, to \SI{-50}{MeV}, resonant.

\Cref{fig:pole(cont)} on the other hand depicts the pole's stability with respect to different complex contours. And that's about it 

%\tikzset{external/remake next}
\tikzsetnextfilename{pole(V0)}
\begin{figure}
  \centering
  \begin{tikzpicture}
    \begin{axis}[
      width = \textwidth,
      height = 9cm,
        xlabel=Re $k/\b{\si{fm^{-1}}}$,
        ylabel=Im $k/\b{\si{fm^{-1}}}$,
  		  axis x line = middle,
        axis y line = left,
        every axis y label/.style={
          at = {(current axis.above origin)},
          anchor = north west,
        },
        every axis x label/.style={
          at = {(current axis.right of origin)},
          anchor = north east,
        },
 	      yticklabel style={/pgf/number format/fixed,
 	                     /pgf/number format/precision=3},
        every x tick label/.append style = {anchor = south, yshift = 3pt},
        xmax=0.5,
        ytickmax = 0.3, xtickmax = 0.9,
        enlarge y limits,
        no markers,
        ]
      	\addplot+[very thick, ->] table  {../figures/res_pole(V0)/poles.data};
      	    \addlegendentry{Pole position}
      	\addplot+[very thick] table {../figures/res_pole(V0)/contour.data};
          \addlegendentry{Contour}
      \end{axis}
  \end{tikzpicture}
  \caption{The pole position as a function of $V_0$.}
  \label{fig:pole(V0)}

\end{figure}

%%%%%%%%%%%%%%%%%%%%%%%%%V0 (CONTOUR) FIGURE
  
\begin{figure}
   \centering{
   \pgfplotsset{
     width = \textwidth,
     height = 7cm,
     axis lines = middle,
     xmin = -0.01, xmax = 0.5,
     xtickmin = 0.15,
     max space between ticks=60pt
     enlargelimits,
     ylabel=Im $k/\b{\si{fm^{-1}}}$,
     every axis y label/.style={
     at = {(current axis.above origin)},
     anchor = north west,
     },
     yticklabel style={/pgf/number format/fixed,
     /pgf/number format/precision=3},
     every axis x label/.style={
     at = {(current axis.right of origin)},
     anchor = north,
     },
     every x tick label/.append style = {anchor = south, yshift = 3pt},
   }
     \subfloat[Square Contour]{
       \tikzset{external/remake next}
   \tikzsetnextfilename{sqrcont}
       \begin{tikzpicture}
         \begin{axis}[
			 ymax = 0.04,
			 legend style={at={(0.6,0.4)}, anchor=north,legend columns=1},
			 xlabel=Re $k/\b{\si{fm^{-1}}}$]
         	\addplot+[only marks,very thick] table [x index =2, y index =3] {../figures/poles(contour)/square.data};
         	    \addlegendentry{Pole position}
         	\addplot+[no marks, very thick] table [x index =0, y index =1] {../figures/poles(contour)/square.data};
             \addlegendentry{Contour}
 
         \end{axis}
       \end{tikzpicture}
     }
 
   \subfloat[Triangle Contour]{
     \tikzset{external/remake next}
 \tikzsetnextfilename{trigcont}
     \begin{tikzpicture}
       \begin{axis}[
		   ymax = 0.04,
		   legend style={at={(0.8,0.2)}, anchor=north,legend columns=1},
		   xlabel=Re $k/\b{\si{fm^{-1}}}$]
         	\addplot+[only marks, very thick] table [x index =2, y index =3] {../figures/poles(contour)/triangle.data};
         	    \addlegendentry{Pole position}
         	\addplot+[no marks, very thick] table [x index =0, y index =1] {../figures/poles(contour)/triangle.data};
			             \addlegendentry{Contour}

       \end{axis}
     \end{tikzpicture}
   }
   
   \subfloat[Sierpinski Contour]{
     \tikzset{external/remake next}
 \tikzsetnextfilename{sierpcont}
     \begin{tikzpicture}
       \begin{axis}[
		   ymax = 0.04,
		   legend style={at={(0.8,0.2)}, anchor=north,legend columns=1},
		   xlabel=Re $k/\b{\si{fm^{-1}}}$]
         	\addplot+[only marks, very thick] table [x index =2, y index =3] {../figures/poles(contour)/sierpinski.data};
         	    \addlegendentry{Pole position}
         	\addplot+[no marks, very thick] table [x index =0, y index =1] {../figures/poles(contour)/sierpinski.data};
			             \addlegendentry{Contour}

       \end{axis}
     \end{tikzpicture}
   }
   }
  \caption{yadayoda} 
   \label{fig:pole(cont)}  
\end{figure}

%%%%%%%%%%%%%%%%%%%%%%%%%V0 (CONTOUR) end


%%%%%%%%%%%%%%%%%%%REAL CONTOUR FIGURE

\begin{figure}[H] %this figure needs to be nudged a little bit to the left
   \centering{
   \pgfplotsset{
          width = 0.45\textwidth,
      height = 7cm,
 	  axis lines = middle,
       xmax = 0.5,
	   ymax = 0.5,
	   xmin = -0.01,
	   xtickmin = 0.15,
	   max space between ticks=60pt
       enlargelimits,
       ylabel=Im $k/\b{\si{fm^{-1}}}$,
       every axis y label/.style={
         at = {(current axis.above origin)},
         anchor = north west,
       },
	   yticklabel style={/pgf/number format/fixed,
	                     /pgf/number format/precision=2},
       every axis x label/.style={
         at = {(current axis.right of origin)},
         anchor = north,
       },
       every x tick label/.append style = {anchor = south, yshift = 3pt},
   }
     \subfloat[Real Contour \SI{-70}{MeV}]{
       %\tikzset{external/remake next}
   \tikzsetnextfilename{realcont70}
       \begin{tikzpicture}
         \begin{axis}[
  		   legend style={at={(0.8,0.5)}, anchor=north,legend columns=1},
  		   xlabel=Re $k/\b{\si{fm^{-1}}}$]
           	\addplot+[only marks, very thick] table [x index =0, y index =1] {../figures/poles(realcontour)/poles.data};
           	    \addlegendentry{Pole position}
           	\addplot+[no marks, very thick] table [x index =0, y index =1] {../figures/poles(realcontour)/contour.data};
  			             \addlegendentry{Contour}

         \end{axis}
       \end{tikzpicture}
     }
     \subfloat[Real Contour \SI{-50}{MeV}]{
       %\tikzset{external/remake next}
   \tikzsetnextfilename{realcont50}
       \begin{tikzpicture}
         \begin{axis}[
  		   legend style={at={(0.8,0.5)}, anchor=north,legend columns=1},
  		   xlabel=Re $k/\b{\si{fm^{-1}}}$]
           	\addplot+[only marks, very thick] table [x index =2, y index =3] {../figures/poles(realcontour)/poles.data};
           	    \addlegendentry{Pole position}
           	\addplot+[no marks, very thick] table [x index =0, y index =1] {../figures/poles(realcontour)/contour.data};
  			             \addlegendentry{Contour}

         \end{axis}
       \end{tikzpicture}
     }
   
   }
  \caption{momentum solutions to the Shrödinger equation ``along'' the real axis for \He{5} with a potential of $V_0 =$ \SI{-70}{MeV} (a) and $V_0 =$ \SI{-50}{MeV} (b).} 
   \label{fig:pole real contour}  
\end{figure}
\todo{define pole - momentum solution}
%%%%%%%%%%%%%%%%%%%REAL CONTOUR FIGURE end


\end{document}
