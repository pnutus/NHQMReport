\documentclass[../main/report.tex]{subfiles}
\begin{document}

\chapter{The Complex Momentum Basis}
\label{cha:berggren}

\todo{momentum/plane wave???}
In \cref{cha:two-body} we studied the \He{5} system and found a special state in the continuum which we believe to be a resonance. 
We will in this chapter detail a method of extending the momentum basis to the complex plane, which gives a more complete description of resonances. 
The \He{5} problem is then examined once more using this new tool and we fit the model parameters to experimental resonance data.

\section{The Complex Momentum Plane}

When solving the Schrödinger equation in the momentum basis, we know that the solutions form a complete basis, expressed as a completeness relation
\begin{eq}
  \sum\sub{bound} \ket{E_n}\bra{E_n} + \fint[0][\inf]{k} k^2 \ket{E_k}\bra{E_k} = 1,
\end{eq}
where $E_n$ are discrete bound states and $E_k$ are continuous.

If we relate the energies $E$ of a system to momenta $k$ as
\begin{eq}
  E = \frac{\hbar^2 k^2}{2\mu}
  \quad\quad
  \textup{or}
  \quad\quad
  k = \frac{\sqrt{2\mu E}}{\hbar},
\end{eq}
we can plot the solutions as $k$ in the complex plane, see 
\cref{fig:complex plane}. 
We then expect bound states, with $E<0$, to be represented by discrete $k$ along the imaginary axis---whereas unbound, scattering states, with $E>0$, are found continuously along the real axis. 
Resonance states, with complex $E = E_0 - i \Gamma /2$, would by this argument appear somewhere in the fourth quadrant. Because the root is a multi-valued function, each state will have a mirror image in the imaginary axis.

\begin{figure}
  \tikzset{
    triangle/.style={regular polygon, regular polygon sides=3},
    nosep/.style={inner sep=0},
    bound/.style={circle,draw,minimum size=2mm,nosep},
    unbound/.style={rectangle,draw,minimum size=2mm,nosep},
    quasibound/.style={triangle,draw,minimum size=2.5mm,nosep}
  }
  \subfloat[]{
  \label{fig:simple_contour}
  %\tikzset{external/remake next}
\tikzsetnextfilename{simple_contour}
  \begin{tikzpicture}[scale = 2.5]
    \draw[->] (-1.2, 0) -- (1.2, 0) node[right] {$\Re k$};
    \draw[->] (0, -0.5) -- (0, 1.2) node[above] {$\Im k$};
    \foreach \y in {0.1, 0.3}
      \node at (0, \y) [bound] {};
    \foreach \x in {0.25, -0.25}
      \node at (\x, -0.15) [quasibound] {};
    \draw[very thick, mid arrows] (0, 0) -- (1,0)
       {[dashed]
         arc (0:90:1) arc (90:180:1) -- cycle};
  \end{tikzpicture}
  }
  \subfloat[]{
  \label{fig:berggren_contour}
  %\tikzset{external/remake next}
\tikzsetnextfilename{berggren_contour}
  \begin{tikzpicture}[scale = 2.5]
    \draw[->] (-1.2, 0) -- (1.2, 0) node[right] {$\Re k$};
    \draw[->] (0, -0.5) -- (0, 1.2) node[above] {$\Im k$};
    \foreach \y in {0.1, 0.3}
      \node at (0, \y) [bound] {};
    \foreach \x in {0.25, -0.25}
      \node at (\x, -0.15) [quasibound] {};
    \draw[very thick, mid arrows, radius=1]
      (1, 0) arc [start angle=0,  end angle=90]
             arc [start angle=90, end angle=180]
             -- (-0.5, 0) 
             -- (-0.25, 0.25) 
             -- (0.25, -0.25)
             -- (0.5, 0)
             -- cycle;
  \end{tikzpicture}
  }
  \caption{The complex $k$-plane. The circles represent 
  bound states and the triangles resonant states. Note the 
  mirroring of the states in the imaginary axis.}
  \label{fig:complex plane}
\end{figure}

We now interpret the bound and resonant $k$ as complex poles and the unbound continuum as a contour, mirrored in the imaginary axis, encircling the upper half plane (\cref{fig:simple_contour}).
The integral to be evaluated along the contour is the one in the radial momentum space Schrödinger equation
\begin{eq}
  \frac{k^2}{2\mu}\phi(k) + \fint[0][\inf]{k'} k'^2 V(k,k') \phi(k') 
  &=
  E\phi(k).
\end{eq}
The result of a contour integration depends on the poles it encircles by the residue theorem. 
The contour in \cref{fig:simple_contour} encircles the bound states, but not the resonance.
We suspect a deformation of the contour, such that it encircles the resonance might have an effect on the solutions.

In fact, this is correct, and was proven in 1968 by Berggren \cite{berggren}. 
The contour, dubbed $L_+$, can be deformed to surround the resonance poles. 
The continuum states along $L_+$ combined with the encircled bound and resonant states form a complete basis. The result can be stated succinctly as the \emph{Berggren} completeness relation
\begin{eq}
  \sum_{\substack{\text{bound} \\ \text{resonant}}} \ket{E_n}\bra{E_n} 
  + \fint[L_+]{k} k^2 \ket{E_k}\bra{E_k} = 1.
\end{eq}


Consider the radial momentum space Schrödinger equation
\begin{eq}
  \frac{k^2}{2\mu}\phi(k) + \fint[0][\inf]{k'} k'^2 V(k,k') \phi(k') 
  &=
  E\phi(k),
\end{eq}
specifically the integral along the positive real axis. 
We now make a leap of faith and extend the integration contour into the complex plane as in \cref{fig:complex_plane}.
\todo[inline]{leap of faith?}

% \begin{figure}
%   \centering
%     \begin{tikzpicture}[scale = 2.5]
%       \draw[->] (-1, 0) -- (1.1, 0) node[below] {$\Re k$};
%       \draw[->] (0, -1) -- (0, 1) node[below right] {$\Im k$};
%       \draw[very thick, mid arrows] (0,0) -- (1, 0);
%       
%       \draw[thick] (1.3, 0.2) edge[out=45, in=135, ->] (2,0.2);
%       
%       \begin{scope}[shift={(3.2,0)}]
%         \draw[->] (-1, 0) -- (1.1, 0) node[below] {$\Re k$};
%         \draw[->] (0, -1) -- (0, 1) node[below right] {$\Im k$};
%         \draw[very thick, mid arrows] 
%           (0,0) -- (0.25, -0.25) -- (0.5, 0) -- (1, 0);
%       \end{scope}
%     \end{tikzpicture}
%   \caption{}
%   \label{fig:complex_plane}
% \end{figure}



\todo[inline]{where is the complete basis equation?}

This is a major result, meaning that resonances are included as proper solutions to open quantum systems. This will be of importance later, when we consider systems of multiple interacting particles. 
In this Chapter, we will instead focus on using contours in the complex momentum plane to find the resonances of \He{5}.

To find the resonance with its complex energy as a solution, it is required that a contour is chosen such that it passes below the momentum corresponding to the resonance. 
This follows from Cauchy's residue theorem, and the fact that the resonance corresponds to a pole in the complex plane
\footnote{See standard quantum mechanics textbooks on scattering theory for more detail, e.g. \cite{sakurai}}.
A sample contour that could be used is visualized in \cref{fig:berggren contour}.
Note that the solutions are mirrored in the imaginary axis. It can be shown that this symmetry allows us to restrict our attention to the segment between $0$ and $+\infty$.\cite{berggren}

A curious, but important, fact that follows from this procedure is that the standard scalar product (and thus the norm)
\begin{eq}
  \braket{\phi}{\phi'} = \fint[0][\inf]{k} k^2 \conj{\phi(k)}\phi'(k)
\end{eq}
is replaced with
\begin{eq}
  \braket{\phi}{\phi'} = \fint[0][\inf]{k} k^2 \phi(k)\phi'(k)
\end{eq} 
without the conjugation. 



\section{The Complex Contour}
We choose to extend our integration along the real axis to 
the simplest possible complex contour, a triangle-shaped downward extrusion. 
The tip of the triangle is placed below the hypothesized resonance.

We are faced with the problem of evaluating this contour integral numerically. 
We can solve this by considering each straight segment of the contour separately,
and write it as a sum of three integrals, each treated with Gauss-Legendre quadrature according to \cref{app:gauss-legendre}.
An example of a contour is seen in \cref{fig:triangle_contour}, marking each quadrature point where the integrand is evaluated. Note the concentration of points near the ends of each segment, resulting from the Gauss-Legendre quadrature (see \cite{gausslegendre}).

With this procedure, we once again get the Schrödinger equation on the same discretized form as \cref{eq:plane_wave_matrix_elements}:
\begin{eq}
  \label{eq:nhqm matrix element}
  H_{ij}' = \frac{k_i^2}{2\mu}\delta_{ij} + \sqrt{w_i w_j}k_i k_j V_{ij},
\end{eq}
but now with complex $k$ and $w$.

%\tikzset{external/remake next}
\tikzsetnextfilename{triangle_contour}
\begin{figure}[H]
  \centering
  \begin{tikzpicture}
    \begin{axis}[
      width = \textwidth,
      height = 7cm,
      xlabel=Re $k$,
      ylabel=Im $k$,
		  axis lines = middle,
      ymax = 0.1,
      enlargelimits,
      only marks,
      ticks = none,
      ]
      \addplot[color=gray] table {../figures/numerical_contour/numerical_contour.data};
    \end{axis}
  \end{tikzpicture}
  \caption{The complex contour used. The points are distributed on each segment according to the Gauss-Legendre quadrature rule.}
  \label{fig:triangle_contour}
\end{figure}

%%%%%%%%%%%%%%%%%%%%%%%%%%VERIFICATION figure
%\tikzset{external/remake next}
\tikzsetnextfilename{verification}
\begin{figure}[H]
  \centering
  \begin{tikzpicture}
    \begin{axis}[
      ymax = 0.1,
      enlargelimits,
      only marks,
	    width = \textwidth, 
  		height = 9cm, 
  		xlabel=Re $k/\b{\si{fm^{-1}}}$, 
  		ylabel=Im $k/\b{\si{fm^{-1}}}$, 
  		axis x line = middle, 
  		axis y line = left, 
  		ymax =  0.04,
  		ymin = -0.12,
  		ytick = {-0.12,-0.08,...,0.04},
  		xtick = {0,0.5,...,3},
  		every axis y label/.style={ 
  			at = {(current axis.above origin)},
	    	anchor = north west, 
  		}, 
  		every axis x label/.style={
  			at = {(current axis.right of origin)}, 
  			anchor = north east, 
  		}, 
  		yticklabel style={/pgf/number format/fixed, /pgf/number format/precision=2}, 
  		every x tick label/.append style = {
  			anchor = south, yshift = 3pt
  		}, 
  		xticklabel style={/pgf/number format/fixed, /pgf/number format/precision=3},
      ]
      \addplot [color=gray] table [x index=0, y index=1]{../figures/verification_momenta/momentum_solutions.data};
	  \addlegendentry{contour points}
	  \addplot table [x index=2, y index=3]{../figures/verification_momenta/momentum_solutions.data};
	  \addlegendentry{solutions}
    \end{axis}
  \end{tikzpicture}
  \caption{The complex contour used and the computed solutions. Notice the resonance in the middle of the triangle.}
  \label{fig:verification}
\end{figure}
\todo[inline]{Mention something about solutions in second figure?}
%%%%%%%%%%%%%%%%%%%%%%%%%%VERIFICATION end

\section{Studying the Resonance}
Armed with this we can continue our study of the \He{5} system. 
Diagonalizing the matrix given by \cref{eq:nhqm matrix element} using a complex contour with vertices $k_0 = 0$, $k_1 = 0.2 - i0.1$, $k_2 = 2.5$ 
with 5 points on each segment, yields the result shown in \cref{fig:verification}.
\todo{tydligare vertices?}
The same Woods-Saxon parameters have been used as before, and we represent the solutions by their momenta, $k=\sqrt{2\mu E}$, corresponding to the eigenvalues. 
We can see that we obtain solutions almost corresponding to the $k$-values used in the discretization, similar to what was found in the real case. 
These are once again interpreted as unbound solutions, not interacting strongly with the potential. 

However, the interesting solution is the one not lying on the contour. If this solution is the resonance, we expect it to be unchanged when the contour is varied. 
Indeed, any contour passing below the resonance should give this solution, with reservation for numerical error. \todo[inline]{cite berggren?}
Solutions from various contours are shown in \cref{fig:pole(cont)}, demonstrating that this is the case.

In addition we can investigate the wavefunctions of these solutions. 
\Cref{fig:mom wavefunctions}\todo{ref ref} shows the momentum distribution of an arbitrary unbound solution and the resonance. We also show the corresponding situation when using the real basis, for comparison. Note how the resonance wavefunction is much more distinguished in the complex basis.  
The unbound solution still correspond to one definite (now complex) momentum. 
On the other hand, the resonance has a broad distribution, reflecting the localized nature of the solution 
(remember Heisenberg --- a broad momentum wavefunction allows a localized position wavefunction). 
In principle, this feature also gives us a way to automatically find the resonance among a large set of solutions: 
simply single out the one with the broadest wavefunction, or equivalently after normalization, with the lowest maximum value.
\todo{move the two-body chapter} 

%%%%%%%%%%%%%%%%%%%%%%%%%WF real/complex FIGURE

\begin{figure}[H]
\centering{
	\subfloat[Wavefunctions in momentum space for solutions along real-axis]{
  		\includegraphics[page=1]{../figures/eigvecs_real_comp/eigvecs.pdf}
	}
	\\
	\subfloat[Wavefunctions in momentum space for solutions along the complex berggren contour]{
  		\includegraphics[page=2]{../figures/eigvecs_real_comp/eigvecs.pdf}
	}
  }
\caption{yada} 
\label{fig:he5_eigvecs}
\end{figure}


%%%%%%%%%%%%%%%%%%%%%%%%%WF real/complex end


We can now properly investigate how the solutions change as the potential is varied. \Cref{fig:pole(V0)} shows how the solutions vary as $V_0$ 
is successively changed from \SI{70}{MeV}, where the potential well can contain a bound state, to \SI{50}{MeV}, where no bound states exist. 
As the potential well grows less attractive, we see that the imaginary part $\Gamma$ of the resonance energy increases, and thus the half-time of the quasi-bound state becomes shorter.  

We may now proceed to fit our model of \He{5} to experimental data of the $p_{3/2}$ and  $p_{1/2}$ resonance provided by \cite{inte_jimmy}, giving values 
\todo{fix citation}
\todo{Make tables? One table \cancel{showing} \emph{to rule} them all?}
$p_{3/2}: E_0 = bla bla, \Gamma = bla bla$

$p_{1/2}: E_0 = bla bla, \Gamma = bla bla$ 

Fitting the parameters $V_0$ and $V\sub{so}$ to these four data points give a best fit when

$V_0 = bla$

$V\sub{so} = bla$

where we get the computed values $bla bla$ for the $p_{3/2}$ resonance and $bla bla$ for the $p_{1/2}$ resonance.

%\tikzset{external/remake next}
\tikzsetnextfilename{pole(V0)}
\begin{figure}
  \centering
  \begin{tikzpicture}
    \begin{axis}[
      width = \textwidth,
      height = 9cm,
        xlabel=Re $k/\b{\si{fm^{-1}}}$,
        ylabel=Im $k/\b{\si{fm^{-1}}}$,
  		  axis x line = middle,
        axis y line = left,
        every axis y label/.style={
          at = {(current axis.above origin)},
          anchor = north west,
        },
        every axis x label/.style={
          at = {(current axis.right of origin)},
          anchor = north east,
        },
 	      yticklabel style={/pgf/number format/fixed,
 	                     /pgf/number format/precision=3},
        every x tick label/.append style = {anchor = south, yshift = 3pt},
        xmax=0.5,
        ytickmax = 0.3, xtickmax = 0.9,
        enlarge y limits,
        no markers,
		     xtickmin = 0.1,
		     xticklabel style={/pgf/number format/fixed,
		     /pgf/number format/precision=3},
        ]
      	\addplot+[color = red, very thick, ->] table  {../figures/res_pole(V0)/poles.data};
      	    \addlegendentry{Pole position}
      	\addplot+[color= gray, very thick] table {../figures/res_pole(V0)/contour.data};
          \addlegendentry{Contour}
      \end{axis}
  \end{tikzpicture}
  \caption{The pole position as a function of $V_0$.}
  \label{fig:pole(V0)}

\end{figure}
\todo{make beautiful}
%%%%%%%%%%%%%%%%%%%%%%%%%V0 (CONTOUR) FIGURE
  
\begin{figure}
   \centering{
   \subfloat[Square Contour]{
   \includegraphics[page=1]{../figures/poles(contour)/poles.pdf}
   }
   \\
   \subfloat[Triangle Contour]{
   \includegraphics[page=2]{../figures/poles(contour)/poles.pdf}
   }
   }
  \caption{yadayoda} 
  \label{fig:pole(cont)}  
\end{figure}

\todo{figures where they belong. Wednesday night: fixa uppslag med figur + tillhörande text}

%%%%%%%%%%%%%%%%%%%%%%%%%V0 (CONTOUR) end


%%%%%%%%%%%%%%%%%%%REAL CONTOUR FIGURE

\begin{figure}[H] %this figure needs to be nudged a little bit to the left
   \centering{
     \subfloat[Real Contour \SI{-70}{MeV}]{
	 	\includegraphics[page=1]{../figures/poles(realcontour)/poles.pdf}
	 }
	 \subfloat[Real Contour \SI{-50}{MeV}]{
		 \includegraphics[page=1]{../figures/poles(realcontour)/poles.pdf}
	 }
   }
   \caption{momentum solutions to the Schrödinger equation ``along'' the real axis for \He{5} with a potential of $V_0 = \SI{-70}{MeV}$ (a) and $V_0 =$ \SI{-50}{MeV} (b).} 
   \label{fig:pole real contour}  
 \end{figure}

 %%%%%%%%%%%%%%%%%%%REAL CONTOUR FIGURE end
 \todo{define pole - momentum solution}

\end{document}
