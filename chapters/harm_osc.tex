The Helium nuclei we want to expand are, for our intents and purposes, spherically symmetric. 
A spherically symmetric basis is therfore preferable, and we begin with the spherical harmonic oscillator (HO).
The treatment below is adapted from \cite{moshinsky}.

\todo{Too much detail in this section?}
We have for the HO, the Hamiltonian
\begin{eq}
  \label{eq:HO hamiltonian}
  H\sub{HO} = \frac{p^2}{2\mu} + \frac{\mu\omega^2 r^2}{2},
\end{eq}
where $\mu$ is the mass of the problem and $\omega$ is the angular frequency of the oscillator. With this Hamiltonian, the TISE has the solutions
\begin{eq}
  H\sub{HO}\ket{nlm} = E_{nl}\ket{nlm} % E_{nl}???
\end{eq}
with
\begin{eq}
  E_{nl} = \hbar\omega(2n + l + \frac{3}{2}), \quad n = 0, 1, 2, \dots
\end{eq}
Since they are eigenstates of $H\sub{HO}$, $\ket{nlm}$ form a complete basis, and it is this basis we will expand our TISE in. 
The procedure is the same as in \cref{eq:expand1,eq:expand2,eq:expand3} and gives us
\begin{eq}
  \label{eq:HOexpanded}
  \sum_{n'l'm'} \bra{nlm} H \ket{n'l'm'} \psi_{n'l'm'} = E \psi_{nlm}\,.
\end{eq}
Since we are considering a spherically symmetric Hamiltonian
\begin{eq}
  \label{eq:spherical hamiltonian}
  H = \frac{p^2}{2\mu} + V(r),
\end{eq}
the $l$ and $m$ in $\ket{nlm}$ commute with the Hamiltonian and will therefore contribute with a $\delta_{ll'}\delta_{mm'}$ factor. 
This means we can drop the primes and write \cref{eq:HOexpanded} as
\begin{eq}
  \sum_{n'} \bra{nlm} H \ket{n'lm} \psi_{n'lm} = E\psi_{nlm}.
\end{eq}
If we see $\bra{nlm} H \ket{n'lm} = H_{nn'}$ as a matrix, this can be expressed as {\it $H$ is diagonal in $l$ and $m$}.

Thus we have a matrix equation, but we need to find the matrix elements $\bra{nlm} H \ket{n'lm}$. 
These require some calculation (see \cref{sec:HO matrix elements} for the details) and the result is
\begin{eq}
  \label{eq:HO matrix elements}
  &
  \bra{nlm} H \ket{n'lm} =
	\frac{\hbar\omega}{2}
	\left(
    \p{2n+l+\frac{3}{2}} \delta_{nn'}
    +
		\sqrt{n(n+l+\frac{1}{2})} \delta_{n,n'-1}\right.
		\\ & + 
		\left.\sqrt{n'(n'+l+\frac{1}{2})} \delta_{n',n-1} 
	\right)
	+
	\fint[0][\inf]{r} 
    r^2 R_{nl}(r) V(r) R_{n'l}(r)
\end{eq}
where $R_{nl}$ are the radial wavefunctions of the harmonic oscillator
\begin{eq}
  \label{eq:HO radial wavefunction}
	R_{nl}(r) 
	= 
  \sqrt{\frac{
    2^{l+2} \gamma^{l + 3/2} (n - l)!!
  }{
    \sqrt\pi (n + l + 1)!!
  }}
	r^l e^{-\gamma r^2 / 2}
	L_{(n-l) / 2}^{(l+\frac{1}{2})}(\gamma r^2),
\end{eq}
$\gamma = \mu\omega/\hbar$ and $L_\mu^\nu(x)$ are the generalized Laguerre polynomials.
\todo{Doublecheck normalization constant.}
The real space wavefunction $R(r)$ for a state will be expressed as a linear combination of the harmonic oscillator radial wavefunctions:
\begin{eq}
  R(r) = \sum_n \psi_{nlm} R_{nl}(r).
\end{eq}

\subsection{Numerical Considerations}
\todo{Is this really needed?}
Since the HO basis is a discrete basis, adapting the formulas for calculation on a computer is straightforward. 
There are a few considerations, however, and we mention them here. 

The $\ket{nlm}$ basis is infinite in $n$, but we need a finite matrix, so we truncate the basis at a finite number $N$, giving us an $N \times N$ matrix.
Since the matrix is diagonal in $l$ and $m$, we do the calculation separately for each value of $l$ and $m$ to reduce the amount of computation needed to solve for the eigenvalues.
The equation we are solving is then, for a given $l$ and $m$,
\begin{eq}
  \sum_{n'= 0}^N \bra{n} H \ket{n'} \psi_{n'} = E\psi_{n},
\end{eq}
or in linear algebra notation
\begin{eq}
  H\psi = E\psi.
\end{eq}
This matrix equation is solved using a standard eigensolver algorithm, which uses the fact that the matrix is hermitian to solve the equation faster.

We calculate the matrix elements using \cref{eq:HO matrix elements}. 
The integration is performed using Gauss-Legendre quadrature and setting the upper limit to a finite number.
