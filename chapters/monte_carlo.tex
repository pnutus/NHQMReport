Another way to determine the ground state energy of the systems at hand is through a statistical Monte Carlo simulation. 
Instead of a rigorous examination of how all the many particle states interact with each other we strive to take a couple of state at random and see what energy only their interaction give. 
This proedure is then repeated until we can determine the ground state energy of the system with a statistical confidence.

In practice we generate a set of random points along the contour at hand, usually one per segment with a uniform probability. 
These points for the contour in the momentum space and we use the tools of \cref{cha:donald duck} to calculate the hamiltonian and find the lowest energy. 
We then calculate the average of these values (weighed by 0 if they differ by more than 0.5\% from the sought value) and we find that our Monte Carlo simulation quickly conveges to the correct value with a 99.5\% confidence.

Great Success
Problems: converges slowly and the imaginary part is not vanishing.

Different avenues to explore would be to use another probability density along the segments, more points per segment or 

 maybe something akin the points used in Gauss-Legendre integration