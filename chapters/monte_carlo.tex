\documentclass[../main/report.tex]{subfiles}
\begin{document}

\chapter{Monte Carlo Method}
\label{cha:monte_carlo}

The combination of matrix diagonalisation and vast matrices is not advantageous.
A way to deal with this manner is through \emph{Monte Carlo Simulations}.
Instead of a rigorous examination of how all particle states interact with each other, we chose only some at random and calculate the energies they give rise to.
This procedure is then repeated until we can determine the ground state energy with confidence.

We tried this Monte Carlo method to calculate the energies for the \He{6} problem.
The procedure starts with the computation of the energies and wavefunctions for \He{5}, since these pose the basis for \He{6}.
Then we take one state from each of the contour segments at random, which in principle is one point from each third of the contour points.
The resonance state is also included in the basis because this is the state that contributes the most towards the final solution.

This basis is then used to calculate the energy eigenvalues in the same manner as before.

By plotting all the energies we get we see that there is a region in the first quadrant with many points that seem to correspond to the  bound state of He{6}, but it has a way too large imaginary part.

\todo[inline]{Interpret the results in a better way.}

\end{document}