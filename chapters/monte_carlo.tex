\documentclass[../main/report.tex]{subfiles}
\begin{document}

\chapter{Monte Carlo Method}
\label{cha:monte_carlo}

The combination of matrix diagonalisation and vast matrices is not advantageous.
A way to deal with this manner is through \emph{Monte Carlo Simulations}.
Instead of a rigorous examination of how all particle states interact with each other, we chose some at random and calculate the energies they give rise to.
This procedure is then repeated until we can determine the ground-state energy with confidence.

We tried this Monte Carlo method to calculate the energies for the \He{6} problem.
The procedure starts with the computation of the energies and wavefunctions for \He{5}, since these pose the basis for \He{6}.
Then we take one state from each of the contour segments at random, which in principle is one point from each third of the contour points.
The resonance state is also included in the basis since this is the state that contributes the most towards the final solution.
These states is then used to calculate the energy eigenvalues in the same manner as before.

By plotting all the energies we get, we see in \cref{fig:m_carlo_J0} that there is a region in the first quadrant with many points that seem to correspond to the  bound state of He{6}, but it has a way too large imaginary part.
Consecutively we see in \cref{fig:m_carlo_J2} that we have also here have a region of gathered points, but this time in the fourth quadrant making us believe that this is the resonance state.
It is indeed the sought after resonance, but it has not converged towards the right value.

By theese facts we deem our implementation of the Monte-Carlo method unusable.
It does not converge towards the right value and adding more randomized points per segment only made it worse, making us believe that we have a bug somewhere in our program when we choose more than one point per segment.
This said, it could still prove to work, but we would recommend other solutions for reducing the matrix dimensions.

\todo[inline]{Interpret the results in a better way.}

\end{document}