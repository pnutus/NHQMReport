\documentclass[../main/report.tex]{subfiles}
\begin{document}

\chapter{Outlook}
\label{cha:outlook}

We have studied resonances in heavy Helium isotopes and given heuristic justifications for the methods used. The \He{5} nucleus was modeled satisfactorily, and we could reproduce the experimental width and XXX of its resonances. 

We were not as successful in modeling the \He{6} nucleus. The interactions used were simple first approximations, and while the ground state was reproducible, we were unable to. Consequently, an obvious way of improving upon our results is to use a more realistic potential. 

Other areas to further explore, covered in more detail below, are: increasing the number of particles in the system, reducing the computational complexity through renormalization techniques, combining the complex basis with another type of shell model etc.
The Monte Carlo method shows some promise, but we suggest looking into other possibilites.
\todo{any more ideas?}

\section{Increased Number of Particles}

We have studied the nuclei \He{5} and \He{6}, a two- and three-body problem respectively. 
A natural extension of these problems is to add more particles, either neutrons of protons.
In the case of neutrons, we end up with even heavier \He{} isotopes,  \He{7} and \He{8}.
We then have have to consider the angular momentum coupling of three or four bodies, but other than that the techniques we employ are general.
\todo[inline]{how general?}
Adding protons, we have to consider the different isospin of the two types of nucleon.
This has implications for the nucleon-nucleon interaction, discussed below.

Another possibility is to study other elements than helium. \ce{^{16} O} and \ce{^{24} O} are good candidates, since they too are doubly magic nuclei.
We could study other light elements like \ce{Li} or \ce{Be}, as they display several interesting properties of open quantum systems.

When expanding \He{6}, our basis consists of only $p_{1/2}$ and $p_{3/2}$ waves. 
This is a good approximation for \He{6} \cite{gamow_shell_model_2008},  but for other systems, we have to include more partial waves (sober physicists ...) in the expansion.

Another way of describing different cores would be to use a \emph{No-Core Shell Model} (NCSM), i.e. treating each nucleon as separate and starting with \ce{^2 H}, just one neutron and proton. This approach would only let us study very light nuclei, but the complex energy treatment could prove interesting.

\section{Reducing Computation Time}

Renormalization is a way to decrease computational time by only including the most important states in the single-particle basis.
There are currently several methods of doing this. Two of them are the Similarity Renormalization Group (SRG) and Density Matrix Renormalization Group (DMRG) \cite{DMRG}.
The point of only selecting a few of the single-particle states is that it reduces the size of the matrix, hence the decreased computational time.
This is close to what we did with the Monte Carlo method.

Another way to decrease computational time is the Monte-Carlo simulations we did in \cref{cha:monte_carlo}.
The problem is to get it to work.
In our simulations we did not see neither the resonances in \ce{^{5,6} He}, nor did we find the bound state in \He{6}.
One theory is that it would show results in larger systems or if we would include more basis states in each iteration, but in our case, this would not be faster than the original algorithm.

\section{Realistic Interactions}

QCD yada yada cite jimmy/christian etc.



\end{document}