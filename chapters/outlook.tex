\documentclass[../main/report.tex]{subfiles}
\begin{document}

\chapter{Outlook}
\label{cha:outlook}

\begin{itemize}
  \item Recap/discussion (very short, mostly for red thread)
  \begin{itemize}
    \item We have been able to find resonances (albeit not very well)
    \item We have explained the procedure
    \item (We have debunked the monte carlo myth)
  \end{itemize}
  \item Renormalization
  \begin{itemize}
    \item Renormalization is a way to decrease computational time by only including the most important states in the single-particle basis \cite{jimmy}. This can be done in several different ways, some more advanced than others. By only selecting a few of the single-particle states we can reduce the size of the matrix, hence the decreased computational time.
    \todo[inline]{Cite Jimmy...}
    \item Reducing the size of the many-body matrices by only keeping the most important states in the sp basis
    \begin{itemize}
      \item cite jimmy
    \end{itemize}
    \item Another way to decrease computational time is the Monte-Carlo simulations we did in \cref{cha:monte-carlo}. The problem is to get it to work. In our simulations we did not see neither the resonances in \ce{^{5,6} He} nor the bound state in \He{6}. One theory is that it would show results in larger systems or if we would use more states in each iteration, but then it would not be faster than the original algorithm in our case.
    \item Other ways of using monte carlo.
    \item Monte Carlo for larger systems.
  \end{itemize}

  \item More nucleons
  \begin{itemize}
    \item We have confined us to the nuclea \He{5} and \He{6} in this report and one natural way to proceed is to continuously add more neutrons to the alpha particle. This would primarily mean the \He{7} and \He{8}. Heavier cores than theese would probably need a significantly faster computer than those we have access to.

   \item There are of course more interesting nuclei than Helium to study, which makes the study of other cores an interesting way to continue. One core that also is interesting is \ce{^{16} O} since it also has a doubly magical core that can be treated as a fixed point. Other than that we could study Li or Be since they displays several interesting boromean and resonance phenomena.
   \todo[inline]{Not shure of last sentence.}

   \item Our method for describing \He{6} is currently consisting of only $p_{1/2}$ and $p_{3/2}$ waves, but we believe that in order to study it more carefully or to study different cores we must increase which waves we include in our expansion. This would be waves in the s, p, d, f, ... shells.

   \item Another way of describing different cores would be to start with only nucleons, ie only use protons and neutrons as base elements. We do not know if this is a good approach, but it would be interesting to try and see if the result is the same, better or worse. It would in either case demand a lot of computational power. This is probably a good reason to use a core and just add neutrons to it since this reduces the size of the problem.
  \end{itemize}

  \item More realistic interaction
  \begin{itemize}
   \item We could also get a better result by using a more realistic potential, this would then be ............... which describes ... in a better way since ... .
    \item Cite stuff, maybe ask jimmy/christian
  \end{itemize}
\end{itemize}


\end{document}