\documentclass[../main/report.tex]{subfiles}
\begin{document}

\chapter{Outlook}
\label{cha:outlook}

We have studied resonances in Helium isotopes and given heuristic justifications for the methods used. The \He{5} nucleus was modeled satisfactorily, and we could reproduce the experimental width and position of its resonances.

The Berggren basis obtained from the \He{5} solutions was used to model the \He{6} nucleus. The interactions used were simple first approximations, but while we were unable to reproduce the \He{6} resonance with the Gaussian interaction, the surface delta interaction gave us meaningful results.
Consequently, an obvious way of improving upon our results is to use a more realistic potential. 
We have limited ourselves to a study of the energy spectrum, but one could also explore the \He{6} system in more detail, e.g.~looking at its density distribution \cite{radii}.

Other areas to further explore, covered in more detail below, are: increasing the number of valence particles to study even more exotic systems, reducing the computational complexity through various techniques, combining the complex basis with other bases, better suited for describing bound states.

\section{Realistic Two-Body Interactions}
The phenomenological two-body interactions used in this thesis were chosen because of their simplicity, and there is much room for improvement. 
One approach is to make more educated guesses as to the form of the interaction. 
For example, replacing the delta functions in the SDI by a spread out function could provide a more realistic description of a system. 

However, while these changes would be relatively easy to implement, it would drastically increase computation time of the matrix elements. 
A requirement for doing this would be to further investigate methods of reducing computation times, discussed more below.  

\section{Additional Nucleons and Other Elements}
We have studied the nuclei \He{5} and \He{6} in a few-body picture as a two- and three-body problem consisting of a core and one or two valence neutrons respectively.
A natural extension would be to stay in the same picture, adding more valence neutrons to study the more exotic \He{7} and \He{8} nuclei.
With more particles one has to consider the angular momentum coupling of three or four particles, which is more complicated than the two-body coupling described here. Other than that the techniques we employ are general.

Another possibility is to study other elements than helium using the same model -- a core with valence neutron. \ce{^{16} O} is a good candidate, since it too is a doubly magic nuclei with significantly higher binding energy than neighboring nuclei.
Other light elements like \ce{Li} or \ce{Be} are of interest as well since they, like He, display several interesting properties of open quantum systems. One has to take into consideration that they have a less well-defined core though, making our method less suitable to describe them.

When expanding \He{6}, our basis consists of only $p_{1/2}$ and $p_{3/2}$ waves. 
This is a good approximation for \He{6} \cite{gamow_shell_model_2008},  but for other systems, more partial waves may be needed.
This poses additional challenges, since one needs to introduce a translationally invariant coordinate system when treating particles of mixed parity. 

\section{Reducing Computation Time}
With an increased number of particles, the size of the Hamiltonian matrix grows exponentially. This leads to more matrix element calculations, memory usage requirements and slower diagonalization.

One way to reduce the matrix size is to select only the most important basis states in computations.
This is possible because certain many-body configurations barely give any contribution to the energy of the (quasi-)bound solutions, and they can be neglected. 
An example of a method doing this is the Density Matrix Renormalization Group (DMRG) \cite{DMRG}.

Another, less systematic, approach is the Monte Carlo method of randomly sampling sp basis states to include in the many-body basis, and taking the mean of the energy solutions thus obtained.
We made a minor study of a Monte-Carlo approach, but cannot present any conclusive results.
We want to encourage further research on this topic as the method, while simple, could potentially prove a useful tool to reduce computation times of larger systems.

While a larger system requires more matrix elements, many of the elements are zero. 
As the size of the matrix grows, it is eventually no longer efficient to store them as an array. 
Instead, a \emph{sparse-matrix} representation is employed, which only stores information about the non-zero elements. 

Furthermore, while working with sparse matrices one has to adapt the diagonalization (eigensolver) algorithms. 
The Lanczos algorithm is the most commonly used method for large, sparse matrices.
It generally extracts the largest or smallest eigenvalues, but since resonances can exist in the middle of the spectrum, some adaptations might be required.
It is also important to note that our complex-momentum approach gives us a non-hermitian symmetric matrix while most eigensolvers expect matrices to be hermitian. A possible development is then to research eigensolvers for non-hermitian (but symmetric) matrices.
\end{document}