\todo{The reason... is a bad start of this section. One previous sentence can fix this.}
The reason that we want to solve the problem in momentum space is that we are studying a system with a short-range potential supporting only a few, if any, bound solutions. 
This means that we will find multiple unbound solutions, corresponding to free particles of various energies.
 These are almost eigenstates of the momentum operator, only slightly disturbed by the small potential well at $r=0$. 

The expansion is done in the same way as before, giving us
\begin{eq}
  \int \rd^3 \vec{k}' \bra{\vec{k}} H \ket{\vec{k}'} \Phi(\vec{k}')
  &= 
  E\Phi(\vec{k}) \, .
\end{eq}
This three-dimensional equation is practically unsolvable, but we can simplify it using the fact that we have a central problem. 
A rather involved calculation (see \cref{sec:radial mom space TISE}) shows that the Schrödinger equation can be written as
\begin{eq} 
  \label{eq:radial mom space TISE}
  \frac{k^2}{2\mu}\phi(k) + \fint[0][\inf]{k} k'^2 V(k,k') \phi(k') 
  &=
  E\phi(k) \, ,
\end{eq}
where $\phi(k)$ is the radial part of the momentum space wavefunction, 
\begin{eq}
  V(k,k') 
  &= 
  \frac{2}{\pi}\int_0^\infty \rd r \, r^2 V(r) j_l(kr) j_l(k'r) 
\end{eq}
and $j_l(kr)$ are the spherical bessel functions of order $l$. $\phi(k)$ is related to the real space radial wavefunction by
\begin{eq}
R(r)=i^l\sqrt{\frac{2}{\pi}} \int_0^\infty \rd k \, k^2 \phi(k)j_l(kr).
\end{eq}

\subsection{Discretization}
The integral equation \cref{eq:radial mom space TISE} can be rewritten as a matrix equation through discretization, turning the integral into a sum over a finite set of points $k_j$ and $\rd{k}$ into a set of weights $w_j$:
\todo{Maybe this delta trickery is unnecessary. Do it after discretization.}
\begin{eq}
  \label{eq:discrete_momentum}
  \frac{k_i^2}{2\mu} \phi(k_i)
  +
  \sum_{j=1}^N w_j
    k_j^2 V(k_i,k_j)
  \phi(k_j)
  =
  E \phi(k_i)
  .
\end{eq}
A particular set of points and corresponding weights is called a \emph{quadrature}, and the choice of quadrature greatly impacts the precision of the result. 
A naïve quadrature with evenly spaced $k_j = j\Delta k$ and a constant weight $w_j=\Delta k$ converges slowly, and should not be used. 
We instead use the Gauss-Legendre quadrature \cite{gauss-legendre}.
\todo{Gauss-Legendre appendix?}

With this approximation the Schrödinger equation may be written as
\begin{eq}
  \sum_j H_{ij} \phi_j &= E \phi_i
\end{eq}
where $\phi_i=\phi(k_i)$ and 
\begin{align}
  \label{eq:mom matrix}
  H_{ij} &= \frac{k_i^2}{2\mu}\delta_{ij} + w_jk_j^2 V_{ij} \\
  V_{ij} &= \frac{2}{\pi} \int_0^\infty \rd r \, r^2 V(r) j_l(k_i r) j_l(k_j r)
\end{align}

Because of the $k_j^2$ in the second term of the matrix elements 
(\cref{eq:mom matrix}), the $H_{ij}$ matrix will not be symmetric. 
Working with a symmetric matrix will be faster, hence we perform 
the transformation
\begin{eq}
  \phi_i &\mapsto
  \phi_i' =  \sqrt{w_i} k_i \phi_i
  \\
  H_{ij} &\mapsto
  H_{ij}' 
  = 
  \sqrt{\frac{w_i}{w_j}} \frac{k_i}{k_j}H_{ij},
\end{eq}
which gives us a symmetric matrix
\begin{eq}
  H_{ij}' = \frac{k_i^2}{2\mu}\delta_{ij} + \sqrt{w_i w_j}k_i k_j V_{ij}.
\end{eq}
The Shrödinger equation then becomes
\begin{eq}
  \sum_j H'_{ij}\phi'_j = E\phi'_i \, ,
\end{eq}
with the same eigenvalues $E$, meaning that we can work with the symmetric $H'_{ij}$ matrix.
 The real space radial wavefunction $R(r)$ is expressed in terms of $\phi'_j$ as
\begin{eq}
  R(r)
  =
  i^l 
  \sqrt{\frac{2}{\pi}}
  \sum_{j=1}^N 
    \sqrt{w_j}k_j \phi'_j j_l(k_j r).
\end{eq}


