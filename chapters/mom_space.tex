The reason that we want to solve the problem in momentum space is that we are studying a system with a short-range potential supporting only a few, if any, bound solutions. This means that we will find multiple unbound solutions, corresponding to free particles of various energies. These are basically already eigenstates of the momentum operator, only slightly disturbed by the small potential well at $r=0$. 

The expansion is done in the same way as before, giving us
\begin{eq}
  \int \rd^3 \vec{k}' \bra{\vec{k}} H \ket{\vec{k}'} \Phi(\vec{k}')
  &= 
  E\Phi(\vec{k}) \, .
\end{eq}
This three-dimensional equation is practically unsolvable, but we can simplify it using the fact that we have a central problem. A rather involved calculation (see \cref{sec:radial_mom_space_TISE}) shows that the Schrödinger equation can be written as
\begin{eq} 
  \frac{k^2}{2\mu}\phi(k) + \int_0^\infty \rd k' \, k'^2 V(k,k') \phi(k') 
  &=
  E\phi(k) \, ,
\end{eq}
where $\phi(k)$ is the radial part of the momentum space wavefunction, 
\begin{eq}
  V(k,k') 
  &= 
  \frac{2}{\pi}\int_0^\infty \rd r \, r^2 V(r) j_l(kr) j_l(k'r) 
\end{eq}
and $j_l(kr)$ are the spherical bessel functions of order $l$. We see that the transformation turned the differential equation into an integral equation.

\subsection{Numerical Considerations}
As with the harmonic oscillator basis we want to rewrite the equation as a finite matrix equation. An integral equation can be rewritten as a matrix equation by approximating the integral with a numerical quadrature, 
\begin{eq}
  \label{eq:discrete_momentum}
  \int_0^\infty \rd k' \, k'^2 V(k,k')\phi(k') 
  \approx
  \sum_{j=1}^N w_j k_j^2 V(k,k_j)\phi(k_j)
\end{eq}
where $w_j$ are the quadrature weights. For the na\"{i}ve rectangular quadrature you would use a constant $w_j=\Delta k_j$, equal to the step length. However, this quadrature converges slowly to the correct value of the integral, and much better alternatives can be employed. We are using the G-L quadrature.

With this approximation the Schrödinger equation may be written
\begin{eq}
  \sum_j H_{ij} \phi_j &= E \phi_i
\end{eq}
where $\phi_i=\phi(k_i)$ and 
\begin{eq}
  H_{ij} &= \frac{k_i^2}{2\mu}\delta_{ij} + w_jk_j^2 V_{ij} \\
  V_{ij} &= \frac{2}{\pi} \int_0^\infty \rd r \, r^2 V(r) j_l(k_i r) j_l(k_j r)
\end{eq}
The equation is now written as a matrix equation of order N -- the number of states included in the basis. The energy eigenvalues $E$ will be obtained by calculating the matrix elements $H_{ij}$ and diagonalizing the resulting matrix. Since there is generally no analytic expression for the terms $V_{ij}$, they will need to be evaluated by numerical integration.

To speed up calculations we can transform the equation so that the matrix will be symmetric. This is achieved by the transformation
\begin{eq}
  \phi_i &\mapsto
  \phi_i' =  \sqrt{w_i} k_i \phi_i
  \\
  H_{ij} &\mapsto
  H_{ij}' 
  = 
  \sqrt{\frac{w_i}{w_j}} \frac{k_i}{k_j}H_{ij}
\end{eq}
We would then have
\begin{eq}
  \sum_j H'_{ij}\phi'_j = E\phi'_i\,,
\end{eq}
meaning that the eigenvalues could be just as well obtained by diagonalizing the symmetric $H'$ matrix, thus saving precious time. We have
\begin{eq}
  H_{ij}' = \frac{k_i^2}{2\mu}\delta_{ij} + \sqrt{w_i w_j}k_i k_j V_{ij}
\end{eq}
with $V_{ij}$ defined as above.

