\documentclass[../main/report.tex]{subfiles}
\begin{document}
  
\chapter{The Two-Body Nuclear System}
\label{cha:two-body}

In this chapter we investigate a simple two-body nuclear system using the basis expansion methods from the previous chapter.
We begin by discussing the shell model, the Woods-Saxon potential and our model system, \He{5}.
The Schrödinger equation is then solved in the HO and plane wave bases.
The solutions are studied by looking at the energies and wavefunctions while varying parameters.	

\section{Model System: the \He{5} Nucleus}
\todo{Not sure I like the title. maybe more subsections here? /Pontus}

Typical examples of open quantum systems are encountered in nuclear physics. 
The atomic nucleus is held together by the very short-range strong interaction acting between all nucleons. \todo[inline]{typical? Short rangeD?}
Nuclei are systems that can be studied within a shell model. \todo[inline]{all nuclei?} This is done by introducing a \emph{mean-field} potential, often done in the following way.\cite{suhonen}

Consider the Hamiltonian for a system of $A$ interacting particles,
\begin{eq}
  H = \sum_{i=1}^A \frac{1}{2m_i} \nabla_i^2 
  + 
  \sum_{j<i=1}^A v(\vec{r}_i,\,\vec{k}_i, \vec{r}_j, \vec{k}_j)
\end{eq}
where $v$ is the nucleon-nucleon interaction. Now add and subtract a potential field $V(\vec{r})$ affecting all particles,
\begin{eq}
  H &= \sum_i \b{ 
    \frac{1}{2m_i} \nabla_i^2 + V(\vec{r}_i) 
    }
  + 
  \sum_{j<i} \b{ 
    v(\vec{r}_i,\,\vec{k}_i, \vec{r}_j, \vec{k}_j) - V(\vec{r_i})
  } \\
  &=
  H\sub{MF} + V\sub{res}
\end{eq} 
where we have split the Hamiltonian into a spherically symmetric mean-field Hamiltonian $H\sub{MF}$ in which particles do notinteract directly, and the \emph{residual interaction} $V\sub{res}$ that can be seen as the new interaction between particles. If the mean-field potential $V$ is chosen carefully, $V\sub{res}$ can become small enough to be treated pertubationally (if at all).

This method has had some success in reproducing the general features of nuclides\cite{suhonen}, especially for lighter nuclei ($A<50$). 
It is found that there are magic numbers of nucleons, where the protons or neutrons form complete shells with $0$ angular momentum. 
Of special interest to us are \emph{doubly magic} nuclei, where both proton are neutron numbers are magic. These nuclei can be very tightly bound.
 
Because of this, we will choose to study the \He{5} nucleus as an example of a two-body system. 
The \He{4} nucleus is doubly magic, with two $s_{1/2}$-neutrons and two $s_{1/2}$-protons forming full shells, creating a stable core. 
This core will be treated as a single particle (subsequently referred to as the $\alpha$-particle), interacting with the valence neutron through the mean-field only.

Other doubly magic light nuclei, such as \ce{^{16}O} and its isotopes, have been studied using methods similar to ours\cite{gamow_shell_model_2008}, but we will restrict ourselves to Helium nuclei. 

For mean-field potential we will use the established Woods-Saxon potential, given by
\begin{eq}
	V(r)=
	-V_0f(r) - 4V\sub{so}\vec{l}\cdot\vec{s}\frac{1}{r}\frac{df}{dr} ,
\end{eq}
where 
\begin{eq}
	f(r)=\b{1+\exp\p{\frac{r-r_0}{d}}}^{-1} ,
\end{eq}
and it is visualized in \cref{fig:woods-saxons}).

We will eventually optimize the parameters to match experimentally determined energy levels for \He{5}, but to begin with we will use 
standard values given by \cite{suhonen,dickhoff} to investigate the general behavior of the solutions. These parameters are: 
\begin{center}
\begin{tabular}{r l}
 Interaction strength          & $V_0 = \SI{47}{MeV}$   \\
 Spin-orbit coupling strength  & $V\sub{so} = \SI{-7.5}{Mev}$ \\
 Range                         & $r_0 = \SI{2}{fm}   $  \\
 Diffuseness                   & $d = \SI{0.65}{fm}$  \\ 
\end{tabular}
\end{center}

Note that the spin-orbit coupling term can give either attractive or repulsive contributions, depending on how the angular momenta couples. Recall
\begin{eq}
  \vec{l}\cdot\vec{s} 
  = 
  \frac{1}{2}
  \bigp{
    j(j+1)-l(l+1)-s(s+1)
  }
  =
  \begin{cases}
    l,    &\text{ if } j = l + \frac{1}{2}\\
    -l-1, &\text{ if } j = l - \frac{1}{2}\\
  \end{cases}
  .
\end{eq}
Because the $s$-shell is already filled in \He{4}, the valence neutron of \He{5} will be a $p$-wave, with $l=1$. 
We see that the $p_{3/2}$-wave will get a negative net contribution from the total spin-orbit term, also shown in \cref{fig:woods-saxons}. 
This means that the ground state of \He{5} will be the $p_{3/2}$-wave, with $p_{1/2}$ an excited state.

Since we approximate the system as a spherically symmetric interaction between two particles, 
we can by standard procedure reduce the problem to a one-dimensional equation by using the relative coordinate 
$r = |\vec{r}_\alpha - \vec{r}_n|$ and the reduced mass
\begin{eq}
  \mu = \frac{m_\alpha m_n}{m_\alpha + m_n}.
\end{eq}
We can now proceed to solve the Schrödinger equation with the specified potential.
\todo{use correct mass}
\todo{search replace for backslash ,}

\begin{figure}
	\newcommand{\diff}{0.65}
	\newcommand{\ro}{2}
	\newcommand{\vo}{47}
	\newcommand{\func}{1/(1 + e^((x-\ro)/\diff))}
	\newcommand{\mass}{0.1}
	  \centering{
	  \pgfplotsset{
	    width = 0.45\textwidth, height = 7cm,
      domain = 0.1:9.8, 
      ymin = -47, 
	    ymax = 9,
      xlabel = $r/\b{\si{fm}}$,
      axis x line = middle,
      axis y line = left
	  }
	    \subfloat[$p_\frac{3}{2}$]{
	      \tikzset{external/remake next}
	  \tikzsetnextfilename{p32}
	      \begin{tikzpicture}
		    \begin{axis}[
				ylabel = $V/\b{\si{MeV}}$,
				xmax = 7.9,]
		      \addplot[black] {2/ (2 * \mass * x^2) + \func * (-\vo - 4 * 7.5 * -2 * (\func -1) / (\diff * x )) };
		    \end{axis}
	      \end{tikzpicture}
	    }
	  \subfloat[$p_\frac{1}{2}$]{
	    \tikzset{external/remake next}
	\tikzsetnextfilename{p12}
	    \begin{tikzpicture}
	      \begin{axis}[
			  xmax = 7.9,]
	        \addplot[black] {2/ (2 * \mass * x^2) + \func * (-\vo - 4 * 7.5 * 1 * (\func -1) / (\diff * x ))};
	      \end{axis}
	    \end{tikzpicture}
	  }
	  }
  \caption{The Woods-Saxon potential for different waves, with $V_0 = \SI{47}{MeV}$, $r_0 = \SI{2}{fm}$ and $d = \SI{0.65}{fm}$}
  \label{fig:woods-saxons}
\end{figure}

\section{Studying the \He{5} Schrödinger Equation}
In \cref{cha:basis_expansion} we described the procedure for numerically solving the Schrödinger equation using basis expansion. 
In this section we will apply those methods to find the eigensolutions of our model system. 

\subsection{Harmonic Oscillator Basis}
\todo{Fix paragraph}
The range $r_0$ of the HO potential determines the rate at which the potential increases, but also affects how rapidly the solutions converge, since a basis where the wavefunctions are similar to the solutions will have better convergence. 
\Cref{fig:energies(r0)} shows how the eigensolutions depend on the range of the basis functions.

%%%%%%%%%%%%%%%%%%%%%%%%%%%%%%%%%%%%%% E(r0) FIGURE

\tikzsetnextfilename{E(r0)}
\begin{figure}[ht!]
  \centering
 	\includegraphics[]{../figures/E(r0)/E(r0).pdf}
  \caption{The lowest energy eigenvalues of the \He{5} problem as a function of the HO range $r_0$. We see that one of the states behaves differently from the others.}
  \label{fig:energies(r0)}
\end{figure}

%%%%%%%%%%%%%%%%%%%%%%%%%%%%%%%%%%%%%%% E(r0) end

The solutions all have energies $E>0$, meaning that they are not bound, and thus have unlimited range. 
However, we can clearly distinguish a minimum in energy for the lowest energy solution when the range of the basis is in the region $r_0 \approx \SI{1}{fm}$, corresponding to radii within the nucleus. 
This is a sign that there is a solution to \He{5} localized in this region. 
Furthermore, we see that the energies for the others strictly decrease when the range of the basis increases, further showing their unbound nature.

Because the harmonic oscillator consists only of bound states and we are trying to study unbound states with infinite range, we can not get much further with this method.
We will have to switch to a basis with wavefunctions of infinite reach to properly describe this system.
\todo{what is infinite range?}

\subsection{Plane Wave Basis}
The fact that we have found solutions lying in the positive energy spectrum, solutions with infinite reach, have brought us to consider the plane wave basis. 
Solving the Schrödinger equation in the way prescribed in \cref{cha:basis_expansion} with the Woods-Saxon potential gives us the eigenvalues and momentum eigenfunctions. 
In \cref{fig:he5_eigvecs} we present all the obtained eigenfunctions. 
We especially marked one of the functions in red, clearly standing out among the others and peaking at $k = \SI{0.17}{fm^{-1}}$. 
Let us discuss the physical significance of the obtained solutions.

We see that we have a background of wavefunctions peaking at different values of $k$. Recall the relation between wavefunctions in position and momentum representation, \cref{eq:radial wavefunction}
\begin{eq}
  R(r)=i^l\sqrt{\frac{2}{\pi}} \fint[0][\inf]{k} k^2 \phi(k)j_l(kr).
\end{eq} 
We see that a wavefunction peaking at a single value $k_i$ corresponds to a radial wavefunction
\begin{eq}
  R_i(r) = i^l\sqrt{\frac{2}{\pi}} j_l(k_i r)
\end{eq}
Now, the defining property of the spherical bessel functions are that they are eigensolutions to the Schrödinger equation for free particles with spherical symmetry. 
Thus we may conclude that these peaked solutions correspond to free particles. 
In addition, the fact that we find unbound solutions corresponding to each value of $k$ that we feed into the equation hints at the continuous nature of these states. 

%%%%%%%%%%%%%%%%%%%%%%%%%WF real/complex FIGURE

\begin{figure}[H]
\centering
	\subfloat[Wavefunctions in momentum space for solutions along real-axis]{
  		\includegraphics[page=1]{../figures/eigvecs_real_comp/eigvecs.pdf}
	}

\caption{Momentum probability distributions for all obtained solutions to the Woods-Saxon potential.} 
\label{fig:he5_eigvecs}
\end{figure}
\todo{Move the legend more up to the right?}
\todo[inline]{Should this be here as well?}
%%%%%%%%%%%%%%%%%%%%%%%%%WF real/complex end


In \cref{fig:resonance wavefunction} the radial probability distributions $r^2|R(r)^2|$ are shown for the broadest momentum wavefunction (the one marked in red in \cref{fig:real mom wavefunctions}) together with an unbound state, for different values of the interaction strength $V_0$. The figure demonstrates how we may have bound solutions ($E<0$) if the potential is sufficiently attractive, and how this solution is replaced by a localized but unbound state as the interaction strength decreases. 

%We can see that for systems with a deep potential well (here represented by the $V_0 = \SI{70}{MeV}$ case) will support a bound solution with $E<0$, characterized by a wavefunction that quickly tends to zero outside the potential well. 
%As the strength is decreased, however, the energy of the bound state will increase. 
%When a certain value is reached (in our case around $V_0 = \SI{52}{MeV}$), the potential well no longer supports any bound states. 
%Instead we find a solution with a very localized wavefunction, but with positive energy and a wavefunction that does not tend to zero outside the core. 
%As the interaction strength is further decreased this unique solution becomes less and less localized, as demonstrated in the two last figures ($V0 = \SI{47}{MeV}$, our \He{5} problem, and $V_0 = \SI{40}{MeV}$.

We have now studied the solutions obtained from solving the nuclear two-body problem, with the \He{5} nucleus not supporting any bound states as a model system. Instead we have found a continuum of unbound particles with positive energies, and one unique solution corresponding to a more localized state, but that still has many characteristics of the unbound states. These are the first signs of quasi-bound states --- the sought-after resonance --- but these are described by complex energies, and the Schrödinger equation only gives us real solutions. Therefore we cannot describe them within this framework. Instead we have to move on to a new one: The complex momentum representation. 



%%%%%%%%%%%%%%%%%%%%%%%%% Wavefunctions

\begin{figure}
\centering{
	\subfloat[$V_0=\SI{70}{MeV}$]{
		\includegraphics[page=1]{../figures/wavefunctions/wavefunctions.pdf}
	}
	\subfloat[$V_0=\SI{52}{MeV}$]{
		\includegraphics[page=2]{../figures/wavefunctions/wavefunctions.pdf}	  
	}
	\\
	\subfloat[$V_0 = \SI{47}{MeV}$]{
		\includegraphics[page=3]{../figures/wavefunctions/wavefunctions.pdf}
	}
	\subfloat[$V_0 = \SI{40}{MeV}$]{
	  	\includegraphics[page=4]{../figures/wavefunctions/wavefunctions.pdf}
	}
}



  \caption{
    \He{5} wavefunctions for varying potential depth. 
    Plotted are the unique localized solution (solid) and, for comparison, an arbitrary continuum solution (dashed).
    With a deep potential $V_0 = \SI{70}{MeV}$ there is a strongly bound state, which gets weaker as the potential depth is decreased.
    At $V_0 = \SI{52}{MeV}$ the wavefunction is highly localized, yet the energy value is in the continuum, a sign of resonance.
    There is still a clearly localized state with $V_0 = \SI{47}{MeV}$, but at $V_0 = \SI{40}{MeV}$ it is practically indistinguishable from the other states.
  } 
  \label{fig:resonance wavefunction}
  \end{figure}

%%%%%%%%%%%%%%%%%%%%%%%%% Wavefunctions END


\end{document}