\documentclass[../main/report.tex]{subfiles}
\begin{document}
  
\chapter{The Two-Body Nuclear System}
\label{cha:two-body}

\todo{read through entire chapter OLA SPILL}

In this chapter we investigate a simple two-body nuclear system using the basis expansion methods from the previous chapter.
We begin by discussing the shell model, the Woods-Saxon potential and our model system, \He{5}.
The Schrödinger equation is then solved in the HO and momentum bases.
The solutions are studied by looking at the energies and wavefunctions while varying parameters.	

\section{The Nuclear Shell Model}

\todo{Why open? OLA}
A typical example of an open quantum system is the atomic nucleus encountered in nuclear physics. 
The atomic nucleus is held together by the short-ranged strong interaction acting between all nucleons and is commonly studied within a shell model.
This is done by introducing a \emph{mean-field} potential, often in the following way.\cite{suhonen}

Consider the Hamiltonian for a system of $A$ interacting particles,
\begin{eq}
  H = \sum_{i=1}^A \frac{1}{2m_i} \nabla_i^2 
  + 
  \sum_{j<i=1}^A v(\vec{r}_i,\vec{k}_i, \vec{r}_j, \vec{k}_j)
\end{eq}
where $v$ is the nucleon-nucleon interaction. Now add and subtract a potential field $V(\vec{r})$ affecting all particles,
\begin{eq}
  H &= \sum_i \b{ 
    \frac{1}{2m_i} \nabla_i^2 + V(\vec{r}_i) 
    }
  + 
  \sum_{j<i} \b{ 
    v(\vec{r}_i,\vec{k}_i, \vec{r}_j, \vec{k}_j) - V(\vec{r_i})
  } \\
  &=
  H\sub{MF} + V\sub{res}
\end{eq} 
where we have split the Hamiltonian into a spherically symmetric mean-field Hamiltonian $H\sub{MF}$ in which particles do not interact directly, and the \emph{residual interaction} $V\sub{res}$ that can be seen as the new interaction between particles. If the mean-field potential $V$ is chosen carefully, $V\sub{res}$ can become small enough to be treated pertubationally (if at all).
 
\subsection{The Woods-Saxon Potential}
For the mean-field potential we will use the established Woods-Saxon potential, visualized in \cref{fig:woods-saxons}, given by
\begin{eq}
	V(r)=
	-V_0f(r) - 4V\sub{so}\vec{l}\cdot\vec{s}\frac{1}{r}\frac{df}{dr},
\end{eq}
where 
\begin{eq}
	f(r)=\b{1+\exp\p{\frac{r-r_0}{d}}}^{-1}.
\end{eq}
The parameters are the potential depth $V_0$, the spin-orbit coupling strength $V\sub{so}$, the range $r_0$ and the diffuseness $d$. 
There are approximate formulas for these values, depending on the number of each kind of nucleon. 
Alternatively, one can fit the parameters to reproduce experimental energy levels.
We will use both approaches.

Note that the spin-orbit coupling term can give either attractive or repulsive contributions, depending on how the angular momenta couples. Recall that
\begin{eq}
  \label{eq:spin-orbit}
  \vec{l}\cdot\vec{s} 
  = 
  \frac{1}{2}
  \bigp{
    j(j+1)-l(l+1)-s(s+1)
  }
  =
  \begin{cases}
    l,    &\text{ if } j = l + \frac{1}{2}\\
    -l-1, &\text{ if } j = l - \frac{1}{2}\\
  \end{cases}
  ,
\end{eq}
where we have stated the result in the case of one valence nucleon.

\subsection{Magic Nuclei}

The shell model has had some success in reproducing the general features of nuclides\cite{suhonen}, especially for lighter nuclei ($A<50$). 
It is found that there are magic numbers of nucleons, where the protons or neutrons form complete shells with $0$ total angular momentum. 
Of special interest to us are \emph{doubly magic} nuclei, where both proton and neutron numbers are magic. 
These nuclei can be very tightly bound, and will therefore interact weakly with nucleons in outer shells.
If we add nucleons to a doubly magic nuclei, we can thus treat it as a rigid \emph{core}, interacting with valence neutrons through the mean-field only.

We will consider the special case of core and one valence neutron, two particles with a spherically symmetric potential. We can then perform the standard procedure of reducing the problem to a one-dimensional equation by using the relative coordinate 
$r = |\vec{r}_\alpha - \vec{r}_n|$ and the reduced mass
\begin{eq}
  \mu = \frac{m_\alpha m_n}{m_\alpha + m_n}.
\end{eq}

\todo{search replace for backslash , SPILL}

\begin{figure}
	\newcommand{\diff}{0.65}
	\newcommand{\ro}{2}
	\newcommand{\vo}{47}
        \newcommand{\so}{-7.5}
	\newcommand{\func}{1/(1 + e^((x-\ro)/\diff))}
	\newcommand{\mass}{0.019272}
	  \centering{
	  \pgfplotsset{
	    width = 0.45\textwidth, height = 7cm,
      domain = 0.5:9.8, 
	  xmax=10,
      ymin = -20,
	    ymax = 9,
      xlabel = $r/\b{\si{fm}}$,
      axis x line = middle,
      axis y line = left
		%       every axis x label/.style={
		%         at = {(current axis.right of origin)},
		%         anchor = north east,
		% }
	  }
	    \subfloat[$p_{1/2}$]{
	      \tikzset{external/remake next}
	  \tikzsetnextfilename{p12}
	      \begin{tikzpicture}
		    \begin{axis}[
				ylabel = $V/\b{\si{MeV}}$,
				xmax = 7.9,]
		      \addplot[black] {2/ (2 * \mass * x^2) + \func * (-\vo - 4 * \so * (-2) * (\func -1) / (\diff * x )) };
		    \end{axis}
	      \end{tikzpicture}
	    }
	  \subfloat[$p_{3/2}$]{
	    \tikzset{external/remake next}
	\tikzsetnextfilename{p32}
	    \begin{tikzpicture}
	      \begin{axis}[
			  xmax = 7.9,]
	        \addplot[black] {2/ (2 * \mass * x^2) + \func * (-\vo - 4 * \so * 1 * (\func -1) / (\diff * x ))};
	      \end{axis}
	    \end{tikzpicture}
	  }
	  }
  \caption{The Woods-Saxon potential with centrifugal barrier $l(l+1)/2\mu r^2$ added, for different waves. Parameters $V_0 = \SI{47}{MeV}$, $r_0 = \SI{2}{fm}$ and $d = \SI{0.65}{fm}$}
  \label{fig:woods-saxons}
\end{figure}

\todo{ticks, xlabel. caption ok? (deep one should be p3/2).  SPILL}

\section{The \He{5} Nucleus}
We choose to study \He{5}, seen as a \He{4} nucleus and a valence neutron. 
The \He{4} nucleus ($\alpha$ particle) is doubly magic, with two $s_{1/2}$-neutrons and two $s_{1/2}$-protons forming full shells, creating a stable core.  
Other doubly magic light nuclei, such as \ce{^{16}O} and its isotopes, have been studied using methods similar to ours\cite{gamow_shell_model_2008}, but we will restrict ourselves to \He nuclei.

Because the $s$-shell is already filled in \He{4}, the valence neutron of \He{5} will be a $p$-wave, with $l=1$. 
We see from [ref?] that the $p_{3/2}$-wave will get a negative net contribution from the total spin-orbit term, also shown in \cref{fig:woods-saxons}. \todo[inline]{fix reference.}
This means that the ground state of \He{5} will be the $p_{3/2}$ wave, with $p_{1/2}$ an excited state.

Both the $p_{3/2}$ and $p_{1/2}$ waves have known resonances, see \cref{tab:resonance_data}.
\todo[inline]{Forward-reffing tables. Spoiler alert?}

The Woods-Saxon paramaters will initially be set to standard values given by \cite{suhonen,dickhoff} 
\todo{Are they given there? v0 = 51, r0 = 2, d=0.67 enl. suhonen, inga värden i dickhoff}
\todo[inline]{Where are the equation for the Woods-Saxon potential.}
to investigate the general behavior of the solutions. The following values are used: 
\begin{center}
\begin{tabular}{r l}
 Potential depth               & $V_0 = \SI{47}{MeV}$   \\
 Spin-orbit coupling strength  & $V\sub{so} = \SI{-7.5}{Mev}$ \\
 Range                         & $r_0 = \SI{2}{fm}   $  \\
 Diffuseness                   & $d = \SI{0.65}{fm}$  \\ 
\end{tabular}
\end{center}
We later optimize the parameters to match experimentally determined energy levels for \He{5}.

We use the procedures described in \cref{cha:basis_expansion} to numerically solve the \He{5} Schrödinger equation for the $p_{3/2}$ wave. When examining the solutions, we are specifically looking for the ground state resonance.

\todo{More about resonances?}

\subsection{Harmonic Oscillator Basis}

We begin by solving the \He{5} Schrödinger equation in the HO basis. 
We plot the solutions in \cref{fig:energies(r0)} as a function of the range $r_0$ of the HO potential.
All solutions have energies $E > 0$, meaning that they are unbound, scattering states, and thus have unlimited range.
However, we see a saddle point in energy for the lowest energy solution when the HO range $r_0 \approx \SI{1}{fm}$, corresponding to radii within the nucleus.
The range $r_0$ is a measure of the 
\todo[inline]{text??????}
\todo{concis tolkning, bad plot}

%%%%%%%%%%%%%%%%%%%%%%%%%%%%%%%%%%%%%% E(r0) FIGURE

\tikzsetnextfilename{E(r0)}
\begin{figure}[ht]
  \centering
 	\includegraphics[]{../figures/E(r0)/E(r0).pdf}
  \caption{The lowest energy eigenvalues of the \He{5} problem as a function of the HO range $r_0$. The lowest energy state behaves differently from the others.}
  \label{fig:energies(r0)}
\end{figure}

%%%%%%%%%%%%%%%%%%%%%%%%%%%%%%%%%%%%%%% E(r0) end

Because the harmonic oscillator consists only of bound states and we are trying to study unbound states, this method cannot take us much further.
We will have to switch to a basis with wavefunctions of infinite range to properly describe this system.

\subsection{Momentum Basis}
\label{sub:momentum_basis}

The momentum basis describes a plane wave, i.e. a free particle, with wave functions of infinite range.
Because all \He{5} solutions appear in the continuum, this basis is better suited to the problem than the HO basis.

\subsubsection{Momementum Space Wavefunctions}

Solving the Schrödinger equation in the momentum basis gives us momentum eigenfunctions $\phi(k)$, presented in \cref{fig:real_momentum_wavefunctions}.
There is a background of wavefunctions peaking at different values of $k$, and one wavefunction standing out as lower and wider. Let us first discuss the peak-shaped wavefunctions.

%%%%%%%%%%%%%%%%%%%%%%%%%WF real/complex FIGURE

\begin{figure}[h]
  \centering
  	\includegraphics[page=1]{../figures/eigvecs_real_comp/eigvecs.pdf}

  \caption{\He{5} momentum probability distributions. The resonance is significantly wider than the surrounding states. We can see a trend toward lower and wider states around \SI{1}{fm^{-1}}, but it is merely a numerical pecularity stemming from the normalization.} 
  \label{fig:real_momentum_wavefunctions}
\end{figure}

%%%%%%%%%%%%%%%%%%%%%%%%%WF real/complex end

Recall the relation between wavefunctions in position and momentum representation, \cref{eq:radial wavefunction}
\begin{eq}
  R(r)=i^l\sqrt{\frac{2}{\pi}} \fint[0][\inf]{k} k^2 \phi(k)j_l(kr).
\end{eq} 
A wavefunction peaking at a single value $k_i$ corresponds to a radial wavefunction
\begin{eq}
  R_i(r) = i^l\sqrt{\frac{2}{\pi}} j_l(k_i r)
\end{eq}
The spherical bessel functions $j_l(k_i r)$ are eigensolutions to the Schrödinger equation for free particles with spherical symmetry. 
We can thus conclude that the peak-shaped solutions correspond to free particles. 
This is not unexpected as we are dealing with an open quantum system that only marginally perturbs passing particles, explaining why we find unbound solutions corresponding to each value of $k$ used in the discretized basis.

The unique state peaks at $k = \SI{0.17}{fm^{-1}}$ and is  wider than the surrounding states. 
The Heisenberg uncertainty relation tells us that a less well-defined momentum corresponds to a more well-defined position.
Consequently, this state should correspond to a localized wavefunction, which is what we expect from a resonance.

\subsubsection{Position Space Wavefunctions}

To investigate the relation between the hypothesized resonance solution and a bound solution, we increase the depth of the potential to $V_0 = \SI{70}{MeV}$, and successively decrease the depth until there is no bound state. 
This is documented in \cref{fig:wavefunctions}, where the radial probability distributions $r^2|R(r)^2|$ are plotted for the widest momentum wavefunction together with an arbitrary unbound state. 
With a deep potential, the bound state wavefunction quickly tends to zero outside the potential well.
As we decrease the depth below a certain threshold, the potential well no longer supports the bound state ($E > 0$), but the wavefunction remains localized.
This is the region of the \He{5} depth (\SI{47}{MeV}) where we should find the resonance.
When $V_0$ is decreased even further, the resonance disappears.

%%%%%%%%%%%%%%%%%%%%%%%%% Wavefunctions

\begin{figure}
\centering{
	\subfloat[$V_0=\SI{70}{MeV}$]{
		\includegraphics[page=1]{../figures/wavefunctions/wavefunctions.pdf}
	}
	\subfloat[$V_0=\SI{52}{MeV}$]{
		\includegraphics[page=2]{../figures/wavefunctions/wavefunctions.pdf}	  
	}
	\\
	\subfloat[$V_0 = \SI{47}{MeV}$]{
		\includegraphics[page=3]{../figures/wavefunctions/wavefunctions.pdf}
	}
	\subfloat[$V_0 = \SI{40}{MeV}$]{
	  	\includegraphics[page=4]{../figures/wavefunctions/wavefunctions.pdf}
	}
}
  \caption{
    \He{5} wavefunctions for varying potential depth. 
    Plotted are the unique localized solution (thick line) and, for comparison, an arbitrary continuum solution (thin line).
    With a deep potential $V_0 = \SI{70}{MeV}$ there is a strongly bound state, which gets weaker as the potential depth is decreased.
    At $V_0 = \SI{52}{MeV}$ the wavefunction is highly localized, yet the energy lie in the continuum, a sign of resonance.
    There is still a clearly localized state for $V_0 = \SI{47}{MeV}$, but at $V_0 = \SI{40}{MeV}$ it is practically indistinguishable from the other continuum states.
  } 
  \label{fig:wavefunctions}
  \end{figure}

%%%%%%%%%%%%%%%%%%%%%%%%% Wavefunctions END
\todo{fix figure placement SPILL}



\end{document}