\documentclass[../main/report.tex]{subfiles}
\begin{document}
  
\chapter{The Two-Body Nuclear System}
\label{cha:two-body}

In this chapter we investigate a simple two-body nuclear system using the basis expansion methods from the previous chapter.
We begin by discussing the shell model, the Woods-Saxon potential and our model system, \He{5}.
The Schrödinger equation is then solved in the HO and plane wave bases.
The solutions are studied by looking at the energies and wavefunctions while varying parameters.	

\section{The Nuclear Shell Model}
\todo{more subsections here PONTUS}

Typical examples of open quantum systems are encountered in nuclear physics. 
The atomic nucleus is held together by the short-ranged strong interaction acting between all nucleons.
Nuclei are systems that can be studied within a shell model. This is done by introducing a \emph{mean-field} potential, often done in the following way.\cite{suhonen}

Consider the Hamiltonian for a system of $A$ interacting particles,
\begin{eq}
  H = \sum_{i=1}^A \frac{1}{2m_i} \nabla_i^2 
  + 
  \sum_{j<i=1}^A v(\vec{r}_i,\,\vec{k}_i, \vec{r}_j, \vec{k}_j)
\end{eq}
where $v$ is the nucleon-nucleon interaction. Now add and subtract a potential field $V(\vec{r})$ affecting all particles,
\begin{eq}
  H &= \sum_i \b{ 
    \frac{1}{2m_i} \nabla_i^2 + V(\vec{r}_i) 
    }
  + 
  \sum_{j<i} \b{ 
    v(\vec{r}_i,\,\vec{k}_i, \vec{r}_j, \vec{k}_j) - V(\vec{r_i})
  } \\
  &=
  H\sub{MF} + V\sub{res}
\end{eq} 
where we have split the Hamiltonian into a spherically symmetric mean-field Hamiltonian $H\sub{MF}$ in which particles do notinteract directly, and the \emph{residual interaction} $V\sub{res}$ that can be seen as the new interaction between particles. If the mean-field potential $V$ is chosen carefully, $V\sub{res}$ can become small enough to be treated pertubationally (if at all).

This method has had some success in reproducing the general features of nuclides\cite{suhonen}, especially for lighter nuclei ($A<50$). 
It is found that there are magic numbers of nucleons, where the protons or neutrons form complete shells with $0$ angular momentum. 
Of special interest to us are \emph{doubly magic} nuclei, where both proton are neutron numbers are magic. These nuclei can be very tightly bound.
 
Because of this, we will choose to study the \He{5} nucleus as an example of a two-body system. 
The \He{4} nucleus is doubly magic, with two $s_{1/2}$-neutrons and two $s_{1/2}$-protons forming full shells, creating a stable core. 
This core will be treated as a single particle (subsequently referred to as the $\alpha$-particle), interacting with the valence neutron through the mean-field only.\todo{Why can we see the alpha core as a particle through the mean-field approximation.}

Other doubly magic light nuclei, such as \ce{^{16}O} and its isotopes, have been studied using methods similar to ours\cite{gamow_shell_model_2008}, but we will restrict ourselves to He nuclei.

For mean-field potential we will use the established Woods-Saxon potential, given by
\begin{eq}
	V(r)=
	-V_0f(r) - 4V\sub{so}\vec{l}\cdot\vec{s}\frac{1}{r}\frac{df}{dr} ,
\end{eq}
where 
\begin{eq}
	f(r)=\b{1+\exp\p{\frac{r-r_0}{d}}}^{-1} ,
\end{eq}
and it is visualized in \cref{fig:woods-saxons}).

We will eventually optimize the parameters to match experimentally determined energy levels for \He{5}, but to begin with we use 
standard values given by \cite{suhonen,dickhoff} to investigate the general behavior of the solutions. These parameters are: 
\begin{center}
\begin{tabular}{r l}
 Potential depth               & $V_0 = \SI{47}{MeV}$   \\
 Spin-orbit coupling strength  & $V\sub{so} = \SI{-7.5}{Mev}$ \\
 Range                         & $r_0 = \SI{2}{fm}   $  \\
 Diffuseness                   & $d = \SI{0.65}{fm}$  \\ 
\end{tabular}
\end{center}
\todo{beautify ordo sista minuten}

Note that the spin-orbit coupling term can give either attractive or repulsive contributions, depending on how the angular momenta couples. Recall
\begin{eq}
  \vec{l}\cdot\vec{s} 
  = 
  \frac{1}{2}
  \bigp{
    j(j+1)-l(l+1)-s(s+1)
  }
  =
  \begin{cases}
    l,    &\text{ if } j = l + \frac{1}{2}\\
    -l-1, &\text{ if } j = l - \frac{1}{2}\\
  \end{cases}
  .
\end{eq}
Because the $s$-shell is already filled in \He{4}, the valence neutron of \He{5} will be a $p$-wave, with $l=1$. 
We see that the $p_{3/2}$-wave will get a negative net contribution from the total spin-orbit term, also shown in \cref{fig:woods-saxons}. 
This means that the ground state of \He{5} will be the $p_{3/2}$-wave, with $p_{1/2}$ an excited state.

Since we approximate the system as a spherically symmetric interaction between two particles, 
we can by standard procedure reduce the problem to a one-dimensional equation by using the relative coordinate 
$r = |\vec{r}_\alpha - \vec{r}_n|$ and the reduced mass
\begin{eq}
  \mu = \frac{m_\alpha m_n}{m_\alpha + m_n}.
\end{eq}
We can now proceed to solve the Schrödinger equation with the specified potential.
\todo{search replace for backslash , SPILL}

\begin{figure}
	\newcommand{\diff}{0.65}
	\newcommand{\ro}{2}
	\newcommand{\vo}{47}
	\newcommand{\func}{1/(1 + e^((x-\ro)/\diff))}
	\newcommand{\mass}{0.1}
	  \centering{
	  \pgfplotsset{
	    width = 0.45\textwidth, height = 7cm,
      domain = 0.1:9.8, 
      ymin = -47, 
	    ymax = 9,
      xlabel = $r/\b{\si{fm}}$,
      axis x line = middle,
      axis y line = left
	  }
	    \subfloat[$p_{3/2}$]{
	      \tikzset{external/remake next}
	  \tikzsetnextfilename{p32}
	      \begin{tikzpicture}
		    \begin{axis}[
				ylabel = $V/\b{\si{MeV}}$,
				xmax = 7.9,]
		      \addplot[black] {2/ (2 * \mass * x^2) + \func * (-\vo - 4 * 7.5 * -2 * (\func -1) / (\diff * x )) };
		    \end{axis}
	      \end{tikzpicture}
	    }
	  \subfloat[$p_{1/2}$]{
	    \tikzset{external/remake next}
	\tikzsetnextfilename{p12}
	    \begin{tikzpicture}
	      \begin{axis}[
			  xmax = 7.9,]
	        \addplot[black] {2/ (2 * \mass * x^2) + \func * (-\vo - 4 * 7.5 * 1 * (\func -1) / (\diff * x ))};
	      \end{axis}
	    \end{tikzpicture}
	  }
	  }
  \caption{The Woods-Saxon potential for different waves, with $V_0 = \SI{47}{MeV}$, $r_0 = \SI{2}{fm}$ and $d = \SI{0.65}{fm}$}
  \label{fig:woods-saxons}
\end{figure}

\todo{change mass, ticks SPILL}

\section{The \He{5} Nucleus}

\todo{read through OLA VINCENT SPILL}

\todo{We want to find p3/2 resonance, and we know its parameters}

In \cref{cha:basis_expansion} we described the procedure for numerically solving the Schrödinger equation using basis expansion. 
In this section we will apply those methods to solve the Schrödinger equation and subsequently study the solutions, specifically searching for the known resonance
\begin{eq}
  bla.
\end{eq}


\subsection{Harmonic Oscillator Basis}

The range $r_0$ of the HO potential determines the rate at which the potential increases, and thus the range of the HO eigenfunctions.
This affects how rapidly the solutions converge, since a basis with wavefunctions more similar to the solutions will have better convergence. 
\Cref{fig:energies(r0)} shows how the eigensolutions depend on the range of the basis functions.

%%%%%%%%%%%%%%%%%%%%%%%%%%%%%%%%%%%%%% E(r0) FIGURE

\tikzsetnextfilename{E(r0)}
\begin{figure}[t]
  \centering
 	\includegraphics[]{../figures/E(r0)/E(r0).pdf}
  \caption{The lowest energy eigenvalues of the \He{5} problem as a function of the HO range $r_0$. The lowest energy state behaves differently from the others.}
  \label{fig:energies(r0)}
\end{figure}

%%%%%%%%%%%%%%%%%%%%%%%%%%%%%%%%%%%%%%% E(r0) end

\todo{use 47, more points between 1 and 2 fm VINCENT}

\todo{redo figure different $V_0 = 47$ VINCENT}

\todo{what is infinite range? OLA}

All solutions have energies $E > 0$, meaning that they are unbound, scattering states, and thus have unlimited range. 
\todo{minimum/saddle/plateau?}
However, we can clearly distinguish a saddle point in energy for the lowest energy solution when the range $r_0 \approx \SI{1}{fm}$, corresponding to radii within the nucleus.
This is a sign that there is a solution localized in this region.
We guess that this is the resonance state.

Because the harmonic oscillator consists only of bound states and we are trying to study unbound states, this method cannot take us much further.
We will have to switch to a basis with wavefunctions of infinite range to properly describe this system.

\subsection{Momentum Basis}

\todo{subsub PONTUS}

The momentum basis describes a plane wave, i.e. a free particle. Because all solutions appear in the continuum, this basis is more suited to the problem.

Solving the Schrödinger equation in the momentum basis gives us momentum eigenfunctions $\phi(k)$, presented in \cref{fig:real_momentum_wavefunctions}.
There is a background of wavefunctions peaking at different values of $k$, and, once again, one of the functions stand out as lower and wider. Let us first discuss the peaked wavefunctions.

%%%%%%%%%%%%%%%%%%%%%%%%%WF real/complex FIGURE

\begin{figure}
  \centering
  	\includegraphics[page=1]{../figures/eigvecs_real_comp/eigvecs.pdf}

  \caption{\He{5} momentum probability distributions. The } 
  \label{fig:real_momentum_wavefunctions}
\end{figure}

%%%%%%%%%%%%%%%%%%%%%%%%%WF real/complex end

Recall the relation between wavefunctions in position and momentum representation, \cref{eq:radial wavefunction}
\begin{eq}
  R(r)=i^l\sqrt{\frac{2}{\pi}} \fint[0][\inf]{k} k^2 \phi(k)j_l(kr).
\end{eq} 
A wavefunction peaking at a single value $k_i$ corresponds to a radial wavefunction
\begin{eq}
  R_i(r) = i^l\sqrt{\frac{2}{\pi}} j_l(k_i r)
\end{eq}
The spherical bessel functions $j_l(k_i r)$ are eigensolutions to the Schrödinger equation for free particles with spherical symmetry. 
We can thus conclude that the peaked solutions correspond to free particles. 
In addition, the fact that we find unbound solutions corresponding to each value of $k$ used in the discretization hints at the continuous nature of these states.

The unique state peaks at $k = \SI{0.17}{fm^{-1}}$ and is slightly wider than the surrounding states. 
The Heisenberg uncertainty relation tells us that a less well-defined momentum corresponds to a more well-defined position.
Consequently, the state should correspond to a localized wavefunction, which is what we expect from a resonance.

To investigate the relation between the hypothesized resonance solution and a bound solution, we increase the depth of the potential to $V_0 = \SI{70}{MeV}$, and successively decrease the depth until there is no bound state. 
This is documented in \cref{fig:wavefunctions}, where the radial probability distributions $r^2|R(r)^2|$ are plotted for the widest momentum wavefunction together with an arbitrary unbound state. 
With a deep potential, the bound state wavefunction quickly tends to zero outside the potential well.
As we decrease the depth below a certain threshold, the potential well no longer supports the bound state ($E > 0$), but the wavefunction is still fairly localized.
This is the region where we find our resonances.
When decreased even further the wavefunction starts to look similar to the unbound states.

\todo{Short summary?}

%%%%%%%%%%%%%%%%%%%%%%%%% Wavefunctions

\begin{figure}
\centering{
	\subfloat[$V_0=\SI{70}{MeV}$]{
		\includegraphics[page=1]{../figures/wavefunctions/wavefunctions.pdf}
	}
	\subfloat[$V_0=\SI{52}{MeV}$]{
		\includegraphics[page=2]{../figures/wavefunctions/wavefunctions.pdf}	  
	}
	\\
	\subfloat[$V_0 = \SI{47}{MeV}$]{
		\includegraphics[page=3]{../figures/wavefunctions/wavefunctions.pdf}
	}
	\subfloat[$V_0 = \SI{40}{MeV}$]{
	  	\includegraphics[page=4]{../figures/wavefunctions/wavefunctions.pdf}
	}
}
  \caption{
    \He{5} wavefunctions for varying potential depth. 
    Plotted are the unique localized solution (solid) and, for comparison, an arbitrary continuum solution (dashed).
    With a deep potential $V_0 = \SI{70}{MeV}$ there is a strongly bound state, which gets weaker as the potential depth is decreased.
    At $V_0 = \SI{52}{MeV}$ the wavefunction is highly localized, yet the energy value is in the continuum, a sign of resonance.
    There is still a clearly localized state with $V_0 = \SI{47}{MeV}$, but at $V_0 = \SI{40}{MeV}$ it is practically indistinguishable from the other states.
  } 
  \label{fig:wavefunctions}
  \end{figure}

%%%%%%%%%%%%%%%%%%%%%%%%% Wavefunctions END

\todo{probability distributions or wavefunctions?}
\todo{dashed/solid SPILL}



\end{document}