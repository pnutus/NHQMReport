\documentclass[../main/report.tex]{subfiles}
\begin{document}
  
\chapter{The Two-Body Nuclear System}
\label{cha:two-body}

In this chapter we investigate a simple two-body nuclear system using the basis expansion methods from the previous chapter.
We begin by discussing the shell model, the Woods-Saxon potential and our model system, \He{5}. 
The Shrödinger equation is then solved in the HO and plane wave bases.
The solutions are studied by looking at the energies and wavefunctions while varying parameters.

\section{Model system -- the \He{5} Nucleus}
%\begin{itemize}
%  \item shell model
%  \item core/valence nucleons
%  \item woods-saxon
%  \item doubly magic nuclei are good
%  \item we can choose O or He, we chose He
%  \item \He{5} specifics (parameters)
%\end{itemize}
We will choose to study the \He{5} nucleus as a concrete\todo{Concrete <=> betong confusion?} example because its solutions have physical significance, and it is a system that can effectively be treated as a two-body system, consisting of the \He{4} nucleus (subsequently referred to as the $\alpha$-particle) and a valence neutron. This is a good approximation, because the $\alpha$ is a tightly bound doubly magic nucleus, where both the neutrons and protons form filled shells, consisting of 2 $s_{1/2}$-neutrons and 2 $s_{1/2}$-protons. The ground state of \He{5} will thus consist of a valence $p_{3/2}$-neutron, with an excited $p_{1/2}$-state. 

Other doubly magic nuclei, such as \ce{^{16}O}, have been studied using similar methods\cite{gamow shell model 2008}, but we will focus on the light Helium nuclei.
%This is common practice and is a corner stone of the nuclear shell model. \todo{maybe a citation here? OLA PONTUS OLA} \todo{why?/cite/explain}

The \He{5} consists of five nucleons, each consisting of three quarks, all interacting with the strong force. The exact interaction of this system is in fact unknown, and we will have to use an effective \emph{mean-field} potential to describe it. These are potentials constructed to approximate the real behavior, with a set of parameters fitted to reproduce experimental data. Throughout this report, we will use the established Woods-Saxon potential to model our systems.
 
The Woods-Saxon potential (shown in \cref{fig:woods-saxons}) is given by
\begin{eq}
	V(r)=
	-V_0f(r) - 4V\sub{so}\vec{l}\cdot\vec{s}\frac{1}{r}\frac{df}{dr}
\end{eq}
where 
\begin{eq}
	f(r)=\b{1+\exp\p{\frac{r-r_0}{d}}}^{-1}.
\end{eq}
We will eventually optimize the parameters to match experimentally determined energy levels for \He{5}, but we will begin with using standard values given by \cite{suhonen,dickhoff} to investigate the general behavior of the solutions. 
These parameters are: 
\begin{align*}
  & \text{Interaction strength}         & V_0       &= \SI{47}{MeV}   \\
  & \text{Spin-orbit coupling strength} & V\sub{so} &= \SI{-7.5}{Mev} \\
  & \text{Range}                        & r_0       &= \SI{2}{fm}     \\
  & \text{Diffuseness}                  & d         &= \SI{0.65}{fm}  \\ 
\end{align*}
where the names of the parameters should be self-explaining. 

Note that the spin-orbit coupling term can give either attractive or repulsive contributions, depending on how the angular momenta couples
\begin{eq}
  \vec{l}\cdot\vec{s} 
  = 
  \frac{1}{2}
  \bigp{
    j(j+1)-l(l+1)-s(s+1)
  }
  =
  \begin{cases}
    l,    &\text{ if } j = l + \frac{1}{2}\\
    -l-1, &\text{ if } j = l - \frac{1}{2}\\
  \end{cases}
  .
\end{eq}

Since we approximate the system as a spherically symmetric interaction 
between two particles, we can by standard procedure reduce the problem to a one-dimensional equation by using the relative coordinate $r = |\vec{r}_\alpha - \vec{r}_n|$ and the reduced mass \todo{why?/cite/explain}
\begin{eq}
  \mu = \frac{m_\alpha m_n}{m_\alpha + m_n}.
\end{eq}
We can now proceed to solve the Schrödinger equation with the specified potential.


\begin{figure}
	\newcommand{\diff}{0.65}
	\newcommand{\ro}{2}
	\newcommand{\vo}{47}
	\newcommand{\func}{1/(1 + e^((x-\ro)/\diff))}
	\newcommand{\mass}{0.1}
	  \centering{
	  \pgfplotsset{
	    width = 0.45\textwidth, height = 7cm,
        domain = 0.1:9.8, 
        ymin = -47, 
		ymax = 9,
        xlabel = $r/\b{\si{fm}}$,
        axis x line = middle,
        axis y line = left
		% 	    xlabel = $r/\b{\si{fm}}$, ylabel = $r^2\absq{R(r)}$,
		% 	    axis x line = bottom,
		% 	    axis y line = left,
		% 	    no markers,
		% 	    ytick = \empty,
		% 	    ymax = 1.7,
		% legend style={anchor=north east,legend columns=1},
	  }
	    \subfloat[$p_\frac{1}{2}$]{
	      \tikzset{external/remake next}
	  \tikzsetnextfilename{p12}
	      \begin{tikzpicture}
		    \begin{axis}[
				ylabel = $V/\b{\si{MeV}}$,
				xmax = 5.9,]
		      \addplot[black] {2/ (2 * \mass * x^2) + \func * (-\vo - 4 * 7.5 * -2 * (\func -1) / (\diff * x )) };
		    \end{axis}
	      \end{tikzpicture}
	    }
	  \subfloat[$p_\frac{3}{2}$]{
	    \tikzset{external/remake next}
	\tikzsetnextfilename{p32}
	    \begin{tikzpicture}
	      \begin{axis}[
			  xmax = 7.9,]
	        \addplot[black] {2/ (2 * \mass * x^2) + \func * (-\vo - 4 * 7.5 * 1 * (\func -1) / (\diff * x ))};
	      \end{axis}
	    \end{tikzpicture}
	  }
	  }
  \caption{The Woods-Saxon potential for different waves, with $V_0 = \SI{47}{MeV}$, $r_0 = \SI{2}{fm}$ and $d = \SI{0.65}{fm}$}
  \label{fig:woods-saxons}
\end{figure}

\section{Solutions to the \He{5} Schrödinger equation}

  \begin{itemize}
    \item Short method recap, like one equation
    \item HO, vary $\omega$
    \item Wavefunctions
    \item k-space wavefunction
  \end{itemize}

\todo{review wf //this// section.}
The best metric of a potential or Hamiltonian are the wave functions it harbors, we present below the wavefunctions that solve the two particle system for different potential strengths.
In \cref{fig:resonance wavefunction} we can see the radial probability distributions $r^2|R(r)^2|$ for two states as the potential is varied.
 We see that one solution (filled line) displays different behavior for diffrent potentials.
 This is the state with the lowest energy and with a very strong potential, $V_0 =$ \SI{70}{MeV} this is a bound state, $E<0$. \todo{lowest energy?}
In the two weaker potentials, though, this state has an energy $E>0$ meaning that it, too, must be unbound. 
However, its wavefunction is highly localized near the center, suggesting a quasi-bound state. 
Additionally, the fact that the solution varies dramatically with the potential implies that it must be a feature of the system.
 
The other (dashed line) solution, however, is essentially unchanged under variation of the potential. 
This can be interpreted as it being an unbound state in the energy continuum. We see that its probability distribution does not decrease for large $r$. \todo{what's the expected bahvior in the continuum? reflect and compare.}


%Since we are working within the realm of real numbers, we can gain no further insight into the nature of these solutions yet. We do not expect the resonance to be properly described until complex energies are introduced.

\todo{kalla det inte resonans, information hiding, vi kommer på det sen för vi är smarta. OLA}

%%%%%%%%%%%%%%%%%%%%%%%%%WAVEUNCTION MOM FIGURE

\begin{figure}
  \pgfplotstableread{../figures/wavefunctions/wavefunctions.data}\wavefunctions
  \centering{
  \pgfplotsset{
    width = 0.49\textwidth, height = 7cm,
    xlabel = $r/\b{\si{fm}}$, ylabel = $r^2\absq{R(r)}$,
    axis x line = bottom,
    axis y line = left,
    no markers,
    ytick = \empty,
    ymax = 1.7,
	  legend style={anchor=north east},
    legend cell align=left,
  }
    \subfloat[$V_0=\SI{70}{MeV}$]{
      \tikzset{external/remake next}
      \tikzsetnextfilename{wavefunction-70MeV}
      \begin{tikzpicture}
        \begin{axis}
          \addplot          table[x index=0, y index=1] {\wavefunctions};
					  \addlegendentry{$E = \SI{-4.85}{MeV}$}
							
          \addplot+[densely dashed] table[x index=0, y index=2] {\wavefunctions};
		  		  \addlegendentry{$E = \phantom{+}\SI{5.63}{MeV}$}
        \end{axis}
      \end{tikzpicture}
    }
  \subfloat[$V_0=\SI{52}{MeV}$]{
    \tikzset{external/remake next}
    \tikzsetnextfilename{wavefunction-60MeV}
    \begin{tikzpicture}
      \begin{axis}
        \addplot          table[x index=0, y index=3] {\wavefunctions};
				  \addlegendentry{$E = \SI{0.062}{MeV}$}
        \addplot+[densely dashed] table[x index=0, y index=4] {\wavefunctions};
				  \addlegendentry{$E = \phantom{0}\SI{5.75}{MeV}$}
      \end{axis}
    \end{tikzpicture}
  } \\
  \subfloat[$V_0 = \SI{47}{MeV}$]{
	\tikzset{external/remake next}
    \tikzsetnextfilename{wavefunction-52MeV}
    \begin{tikzpicture}
      \begin{axis}
        \addplot table[x index=0, y index=5] {\wavefunctions};
				  \addlegendentry{$E = \SI{0.94}{MeV}$}
        \addplot+[densely dashed] table[x index=0, y index=6] {\wavefunctions};
				  \addlegendentry{$E = \SI{5.80}{MeV}$}
      \end{axis}
    \end{tikzpicture}
  	}
  \subfloat[$V_0 = \SI{40}{MeV}$]{
    \tikzset{external/remake next}
    \tikzsetnextfilename{wavefunction-47MeV}
    \begin{tikzpicture}
      \begin{axis}
        \addplot table[x index=0, y index=7] {\wavefunctions};
				  \addlegendentry{$E = \SI{1.02}{MeV}$}
        \addplot+[densely dashed] table[x index=0, y index=8] {\wavefunctions};
				  \addlegendentry{$E = \SI{5.89}{MeV}$}
      \end{axis}
    \end{tikzpicture}
  	}

  }
  \caption{
    \He{5} wavefunctions for varying potential depth. 
    Plotted are the unique localized solution (solid) and, for comparison, an arbitrary continuum solution (dashed).
    With a deep potential $V_0 = \SI{70}{MeV}$ there is a strongly bound state, which gets weaker as the potential depth is decreased.
    At $V_0 = \SI{52}{MeV}$ the wavefunction is highly localized, yet the energy value is in the continuum, a sign of resonance.
    There is still a clearly localized state with $V_0 = \SI{47}{MeV}$, but at $V_0 = \SI{40}{MeV}$ it is practically indistinguishable from the other states.
  } 
  \label{fig:resonance wavefunction}
  \end{figure}

%%%%%%%%%%%%%%%%%%%%%%%%%WAVEUNCTION MOM end





%%%%%%%%%%%%%%%%%%%%%%%%%EIGVECS FIGURE

\begin{figure}[H]
	  \pgfplotstableread{../figures/he5_mom_eigvecs/physical_eigvecs.data}\phys
	    \pgfplotstableread{../figures/he5_mom_eigvecs/symmetric_eigvecs.data}\symm
  \centering{
  \pgfplotsset{
    width = \textwidth, height = 7cm,
    xlabel = $k$, 
    axis x line = bottom,
    xmin = 0,
    xmax=1,
    axis y line = left,
    no markers,
    ytick = \empty,
    legend style={at={(0.5,0.6)}, anchor=north,legend columns=1},
  }
  \subfloat[Squared absolute value of the eigenvectors to the hamiltonian]{
    %\tikzset{external/remake next}
    \tikzsetnextfilename{eigvecs}
    \begin{tikzpicture}
      \begin{axis}[	ylabel = $|\phi_i|^2$,]
        \foreach \yindex in {1,15,8,54}
        {   
        \addplot table [x index= 0, y index = \yindex] {\phys};
        }
      \end{axis}
    \end{tikzpicture}
  }
	
  \subfloat[Squared absolute value weighted by k squared and integration weights of the eigenvectors to the hamiltonian]{
    % \tikzset{external/remake next}
    \tikzsetnextfilename{symmvecs}
    \begin{tikzpicture}
      \begin{axis}[	ylabel = $k^2 w |\phi_i|^2$,]
        \foreach \yindex in {1,15,8,54}
        {   
        \addplot table [x index= 0, y index = \yindex] {\symm};
        }
      \end{axis}
    \end{tikzpicture}
  }
	
}



\caption{yada} 
\label{fig:he5_eigvecs}
\end{figure}

%%%%%%%%%%%%%%%%%%%%%%%%%EIGVECS end


%%%%%%%%%%%%%%%%%%%%%%%%%WF real/complex FIGURE

\begin{figure}[H]
	    \pgfplotstableread{../figures/eigvecs_real_comp/eigvecs_real.data}\wfs
  \centering{
  \pgfplotsset{
    width = \textwidth, height = 7cm,
 xlabel = $k/\b{\si{fm^{-1}}}$, ylabel = $k^2 \absq{\Phi(k)}$,
    axis x line = bottom,
    xmin = 0,
    xmax=0.5,
    axis y line = left,
    no markers,
    %ytick = \empty,
    legend style={at={(0.5,0.6)}, anchor=north,legend columns=1},
  }
  \subfloat[Wavefunctions in momentum space for solutions along real-axis]{
    \tikzset{external/remake next}
    \tikzsetnextfilename{wfs-real}
    \begin{tikzpicture}
      \begin{axis}[xmax=3]
		  
  		 \addplot[color=red, ultra thick] table [x index= 0, y index = 9] {\wfs};
  		 \addlegendentry{Resonance}

        \foreach \yindex in {1,...,60}
        {   
        \addplot[color=gray] table [x index= 0, y index = \yindex] {\wfs};
        }
		\addlegendentry{Continuum state}
		
		 \addplot[color=red, ultra thick] table [x index= 0, y index = 9] {\wfs};


      \end{axis}
    \end{tikzpicture}
  }
	
  \subfloat[Wavefunctions in momentum space for solutions along the complex berggren contour]{
     \tikzset{external/remake next}
    \tikzsetnextfilename{wfs-comp}
    \begin{tikzpicture}
					  \pgfplotstableread{../figures/eigvecs_real_comp/eigvecs_complex.data}\wfs
      \begin{axis}[xmax=3]
 		 \addplot[color=red, ultra thick] table [x index= 0, y index = 13] {\wfs};
 		 \addlegendentry{Resonance}
        \foreach \yindex in {1,...,60}
        {   
        \addplot[color=gray] table [x index= 0, y index = \yindex] {\wfs};
		
        }
		\addlegendentry{Continuum state}
 		 \addplot[color=red, ultra thick] table [x index= 0, y index = 13] {\wfs};

      \end{axis}
    \end{tikzpicture}
  }
	
}



\caption{yada} 
\label{fig:he5_eigvecs}
\end{figure}

%%%%%%%%%%%%%%%%%%%%%%%%%WF real/complex end




\section{Varying $\omega$} \todo{the black sheep section}
%sentiment: last ditch effort to find the resonnance using hermitian QM. this doesn't / almost work. need to tackle the problem from a diffrent (Non-H) angle
Theory suggests that the HO solutions will hint at the existance of a pole as omega is varried. 
Supposedly \cite{dolan} one wavefunction will enter a plateau as omega is increased while the other wavefunctions will continue to grow unabatedly. 
Our results from investigating this is presented in \cref{fig:energies(omega)}.
It is clear that no such behavior was observed.

\end{document}