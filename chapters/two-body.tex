\documentclass[../main/report.tex]{subfiles}
\begin{document}
  
\chapter{The Two-Body Nuclear System}
\label{cha:two-body}

In this chapter we investigate a simple two-body nuclear system using the basis expansion methods from the previous chapter.
We begin by discussing the shell model, the Woods-Saxon potential and our model system, \He{5}. 
The Shrödinger equation is then solved in the HO and plane wave bases.
The solutions are studied by looking at the energies and wavefunctions while varying parameters.

\section{Model System: the \He{5} Nucleus}
Typical examples of open quantum systems are encountered in nuclear physics. The atomic nucleus is held together by the very short-range strong interaction, acting between the nucleuons. 
This can be studied within a shell model. 
In this framework, the nucleons are modelled as interacting with a \emph{mean-field} potential, approximating an effective interaction the nucleons experience. 
The remaining potential is left as a weak \emph{residual} interaction between nucleons. 

This method has had some success in reproducing the general features of nuclides\cite{suhonen}, especially for lighter nuclei ($A<50$). 
It is found that there are magic numbers of nucleons, where the protons or neutrons form complete shells with $0$ angular momentum. 
Of special interest to us are \emph{doubly magic} nuclei, where both proton are neutron numbers are magic. These nuclei can be very tightly bound.  
 
Because of this, we will choose to study the \He{5} nucleus as an example of a two-body system. 
The \He{4} nucleus is doubly magic, with two $s_{1/2}$-neutrons and two $s_{1/2}$-protons forming full shells, creating a stable core. 
This core will be treated as a single particle (subsequently referred to as the $\alpha$-particle), interacting with the valence neutron through the mean-field only. 

Other doubly magic light nuclei, such as \ce{^{16}O}, have been studied using similar methods\cite{gamow shell model 2008}, but we will here focus on the Helium nucleus.

For mean-field potential we will use the established Woods-Saxon potential. It is given by
\begin{eq}
	V(r)=
	-V_0f(r) - 4V\sub{so}\vec{l}\cdot\vec{s}\frac{1}{r}\frac{df}{dr} ,
\end{eq}
where 
\begin{eq}
	f(r)=\b{1+\exp\p{\frac{r-r_0}{d}}}^{-1} ,
\end{eq}
and it is visualized in \cref{fig:woods-saxons}).

We will eventually optimize the parameters to match experimentally determined energy levels for \He{5}, but to begin with we will use 
standard values given by \cite{suhonen,dickhoff} to investigate the general behavior of the solutions. These parameters are: 
\todo{fix table instead}\begin{align*}
  & \text{Interaction strength}         & V_0       = \SI{47}{MeV}   \\
  & \text{Spin-orbit coupling strength} & V\sub{so} = \SI{-7.5}{Mev} \\
  & \text{Range}                        & r_0       = \SI{2}{fm}     \\
  & \text{Diffuseness}                  & d         = \SI{0.65}{fm}  \\ 
\end{align*}
Note that the spin-orbit coupling term can give either attractive or repulsive contributions, depending on how the angular momenta couples
\begin{eq}
  \vec{l}\cdot\vec{s} 
  = 
  \frac{1}{2}
  \bigp{
    j(j+1)-l(l+1)-s(s+1)
  }
  =
  \begin{cases}
    l,    &\text{ if } j = l + \frac{1}{2}\\
    -l-1, &\text{ if } j = l - \frac{1}{2}\\
  \end{cases}
  .
\end{eq}
Because the $s$-shell is already filled in \He{4}, the valence neutron of \He{5} will be a $p$-wave, with $l=1$. 
We see that the $p_{3/2}$-wave will get a negative net contribution from the total spin-orbit term, also shown in \cref{fig:woods-saxons}. 
This means that the ground state of \He{5} will be the $p_{3/2}$-wave, with $p_{1/2}$ an excited state.

Since we approximate the system as a spherically symmetric interaction between two particles, 
we can by standard procedure reduce the problem to a one-dimensional equation by using the relative coordinate 
$r = |\vec{r}_\alpha - \vec{r}_n|$ and the reduced mass \todo{why?/cite/explain}
\begin{eq}
  \mu = \frac{m_\alpha m_n}{m_\alpha + m_n}.
\end{eq}
We can now proceed to solve the Schrödinger equation with the specified potential.

\begin{figure}
	\newcommand{\diff}{0.65}
	\newcommand{\ro}{2}
	\newcommand{\vo}{47}
	\newcommand{\func}{1/(1 + e^((x-\ro)/\diff))}
	\newcommand{\mass}{0.1}
	  \centering{
	  \pgfplotsset{
	    width = 0.45\textwidth, height = 7cm,
      domain = 0.1:9.8, 
      ymin = -47, 
	    ymax = 9,
      xlabel = $r/\b{\si{fm}}$,
      axis x line = middle,
      axis y line = left
      % xlabel = $r/\b{\si{fm}}$, ylabel = $r^2\absq{R(r)}$,
      % axis x line = bottom,
      % axis y line = left,
      % no markers,
      % ytick = \empty,
      % ymax = 1.7,
      % legend style={anchor=north east,legend columns=1},
	  }
	    \subfloat[$p_\frac{3}{2}$]{
	      \tikzset{external/remake next}
	  \tikzsetnextfilename{p32}
	      \begin{tikzpicture}
		    \begin{axis}[
				ylabel = $V/\b{\si{MeV}}$,
				xmax = 5.9,]
		      \addplot[black] {2/ (2 * \mass * x^2) + \func * (-\vo - 4 * 7.5 * -2 * (\func -1) / (\diff * x )) };
		    \end{axis}
	      \end{tikzpicture}
	    }
	  \subfloat[$p_\frac{1}{2}$]{
	    \tikzset{external/remake next}
	\tikzsetnextfilename{p12}
	    \begin{tikzpicture}
	      \begin{axis}[
			  xmax = 7.9,]
	        \addplot[black] {2/ (2 * \mass * x^2) + \func * (-\vo - 4 * 7.5 * 1 * (\func -1) / (\diff * x ))};
	      \end{axis}
	    \end{tikzpicture}
	  }
	  }
  \caption{The Woods-Saxon potential for different waves, with $V_0 = \SI{47}{MeV}$, $r_0 = \SI{2}{fm}$ and $d = \SI{0.65}{fm}$}
  \label{fig:woods-saxons}
\end{figure}

\section{The \He{5} Spectrum}
  \begin{itemize}
    \item Short method recap, like one equation
    \item HO, vary $\omega$
    \item Wavefunctions
    \item k-space wavefunction
  \end{itemize}
In \cref{cha:basis_expansion} we described the procedure for numerically solving the Schrödinger equation using basis expansion. 
In this section we will apply those methods to find the eigensolutions of our model system. 

\subsection{Harmonic Oscillator Basis }
To begin with, we will solve the problem in HO basis. \todo[inline]{convergence plot or not?} \Cref{fig:he5 conv HO} shows how the obtained energies converge when the number of basis states $N$ included in the expansion increases. 

In addition, the parameter $\omega$ in the HO potential may be varied. 
This parameter determines the rate at which the potential increases, but also (as can be seen in \cref{eq:HO radial wavefunction}) the range of the wavefunctions which include a factor $e^{-\gamma r^2}$, $\gamma = \mu \omega/\hbar$. We can define a range 
\begin{eq}
  r_0 = \frac{1}{\sqrt{\gamma}} = \sqrt{\frac{\hbar}{\mu \omega}} ,
\end{eq}  
such that for $r>r_0$ the basis wavefunctions quickly approach zero. 
This parameter will affect how rapidly the solutions converge, because a basis where the wavefunctions are similar to the solutions will have better convergence. 
\Cref{fig:HO omega} shows how the eigensolutions depend on the range of the basis functions.

The solutions all have energies $E>0$, meaning that they are not bound, and thus have unlimited range. 
However, we can clearly distinguish a minimum in energy for the lowest energy solution when the range of the basis is in the region $r_0 \approx \SI{1.3}{fm}$, corresponding to radii within the nucleus. 
This is a sign that there is a solution to \He{5} localized in this region. 
Furthermore, we see that the energies for the others strictly decrease when the range of the basis increases, further showing their unbound nature.

Because the harmonic oscillator consists only of bound states and we are trying to study unbound states with infinite range, we can not get much further with this method. 
We will have to switch to a basis with wavefunctions of infinite reach to properly describe this system.

\subsection{Plane Wave Basis}
The best metric of a potential or Hamiltonian are the wave functions it harbors, we present below the wavefunctions that solve the two particle system for different potential strengths.
In \cref{fig:resonance wavefunction} we can see the radial probability distributions $r^2|R(r)^2|$ for two states as the potential is varied.
 We see that one solution (filled line) displays different behavior for diffrent potentials.
 This is the state with the lowest energy and with a very strong potential, $V_0 =$ \SI{70}{MeV} this is a bound state, $E<0$. \todo{lowest energy?}
In the two weaker potentials, though, this state has an energy $E>0$ meaning that it, too, must be unbound. 
However, its wavefunction is highly localized near the center, suggesting a quasi-bound state. 
Additionally, the fact that the solution varies dramatically with the potential implies that it must be a feature of the system.
 
The other (dashed line) solution, however, is essentially unchanged under variation of the potential. 
This can be interpreted as it being an unbound state in the energy continuum. We see that its probability distribution does not decrease for large $r$. \todo{what's the expected bahvior in the continuum? reflect and compare.}


%Since we are working within the realm of real numbers, we can gain no further insight into the nature of these solutions yet. We do not expect the resonance to be properly described until complex energies are introduced.

\todo{kalla det inte resonans, information hiding, vi kommer på det sen för vi är smarta. OLA}

%%%%%%%%%%%%%%%%%%%%%%%%% Wavefunctions

\begin{figure}
  \pgfplotstableread{../figures/wavefunctions/wavefunctions.data}\wavefunctions
  \centering{
  \pgfplotsset{
    width = 0.49\textwidth, height = 7cm,
    xlabel = $r/\b{\si{fm}}$, ylabel = $r^2\absq{R(r)}$,
    axis x line = bottom,
    axis y line = left,
    no markers,
    ytick = \empty,
    ymax = 1.7,
	  legend style={anchor=north east},
    legend cell align=left,
  }
    \subfloat[$V_0=\SI{70}{MeV}$]{
      \tikzset{external/remake next}
      \tikzsetnextfilename{wavefunction-70MeV}
      \begin{tikzpicture}
        \begin{axis}
          \addplot          table[x index=0, y index=1] {\wavefunctions};
					  \addlegendentry{$E = \SI{-4.85}{MeV}$}
							
          \addplot+[densely dashed] table[x index=0, y index=2] {\wavefunctions};
		  		  \addlegendentry{$E = \phantom{+}\SI{5.63}{MeV}$}
        \end{axis}
      \end{tikzpicture}
    }
  \subfloat[$V_0=\SI{52}{MeV}$]{
    \tikzset{external/remake next}
    \tikzsetnextfilename{wavefunction-60MeV}
    \begin{tikzpicture}
      \begin{axis}
        \addplot          table[x index=0, y index=3] {\wavefunctions};
				  \addlegendentry{$E = \SI{0.062}{MeV}$}
        \addplot+[densely dashed] table[x index=0, y index=4] {\wavefunctions};
				  \addlegendentry{$E = \phantom{0}\SI{5.75}{MeV}$}
      \end{axis}
    \end{tikzpicture}
  } \\
  \subfloat[$V_0 = \SI{47}{MeV}$]{
	\tikzset{external/remake next}
    \tikzsetnextfilename{wavefunction-52MeV}
    \begin{tikzpicture}
      \begin{axis}
        \addplot table[x index=0, y index=5] {\wavefunctions};
				  \addlegendentry{$E = \SI{0.94}{MeV}$}
        \addplot+[densely dashed] table[x index=0, y index=6] {\wavefunctions};
				  \addlegendentry{$E = \SI{5.80}{MeV}$}
      \end{axis}
    \end{tikzpicture}
  	}
  \subfloat[$V_0 = \SI{40}{MeV}$]{
    \tikzset{external/remake next}
    \tikzsetnextfilename{wavefunction-47MeV}
    \begin{tikzpicture}
      \begin{axis}
        \addplot table[x index=0, y index=7] {\wavefunctions};
				  \addlegendentry{$E = \SI{1.02}{MeV}$}
        \addplot+[densely dashed] table[x index=0, y index=8] {\wavefunctions};
				  \addlegendentry{$E = \SI{5.89}{MeV}$}
      \end{axis}
    \end{tikzpicture}
  	}

  }
  \caption{
    \He{5} wavefunctions for varying potential depth. 
    Plotted are the unique localized solution (solid) and, for comparison, an arbitrary continuum solution (dashed).
    With a deep potential $V_0 = \SI{70}{MeV}$ there is a strongly bound state, which gets weaker as the potential depth is decreased.
    At $V_0 = \SI{52}{MeV}$ the wavefunction is highly localized, yet the energy value is in the continuum, a sign of resonance.
    There is still a clearly localized state with $V_0 = \SI{47}{MeV}$, but at $V_0 = \SI{40}{MeV}$ it is practically indistinguishable from the other states.
  } 
  \label{fig:resonance wavefunction}
  \end{figure}

%%%%%%%%%%%%%%%%%%%%%%%%% Wavefunctions END


%%%%%%%%%%%%%%%%%%%%%%%%%WF real/complex FIGURE

\begin{figure}[H]
	    \pgfplotstableread{../figures/eigvecs_real_comp/eigvecs_real.data}\wfs
  \centering{
  \pgfplotsset{
    width = \textwidth, height = 7cm,
 xlabel = $k/\b{\si{fm^{-1}}}$, ylabel = $k^2 \absq{\Phi(k)}$,
    axis x line = bottom,
    xmin = 0,
    xmax=0.5,
    axis y line = left,
    no markers,
    %ytick = \empty,
	xtick = {0,0.5,...,3},
    legend style={at={(0.5,0.6)}, anchor=north,legend columns=1},
  }
  \subfloat[Wavefunctions in momentum space for solutions along real-axis]{
    \tikzset{external/remake next}
    \tikzsetnextfilename{wfs-real}
    \begin{tikzpicture}
      \begin{axis}[xmax=3]
		  
  		 \addplot[color=red, ultra thick] table [x index= 0, y index = 9] {\wfs};
  		 \addlegendentry{Resonance}

        \foreach \yindex in {1,...,60}
        {   
        \addplot[color=gray] table [x index= 0, y index = \yindex] {\wfs};
        }
		\addlegendentry{Continuum state}
		
		 \addplot[color=red, ultra thick] table [x index= 0, y index = 9] {\wfs};


      \end{axis}
    \end{tikzpicture}
  }
	
  \subfloat[Wavefunctions in momentum space for solutions along the complex berggren contour]{
     \tikzset{external/remake next}
    \tikzsetnextfilename{wfs-comp}
    \begin{tikzpicture}
					  \pgfplotstableread{../figures/eigvecs_real_comp/eigvecs_complex.data}\wfs
      \begin{axis}[xmax=3]
 		 \addplot[color=red, ultra thick] table [x index= 0, y index = 13] {\wfs};
 		 \addlegendentry{Resonance}
        \foreach \yindex in {1,...,60}
        {   
        \addplot[color=gray] table [x index= 0, y index = \yindex] {\wfs};
		
        }
		\addlegendentry{Continuum state}
 		 \addplot[color=red, ultra thick] table [x index= 0, y index = 13] {\wfs};

      \end{axis}
    \end{tikzpicture}
  }
	
}



\caption{yada} 
\label{fig:he5_eigvecs}
\end{figure}

%%%%%%%%%%%%%%%%%%%%%%%%%WF real/complex end




\section{Varying $\omega$} \todo{Needs plot}
%sentiment: last ditch effort to find the resonnance using hermitian QM. this doesn't / almost work. need to tackle the problem from a diffrent (Non-H) angle
Theory suggests that the HO solutions will hint at the existance of a pole as omega is varried. 
Supposedly \cite{dolan} one wavefunction will enter a plateau as omega is increased while the other wavefunctions will continue to grow unabatedly. 
Our results from investigating this is presented in \cref{fig:energies(omega)}.
It is clear that no such behavior was observed.

%%%% E(r0)
\tikzsetnextfilename{triangle_contour}
\begin{figure}[H]
  \centering
  \begin{tikzpicture}
    \begin{axis}[
      width = \textwidth,
      height = 7cm,
      xlabel= $r_0$,
      ylabel=E,
		  axis lines = middle,
      ymax = 2.5,
      no markers,
      enlargelimits,
      ticks = none,
      ]
      \addplot[color=red, ultra thick] table[x index = 0, y index =1] {../figures/E(omega)/E(omega).data};
      \foreach \y in {2,...,5}
         \addplot[color=gray] table[x index = 0, y index = \y] {../figures/E(omega)/E(omega).data};
    \end{axis}
  \end{tikzpicture}
  \caption{The complex contour used. The points are distributed on each segment according to the Gauss-Legendre quadrature rule.}
  \label{fig:triangle contour}
\end{figure}
\end{document}