Equipped with the Fock-space theory and the solutions to the \He{5} nucleous we are ready to solve the \He{6} nucleus. We will describe the \He{6} nucleus as $\alpha+n+n$, where the neutrons are treated as moving in the potential field generated by the $\alpha$. In addition there will be an attractive interaction between the neutrons, which will allow for bound states in \He{6} where there were none in \He{5}. 

To calculate the \He{6} Hamilton matrix we will use the eigensolutions $\ket{\phi_i}$ from the \He{5} problem as single-particle basis states. From these we form two-particle (fermionic) states $\ket{\psi_i,\, \psi_j}$ in accordance to the theory in the previous chapter, that will be used to solve the problem. The benefit of this choice is that the Hamiltonian is written as
\begin{eq}
\hat{H} = \hat{H_1} + \hat{H_2}
\end{eq}
where $H_1$ is the interaction between neutron and core. Since we are only considering valence neutrons in the $p$ orbitals ($l=1$), it has been shown\cite{suzuki} that one may use the same reduced mass as in the two-body problem. Because of this, $H_1$ is the Hamiltonian that was used in the solution of \He{5}. This means that our basis states will be eigensolutions to this component, and it will only give diagonal contributions to the hamilton matrix. The non-diagonal part of the matrix will come from the two-body interaction $H_2$.

\section{Neutron - Neutron Interaction}
The interaction between neutrons is a complex feature. This is because it arises from the strong interaction between the quarks constituting the neutrons. It depends on both relative coordinates and momenta, and does not have a known analytical expression. In numerical calculations various approximations are used. To make the calculations more managable, we will use a separable interaction:
\begin{equation}
V\sub{n-n}(r_{1} , r_{2}) = V_{0}e^{ - \beta (r_1^2 + r_2^2)},
\end{equation}
where $r_1$ and $r_2$ are the radial coordinates of the neutrons relative to the core. $\beta$ (range) and $V_{0}$ (strength) are parameters that have to be fitted after experimential data. 

To solve the problem we need to determine what contribution the n-n interaction give to the Hamiltonian. We present below the key points in this derivation. Details can be found in \cref{app: n-n}. 
\begin{eq}
  \hat{H}_2 
  = 
  \frac{1}{4}
  \sum_{\alpha\beta\gamma\delta}
  \bra{\alpha\beta} V\sub{n-n} \ket{\gamma\delta} 
  a^\dag_\alpha a^\dag_\beta a_\delta a_\gamma
  =
  \sum_{\substack{\alpha < \beta \\ \gamma < \delta}}
  \bra{\alpha\beta} V\sub{n-n} \ket{\gamma\delta} 
  a^\dag_\alpha a^\dag_\beta a_\delta a_\gamma,
\end{eq}
where
\begin{eq}
  \bra{ab} V\sub{n-n} \ket{cd} 
  =
  \pbra{ab} V\sub{n-n} \pket{cd}
  -
  \pbra{ab} V\sub{n-n} \pket{dc}.
\end{eq}
%(Rounded bras and kets are non-antisymmetrised product states.)
We have
\begin{eq}
  \pbra{ab} V\sub{n-n} \pket{cd}  
  &=
  V_0 \pbra{ab} e^{-\beta(r_1^2+r_2^2)}\pket{cd} \\
  &=
  V_0 \bra{a}V\sub{sep}\ket{c}\bra{b}V\sub{sep}\ket{d}
  %V_0 \int \rd^3 \vec{r}_1 \int \rd^3 \vec{r}_2 \conj{\psi_a}(\vec{r}_1)\conj{\psi_b}(\vec{r}_2) e^{-\beta r_1^2}e^{-\beta r_2^2}  \psi_c(\vec{r_1})\psi_d(\vec{r_2}) \\
  %&=
  %V_0 \int \rd^3 \vec{r} \conj{\psi_a}(\vec{r})e^{-\beta r^2} \psi_c(\vec{r}) \cdot \int \rd^3 \vec{r} \conj{\psi_b}(\vec{r})e^{-\beta r^2} \psi_d(\vec{r}) \\
\end{eq}
where $V\sub{sep} = e^{-\beta r^2}$. 

We have here denoted single-particle states $a,\, b,...$ etc. for simplicity. For the \He{5} eigensolutions these consist of the quantum numbers
\begin{eq}
\ket{a} = \ket{E l s j m} .
\end{eq}
We will only be using $p_{3/2}$ states in our expansion, meaning that $(lsj)$ will be common for all states and will not be written out explicitly, so single-particle states will be written as $\ket{Em}$. 

We can now appreciate the benefits of choosing a spherically symmetric separable potential. To calculate the hamilton matrix we need only evaluate the interaction matrix $V_{ij} = \bra{E_i m_i}V\sub{sep}\ket{E_j m_j}$. Since we are solving the \He{5} problem in some basis expansion, we generally do not know the wavefunctions in coordinate space. Instead we transform the integral to the representation that is currently used. For plane wave expansion we could do it explicitly by inserting \cref{eq:radial wavefunction} twice:
\begin{eq}
  \bra{E_i m_i}V\sub{sep}\ket{E_j m_j} 
  &=
  \int \rd^3 \vec{r} \conj{\psi_i}(\vec{r}) V\sub{sep}(r) \psi_j(\vec{r}) \\
  &= 
  \int \rd k k^2 \phi_i(k) \int \rd k' k'^2 \phi_j(k') V\sub{sep}(k,k') \delta_{m_i m_j}
\end{eq}
where 
\begin{eq}
  V\sub{sep}(k,k') = \frac{2}{\pi}\int \rd r r^2 V(r) j_l(kr)j_l(k'r).
\end{eq}
Putting it all together, we have 
\begin{eq}
  &\bra{E_1 m_1 E_2 m_2} V\sub{n-n} \ket{E'_1 m'_1 E'_2 m'_2} \\
   = &V_0 \bra{E_1 m_1} V\sub{sep} \ket{E'_1 m'_1} \bra{E_2 m_2} V\sub{sep} \ket{E'_2 m'_2}
  -
  V_0 \bra{E_1 m_1} V\sub{sep} \ket{E'_2 m'_2} \bra{E_2 m_2} V\sub{sep} \ket{E'_1 m'_1} \\
   =&
  V_0 \bigp{ V(E_1, E'_1)V(E_2, E'_2) \delta_{m_1 m'_1} \delta_{m_2 m'_2}
  -
  V(E_1, E'_2)V(E_2, E'_1) \delta_{m_1 m'_2} \delta_{m_2 m'_1} }.
\end{eq}
where
\begin{eq}
V(E, E') = \int \rd k \, k^2 \phi(k) \int \rd k' \, k'^2 \phi'(k') V\sub{sep}(k,k').
\end{eq}
$\phi$ and $\phi'$ are the eigenfunctions with eigenvalues $E$ and $E'$ respectively. To evaluate the integral numerically it is discretized as
\begin{eq}
V(E_1, E_2) = \sum_i \sqrt{w_i}k_i \phi_i \sum_j \sqrt{w_j}k_j \phi'_j V\sub{sep}(k_i, k_j)
\end{eq}
where part of the quadrature weight is baked into the eigenvectors (see \cref{sec:mom discretization}) obtained from solving the single-particle problem.

\section{One Body hamiltonian...}
\begin{eq}
	\op{H_1} = 
	\sum_{\alpha \beta} \bra{\alpha}H_1\ket{\beta} a_{\alpha}^{\dagger} a_{\beta}
\end{eq}	 
where 
\[
\bra{\alpha}H_1\ket{\beta} 
= 
\bra{E m}H_1\ket{E' m'}
\\=
\bra{E}H_1\ket{E'} \delta_{m m'}
= 
\bra{E}E'\ket{E'} \delta_{m m'}
=
E \delta_{E E'}\delta_{m m'}
\]
We have
\begin{eq}
	\bra{a b}\op{H_1}\ket{c d} 
	=
\bra{a b} \sum_{\alpha} E_{\alpha} a_{\alpha}^{\dagger} a_{\alpha} \ket{c d} 	
	\\=
	\bra{a b}E_c + E_d \ket{cd}
	=
	(E_c + E_d)\bra{a b}\ket{c d}
	=
	(E_c + E_d)\delta_{a c}\delta{b d}
\end{eq}
\begin{eq}
\bra{E_1 E_2}\op{H_1}\ket{E_1' E_2'}
	=
\frac{1}{\sqrt{(1+\delta_{E_1 E_2})(1+\delta_{E_1' E_2'})}} \sum_{\substack{m_1 m_2 \\ m_1' m_2'}}c_{m_1 m_2}c_{m_1' m_2'}^{*} \bra{a b}\op{H_1}\ket{c d} 
	\\=
\frac{1}{\sqrt{(1+\delta_{E_1 E_2})(1+\delta_{E_1' E_2'})}} \sum_{\substack{m_1 m_2 \\ m_1' m_2'}} (E_1 +E_2) c_{m_1 m_2}c_{m_1' m_2'}^{*} \delta_{E_1 E_1'} \delta_{m_1 m_1'} \delta_{E_2 E_2'} \delta_{m_2 m_2'}
\\=
(E_1 + E_2 )\frac{\delta_{E_1 E_1'} \delta_{E_2 E_2'}}{\sqrt{(1+\delta_{E_1 E_2})(1+\delta_{E_1' E_2'})}} \sum |c_{m_1 m_2}|^2
\\=
(E_1 + E_2 )\frac{\delta_{E_1 E_1'} \delta_{E_2 E_2'}}{\sqrt{(1+\delta_{E_1 E_2})(1+\delta_{E_1' E_2'})}}	

\section{mb Hamiltonian}
We have now developed the tools we need to determine the mb Hamiltonian. 
The \He{6} nucleus can be solved in two ways, either we couple the angular momenta of the two sp states to reduce the amount of calculations needed or we generate all mb states to get calculations that are more straight-forward. 
We solve the problem using both schemes, the coupled scheme because that's what we want to use but we also solve - in the uncoupled scheme to verify the answer. 
Both schemes can be used together with the HO basis or the k basis. 
We will cover all four approaches in this chapter.

\subsection{Uncoupled Scheme}
In the uncoupled scheme we use a mb basis, $\alpha$, formed from two sp solutions, $a,b$, and give each sp state a m quantum number. In order to reduce the calculatins needed we filter the mb states so the sum of m numbers equals M.

\subsection{Coupled Scheme}

\end{eq}




