We want to study the nuclei of Helium isotopes by solving the time independent Schrödinger equation (TISE)
\begin{eq}
  \label{eq:TISE}
  H \ket\psi = E \ket\psi.
\end{eq}

For the nuclear systems we are looking at, the TISE has no known analytical solutions, and we need to use numerical methods to solve it.
However, written as in \cref{eq:TISEpos}, it is not suitable for numerical calculations.
Instead, we would  like to write it as a matrix equation
\begin{eq}
  \label{eq:matrix equation}
  \sum_j H_{ij}\psi_j = E \psi_i
\end{eq}
with a finite matrix $H$ that we can diagonalize to find the eigenvalues $E$.

To write the TISE as a matrix equation we use \emph{basis expansion}. 
Basis expansion is how we make sense of the abstract Hilbert spaces, operators and state vectors of QM. 
By expanding these abstract objects in a basis, we can relate them to the physical world. 
For example, equation  the TISE for a particle, expanded in the position basis T
\begin{eq}
  \label{eq:TISEpos}
  \p{-\frac{\hbar^2}{2m}\nabla^2 + V(\vec{r})}\psi(\vec{r}) = E\psi(\vec{r}),
\end{eq}

This is the only basis in which we can express the potential, which is why we will always relate our new bases to this one.

Before we begin, we briefly recap some well known QM facts. 
First we need a \emph{complete basis}, either discrete $\ket{n}$ or continuous $\ket{x}$. 
A discrete basis means that any state $\ket\psi$ can be written as a linear combination of the basis states
\begin{eq}
  \label{eq:lincomb}	
  \ket\psi = \sum_n \psi_n \ket{n}
  \quad
  \textup{or}
  \quad
  \ket\psi = \fint{x} \psi(x) \ket{x}.
\end{eq}
The complete bases we will use in this report are the \emph{position basis} $\ket{\vec{r}}$, the \emph{momentum basis} $\ket{\vec{k}}$, the \emph{harmonic oscillator basis} $\ket{nlm}$ and the elusive \emph{Berggren basis} \cite{berggren}. All these bases are orthonormal, i.e. all the basis vectors satisfy 
\begin{eq}
  \braket{n}{n'} = \delta_{nn'}
  \quad
  \textup{or}
  \quad
  \braket{x}{x'} = \delta(x - x').
\end{eq}
With a complete basis $\ket{n}$, we get the very useful \emph{completeness relation}
\begin{eq}
  I = \sum_n \ket{n} \bra{n}
  \quad
  \textup{or}
  \quad
  I = \fint{x} \ket{x}\bra{x},
\end{eq}
where $I$ is the identity operator. This relation can therefore be inserted anywhere in any equation, and will find frequent use in this report. 

Let us now expand the TISE in the abstract $\ket{n}$ basis. We start by inserting the completeness relation for $\ket{n}$ in \cref{eq:TISE}
\begin{eq}
  \label{eq:expand1}
  H
  \p{
    \sum_{n'} \ket{n'} \bra{n'}
  }
  \ket\psi
  =
  \sum_{n'} H \ket{n'} \braket{n'}{\psi}
  =
  E \ket\psi.
\end{eq}
By closing \cref{eq:lincomb} with $\bra{n}$ on the left side and using orthonormality, we see that $\braket{n'}{\psi} = \psi_{n'}$. Now we close \cref{eq:expand1} with $\bra{n}$ on the left
\begin{eq}
  \label{eq:expand2}
  \sum_{n'} \bra{n} H \ket{n'} \psi_{n'}
  = 
  E \braket{n}{\psi},
\end{eq}
and if we write $H_{nn'} = \bra{n} H \ket{n'}$, we get
\begin{eq}
  \label{eq:expand3}
  \sum_{n'} H_{nn'} \psi_{n'} = E \psi_n,
\end{eq}
which is equivalent to the matrix \cref{eq:matrix equation}. This is the basic method of expanding the TISE in a basis. Expanding in an abstract basis will not get us very far, however, so we move on to the epitomic example: the harmonic oscillator.

\section{The Spherical Harmonic Oscillator}
\label{sec:harm_osc}
The Helium nuclei we want to expand are, for our intents and purposes, spherically symmetric. A spherically symmetric basis is therfore preferable, and we begin with the spherical harmonic oscillator (HO). The treatment below is adapted from \cite{moshinsky}.

We have for the HO, the Hamiltonian
\begin{eq}
  \label{eq:HO hamiltonian}
  H\sub{HO} = \frac{p^2}{2\mu} + \frac{\mu\omega^2 r^2}{2},
\end{eq}
where $\mu$ is the mass of the problem and $\omega$ is the angular frequency of the oscillator. With this Hamiltonian, the TISE has the solutions
\begin{eq}
  H\sub{HO}\ket{nlm} = E_{nl}\ket{nlm} % E_{nl}???
\end{eq}
with
\begin{eq}
  E_{nl} = \hbar\omega(2n + l + \frac{3}{2}), \quad n = 0, 1, 2, \dots
\end{eq}
Since they are eigenstates of $H\sub{HO}$, $\ket{nlm}$ form a complete basis, and it is this basis we will expand our TISE in. The procedure is the same as in \cref{eq:expand1,eq:expand2,eq:expand3} and gives us
\begin{eq}
  \label{eq:HOexpanded}
  \sum_{n'l'm'} \bra{nlm} H \ket{n'l'm'} \psi_{n'l'm'} = E \psi_{nlm}\,.
\end{eq}
Since we are considering a spherically symmetric Hamiltonian
\begin{eq}
  \label{eq:spherical hamiltonian}
  H = \frac{p^2}{2\mu} + V(r),
\end{eq}
the $l$ and $m$ in $\ket{nlm}$ commute with the Hamiltonian and will therefore contribute with a $\delta_{ll'}\delta_{mm'}$ factor. This means we can drop the primes and write \cref{eq:HOexpanded} as
\begin{eq}
  \sum_{n'} \bra{nlm} H \ket{n'lm} \psi_{n'lm} = E\psi_{nlm}.
\end{eq}
If we see $\bra{nlm} H \ket{n'lm} = H_{nn'}$ as a matrix, this can be expressed as {\it $H$ is diagonal in $l$ and $m$}.

Now we have our matrix equation, but we need to find the matrix elements $\bra{nlm} H \ket{n'lm}$. These require some calculation (see \cref{sec:HO matrix elements} for the details) and the result is
\begin{eq}
  \label{eq:HO matrix elements}
  &
  \bra{nlm} H \ket{n'lm} =
	\frac{\hbar\omega}{2}
	\left(
    \p{2n+l+\frac{3}{2}} \delta_{nn'}
    +
		\sqrt{n(n+l+\frac{1}{2})} \delta_{n,n'-1}\right.
		\\ & + 
		\left.\sqrt{n'(n'+l+\frac{1}{2})} \delta_{n',n-1} 
	\right)
	+
	\fint[0][\inf]{r} 
    r^2 R_{nl}(r) V(r) R_{n'l}(r)
\end{eq}
where $R_{nl}$ are the radial wavefunctions of the harmonic oscillator
\begin{eq}
  \label{eq:HO radial wavefunction}
	R_{nl}(r) 
	= 
	N r^l e^{-\gamma r^2 / 2}
	L_{(n-l) / 2}^{(l+\frac{1}{2})}(\gamma r^2)
\end{eq}
with $\gamma = \mu\omega/\hbar$, $L_n^\alpha(x)$ are the generalized Laguerre polynomials and $N$ is a normalization constant,
\begin{eq}
	N = 
  \sqrt{\frac{
    2^{n+l+2}\gamma^{l+\frac{3}{2}}
  }{
    \sqrt{\pi}
  }}
  \sqrt{\frac{
    \p{\frac{1}{2}(n-l)}!
    \p{\frac{1}{2}(n+l)}!
  }{
    (n+l+1)!
  }}.
\end{eq}

\section{Implementation}

Since the HO basis is a discrete basis, implementing it in code is straightforward. There are a few considerations, however, and we mention them here. 

The $\ket{nlm}$ basis is infinite in $n$, but we need a finite matrix, so we truncate the basis at a finite number $N$ giving us an $N \times N$ matrix. Since the matrix is diagonal in $l$ and $m$, we do the calculation separately for each value of $l$ and $m$ to reduce the amount of computation needed to solve for the eigenvalues
The equation we are solving is then, for a given $l$ and $m$,
\begin{eq}
  \sum_{n'= 0}^N \bra{n} H \ket{n'} \psi_{n'} = E\psi_{n}
\end{eq}
or in linear algebra notation
\begin{eq}
  H\psi = E\psi.
\end{eq}
This matrix equation is solved using a standard eigensolver algorithm, which uses the fact that the matrix is hermitian to solve the equation faster.

We calculate the matrix elements using \cref{eq:HO matrix elements}. The integration is performed using Gauss-Legendre quadrature and setting the upper limit to a finite number.

\section{Expanding the \He{5} Nucleus}


\todo{Possible name change: Plane Wave Expansion?}
\section{Discretized Momentum Space}
\label{sec:mom_space}
The reason that we want to solve the problem in momentum space is that we are studying a system with a short-range potential supporting only a few, if any, bound solutions. This means that we will find multiple unbound solutions, corresponding to free particles of various energies. These are basically already eigenstates of the momentum operator, only slightly disturbed by the small potential well at $r=0$. 

The expansion is done in the same way as before, giving us
\begin{eq}
  \int \rd^3 \vec{k}' \bra{\vec{k}} H \ket{\vec{k}'} \Phi(\vec{k}')
  &= 
  E\Phi(\vec{k}) \, .
\end{eq}
This three-dimensional equation is practically unsolvable, but we can simplify it using the fact that we have a central problem. A rather involved calculation (see \cref{sec:radial_mom_space_TISE}) shows that the Schrödinger equation can be written as
\begin{eq} 
  \frac{k^2}{2\mu}\phi(k) + \int_0^\infty \rd k' \, k'^2 V(k,k') \phi(k') 
  &=
  E\phi(k) \, ,
\end{eq}
where $\phi(k)$ is the radial part of the momentum space wavefunction, 
\begin{eq}
  V(k,k') 
  &= 
  \frac{2}{\pi}\int_0^\infty \rd r \, r^2 V(r) j_l(kr) j_l(k'r) 
\end{eq}
and $j_l(kr)$ are the spherical bessel functions of order $l$. We see that the transformation turned the differential equation into an integral equation.

\subsection{Numerical Considerations}
As with the harmonic oscillator basis we want to rewrite the equation as a finite matrix equation. An integral equation can be rewritten as a matrix equation by approximating the integral with a numerical quadrature, 
\begin{eq}
  \label{eq:discrete_momentum}
  \int_0^\infty \rd k' \, k'^2 V(k,k')\phi(k') 
  \approx
  \sum_{j=1}^N w_j k_j^2 V(k,k_j)\phi(k_j)
\end{eq}
where $w_j$ are the quadrature weights. For the na\"{i}ve rectangular quadrature you would use a constant $w_j=\Delta k_j$, equal to the step length. However, this quadrature converges slowly to the correct value of the integral, and much better alternatives can be employed. We are using the G-L quadrature.

With this approximation the Schrödinger equation may be written
\begin{eq}
  \sum_j H_{ij} \phi_j &= E \phi_i
\end{eq}
where $\phi_i=\phi(k_i)$ and 
\begin{eq}
  H_{ij} &= \frac{k_i^2}{2\mu}\delta_{ij} + w_jk_j^2 V_{ij} \\
  V_{ij} &= \frac{2}{\pi} \int_0^\infty \rd r \, r^2 V(r) j_l(k_i r) j_l(k_j r)
\end{eq}
The equation is now written as a matrix equation of order N -- the number of states included in the basis. The energy eigenvalues $E$ will be obtained by calculating the matrix elements $H_{ij}$ and diagonalizing the resulting matrix. Since there is generally no analytic expression for the terms $V_{ij}$, they will need to be evaluated by numerical integration.

To speed up calculations we can transform the equation so that the matrix will be symmetric. This is achieved by the transformation
\begin{eq}
  \phi_i &\mapsto
  \phi_i' =  \sqrt{w_i} k_i \phi_i
  \\
  H_{ij} &\mapsto
  H_{ij}' 
  = 
  \sqrt{\frac{w_i}{w_j}} \frac{k_i}{k_j}H_{ij}
\end{eq}
We would then have
\begin{eq}
  \sum_j H'_{ij}\phi'_j = E\phi'_i\,,
\end{eq}
meaning that the eigenvalues could be just as well obtained by diagonalizing the symmetric $H'$ matrix, thus saving precious time. We have
\begin{eq}
  H_{ij}' = \frac{k_i^2}{2\mu}\delta_{ij} + \sqrt{w_i w_j}k_i k_j V_{ij}
\end{eq}
with $V_{ij}$ defined as above.



\section{Verfying with the Hydrogen Atom} 

To test our HO and momentum-space bases, we expand the well-known hydrogen atom. It has the potential
\begin{eq}
  \label{eq:hydrogen potential}
	V(r)=
	-\frac{q_e^2}{4\pi \epsilon_0 r},
\end{eq}
and is analytically solvable with a ground state energy $E_0 = -\SI{13.6}{eV}$.
In \cref{fig:hydrogen HOvsMom} \todo{N size of basis // numbre of basis states included?}we plot the convergence of the ground state energy as a function of $N$, the size of the basis.
We see that the plane-wave expansion converges faster than the harmonic oscillator, but toward the wrong value.\todo{verify this and write it in a better fashion}
This is because the momentum space expansion does not support the use of infinite potentials, and since \cref{eq:hydrogen potential} diverges as $r \to 0$,  a small deviation is not surprising.

\begin{figure}
  \hspace{-2cm}
    \includegraphics[width = 1.25\textwidth]{figures/hydrogen_convergence.pdf}
  \caption{Ground state energy of hydrogen atom as a function of $N$, the number of states included in the basis.}
  \label{fig:hydrogen HOvsMom}
\end{figure}


