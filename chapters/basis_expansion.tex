\documentclass[../main/report.tex]{subfiles}
\begin{document}

\chapter{Basis Expansion}
\label{cha:basis_expansion}

\todo{Intro to chapter here PONTUS}
We want to study loosely bound quantum systems by solving the \emph{Time Independent Schrödinger Equation} (TISE)
\todo{why TISE?}
\begin{eq}
  \label{eq:TISE}
  H \ket\psi = E \ket\psi,
\end{eq}
commonly written in the position basis as
\begin{eq}
  \label{eq:TISEpos}
  \p{-\frac{\hbar^2}{2m}\nabla^2 + V(\vec{r})}\psi(\vec{r}) = E\psi(\vec{r}).
\end{eq}

For the nuclear systems to be treated, the TISE has no known analytical solutions, and we need to use numerical methods to solve it.
There exists some methods for this, one can solve a system of coupled differential equations, make it a problem of evaluate a multi-dimensional integral or as we will do, express it as a matrix equation \todo{would something like this be ok?}
\begin{eq}
  \label{eq:matrix equation}
  \sum_j H_{ij}\psi_j = E \psi_i
\end{eq}
with a finite matrix $H$ that we can diagonalize to find the eigenvalues $E$.

To write the TISE as a matrix equation we use \emph{basis expansion}. 
Basis expansion is how we make sense of the abstract Hilbert spaces, operators and state vectors of Quantum Mechanics (QM).
By expanding these abstract objects in a basis, we can relate them to the physical world. 
For example, \cref{eq:TISEpos} is the TISE for a single particle, expanded in the position basis. 
We will not expand our problems in the position basis, but it will still be important, since it is the most natural basis to express the potential in.

Before we begin, we briefly recap some well known QM facts. 
First we need a \emph{complete basis}, either discrete, $\ket{n}$, or continuous, $\ket{x}$. 
A complete basis means that any state $\ket\psi$ can be written as a linear combination of the basis states
\begin{eq}
  \label{eq:lincomb}	
  \ket\psi = \sum_n \psi_n \ket{n}
  \quad
  \textup{or}
  \quad
  \ket\psi = \fint{x} \psi(x) \ket{x}.
\end{eq}
The  complete bases we will use in this thesis are the \emph{position basis} $\ket{\vec{r}}$, the \emph{plane wave basis} $\ket{\vec{k}}$, the \emph{harmonic oscillator basis} $\ket{nlm}$ and the elusive \emph{Berggren basis} \cite{berggren}. 
All these bases are orthonormal, i.e.~all the basis vectors satisfy 
\begin{eq}
  \braket{n}{n'} = \delta_{nn'}
  \quad
  \textup{or}
  \quad
  \braket{x}{x'} = \delta(x - x')
\end{eq}
depending on if the base is discrete or continuous.
With a complete basis $\ket{n}$, we get the very useful \emph{completeness relation}
\begin{eq}
  I = \sum_n \ket{n} \bra{n}
  \quad
  \textup{or}
  \quad
  I = \fint{x} \ket{x}\bra{x},
\end{eq}
where $I$ is the identity operator. This relation can thus be inserted anywhere in any equation, and will find frequent use in this thesis.

Let us now expand the TISE in the abstract $\ket{n}$ basis. We start by inserting the completeness relation for $\ket{n}$ in \cref{eq:TISE}
\begin{eq}
  \label{eq:expand1}
  H
  \p{
    \sum_{n'} \ket{n'} \bra{n'}
  }
  \ket\psi
  =
  \sum_{n'} H \ket{n'} \braket{n'}{\psi}
  =
  E \ket\psi.
\end{eq}
Multiplying \cref{eq:lincomb} with $\bra{n}$ from the left and using orthonormality, we see that $\braket{n'}{\psi} = \psi_{n'}$. If we now close \cref{eq:expand1} with $\bra{n}$ on the left
\begin{eq}
  \label{eq:expand2}
  \sum_{n'} \bra{n} H \ket{n'} \psi_{n'}
  = 
  E \braket{n}{\psi},
\end{eq}
and write $H_{nn'} = \bra{n} H \ket{n'}$, we get
\begin{eq}
  \label{eq:expand3}
  \sum_{n'} H_{nn'} \psi_{n'} = E \psi_n,
\end{eq}
which is equivalent to the matrix equation \cref{eq:matrix equation}. This is 
the basic method of expanding the TISE in a basis.
\todo{Only use (1.4) for equations? DOUBLECHECK}

\section{Spherical Symmetry}
\label{sec:spherical symmetry}

We limit ourselves to spherically symmetric systems, i.e. systems with a potential $V(r)$ that only depends on the radial distance $r$.
If we considered three-dimensional systems with arbitrary potentials, the matrices would be very large and solving the problem would become infeasible.

Spherical symmetry allows us to write the wavefunction $\psi(\vec{r})$ as a product of a radial wavefunction $R(r)$ and the spherical harmonics $Y_l^m(\Omega_r)$, $\Omega_r = (\theta,\, \varphi)$.
\begin{eq}
<<<<<<< HEAD
  \psi(\vec{r}) = R(r) Y_l^m(\theta, \varphi).
=======
  \psi(\vec{r}) = R_l(r) Y_l^m(\Omega_r).
>>>>>>> f05a509298b040f9698a285eaf09835448519b33
\end{eq}
Here $l$ and $m$ are the quantum numbers for the orbital angular momentum and its projection along an arbitrary $z$-axis. \todo{$j$ in here somewhere?}
For the matrix elements we find, using the orthonormality of the spherical harmonics,
\begin{eq}
  \bra{nlm}V\ket{n'l'm'} 
  &= \fint[0][\inf]{r} 
    r^2 \conj{R_l(r)}R_{l'}(r)V(r) 
    \fint{\Omega_r} 
      \conj{Y_l^m(\Omega_r)}Y_{l'}^{m'}(\Omega_r)
  \\ & = 
  \delta_{mm'}\delta_{ll'}\fint[0][\inf]{r} 
    r^2 \conj{R_l(r)}R_{l'}(r)V(r)
\end{eq}
meaning that the matrix will be \emph{block diagonal}, illustrated in \cref{fig:bmatrix}. This means that systems with different $l$ and $m$ can be treated separately, thus significantly reducing computation time. We say that \emph{$H$ is diagonal in $l$ and $m$}.
%Their corresponding operators, $L^2$ and $L_z$, commute with the Hamiltonian $H$, and will result in a factor of $\delta_{ll'}\delta_{mm'}$.
%This delta factor puts the restraint $l=l'$ and $m=m'$ on the SE, giving us

%\begin{eq}
%  \sum_{n'} \bra{nlm} H \ket{n'lm} \psi_{n'lm} = E\psi_{nlm}.
%\end{eq}
%\todo{ket nlm can be confused with the harmonic oscillator.}
%If we now treat $\bra{nlm} H \ket{n'lm} = H_{nn'}$ as a matrix,
%we notice that the the matrix is block diagonal in $l$ and $m$, i e there are no terms between states with $lm != l'm'$, this is presented graphically in \cref{fig:bmatrix}.
%We say that \emph{$H$ is diagonal in $l$ and $m$}. \todo[inline]{first step...}

\begin{figure}
\center
\includegraphics{../figures/block_matrix/matrix.pdf}	
\caption{An illustration of a block diagonal matrix. The eigenvalues of different blocks are independent of each other. Thus the blocks can be diagonalized separately.}
\label{fig:bmatrix}
\end{figure}

\section{The Harmonic Oscillator Basis}
\label{sec:harmosc}

We now expand the TISE in the spherically symmetric Harmonic Oscillator (HO) basis. The basis consists of the eigenstates $\ket{nlm}$ of the HO Hamiltonian
\begin{eq}
  \label{eq:HO_hamiltonian}
  H\sub{HO} = \frac{p^2}{2\mu} + \frac{\mu\omega^2 r^2}{2},
\end{eq}
where $\mu$ is the mass of the problem and $\omega$ is the angular frequency of the oscillator. 
The expansion procedure is the same as in \cref{eq:expand1,eq:expand2,eq:expand3} and gives us
\begin{eq}
  \sum_{n'l'm'} \bra{nlm} H \ket{n'l'm'} \psi_{n'l'm'} = E \psi_{nlm}
\end{eq}
and if we use the fact that $H$ is diagonal in $l$ and $m$ we get
\begin{eq}
  \sum_{n'} \bra{nlm} H \ket{n'lm} \psi_{n'lm} = E\psi_{nlm}.
\end{eq}
We now have a matrix equation, but we need to find the matrix elements $\bra{nlm} H \ket{n'lm}$. These require some calculation (details are in \cref{sec:HO matrix elements}) and the result is
\begin{eq}
  \label{eq:HO_matrix_elements}
  &
  \bra{nlm} H \ket{n'lm} =
	\frac{\hbar\omega}{2}
	\left(
    \p{2n+l+\frac{3}{2}} \delta_{nn'}
    +
		\sqrt{n(n+l+\frac{1}{2})} \delta_{n,n'-1}\right.
		\\ & + 
		\left.\sqrt{n'(n'+l+\frac{1}{2})} \delta_{n',n-1} 
	\right)
	+
	\fint[0][\inf]{r} 
    r^2 R_{nl}(r) V(r) R_{n'l}(r)
\end{eq}
where $R_{nl}$ are the radial wavefunctions of the harmonic oscillator,
\begin{eq}
  \label{eq:HO_radial_wavefunction}
	R_{nl}(r) 
	= 
  \sqrt{\frac{
    2^{l+2}  (n - l)!!
  }{
    \sqrt\pi r_0^{2l + 3} (n + l + 1)!!
  }}
	r^l \exp\p{-\frac{r^2}{2r_0^2}}
	L_{(n-l) / 2}^{(l+\frac{1}{2})}\p{\frac{r^2}{r_0^2}},
\end{eq}
$r_0 = \sqrt{\hbar/\mu\omega}$ is the range and $L_\nu^\lambda(x)$ are the generalized Laguerre polynomials.
\todo{Doublecheck normalization constant.}
The radial wavefunction $R(r)$ of a state will be expressed as a linear combination of the harmonic oscillator radial wavefunctions:
\begin{eq}
  R(r) = \sum_n \psi_{nl} R_{nl}(r).
\end{eq}

\section{The Plane Wave Basis}
\label{sec:mom_space}
\todo{Relation to momentum space $p = \hbar k$, plane wave is what?}
\todo{Do we introduce this earlier, since we use the word since? (open system, few/any bound solutions)}
The plane wave basis $\ket{\vec{k}}$ is of great importance for us since we are studying a system with a short-range potential supporting only a few, if any, bound solutions.
This means that we will find multiple unbound solutions, corresponding to free particles with various momenta.
These are almost eigenstates of the momentum operator, only slightly perturbed by the small potential well at $r=0$. 

The expansion is done in the same way as before, giving us
\begin{eq}
  \int \rd^3 \vec{k}' \bra{\vec{k}} H \ket{\vec{k}'} \Phi(\vec{k}')
  &= 
  E\Phi(\vec{k}),
\end{eq}
where we denote the wavefunctions in momentum space with $\Phi$.
This three-dimensional equation is practically unsolvable in general, but we can simplify it using the fact that we have a central problem. 
\todo[inline]{Redundant? same stuff in sphere symm section}
A rather involved calculation (see \cref{app:radial_mom_TISE}) shows that the Schrödinger equation can be written as
\begin{eq} 
  \label{eq:radial mom space TISE}
  H\phi(k)
  =
  \frac{k^2}{2\mu}\phi(k) + \fint[0][\inf]{k'} k'^2 V(k,k') \phi(k') 
  =
  E\phi(k)
  \\
  V(k,k') 
  = 
  \frac{2}{\pi}\fint[0][\inf]{r} r^2 V(r) j_l(kr) j_l(k'r),
\end{eq}
\todo{align -is'nt this fixed now, remove it if that is the case.}
where $\phi(k)$ is the radial part of the momentum space wavefunction
and $j_l(kr)$ are the spherical bessel functions of order $l$. 
The momentum space radial wavefunction $\phi(k)$ is related to the position space radial wavefunction by
\begin{eq}
  R(r)=i^l\sqrt{\frac{2}{\pi}} \fint[0][\inf]{k} k^2 \phi(k)j_l(kr).
  \label{eq:radial wavefunction}
\end{eq}

\subsection{Discretization and Symmetrization}
\label{sec:mom discretization}
The integral equation \cref{eq:radial mom space TISE} can be rewritten as a matrix equation through discretization, turning the integral into a sum over a finite set of points $k_j$ and $\rd{k}$ into a set of weights $w_j$:
\begin{eq}
  \label{eq:discrete_momentum}
  \frac{k_i^2}{2\mu} \phi(k_i)
  +
  \sum_{j=1}^N w_j
    k_j^2 V(k_i,k_j)
  \phi(k_j)
  =
  E \phi(k_i)
  .
\end{eq}
\todo{source on this or explain that we tried the naive way to do it.}
A particular set of points and corresponding weights is called a \emph{quadrature}, and the choice of quadrature greatly impacts the precision of the result. 
A naïve quadrature with evenly spaced $k_j = j\Delta k$ and a constant weight $w_j=\Delta k$ converges slowly, and should not be used.
We instead use the Gauss-Legendre quadrature, for details see \cref{app:gauss-legendre}.

With this discretization the Schrödinger equation may be written as 
\begin{eq}
  \sum_j H_{ij} \phi(k_j) &= E \phi(k_i)
\end{eq}
where
\begin{align}
  \label{eq:mom_matrix}
  H_{ij} &= \frac{k_i^2}{2\mu}\delta_{ij} + w_jk_j^2 V_{ij} \\
  \label{eq:potential matrix}
  V_{ij} &= \frac{2}{\pi} \fint[0][\inf]{r} r^2 V(r) j_l(k_i r) j_l(k_j r).
\end{align}

Because of the $k_j^2$ in the second term of the matrix elements 
(\cref{eq:mom_matrix}), the $H_{ij}$ matrix will not be symmetric. 
Working with a symmetric matrix is preffered since it halves the amount of time the eigensolver takes, hence we perform \todo{why?/explain more}\todo{where do we mention the need for symmetrized matrix in he6?}
the transformation
\begin{eq}
  \phi(k_i) &\mapsto
  \phi'(k_i) =  \sqrt{w_i} k_i \phi(k_i)
  \\
  H_{ij} &\mapsto
  H_{ij}' 
  = 
  \sqrt{\frac{w_i}{w_j}} \frac{k_i}{k_j}H_{ij},
\end{eq}
which gives us a symmetric matrix
\begin{eq}
  \label{eq:plane_wave_matrix_elements}
  H_{ij}' = \frac{k_i^2}{2\mu}\delta_{ij} + \sqrt{w_i w_j}k_i k_j V_{ij}.
\end{eq}
The Schrödinger equation then becomes
\begin{eq}
  \sum_j H'_{ij}\phi'(k_j) = E\phi'(k_j),
\end{eq}
with the same eigenvalues $E$, meaning that we can work with the symmetric $H'_{ij}$ matrix.
 The real space radial wavefunction $R(r)$ is expressed in terms of $\phi'(k_j)$ as
\begin{eq}
  R(r)
  =
  i^l 
  \sqrt{\frac{2}{\pi}}
  \sum_{j=1}^N 
    \sqrt{w_j}k_j \phi'_j j_l(k_j r).
\end{eq}

\section{Numerical Considerations}
\todo{point out that symmetrization gives correct normalization}
\todo{calculation vs computation (entire report)}

\todo{order of sections}
In order to perform basis expansion on a computer, we need to consider the numerical aspects of the problem. 
This includes truncation of the basis, matrix size reduction, numerical integration and eigensolver optimizations.
The plane wave basis is continuous and thus requires special treatment.

The $\ket{nlm}$ and $\ket{k}$ bases are infinite, so we truncate them by only including a finite number $N$ of states in the basis. 
For an orthonormal basis, the best approximation is to include the $N$ first states.
The more states we include in the basis, the more accurate results we get. This is illustrated in \cref{fig:hydrogen_convergence}.

\todo{should the last things of the figure (ho similar to hydrogen etc.) be mentioned in the text as well?}
\todo{two figures}

The truncation gives us an $N \times N$ Hamiltonian matrix $H$.
Since $H$ is diagonal in $l$ and $m$, we can consider each value of $l$ and $m$ separately, reducing the size of the matrix and thus the amount of computation needed to solve for the eigenvalues.
The equation is then, for a given $l$ and $m$,
\begin{eq}
  \label{eq:symm matrix elem}
  \sum_{n'= 0}^N \bra{n} H \ket{n'} \psi_{n'} = E\psi_{n},
\end{eq}
or in matrix notation
\begin{eq}
  \label{eq:matrix eq}
  H\psi = E\psi.
\end{eq}

We compute the matrix elements $H_{ij}$ with \cref{eq:HO_matrix_elements,eq:plane_wave_matrix_elements}, carrying out the integrals with the Gauss-Legendre quadrature rule (see \cref{app:gauss-legendre}) and setting the upper limit to a finite number.
\todo{Gauss-Legendre in many places, where is canon?}
If the matrix is hermitian or symmetric, we only need to consider the elements in the upper triangle including the diagonal, roughly halving the number of computed elements. 

When the matrix elements have been computed, the matrix is diagonalized using a standard eigensolver algorithm. For hermitian matrices we use a faster specialized algorithm.

\end{document}