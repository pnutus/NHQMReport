As we previously mentioned (depends on where this text is placed.), NHQM arises when one observes the binding momentum, related to the energy eigenvalues as
\begin{equation}
    p=\sqrt{2mE},
\end{equatoin}
where the energy $E$ might be complex.
This energy will be complex if we expand the hamiltonian in a Berggren-basis of a state that includes a resonance.

\section{The Berggren basis}
Previously, we calculated the energy eigenvalues in momentum-space \cref{sec:mom-space}, but now we introduce a generalization of this.
If we look at our momentum-space expansion, we see that it consists of some discrete momentum-values along the real axis.
This is fine as long as the states we observe are bound.
If they on the other hand are quasi-bound resonance-states, they will appear as poles in the complex momentum-plane.
With knowledge of Cauchy's Residue theorem, we know that we must integrate around this pole to see to it's contribution, which in our case is the resonance's behavior.

The way to do this is to instead of using a strictly real basis introduce a complex one, called a Berggren-basis (Should we cite here?).
The Berggren-basis is expressed as a contour in the complex plane, the one we used, with slight modifications, is seen in \cref{fig:}.
By lowering this contour to contain the pole we include the resonance in the calculations and are thus able to predict its behavior.
Last, one have to include the resonance pole in the basis for it to be complete\cite{berggren}.

{\Large Should this above move to introduction, or should we have it here?}

\section{Results}
By remodelling our old mom-space solution to allow different contours, we were ready to get som results to see if this new model was correct.

{\Large RESULTS!!!!!!!!!!!!!!!}
Får Ola ordning på plottarna är denna sectionen snart klar, tror jag...