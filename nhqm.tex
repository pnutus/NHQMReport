As we previously mentioned, NHQM arises when one observes the binding momentum, related to the energy eigenvalues as
\begin{equation}
    p=\sqrt{2mE}.
\end{equatoin}
This energy will be complex if we use the NHQM-approach by expanding the hamiltonian in a Berggren-basis to describe the resonance.

\section{The Berggren basis}
Previously, we calculated the energy eigenvalues in momentum-space \cref{sec:mom-space}, now we will introduce a generalization of this.
If we look at our momentum-space expansion, we see that it consists of some discrete momentum-values along the real axis.
This is fine as long as the states we observe are bound.
If they on the other hand are quasi-bound resonance-states, they will appear as poles in the complex momentum-plane.
By knowing of Cauchy's Residue theorem, we know that we must integrate around this pole to take care of it's contribution.

The way of doing this is to instead of using a strictly real basis introduce a complex one, called a Berggren-basis (Should we cite here).
The Berggren-basis is expressed as a contour in the complex plane, the one we used, with slight modifications, is seen in \cref{fig:}.