\documentclass[12pt,a4paper]{article}
\usepackage[swedish]{babel}

\usepackage{../main/NHQM}

\begin{document}
  
%\numberwithin{equation}{chapter}
%\numberwithin{figure}{chapter}

\listoftodos


\title{Kvantresonanser behandlade i en komplex rörelsemängdsbas \\ 
\Large Kandidatrapport i fysik, svensk sammanfattning}
\author{Jonathan Bengtsson \and Ola Embréus \and Vincent Ericsson \and Pontus Granström \and Nils Wireklint}
\date{\today}
\maketitle
\newpage

\todo{Bör ha med kurskod}


Beteendet hos ett kvantmekaniskt system beskrivs av dess Hamiltonian som består av en rörelsmängdsterm och en potentialterm.
Beroende på potentialens form kan systemet ge upphov till bundna och obundna tillstånd, som svarar mot diskreta respektive kontinuerliga energispektrum.
Partiklar i ett så kallat \emph{stängt kvantsystem} är alltid bundna och lokaliserade i en ändlig omgivning av potentialen. 
Ett \emph{öppet kvantsystem} kan ha ett spektrum som är både diskret och kontinuerligt.
I det kontinuerliga spektrumet hos ett öppet kvantsystem kan uppstå något som kallas \emph{resonans}.
Ett resonanstillstånd är obundet, men diskret och lokaliserat, och brukar kallas \emph{kvasibundet}.

Syftet med detta projekt är att studera dessa resonanser, specifikt \He{5} och \He{6}, lätta atomkärnor som är obunden resp. löst bunden.
Vi gör detta genom att betrakta den tidsoberoende Schrödingerekvationen, ett egenvärdesproblem
\begin{eq}
  H \ket\psi = E \ket\psi.
\end{eq}
I denna formalism kan ett resonanstillstånd beskrivas med en komplex energi $E = E_0 - i\Gamma/2$, vilket modellerar tillståndets korta livslängd.

Vi studerar Schrödingerekvationen med numeriska beräkningar. Då behöver ekvationen skrivas om på en passande form.
Detta görs genom att utveckla ekvationen i en fullständig bas där egenvärdesproblemet kan skrivas med en ändlig Hamiltonianmatris $H$. 
Då kan egenvärdesproblemet lösas med en så kallad egenlösaralgoritm.
Vi gör en naturlig inskränkning till sfäriskt symmetriska problem, vilket minskar komplexiteten hos problemet och leder till mindre matriser och snabbare beräkningar.

Rapporten behandlar två fullständiga baser, den sfäriska harmoniska oscillatorns bas och rörelsemängdsbasen. 
Rörelsemängdsbasen är kontinuerlig och kräver diskretisering som utförs med hjälp av Gauss-Legendrekvadratur. 
Kvadraturen används även för att beräkna de integraler vid beräkning av matriselement.

Genom en skalmodell kan \He{5}-kärnan betraktas två partiklar, en alfapartikel och en valensneutron. Detta tillåter att problemet reduceras till en dimension.
Alfapartikeln och neutronen växelverkar med en Woods-Saxonpotential:
\begin{eq}
	w-s.
\end{eq}
De parametrar som använts är: PARAMS
\todo{Bra upplägg med potential??}

Den bas vi börjar med att använda består av harmoniska oscillatorer, vilket ger oss matriselement på formen
\begin{eq}
2.15, ev 2.16 också.
\end{eq}
Det visade sig snart dock att denna bas inte räckte till när vi ville betrakta resonanser, då det enda vi kan se är en avvikelse hos den energinivån när vi plottar de lägsta energinivåerna mot oscillatorns räckvidd.

För att råda bot på detta introducerar vi en planvågsexpansion som går ut på att vi väljer vissa diskreta punkter i rörelsemängdsrummet och utvecklar problemet i denna bas.
Efter detta kunde vi nu se tydliga tendenser till en resonans hos \He{5}, men hade inget sätt att bestämma dess komplexa delar.

För att beskriva resonansen på ett bra sätt använder vi Tore \emph{Berggren}s teori och utökar basen till att även innehålla komplexa rörelsemängdsmoment.
Detta betyder att vi istället för att låta våra planvågspunkter ligga längs den reella axeln låter de gå ner i den fjärde kvadranten och runt den pol som svarar mot resonansen.
Denna går då att lokalisera och vi varrierar våra parametrar för att polen ska ligga så nära den experimentellt bestämda resonansen som möjligt.
Resulattet av detta blev:
\todo[inline]{tabellsaker.}

\subsection{Flerkroppssaker}
Nu när vi har lokaliserat resonansen i \He{5} kan vi anse oss ha löst tvåkropparsfallet och vill då naturligtvis gå vidare med tre.
Detta är svårare då vi inte längre kan reducera problemet till en dimension och det faktum att matriserna nu tar en betydande tid att diagonalisera.

Det trekropparssystem vi vill undersöka är \He{6}, vilket är vår naturliga övergång ifrån \He{5}.
Denna kärna moddelerar vi som en alfapartikel med två valensneutroner.
Vi kan plocka interaktionen mellan dessa valensneutroner och kärnan från våra lösningar av \He{5}, men måste ta hänsyn till att de är fermioner och därför inte kan befinna sig i samma tillstånd.
Interaktionen mellan de två neutronerna beskriver vi med en grovt approximerad sfärisk potential.

I ett första steg bygger vi upp en bas utav energierna och vågfunktionerna vi fann i tvåkropparssystemet och avänder dessa för att konstruera möjliga simultana tillstånd för de två neutronerna.
För att beskriva flerkropparstillstånden kan vi antingen koppla de två enpartikeltillståndens rörelsemängsmoment eller räkna på en större uppsättning okopplade tillstånd. För att koppla rörelsemängdsmomenten utnyttjar vi ... 
\begin{eq}
	1+2+... = -\frac{1}{12}
\end{eq}
Vi får då kravet att den totala rörelsemängden ... 
Detta för att undvika att i onödan behadla kombinationer som inte ger bidrag till hamiltonianen.
Medan vi i det okopplade fallet testar alla möjligheter och inkluderar snävare krav i vilka tillstånd som ger bidrag till hamiltonianen. Beräkningstekniskt är det den kopplade konfigurationen som de facto används.

När vi sedan beräknar hamiltonianen kommer neutronerna växelvärka med kärnan på samma sätt som förut, men de kommer även att växelvärka med varandra.
Vi delar upp hamiltonianen i två delar, den ena är en operator för kinetisk energi och behandlar enpartikeltillstånden oberoende av varandra, den andra delen beskriver är även den en operator och beskriver växelvärkan mellan de två neutronerna.
Denna växelvärkan är extremt komplicerad och är inte heller helt känd, istället använder vi oss utav grova förenklingar genom att använda först en gaussisk interaktion och sedan en mer fysikalisk ytdeltainteraktion.
\todo[inline]{Upprepning från ovan?}


\subsection{saker}


\begin{figure}
\centering{
	\subfloat[$V_0=\SI{70}{MeV}$]{
		\includegraphics[page=1]{../figures/wavefunctions/wavefunctions.pdf}
	}
	\subfloat[$V_0 = \SI{47}{MeV}$]{
		\includegraphics[page=3]{../figures/wavefunctions/wavefunctions.pdf}
}



  \caption{
    \He{5} Vågfunktioner för olika starka potentialer. Den unika lokaliserade lösningen till SE markers i rött och en vanlig obunden tillstånd visas i grått.
    Till vänster uppvisar en stark potential $V_0 = \SI{70}{MeV}$ ett bundet tillstånd medan vi har ett lokaliserat obundet tillstånd tillvänster vid potentialen $V_0 = \SI{47}{MeV}$.
  } 
  }
  \label{fig:resonance wavefunction}
  \end{figure}


\end{document}